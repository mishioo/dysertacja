\section{Stereochemia przemiany}\label{numeric:stereo}

\begin{marginfigure}
  \includesvg{direction}
  \caption{
    Możliwe kierunki ataku nukleofila na~przejściowy kation iminiowy \refcmpd{glu-iminium}.
    W~obserwowanej przemianie, wbrew intuicji, preferowana jest addycja od~strony \textit{re}.
  }\label{fig:direction}
\end{marginfigure}

\Cref{fig:direction} przedstawia możliwe kierunki ataku nukleofila na~kation iminiowy
  \refcmpd{glu-iminium} \--- wywiedziony z~glukozy odpowiednik produktu pośredniego \refcmpd{int-4}
  w~proponowanym wcześniej mechanizmie.
Na~podstawie płaskiej reprezentacji tego związku, intuicja może podpowiadać, że ów atak
  nastąpi raczej od~strony \textit{si}, czyli przeciwnej do~sąsiedniego podstawnika
  benzoksylowego w~pozycji 3.
W~końcu stanowi on stosunkowo dużą zawadę steryczną, jego przeciwna strona powinna być
  łatwiej dostępna.

\begin{figure}
  \includesvg{xray}
  \caption{
    Uproszczone struktury związków \refcmpd{glu-tet.cy, glu-tet.pmb} przedstawione
      w~stylu ORTEP, w~reprezentacji elipsoid termalnych na~poziomie prawdopodobieństwa
      \SI{35}{\percent}.
    Pominięte zostały atomy wodoru oraz grupy benzylowe.
  }\label{fig:xray}
\end{figure}

A~jednak, dominującym produktem addycji do~wywiedzionego z~glukozy związku \refcmpd{glu-iminium},
  a~w~większości przypadków nawet jedynym, jest ten powstający w~wyniku ataku od~strony \textit{re}.
Niezbitym na~to dowodem jest analiza rentgenostrukturalna związków
  \refcmpd{glu-tet.cy, glu-tet.pmb}, której wyniki w~uproszczeniu przedstawia \cref{fig:xray},
  a~szczegółowy ich opis znajduje się w~sekcji \secref{experimental:xray}.
Fenomen nieoczekiwanego kierunku tej addycji wyjaśnić można na~podstawie modelu Woerpla,
  opisującego stereochemię analogicznej nukleofilowej addycji do~kationów oksokarboniowych,
  będących częścią sześcioczłonowego pierścienia.
Kluczową rolę w~tym modelu odgrywa stabilność poszczególnych konformerów kationu i~produktu.
Aby ułatwić dyskusję na~ten temat,
  na~\cref{fig:thp-conformers} przypominam ich nazewnictwo\sidecite[-2\baselineskip]{iupac81}.

\begin{figure}
  \includesvg{thp-conformers}
  \caption{
    Nomenklatura konformerów pochodnych tetrahydropiranu.
    Aby nazwać daną konformację niezbędne jest ustalenie jej kształtu:
      krzesełka (\textit{C}), pół-krzesełka (\textit{H}), łódki (\textit{B}),
      lub skręconej łódki (\textit{S}).
    Podanego w~nawiasie symbolu używa się do~oznaczenia danego kształtu.
    Należy ułożyć pierścień w~taki sposób, aby, patrząc \enquote{z~góry},
      numery pozycji rosły zgodnie z~ruchem wskazówek zegara.
    Następnym krokiem jest znalezienie czterech atomów pierścienia położonych
      w~jednej płaszczyźnie.
    Numery atomów położonych poza wyznaczoną płaszczyzną odniesienia należy dodać
      do~symbolu kształtu \--- numer atomu nad płaszczyzną podaje się przed symbolem w~indeksie
      górnym, a~atomu pod płaszczyzną za~symbolem w~indeksie dolnym.
  }\label{fig:thp-conformers}
\end{figure}

\subsection{Model Woerpla}
W~modelu Woerpla\sidecite[2em]{woerpel00} kierunek addycji nukleofila do~cyklicznego kationu
  oksokarboniowego zależny jest nie od~bezpośredniego otoczenia centrum reakcyjnego,
  ale od~stabilności konformacyjnej powstających produktów.
Ze~względu na~płaską geometrię wiązania \ch{-C=O^+\bond{single}}, sześcioczłonowy cykliczny
  kation oksokarboniowy będzie znajdował się zawsze w~konformacji pół-krzesełkowej
  \cycloconf{3}{H}{4} lub \cycloconf{4}{H}{3}.
Atak nukleofila na~to wiązanie, zależnie od~strony, z~której nastąpi, może prowadzić do~powstania
  produktu w~konformacji krzesełkowej lub w~konformacji skręconej łódki,
  tak jak obrazuje to \cref{sch:woerpel}.
Preferowany będzie zawsze atak prowadzący do~powstania konformacji krzesełkowej,
  ponieważ biegnie on poprzez niżej energetyczny stan przejściowy o~analogicznej konformacji.

\begin{scheme}
  \includesvg{woerpel}
  \caption{
    Kierunek addycji nukleofila do~kationu oksokarboniowego według modelu Woerpla zależny jest
      od~stabilności konformacyjnej potencjalnego produktu addycji oraz samego kationu.
    Możliwe ścieżki przemiany przedstawiam na~przykładzie \iupac{4-podstawionej} piranozy.
  }\label{sch:woerpel}
\end{scheme}

Aby przewidzieć przebieg reakcji w~tym modelu, należy wpierw ustalić, która z~konformacji
  rozważanego kationu będzie stabilniejsza \--- \cycloconf{3}{H}{4} czy \cycloconf{4}{H}{3}.
Gdy to już wiadomo, można wnioskować czy w~przewadze będzie powstawał produkt ataku, odpowiednio,
  \enquote{z~góry}, czyli \textit{anti} do~podstawnika w~pozycji 4 w~przykładzie
  ze~\cref{sch:woerpel}, czy \enquote{z~dołu}, czyli \textit{syn} do~tego podstawnika\sidenote{
    Z~rozmysłem nie używam tu terminologi \textit{re} i~\textit{si}, gdyż nazwy stron wiązania
      mogą być różne, zależnie od~faktycznych podstawników w~pierścieniu.
  }.
\citeauthor{woerpel99} zaproponowali również taki model dla pierścieni
  pięcioczłonowych\sidecite{woerpel99}, nie będę go tu jednak omawiał.
Warto natomiast wspomnieć, że rozstrzygnięcie, który z~konformerów cyklicznego kationu
  oksokarboniowego będzie trwalszy, może nie być trywialne ze~względu na~występowanie
  efektu anomerycznego w~pierścieniach zawierających heteroatomy\sidecite{grindley08}.
Ponadto, intuicja, podpowiadająca, że podstawniki w~pozycji aksjalnej nie będą korzystne może
  zawieść, bowiem w~przypadku takich pierścieni zależy to od~elektronowego charakteru
  podstawnika\sidecite{woerpel00}.

Wykonana wcześniej w~zespole badawczym Furmana analiza konformacyjna wywiedzionej z~glukozy
  iminy\sidecite{furman14acs} sugeruje zgodność wyników z~modelem Woerpla.
Analiza została przeprowadzona na~uproszczonym modelu, w~którym podstawniki benzoksylowe
  zostały zastąpione zbliżonymi elektronowo, ale nie tak labilnymi grupami metoksylowymi.
Niżej energetyczny jest konformer \cycloconf{4}{H}{3}, w~przypadku którego preferowany jest
  atak nukleofila \enquote{od~dołu}, prowadzący do~powstania produktu o~takiej samej konfiguracji,
  jak wydzielany przeze mnie związek \refcmpd{glu-tet.cy}.

\begin{scheme}
  \includesvg{gluco-imine-confs}
  \caption{
    Porównanie stabilności konformerów cyklicznej iminy o~konfiguracji glukozy,
      według badań DFT przeprowadzonych przez zespół badawczy Furmana.
    Addycja nukleofilowa do~stabilniejszego konformeru prowadziłaby do~powstania centrum
      chiralnego o~konfiguracji \iupac{\cip{2R}}.
  }\label{sch:gluco-imine-confs}
\end{scheme}

Różnica w~stabilności konformerów \cycloconf{3}{H}{4} i~\cycloconf{4}{H}{3}, pokazanych
  na~\cref{sch:gluco-imine-confs} jest znaczna i~wynosi \SI{4.5}{\kcalpm}.
Stosując analizę rozkładu Boltzmanna można oszacować w~jakiej proporcji te konformery
  będą występowały w~mieszanie reakcyjnej\sidenote{
    O~detalach technicznych tej analizy piszę więcej w~sekcji \secref{tesliper:boltzmann}.
  }.
Wskazuje on, że w~temperaturze pokojowej stabilniejszy z~konformerów będzie stanowił zdecydowaną
  większość, około \SI{99.95}{\percent} składu.
Estymacja ta dobrze koresponduje z~eksperymentem, w~którym obserwowałem tylko jeden
  z~diastereomerów produktu.

\subsection{Konfiguracja produktów wywiedzionych z~galaktozy}
Ustalenie konfiguracji absolutnej centrum stereogenicznego w~pozycji 2. wywiedzionych
  z~galaktozy związków o~strukturze \refcmpd{gal-tet} okazało się znacznie większym wyzwaniem.
Wszystkie próby otrzymania monokryształu któregoś z~tych produktów zakończyły się fiaskiem,
  uniemożliwiając wykorzystanie najbardziej bezpośredniej metody,
  czyli rentgenografii strukturalnej.
Korzystając ze zmodyfikowanej procedury syntetycznej\sidenote{
    Opisuję ją w~sekcji \textit{\nameref{experimental:iminosugars}},
      str.~\pageref{syn:gal-epi-tet.cy}.
  } udało mi się otrzymać niewielką ilość związku \refcmpd{gal-epi-tet.cy},
  będącego diastereomerem produktu \refcmpd{gal-tet.cy}.
Mogłem dzięki temu uciec się do~innych dostępnych metod \--- analizy \NMR*{} i~efektów \gls{noe}
  oraz spektroskopii chiralooptycznej, które dalej opisuję.

Zwykle można wnioskować o~konfiguracji absolutnej centrum stereogenicznego w~obrębie pierścienia
  na~podstawie analizy stałych sprzężenia położonego przy nim protonu\sidenote{
    Zakładając, oczywiście, że położony jest w~sąsiedztwie innego protonu,
      z~którym może się sprzęgać.
    W~opisywanym przypadku rolę tę pełni proton w~pozycji 3.
  }.
Sama znajomość tej stałej sprzężenia jest zazwyczaj niewystarczająca, ale porównanie stałych,
  z~jakimi sprzęgają protony w~obydwu konfiguracjach, często pozwala skutecznie zidentyfikować
  te konfiguracje.
Stała sprzężenia między protonami \ch{H^2} oraz \ch{H^3} w~związku \refcmpd{gal-tet.cy} wynosi
  $J = \SI{8.5}{\hertz}$, co nie może zostać wprost powiązane z~konkretną konfiguracją
  centrum w~pozycji 2.
Stała w~diastereomerycznym związku \refcmpd{gal-epi-tet.cy} jest, niestety,
  niemożliwa do~odczytania ze~względu na~poszerzenie i~nakładanie się sygnałów.

\begin{marginfigure}
  \includesvg{gal-noe}
  \caption{
    Proponowane struktury związków \refcmpd{gal-tet.cy, gal-epi-tet.cy} z~zaznaczonymi
      efektami \gls{noe}.
  }\label{fig:gal-noe}
\end{marginfigure}

Ten sam problem uniemożliwia skuteczne przeprowadzenie analizy efektów \gls{noe},
  wizualizowanych na~\cref{fig:gal-noe}.
O~ile w~związku \refcmpd{gal-tet.cy} efekty między poszczególnymi protonami są łatwe
  do~zidentyfikowania, to w~związku \refcmpd{gal-epi-tet.cy} są niejednoznaczne.
Ponadto, analiza zidentyfikowanych efektów nie jest wcale łatwa do~interpretacji \---
  widoczne są oddziaływania protonu \ch{H^2} zarówno z~protonem \ch{H^4} jak i~protonami
  \ch{H^7} po~przeciwnej stronie pierścienia.
Struktury obydwu związków, przedstawione na~\cref{fig:gal-noe}, są propozycjami wstępnymi,
  opartymi o~przypuszczenie, że protony \ch{H^2} i~\ch{H^4}, znajdujące się w~obrębie
  pierścienia, powinny być położone po~tej samej jego stronie.

Inną współczesną metodą ustalania konfiguracji absolutnej związków chiralnych jest analiza
  ich widm chiralooptycznych, na~przykład widm \gls{ecd}\sidecite{pescitelli22}.
Polega to zwykle na~porównaniu widma zarejestrowanego z~widmami symulowanymi na~podstawie
  rozważanych konfiguracji absolutnych badanego związku.
Widma symulowane otrzymuje się, opracowując dane otrzymane na~drodze analizy konformacyjnej
  i~obliczeń kwantowo chemicznych\sidecite{pescitelli16}.
Typowy przebieg tego procesu dokładniej opisuję w~kolejnym rozdziale\sidenote{%
    W~sekcji \secref{tesliper:process}.}. 
W~tym miejscu chcę jedynie zaznaczyć, że zarówno dokładność analizy konformacyjnej,
  jak i~poziom teorii przyjęty podczas obliczeń ma znaczny wpływ na~skuteczność tych wysiłków.

\Cref{fig:spectra-ecd} przedstawia porównanie eksperymentalnych widm \gls{ecd} związków
  \refcmpd{gal-tet.cy, gal-epi-tet.cy} z~obliczonymi dla proponowanych struktur tych
  związków\sidenote{Detale techniczne odnośnie rejestracji i~symulacji tych widm
    znajdują się na~końcu niniejszej dysertacji w~sekcji \secref{experimental:spectra}.}.
Niestety, już pierwszy rzut oka pozwala sądzić, że porównanie to nie wnosi wiele.
Dominujące pasmo w~zakresie \SIrange{180}{200}{\nano\meter} pochodzi najpewniej od~grup
  \ch{-OBn} i~nie wnosi nic do~analizy.
Pozostałe pasma, lepiej widoczne na~\cref{fig:spectra-ecd-closeup}, są słabo wykształcone
  i~trudne do~interpretacji.
Żadne z~symulowanych widm nie koresponduje dobrze z~eksperymentem, nie pomagając tym samym
  w~ustaleniu konfiguracji absolutnej dyskutowanych tu diastereomerów.

\begin{figure}
  \begin{tikzpicture}
    \begin{axis}[
      ylabel={$\Delta\epsilon\ /\si{\deci\meter\cubed\per\mole\per\centi\meter}$},
      xlabel={$\lambda\ /\si{\nm}$},
      ymin=-15, ymax=18,
      xmin=175, xmax=300,
      axis lines=left,
      axis line style={-},
      legend style={at={(0.97,1.05)}, anchor={north east}, draw=none},
      width=\textwidth,
      height=15em,
    ]
    % zero line
    \draw[ultra thin,color=gray]
      (axis cs:\pgfkeysvalueof{/pgfplots/xmin},0) -- (axis cs:\pgfkeysvalueof{/pgfplots/xmax},0);
    
    \addplot[color=wongvermillion]
      table {chapter-3/stereochemistry/gal-ecd195-data.txt};
    \addlegendentry{\refcmpd{gal-tet.cy}}
      
    \addplot[color=wongblue]
      table [y expr=\thisrow{y}*10] {chapter-3/stereochemistry/epi-ecd195-data.txt};
    \addlegendentry{\refcmpd{gal-epi-tet.cy}$\ /0.1$}
    
    \addplot[color=wongpurple]
      table [x expr=\thisrow{x-s-ecd}-15, y=y-s-ecd]
      {chapter-3/stereochemistry/simulated-data.txt};
    \addlegendentry{Sym. \iupac{\cip{2S}}}
    
    \addplot[color=wonggreen]
      table [x expr=\thisrow{x-r-ecd}-15, y=y-r-ecd]
      {chapter-3/stereochemistry/simulated-data.txt};
    \addlegendentry{Sym. \iupac{\cip{2R}}}
      
    \end{axis}
  \end{tikzpicture}

  \caption{
    Zarejestrowane widma \gls{ecd} badanych związków \refcmpd{gal-tet.cy, gal-epi-tet.cy}
      w~porównaniu z~ich symulowanymi odpowiednikami.
    Dla lepszej czytelności wartości intensywności poszczególnych widm zostały przeskalowane
      o~współczynnik podany w~legendzie.
  }
  \label{fig:spectra-ecd}
\end{figure}

\begin{figure}
  \begin{tikzpicture}
    \begin{axis}[
      ylabel={$\Delta\epsilon\ /\si{\deci\meter\cubed\per\mole\per\centi\meter}$},
      xlabel={$\lambda\ /\si{\nm}$},
      ymin=-0.3, ymax=0.3,
      xmin=215, xmax=300,
      axis lines=left,
      axis line style={-},
      legend style={at={(0.97,1.1)}, anchor={north east}, draw=none},
      width=\textwidth,
      height=15em,
    ]
    % zero line
    \draw[ultra thin,color=gray]
      (axis cs:\pgfkeysvalueof{/pgfplots/xmin},0) -- (axis cs:\pgfkeysvalueof{/pgfplots/xmax},0);
    
    \addplot[color=wongvermillion]
      table {chapter-3/stereochemistry/gal-ecd220-data.txt};
    \addlegendentry{\refcmpd{gal-tet.cy}}
      
    \addplot[color=wongblue]
      table {chapter-3/stereochemistry/epi-ecd220-data.txt};
    \addlegendentry{\refcmpd{gal-epi-tet.cy}}

    \addplot[color=wongpurple]
      table [x expr=\thisrow{x-s-ecd}-15, y expr=\thisrow{y-s-ecd}*0.1]
      {chapter-3/stereochemistry/simulated-data.txt};
    \addlegendentry{Sym. \iupac{\cip{2S}}$\ /10$}
    
    \addplot[color=wonggreen]
      table [x expr=\thisrow{x-r-ecd}-15, y expr=\thisrow{y-r-ecd}*0.1]
      {chapter-3/stereochemistry/simulated-data.txt};
    \addlegendentry{Sym. \iupac{\cip{2R}}$\ /10$}
      
    \end{axis}
  \end{tikzpicture}
  \caption{
    Zarejestrowane w~węższym zakresie widma \gls{ecd} badanych związków
      \refcmpd{gal-tet.cy, gal-epi-tet.cy} w~porównaniu z~ich symulowanymi odpowiednikami.
    Dla lepszej czytelności wartości intensywności poszczególnych widm zostały przeskalowane
      o~współczynnik podany w~legendzie.
  }
  \label{fig:spectra-ecd-closeup}
\end{figure}

Wspomniana wcześniej znaczna labilność badanych układów gra istotną rolę w~niepowodzeniu
  opisanych analiz.
Analiza konformacyjna proponowanych struktur prowadzi do~otrzymania tysięcy konformerów
  do~optymalizacji, a~nawet po~restrykcyjnym odsiewie tych podobnych i~mniej stabilnych,
  wciąż pozostają setki konformerów o~potencjalnie istotnym wpływie na~kształt widma.
Prawdopodobnie korzystny byłby wybór wyższego poziomu teorii do~symulacji aktywności
  optycznej lub objęcie nimi szerszego zakresu konformerów, jednak czas trwania i~koszt
  takich obliczeń byłby niewspółmierny do~ich wartości dla tej pracy.
Alternatywą mogłoby być uproszczenie modelu przez zamianę labilnych grup
  benzoksylowych na~prostsze, na~przykład metoksylowe, nie ma jednak gwarancji,
  że taki eksperyment byłby owocny\sidenote[][-12em]{%
    Krytyczny czytelnik może zastanawiać się dlaczego opisuję te próby, skoro wyraźnie
      nie przyniosły oczekiwanych rezultatów i~nie dały żadnych odpowiedzi.
    Otóż badania te \--- a~w~szczególności uciążliwość i~mozolność analizy wyników obliczeń
      obejmujących dużą ilość konformerów \--- były dla mnie bodźcem do~stworzenia
      oprogramowania komputerowego, mającego ułatwić taką pracę.
    Jest mu poświęcony kolejny rozdział nieniejszej dysertacji.
  }.

W~obliczu niepowodzenia analizy chiralooptycznej i~wątpliwości w~kwestii inwestowania 
  w~nią dalszych wysiłków, odwołałem się ponownie do~modelu Woerpla.
Cytowane wcześniej badania \gls{dft} przeprowadzone na~uproszczonym modelu przez grupę
  Furmana\sidecite{furman14acs} obejmują również układ o~konfiguracji galaktozy.
Jak widać na~\cref{sch:galacto-imine-confs} równowaga między konformerami iminy jest odwrotna niż 
  w~przypadku pochodnej o~konfiguracji glukozy.

\begin{scheme}
  \includesvg{galacto-imine-confs}
  \caption{
    Porównanie stabilności konformerów cyklicznej iminy o~konfiguracji galaktozy,
      według badań DFT przeprowadzonych przez zespół badawczy Furmana.
    Addycja nukleofilowa do~stabilniejszego konformeru prowadziłaby do~powstania centrum
      chiralnego o~konfiguracji \iupac{\cip{2S}}.
  }\label{sch:galacto-imine-confs}
\end{scheme}
  
Różnica energii wynosi \SI{1.5}{\kcalpm} na~korzyść konformeru \cycloconf{3}{H}{4}.
Ulegałby on addycji nukleofila, tworząc produkt o~konfiguracji \iupac{\cip{2S}},
  czyli takiej, jaką wstępnie przypisałem powstającemu w~przewadze związkowi \refcmpd{gal-tet.cy}.
Ponadto, \citeauthor{furman14acs} w~wyniku innych reduktywnych funkcjonalizacji wywiedzionego
  z~galaktozy laktamu \refcmpd{gal-lactam} uzyskiwali jako główne produkty właśnie te
  o~konfiguracji \iupac{\cip{2S}}\sidecite{furman14acs}.
Choć prezentowane dowody nie są mocne, uważam, że ich suma wystarcza aby podtrzymać
  propozycję przypisania konfiguracji \iupac{\cip{2S}} związkowi \refcmpd{gal-tet.cy}\sidenote{%
    A~przez analogię również związkowi \refcmpd{gal-tet.est}
  } i~konfiguracji \iupac{\cip{2R}} diastereomerycznemu związkowi \refcmpd{gal-epi-tet.cy}.
