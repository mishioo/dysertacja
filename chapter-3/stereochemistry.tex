\section{Stereochemia przemiany}\label{numeric:stereo}

\begin{marginfigure}[10\baselineskip]
  \includesvg{direction}
  \caption{
    Możliwe kierunki ataku nukleofila na~przejściowy kation iminiowy \refcmpd{glu-iminium}.
    W~obserwowanej przemianie, wbrew intuicji, preferowana jest addycja od~strony \textit{re}.
  }\label{fig:direction}
\end{marginfigure}

\Cref{fig:direction} przedstawia możliwe kierunki ataku nukleofila na~kation iminiowy
  \refcmpd{glu-iminium} \--- wywiedziony z~glukozy odpowiednik produktu pośredniego \refcmpd{int-4}
  w~proponowanym wcześniej mechanizmie.
Na~podstawie płaskiej reprezentacji tego związku, intuicja może podpowiadać, że ów atak
  nastąpi raczej od~strony \textit{si}, czyli przeciwnej do~sąsiedniego podstawnika
  benzoksylowego w~pozycji 3.
W~końcu stanowi on stosunkowo dużą zawadę steryczną, jego przeciwna strona powinna być
  łatwiej dostępna.

\begin{figure}
  \includesvg{xray}
  \caption{
    Uproszczone struktury związków \refcmpd{glu-tet.cy, glu-tet.pmb} przedstawione
      w~stylu ORTEP, w~reprezentacji elipsoid termalnych na~poziomie prawdopodobieństwa
      \SI{35}{\percent}.
    Pominięte zostały atomy wodoru oraz grupy benzylowe.
  }\label{fig:xray}
\end{figure}

A~jednak, dominującym produktem, a~w~większości przypadków nawet jedynym,
  jest ten powstający w~wyniku addycji od~strony \textit{re}.
Niezbitym na~to dowodem jest analiza rentgenostrukturalna związków
  \refcmpd{glu-tet.cy, glu-tet.pmb}, której wyniki w~uproszczeniu przedstawia \cref{fig:xray},
  a~szczegółowy ich opis znajduje się w~sekcji \secref{experimental:xray}.
Fenomen nieoczekiwanego kierunku tej addycji wyjaśnić można na~podstawie modelu Woerpla,
  opisującego stereochemię analogicznej nukleofilowej addycji do~kationów oksokarboniowych,
  będących częścią sześcioczłonowego pierścienia.
Kluczową rolę w~tym modelu odgrywa stabilność poszczególnych konformerów kationu i~produktu.
Aby ułatwić dyskusję na~ten temat,
  na~\cref{fig:thp-conformers} przypominam ich nazewnictwo\sidecite[-2\baselineskip]{iupac81}.

\begin{figure}
  \includesvg{thp-conformers}
  \caption{
    Nomenklatura konformerów pochodnych tetrahydropiranu.
    Aby nazwać daną konformację niezbędne jest ustalenie jej kształtu:
      krzesełka (\textit{C}), pół-krzesełka (\textit{H}), łódki (\textit{B}),
      lub skręconej łódki (\textit{S}).
    Podanego w~nawiasie symbolu używa się do~oznaczenia danego kształtu.
    Należy ułożyć pierścień w~taki sposób, aby, patrząc \enquote{z~góry},
      numery pozycji rosły zgodnie z~ruchem wskazówek zegara.
    Następnym krokiem jest znalezienie czterech atomów pierścienia położonych
      w~jednej płaszczyźnie.
    Numery atomów położonych poza wyznaczoną płaszczyzną odniesienia należy dodać
      do~symbolu kształtu \--- numer atomu nad płaszczyzną podaje się przed symbolem w~indeksie
      górnym, a~atomu pod płaszczyzną za~symbolem w~indeksie dolnym.
  }\label{fig:thp-conformers}
\end{figure}

% model Woerpla
% porównanie stabilności gluko z modelem
% próby ustalenia stereochemii galakto
% symulacja widm optycznych
