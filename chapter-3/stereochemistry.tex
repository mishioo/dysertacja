\section{Stereochemia przemiany}\label{numeric:stereo}

\begin{marginfigure}[10\baselineskip]
  \includesvg{direction}
  \caption{
    Możliwe kierunki ataku nukleofila na~przejściowy kation iminiowy \refcmpd{glu-iminium}.
    W~obserwowanej przemianie, wbrew intuicji, preferowana jest addycja od~strony \textit{re}.
  }\label{fig:direction}
\end{marginfigure}

\Cref{fig:direction} przedstawia możliwe kierunki ataku nukleofila na~kation iminiowy
  \refcmpd{glu-iminium} \--- wywiedziony z~glukozy odpowiednik produktu pośredniego \refcmpd{int-4}
  w~proponowanym wcześniej mechanizmie.
Na~podstawie płaskiej reprezentacji tego związku, intuicja może podpowiadać, że ów atak
  nastąpi raczej od~strony \textit{si}, czyli przeciwnej do~sąsiedniego podstawnika
  benzoksylowego w~pozycji 3.
W~końcu stanowi on stosunkowo dużą zawadę steryczną, jego przeciwna strona powinna być
  łatwiej dostępna.

\begin{figure}
  \includesvg{xray}
  \caption{
    Uproszczone struktury związków \refcmpd{glu-tet.cy, glu-tet.pmb} przedstawione
      w~stylu ORTEP, w~reprezentacji elipsoid termalnych na~poziomie prawdopodobieństwa
      \SI{35}{\percent}.
    Pominięte zostały atomy wodoru oraz grupy benzylowe.
  }\label{fig:xray}
\end{figure}

A~jednak, dominującym produktem addycji do~wywiedzionego z~glukozy związku \refcmpd{glu-iminium},
  a~w~większości przypadków nawet jedynym, jest ten powstający w~wyniku ataku od~strony \textit{re}.
Niezbitym na~to dowodem jest analiza rentgenostrukturalna związków
  \refcmpd{glu-tet.cy, glu-tet.pmb}, której wyniki w~uproszczeniu przedstawia \cref{fig:xray},
  a~szczegółowy ich opis znajduje się w~sekcji \secref{experimental:xray}.
Fenomen nieoczekiwanego kierunku tej addycji wyjaśnić można na~podstawie modelu Woerpla,
  opisującego stereochemię analogicznej nukleofilowej addycji do~kationów oksokarboniowych,
  będących częścią sześcioczłonowego pierścienia.
Kluczową rolę w~tym modelu odgrywa stabilność poszczególnych konformerów kationu i~produktu.
Aby ułatwić dyskusję na~ten temat,
  na~\cref{fig:thp-conformers} przypominam ich nazewnictwo\sidecite[-2\baselineskip]{iupac81}.

\begin{figure}
  \includesvg{thp-conformers}
  \caption{
    Nomenklatura konformerów pochodnych tetrahydropiranu.
    Aby nazwać daną konformację niezbędne jest ustalenie jej kształtu:
      krzesełka (\textit{C}), pół-krzesełka (\textit{H}), łódki (\textit{B}),
      lub skręconej łódki (\textit{S}).
    Podanego w~nawiasie symbolu używa się do~oznaczenia danego kształtu.
    Należy ułożyć pierścień w~taki sposób, aby, patrząc \enquote{z~góry},
      numery pozycji rosły zgodnie z~ruchem wskazówek zegara.
    Następnym krokiem jest znalezienie czterech atomów pierścienia położonych
      w~jednej płaszczyźnie.
    Numery atomów położonych poza wyznaczoną płaszczyzną odniesienia należy dodać
      do~symbolu kształtu \--- numer atomu nad płaszczyzną podaje się przed symbolem w~indeksie
      górnym, a~atomu pod płaszczyzną za~symbolem w~indeksie dolnym.
  }\label{fig:thp-conformers}
\end{figure}

\subsection{Model Woerpla}
W~modelu Woerpla\sidecite[2em]{woerpel00} kierunek addycji nukleofila do~cyklicznego kationu
  oksokarboniowego zależny jest nie od~bezpośredniego otoczenia centrum reakcyjnego,
  ale od~stabilności konformacyjnej powstających produktów.
Ze~względu na~płaską geometrię wiązania \ch{-C=O^+\bond{single}}, sześcioczłonowy cykliczny
  kation oksokarboniowy będzie znajdował się zawsze w~konformacji pół-krzesełkowej
  \cycloconf{3}{H}{4} lub \cycloconf{4}{H}{3}.
Atak nukleofila na~to wiązanie, zależnie od~strony, z~której nastąpi, może prowadzić do~powstania
  produktu w~konformacji krzesełkowej lub w~konformacji skręconej łódki,
  tak jak obrazuje to \cref{sch:woerpel}.
Preferowany będzie zawsze atak prowadzący do~powstania konformacji krzesełkowej,
  ponieważ biegnie on poprzez niżej energetyczny stan przejściowy o~analogicznej konformacji.

\begin{scheme}
  \includesvg{woerpel}
  \caption{
    Kierunek addycji nukleofila do~kationu oksokarboniowego według modelu Woerpla zależny jest
      od~stabilności konformacyjnej potencjalnego produktu addycji oraz samego kationu.
    Możliwe ścieżki przemiany przedstawiam na~przykładzie \iupac{4-podstawionej} piranozy.
  }\label{sch:woerpel}
\end{scheme}

Aby przewidzieć przebieg reakcji w~tym modelu, należy wpierw ustalić, która z~konformacji
  rozważanego kationu będzie stabilniejsza \--- \cycloconf{3}{H}{4} czy \cycloconf{4}{H}{3}.
Gdy to już wiadomo, można wnioskować czy w~przewadze będzie powstawał produkt ataku, odpowiednio,
  \enquote{z~góry}, czyli \textit{anti} do~podstawnika w~pozycji 4 w~przykładzie
  ze~\cref{sch:woerpel}, czy \enquote{z~dołu}, czyli \textit{syn} do~tego podstawnika\sidenote{
    Z~rozmysłem nie używam tu terminologi \textit{re} i~\textit{si}, gdyż nazwy stron wiązania
      mogą być różne, zależnie od~faktycznych podstawników w~pierścieniu.
  }.
\citeauthor{woerpel99} zaproponowali również taki model dla pierścieni
  pięcioczłonowych\sidecite{woerpel99}, nie będę go tu jednak omawiał.
Warto natomiast wspomnieć, że rozstrzygnięcie, który z~konformerów cyklicznego kationu
  oksokarboniowego będzie trwalszy, może nie być trywialne ze~względu na~występowanie
  efektu anomerycznego w~pierścieniach zawierających heteroatomy\sidecite{grindley08}.
Ponadto, intuicja, podpowiadająca, że podstawniki w~pozycji aksjalnej nie będą korzystne może
  zawieść, bowiem w~przypadku takich pierścieni zależy to od~elektronowego charakteru
  podstawnika\sidecite{woerpel00}.

Wykonana wcześniej w~zespole badawczym Furmana analiza konformacyjna wywiedzionej z~glukozy
  iminy\sidecite{furman14acs} sugeruje zgodność wyników z~modelem Woerpla.
Analiza została przeprowadzona na~uproszczonym modelu, w~którym podstawniki benzoksylowe
  zostały zastąpione zbliżonymi elektronowo, ale nie tak labilnymi grupami metoksylowymi.
Niżej energetyczny jest konformer \cycloconf{4}{H}{3}, w~przypadku którego preferowany jest
  atak nukleofila \enquote{od~dołu}, prowadzący do~powstania produktu o~takiej samej konfiguracji,
  jak wydzielany przeze mnie związek \refcmpd{glu-tet.cy}.

\begin{scheme}
  \includesvg{gluco-imine-confs}
  \caption{
    Porównanie stabilności konformerów cyklicznej iminy o~konfiguracji glukozy,
      według badań DFT przeprowadzonych przez zespół badawczy Furmana.
    Addycja nukleofilowa do~stabilniejszego konformeru prowadziłaby do~powstania centrum
      chiralnego o~konfiguracji \iupac{2-\R}.
  }\label{sch:gluco-imine-confs}
\end{scheme}

Różnica w~stabilności konformerów \cycloconf{3}{H}{4} i~\cycloconf{4}{H}{3}, pokazanych
  na~\cref{sch:gluco-imine-confs} jest znaczna i~wynosi \SI{4.5}{\kcalpm}.
Stosując analizę rozkładu Boltzmanna można oszacować w~jakiej proporcji te konformery
  będą występowały w~mieszanie reakcyjnej\sidenote{
    O~detalach technicznych tej analizy piszę więcej w~sekcji \secref{tesliper:boltzmann}.
  }.
Wskazuje on, że w~temperaturze pokojowej stabilniejszy z~konformerów będzie stanowił zdecydowaną
  większość, około \SI{99.95}{\percent} składu.
Estymacja ta dobrze koresponduje z~eksperymentem, w~którym obserwowałem tylko jeden
  z~diastereomerów produktu.

\subsection{Konfiguracja produktów wywiedzionych z~galaktozy}
Ustalenie konfiguracji absolutnej centrum stereogenicznego w~pozycji 2. wywiedzionych
  z~galaktozy związków o~strukturze \refcmpd{gal-tet} okazało się znacznie większym wyzwaniem.
Wszystkie próby otrzymania monokryształu któregoś z~tych produktów zakończyły się fiaskiem,
  uniemożliwiając wykorzystanie najbardziej bezpośredniej metody,
  czyli rentgenografii strukturalnej.
Korzystając ze zmodyfikowanej procedury syntetycznej udało mi się otrzymać niewielką ilość
  związku \refcmpd{gal-epi-tet.cy}, będącego diastereomerem produktu \refcmpd{gal-tet.cy}.
Mogłem dzięki temu uciec się do~innych dostępnych metod \--- analizy \NMR*{} i~efektów \gls{noe}
  oraz spektroskopii chiralooptycznej, które dalej opisuję.

% próby ustalenia stereochemii galakto
% symulacja widm optycznych
