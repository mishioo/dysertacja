\section{Stereochemia przemiany}\label{numeric:stereo}

\begin{marginfigure}[10\baselineskip]
  \includesvg{direction}
  \caption{
    Możliwe kierunki ataku nukleofila na~przejściowy kation iminiowy \refcmpd{glu-iminium}.
    W~obserwowanej przemianie, wbrew intuicji, preferowana jest addycja od~strony \textit{re}.
  }\label{fig:direction}
\end{marginfigure}

\Cref{fig:direction} przedstawia możliwe kierunki ataku nukleofila na~kation iminiowy
  \refcmpd{glu-iminium} \--- wywiedziony z~glukozy odpowiednik produktu pośredniego \refcmpd{int-4}
  w~proponowanym wcześniej mechanizmie.
Na~podstawie płaskiej reprezentacji tego związku, intuicja może podpowiadać, że ów atak
  nastąpi raczej od~strony \textit{si}, czyli przeciwnej do~sąsiedniego podstawnika
  benzoksylowego w~pozycji 3.
W~końcu stanowi on stosunkowo dużą zawadę steryczną, jego przeciwna strona powinna być
  łatwiej dostępna.

\begin{figure}
  \includesvg{xray}
  \caption{
    Uproszczone struktury związków \refcmpd{glu-tet.cy, glu-tet.pmb} przedstawione
      w~stylu ORTEP, w~reprezentacji elipsoid termalnych na~poziomie prawdopodobieństwa
      \SI{35}{\percent}.
    Pominięte zostały atomy wodoru oraz grupy benzylowe.
  }\label{fig:xray}
\end{figure}

A~jednak, dominującym produktem, a~w~większości przypadków nawet jedynym,
  jest ten powstający w~wyniku addycji od~strony \textit{re}.
Niezbitym na~to dowodem jest analiza rentgenostrukturalna związków
  \refcmpd{glu-tet.cy, glu-tet.pmb}, której wyniki w~uproszczeniu przedstawia \cref{fig:xray},
  a~szczegółowy ich opis znajduje się w~sekcji \secref{experimental:xray}.

\section{Symulacja widm optycznych}\label{numeric:spectra}
