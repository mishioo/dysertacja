\chapter{Dociekania wspomagane komputerowo}\label{chapter:numeric}

Opisana w~sekcji \titleref{synthesis:sugars} przemiana ma dwa istotne aspekty, które mogą
  się wydać niejasne lub zaskakujące.
Pierwszym jest jej mechanizm \--- raportowane w~literaturze przykłady reakcji tworzenia
  tetrazoli z~izocyjanku i~azydku trimetylosililu biegną w~obecności protonowego medium.
Zwykle jest to protonowy rozpuszczalnik, najczęściej metanol, ale może też być to woda powstała
  z~kondensacji aminy z~karbonylem, jeśli jest to wariant czteroskładnikowej reakcji Ugiego.
Proponowane mechanizmy tych przemian zakładają zaangażowanie protonu w~hydrolizie \ch{TMSN3}
  i~udział anionu azydkowego w~cyklizacji do~tetrazolu.
W~środowisku reakcji opisywanej tutaj brakuje takiego protonowego medium, a~jednak reakcja
  ta biegnie dość wydajnie.

Drugą nieoczywistą sprawą jest kwestia stereochemii tej reakcji \--- przebiegającej zwykle
  z~utworzeniem tylko jednego diastereomeru, a~ponadto, powstający produkt ma konfigurację
  przeciwną, niż może początkowo podpowiadać intuicja.
Jeden i~drugi z~tych wątków rozwijam w~tej części niniejszej dysertacji.
Oprócz eksperymentów syntetycznych i~przesłanek literaturowych, istotną rolę odgrywają w~tych
  rozważaniach badania \latin{in silico}, a~w~szczególności obliczenia kwantowo-chemiczne,
  wykonane przeze mnie z~użyciem oprogramowania Gaussian.
Końcówka tego fragmentu pracy, poświęcona symulacji widm spektroskopii optycznej badanych
  cząsteczek, będzie natomiast punktem wyjścia do~opisu stworzonego przeze mnie
  oprogramowania komputerowego.

\section{Przebieg klasycznej reakcji azydo-Ugiego}\label{numeric:classical}
\section{Rozważania mechanistyczne}\label{numeric:mechanism}
\section{Stereochemia przemiany}\label{numeric:stereo}
\section{Symulacja widm optycznych}\label{numeric:spectra}

