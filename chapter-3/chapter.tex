\chapter{Dociekania wspomagane komputerowo}\label{chapter:numeric}

Opisana w~sekcji \textit{\nameref{synthesis:sugars}} przemiana ma dwa istotne aspekty,
  które mogą się wydać niejasne lub zaskakujące.
Pierwszym jest jej mechanizm \--- raportowane w~literaturze przykłady reakcji tworzenia tetrazoli
  z~izocyjanku i~azydku trimetylosililu biegną w~obecności protonowego medium\sidecite{maleki15}.
Zwykle jest to protonowy rozpuszczalnik, najczęściej metanol, ale może też być to woda
  powstała z~kondensacji aminy z~aldehydem lub ketonem, jeśli jest to wariant
  czteroskładnikowej reakcji Ugiego.
Proponowane mechanizmy tych przemian zakładają zaangażowanie protonu w~hydrolizę \ch{TMSN3}
  i~udział anionu azydkowego w~cyklizacji do~tetrazolu.
W~środowisku reakcji opisywanej tutaj brakuje protonowego medium, a~jednak reakcja
  ta biegnie dość wydajnie.
Drugą nieoczywistą sprawą jest kwestia stereochemii tej reakcji \--- przebiega ona zwykle
  z~utworzeniem tylko jednego diastereomeru, którego konfiguracji nie sposób przewidzieć
  na~podstawie intuicji\sidenote{
    Uwaga ta dotyczy konfigutacji nowopowstającego centrum stereogenicznego w~pozycji 2,
      \see{tab:sugars-scope}.
  }.

Jeden i~drugi z~tych wątków rozwijam w~tej części niniejszej dysertacji.
Oprócz eksperymentów syntetycznych i~przesłanek literaturowych, istotną rolę odgrywają w~tych
  rozważaniach badania \latin{in silico}, a~w~szczególności obliczenia kwantowo-chemiczne,
  wykonane przeze mnie z~użyciem oprogramowania Gaussian\sidecite{gaussian09,gaussian16}.
Końcówka tego fragmentu pracy, poświęcona symulacji widm \gls{ecd} badanych
  cząsteczek, będzie jednocześnie punktem wyjścia do~opisu oprogramowania komputerowego,
  stworzonego przeze mnie aby ułatwić prowadzenie takich prac.

\subimport{./}{mechanism}
\subimport{./}{stereochemistry}

