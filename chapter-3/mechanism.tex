\section{Przebieg klasycznej reakcji azydo-Ugiego}\label{numeric:classical}
W~sekcji \secref{synthesis:sugars:ugi} cytowałem pracę Ugiego i~Steinbr{\"u}cknera\sidecite{ugi61},
  którzy po~raz pierwszy zaprezentowali reakcję syntezy tetrazoli, bazującą
  na~wieloskładnikowej reakcji Ugiego.
Autorzy użyli do~jej przeprowadzenia, zależnie od~przykładu, azotowodoru (\ch{HN3}) w~benzenie
  z~metanolem albo azydku sodu w~acetonie z~dodatkiem wody i~\ch{HCl}, generując azotowodór \insitu.
Późniejsze prace często używają azydku trimetylosililu (\ch{\acrshort{tms}N3}) jako źródła
  anionu azydkowego, zazwyczaj w~obecności metanolu jako rozpuszczalnika.

\begin{scheme}
  \includesvg{mechanism-classic}
  \caption{
    Ogólnie przyjęty mechanizm reakcji azydo-Ugiego, wykorzystujący \ch{\acrshort{tms}N3}.
    Metanol (lub inne źródło protonu) jest niezbędny zarówno do~aktywacji
      iminy~\refcmpd{w:imine} na~atak izocyjanku, jak i~do~wytworzenia anionu azydkowego.
    W~takich warunkach reakcji może powstawać też produkt uboczny \refcmpd{n-tetrazole}
      w~wyniku cyklizacji samego izocyjanku z~anionem azydkowym.
  }\label{sch:mechanism-classic}
\end{scheme}

Badacze zdają się być zgodni co~do~konieczności użycia źródła protonu do~przeprowadzenia tej reakcji
  \--- czy to w~postaci protonowego medium reakcyjnego, czy dodatku kwasu\sidecite{neochoritis19}.
Owo źródło protonu przyczynia się zarówno do~aktywacji iminy na~atak izocyjanku, jak i~do~hydrolizy
  \ch{\acrshort{tms}N3} do~anionu azydkowego, który to jest faktycznym reagentem.
\Cref{sch:mechanism-classic} prezentuje ogólnie przyjęty mechanizm powstawania tetrazolu
  w~klasycznej wersji reakcji azydo-Ugiego, wykorzystującej kondensację pierwszorzędowej
  aminy i~aldehydu do~utworzenia iminy~\refcmpd{w:imine}.
Kation iminiowy \refcmpd{w:iminium-h} łatwo ulega addycji izocyjanku, prowadząc do~powstania
  kationu nitryliowego, który to ulega addycji anionu azydkowego.
Cyklizacja do~\refcmpd{aminotetrazole-sec} następuje najpewniej poprzez stan
  przejściowy~\refcmpd{amino-imino-azide} niż w~jednym etapie
  z~\refcmpd{nitrilium-sec}\sidecite{sharpless02}.

\citeauthor{kutovaya19}, którzy wnikliwie badali różne warianty tego typu reakcji,
  pokazali, że w~opisywanych warunkach może powstawać również \textit{C}\-/niepodstawiony
  tetrazol~\refcmpd{n-tetrazole}, jako produkt ubocznej, bezpośredniej reakcji izocyjanku
  i~anionu azydkowego\sidecite{kutovaya19,shmatova13}.
Pokazali też, że możliwe jest jest przeprowadzenie reakcji azydo-Ugiego w~wariancie
  Joulli{\'e}-Ugiego, czyli używając przygotowanej zawczasu
  iminy\sidecite{nenajdenko13,shmatova13}.
Wspominałem o~tym już wcześniej\sidenote{We wstępie do~sekcji \secref{synthesis:sugars:opt}.},
  tutaj jednak chcę podkreślić, że używają zawsze protonowych rozpuszczalników do~przeprowadzenia
  tych przemian.

\section{Rozważania mechanistyczne}\label{numeric:mechanism}
Opisana w~niniejszej dysertacji synteza tetrazoli biegnie wydajnie w~warunkach ściśle
  bezprotonowych, nosząc znamiona bezprecedensowości w~świetle przytoczonych danych z~literatury.
Zaintrygowany, pochyliłem się niżej nad tą kwestią, starając się dociec,
  jak powstają obserwowane przeze mnie produkty.
Punktem wyjścia do~tych rozważań była myśli, że azydek trimetylosililu może pełnić rolę
  nie tylko źródła anionu azydkowego, ale również aktywatora iminy.
Przyjmując takie założenie, proponuję mechanizm przemiany widoczny na~\cref{sch:mechanism-our}.

\begin{scheme}
  \includesvg{mechanism-our}
  \caption{
    Propozycja mechanizmu powstawania aminotetrazolu \refcmpd{mech-9} w~opisywanym, bezportonowym
      wariancie reakcji azydo-Ugiego.
  }\label{sch:mechanism-our}
\end{scheme}

Sugeruję, że kompleks \refcmpd{mech-2}, powstały w~wyniku redukcji laktamu~\refcmpd{mech-1},
  ulega samoistnemu rozpadowi z~uwolnieniem iminy~\refcmpd{mech-3}.
Wolna para elektronowa atomu azotu owej iminy jest czynnikiem prowadzącym do~uwolnienia anionu
  azydkowego z~\ch{\acrshort{tms}N3}, tak jak obrazuje to struktura \refcmpd{mech-4}.
Powstający sililowany kation iminiowy \refcmpd{mech-5} ulega addycji izocyjanianu, by następnie
  przyłączyć uwolniony wcześniej anion \ch{N3-}.
Obojętna elektronowo struktura przejściowa \refcmpd{mech-7} ulega cyklizacji do~stabilniejszego
  \iupac{\N-sililowanego} aminotetrazolu~\refcmpd{mech-8}.
Finalny związek \refcmpd{mech-9} powstaje na~etapie terminacji reakcji,
  w~wyniku spontanicznej hydrolizy.
W~kolejnych akapitach staram się obronić tę propozycję, omawiając przesłanki literaturowe
  i~wyniki moich eksperymentów, przemawiające na~jej korzyść.

\subsection{Samoistny rozpad do~iminy}
Jedne z~pierwszych badań poświęconych redukcji amidów do~imin\sidecite{schedler93},
  przeprowadzone przez \citeauthor{schedler93}, sugerują, że rozpad cyrkonowego kompleksu
  takiego jak \refcmpd{mech-2} nie następuje spontanicznie.
Autorzy zwracają uwagę, że użycie mniej niż \SI{2.0}{\equiv} odczynnika Schwartza do~redukcji
  amidu skutkuje znacznym obniżeniem wydajności wydzielanej przez nich iminy.
Wnoszą na~tej podstawie, że drugi ekwiwalent \schwartz{} jest czynnikiem
  powodującym rozkład cyrkonowego kompleksu i~utworzenie właściwego produktu.

Kolejne prace z~tej dziedziny, opisane gruntownie w~sekcji \secref{literature:schwartz},
  jak i~doświadczenie zespołu badawczego, w~którym wykonane zostały badania zawarte
  w~niniejszej dysertacji, zdają się dawać podobne świadectwo.
Zazwyczaj motorem rozważanej przemiany jest raczej kwas Lewisa lub kwas protonowy,
  niż kolejny reduktor, jednak wspomniane źródła istotnie wskazują,
  że niezbędny jest dodatkowy bodziec, aby nastąpiła rozważana przemiana.
Choć we~warunkach stosowanych przeze mnie nie występuje żaden z~tych czynników\sidenote{
    Z wyjątkiem \SI{0.6}{\equiv} nadmiaru odczynnika Schwartza, które jednak nie byłoby
      wystarczające do~otrzymania średnio \SI{86}{\percent} wydajności, jaką obserwowałem
      w~eksperymentach optymalizacyjnych (\see{fig:sugars-opt-plot}).
  }, to trudno wyobrazić sobie, by proces ten biegł inaczej niż poprzez iminę.
Mimo braku jej solidnego poparcia w~dostępnej literaturze, postanowiłem poddać moją teorię
  próbie eksperymentalnej.

\begin{figure}
  \includesvg{nmr-imine-trace}
  \caption{
    Widma \NMR*{} substratu \refcmpd{glu-lactam} oraz jego mieszanin z~odczynnikiem Schwartza:
      po~homogenizacji roztworu (\refcmpd{glu-lactam} + \schwartz{}) i~po kolejnych trzech dniach
      (\refcmpd{glu-lactam} + \schwartz{} \SI{3}{\day}).
    Pomiar został wykonany w~szczelnie zamkniętej probówce w~atmosferze argonu,
      w~d\textsubscript{8}\-/\acrshort{thf}.
  }\label{fig:nmr-imine-trace}
\end{figure}

\begin{marginfigure}[-17.5em]
  \includesvg{nmr-imine-expand}
  \caption{
    Zbliżenie na~dublet widoczny na~\protect\cref{fig:nmr-imine-trace}, będący najpewniej
      sygnałem pochodzącym od~iminowego protonu, zaznaczonego kolorem pomarańczowym
      na~\protect\cref{sch:mechanism-our} (struktura~\refcmpd{mech-3}).
  }\label{fig:nmr-imine-expand}
\end{marginfigure}

Przeprowadziłem reakcję redukcji modelowego substratu \refcmpd{glu-lactam} za~pomocą odczynnika
  Schwartza w~szczelnie zamkniętej probówce NMR, w~atmosferze argonu, używając suszonego nad~sitami
  molekularnymi d\textsubscript{8}\-/\acrshort{thf} jako rozpuszczalnika.
Zarejestrowałem widmo \NMR*{} po~sklarowaniu się (homogenizacji) mieszaniny, przedstawia
  je \cref{fig:nmr-imine-trace} (środkowa linia).
Jest na~nim widoczny jest niewysoki dublet przy przesunięciu chemicznym
  \SIrange{7.59}{7.60}{\ppm}, nieobecny w~widmie związku \refcmpd{glu-lactam} (górna linia).
Intensywność tego sygnału nasila się wraz z,~widocznym po~kolejnych 3~dniach, rozkładzie
  cyrkonowego kompleksu \refcmpd{mech-2} (dolna linia).

\begin{scheme*}
  \includesvg{zr-to-imine}
  \caption{
    Samoistny rozpad cyrkonowego kompleksu o~strukturze~\refcmpd{mech-2} do~iminy
      na~przykładzie modelowej reakcji z~laktamem \refcmpd{glu-lactam} wywiedzionym z~glukozy.
    Zaznaczone na~schemacie orientacyjne wartości przesunięć chemicznych pochodzą
      z~cytowanej literatury.
  }\label{sch:zr-to-imine}
  \setfloatalignment{b}
\end{scheme*}
% TODO: fix cmpd references to match this scheme

Sygnał ten, widoczny w~powiększeniu na~\cref{fig:nmr-imine-expand}, jest położony przy
  przesunięciu chemicznym, w~pobliżu którego zazwyczaj znajduje się sygnał protonu połączonego
  z~iminowym atomem węgla\sidecite{reich22}.
Przeprowadzone równolegle eksperymenty, w~których do~mieszaniny po~sklarowaniu dodałem jeszcze
  \ch{\acrshort{tms}N3} albo izocyjanianu, nie prowadziły do~zarejestrowania sygnału o~większej
  intensywności\sidenote{
    Co więcej, w~przypadku dodatku \ch{\acrshort{tms}N3} do~mieszaniny po~redukcji,
      intensywność tego sygnału i~nie rośnie z~czasem.
    Obserwacja ta popiera tezę o~reakcji iminy z~azydkiem, której będę bronił w~kolejnej sekcji.
  }.
Obserwacje te mogą świadczyć, że wolna imina jest rzeczywiście obecna w~środowisku reakcji,
  i~że jej tworzenie rzeczywiście następuje w~wyniku samoistnego rozpadu cyrkonowego kompleksu.

% my DFT of spontaneous imine formation
% ? compare with calculations of iminium cation from Schwartz complex formation ?
% unsuccessful attempts at Glu-N-TMS synthesis
% my DFT of tetrazole formation
% compare with sharpless02 DFT calculationss
 