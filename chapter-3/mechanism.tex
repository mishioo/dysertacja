\section{Przebieg klasycznej reakcji azydo-Ugiego}\label{numeric:classical}
W~sekcji \secref{synthesis:sugars:ugi} cytowałem pracę Ugiego i~Steinbr{\"u}cknera\sidecite{ugi61},
  którzy po~raz pierwszy zaprezentowali reakcję syntezy tetrazoli, bazującą
  na~wieloskładnikowej reakcji Ugiego.
Autorzy użyli do~jej przeprowadzenia, zależnie od~przykładu, azotowodoru (\ch{HN3}) w~benzenie
  z~metanolem albo azydku sodu w~acetonie z~dodatkiem wody i~\ch{HCl}, generując azotowodór \insitu.
Późniejsze prace często używają azydku trimetylosililu (\ch{\acrshort{tms}N3}) jako źródła
  anionu azydkowego, zazwyczaj w~obecności metanolu jako rozpuszczalnika.

\begin{scheme}
  \includesvg{mechanism-classic}
  \caption{
    Ogólnie przyjęty mechanizm reakcji azydo-Ugiego, wykorzystujący \ch{\acrshort{tms}N3}.
    Metanol (lub inne źródło protonu) jest niezbędny zarówno do~aktywacji
      iminy~\refcmpd{w:imine} na~atak izocyjanku, jak i~do~wytworzenia anionu azydkowego.
    W~takich warunkach reakcji może powstawać też produkt uboczny \refcmpd{n-tetrazole}
      w~wyniku cyklizacji samego izocyjanku z~anionem azydkowym.
  }\label{sch:mechanism-classic}
\end{scheme}

Badacze zdają się być zgodni co~do~konieczności użycia źródła protonu do~przeprowadzenia tej reakcji
  \--- czy to w~postaci protonowego medium reakcyjnego, czy dodatku kwasu\sidecite{neochoritis19}.
Owo źródło protonu przyczynia się zarówno do~aktywacji iminy na~atak izocyjanku, jak i~do~hydrolizy
  \ch{\acrshort{tms}N3} do~anionu azydkowego, który to jest faktycznym reagentem.
\Cref{sch:mechanism-classic} prezentuje ogólnie przyjęty mechanizm powstawania tetrazolu
  w~klasycznej wersji reakcji azydo-Ugiego, wykorzystującej kondensację pierwszorzędowej
  aminy i~aldehydu do~utworzenia iminy~\refcmpd{w:imine}.
Kation iminiowy \refcmpd{w:iminium-h} łatwo ulega addycji izocyjanku, prowadząc do~powstania
  kationu nitryliowego, który to ulega addycji anionu azydkowego.
Cyklizacja do~\refcmpd{aminotetrazole-sec} następuje najpewniej poprzez stan
  przejściowy~\refcmpd{amino-imino-azide} niż w~jednym etapie
  z~\refcmpd{nitrilium-sec}\sidecite{sharpless02}.

\citeauthor{kutovaya19}, którzy wnikliwie badali różne warianty tego typu reakcji,
  pokazali, że w~opisywanych warunkach może powstawać również \textit{C}\-/niepodstawiony
  tetrazol~\refcmpd{n-tetrazole}, jako produkt ubocznej, bezpośredniej reakcji izocyjanku
  i~anionu azydkowego\sidecite{kutovaya19,shmatova13}.
Pokazali też, że możliwe jest jest przeprowadzenie reakcji azydo-Ugiego w~wariancie
  Joulli{\'e}-Ugiego, czyli używając przygotowanej zawczasu
  iminy\sidecite{nenajdenko13,shmatova13}.
Wspominałem o~tym już wcześniej\sidenote{We wstępie do~sekcji \secref{synthesis:sugars:opt}.},
  tutaj jednak chcę podkreślić, że używają zawsze protonowych rozpuszczalników do~przeprowadzenia
  tych przemian.

\section{Rozważania mechanistyczne}\label{numeric:mechanism}
Opisana w~niniejszej dysertacji synteza tetrazoli biegnie wydajnie w~warunkach ściśle
  bezprotonowych, nosząc znamiona bezprecedensowości w~świetle przytoczonych danych z~literatury.
Zaintrygowany, pochyliłem się niżej nad tą kwestią, starając się dociec,
  jak powstają obserwowane przeze mnie produkty.
Punktem wyjścia do~tych rozważań była myśli, że azydek trimetylosililu może pełnić rolę
  nie tylko źródła anionu azydkowego, ale również aktywatora iminy.
Przyjmując takie założenie, proponuję mechanizm przemiany widoczny na~\cref{sch:mechanism-our}.

\begin{scheme}
  \includesvg{mechanism-our}
  \caption{
    Propozycja mechanizmu powstawania aminotetrazolu \refcmpd{mech-9} w~opisywanym, bezportonowym
      wariancie reakcji azydo-Ugiego.
  }\label{sch:mechanism-our}
\end{scheme}

Sugeruję, że kompleks \refcmpd{mech-2}, powstały w~wyniku redukcji laktamu~\refcmpd{mech-1},
  ulega samoistnemu rozpadowi z~uwolnieniem iminy~\refcmpd{mech-3}.
Wolna para elektronowa atomu azotu owej iminy jest czynnikiem prowadzącym do~uwolnienia anionu
  azydkowego z~\ch{\acrshort{tms}N3}, tak jak obrazuje to struktura \refcmpd{mech-4}.
Powstający sililowany kation iminiowy \refcmpd{mech-5} ulega addycji izocyjanianu, by następnie
  przyłączyć uwolniony wcześniej anion \ch{N3-}.
Obojętna elektronowo struktura przejściowa \refcmpd{mech-7} ulega cyklizacji do~stabilniejszego
  \iupac{\N-sililowanego} aminotetrazolu~\refcmpd{mech-8}.
Finalny związek \refcmpd{mech-9} powstaje na~etapie terminacji reakcji,
  w~wyniku spontanicznej hydrolizy.
W~kolejnych akapitach staram się obronić tę propozycję, omawiając przesłanki literaturowe
  i~wyniki moich eksperymentów, przemawiające na~jej korzyść.

\subsection{Samoistny rozpad do~iminy}
Jedne z~pierwszych badań poświęconych redukcji amidów do~imin\sidecite{schedler93},
  przeprowadzone przez \citeauthor{schedler93}, sugerują, że rozpad cyrkonowego kompleksu
  takiego jak \refcmpd{mech-2} nie następuje spontanicznie.
Autorzy zwracają uwagę, że użycie mniej niż \SI{2.0}{\equiv} odczynnika Schwartza do~redukcji
  amidu skutkuje znacznym obniżeniem wydajności wydzielanej przez nich iminy.
Wnoszą na~tej podstawie, że drugi ekwiwalent \schwartz{} jest czynnikiem
  powodującym rozkład cyrkonowego kompleksu i~utworzenie właściwego produktu.

Kolejne prace z~tej dziedziny, opisane gruntownie w~sekcji \secref{literature:schwartz},
  jak i~doświadczenie zespołu badawczego, w~którym wykonane zostały badania zawarte
  w~niniejszej dysertacji, zdają się dawać podobne świadectwo.
Zazwyczaj motorem rozważanej przemiany jest raczej kwas Lewisa lub kwas protonowy,
  niż kolejny reduktor, jednak wspomniane źródła istotnie wskazują,
  że niezbędny jest dodatkowy bodziec, aby nastąpiła rozważana przemiana.
Choć we~warunkach stosowanych przeze mnie nie występuje żaden z~tych czynników\sidenote{
    Z wyjątkiem \SI{0.6}{\equiv} nadmiaru odczynnika Schwartza, które jednak nie byłoby
      wystarczające do~otrzymania średnio \SI{86}{\percent} wydajności, jaką obserwowałem
      w~eksperymentach optymalizacyjnych (\see{fig:sugars-opt-plot}).
  }, to trudno wyobrazić sobie, by proces ten biegł inaczej niż poprzez iminę.
Mimo braku jej solidnego poparcia w~dostępnej literaturze, postanowiłem poddać moją teorię
  próbie eksperymentalnej.

\begin{figure}
  \includesvg{nmr-imine-trace}
  \caption{
    Widma \NMR*{} substratu \refcmpd{glu-lactam} oraz jego mieszanin z~odczynnikiem Schwartza:
      po~homogenizacji roztworu (\refcmpd{glu-lactam} + \refcmpd{w:schwartz}) i~po kolejnych trzech dniach
      (\refcmpd{glu-lactam} + \refcmpd{w:schwartz}, \SI{3}{\day}).
    Pomiar został wykonany w~szczelnie zamkniętej probówce w~atmosferze argonu,
      w~d\textsubscript{8}\-/\acrshort{thf}.
  }\label{fig:nmr-imine-trace}
\end{figure}

\begin{marginfigure}[-17.5em]
  \includesvg{nmr-imine-expand}
  \caption{
    Zbliżenie na~dublet widoczny na~\protect\cref{fig:nmr-imine-trace}, będący najpewniej
      sygnałem pochodzącym od~iminowego protonu, zaznaczonego zielonym kolorem w~strukturze
      na~\protect\cref{sch:zr-to-imine} (struktura~\refcmpd{glu-imine}).
  }\label{fig:nmr-imine-expand}
\end{marginfigure}

Przeprowadziłem reakcję redukcji modelowego substratu \refcmpd{glu-lactam} za~pomocą odczynnika
  Schwartza w~szczelnie zamkniętej probówce NMR, w~atmosferze gazu obojętnego, używając
  suszonego nad~sitami molekularnymi d\textsubscript{8}\-/\acrshort{thf} jako rozpuszczalnika.
Zarejestrowałem widmo \NMR*{} po~sklarowaniu się (homogenizacji) mieszaniny \--- przedstawia
  je \cref{fig:nmr-imine-trace} (\refcmpd{glu-lactam} + \refcmpd{w:schwartz}).
Jest na~nim widoczny niewysoki dublet przy przesunięciu chemicznym \SIrange{7.59}{7.60}{\ppm},
  nieobecny w~widmie związku \refcmpd{glu-lactam}.
Intensywność tego sygnału nasila się wraz z~rozkładem cyrkonowego kompleksu \refcmpd{mech-2},
  wyraźnie widocznym po~kolejnych 3~dniach
  (\refcmpd{glu-lactam} + \refcmpd{w:schwartz}, \SI{3}{\day}).

\begin{scheme*}
  \includesvg{zr-to-imine}
  \caption{
    Samoistny rozpad cyrkonowego kompleksu o~strukturze~\refcmpd{mech-2} do~iminy
      na~przykładzie modelowej reakcji z~wywiedzionym z~glukozy laktamem \refcmpd{glu-lactam}.
    Zaznaczone na~schemacie orientacyjne wartości przesunięć chemicznych pochodzą
      z~cytowanej literatury \textcolor{\citationcolor}{(\shortcite{spletstoser07,reich22})}.
  }\label{sch:zr-to-imine}
  \setfloatalignment{b}
\end{scheme*}

Sygnał ten, pokazany w~powiększeniu na~\cref{fig:nmr-imine-expand}, jest położony przy
  przesunięciu chemicznym, w~pobliżu którego zazwyczaj znajduje się sygnał protonu połączonego
  z~iminowym atomem węgla\sidecite{reich22}.
Proton ten wyszczególniłem zielonym kolorem w~strukturze
  \refcmpd{glu-imine} na~\cref{sch:zr-to-imine}.
Przeprowadzone równolegle eksperymenty, w~których do~mieszaniny po~sklarowaniu dodałem jeszcze
  \ch{\acrshort{tms}N3} albo izocyjanian, nie prowadziły do~zwiększenia intensywności tego
  sygnału\sidenote{
    Co więcej, w~przypadku dodatku \ch{\acrshort{tms}N3} do~mieszaniny po~redukcji,
      intensywność tego sygnału nie rośnie z~czasem.
    Obserwacja ta popiera tezę o~reakcji iminy z~azydkiem, której będę bronił w~kolejnej sekcji.
  }.
Obserwacje te mogą świadczyć, że wolna imina jest rzeczywiście obecna w~środowisku reakcji,
  i~że jej tworzenie rzeczywiście następuje w~wyniku samoistnego rozpadu cyrkonowego kompleksu.

Przeprowadziłem również wstępne badania \gls{dft} celem dalszej weryfikacji proponowanego
  mechanizmu.
Obliczenia wykonałem na~uproszczonym modelu\sidenote{
    Uproszczenie to polega przede wszystkim na~zastąpieniu struktury wywiedzionego z~glukozy
      związku \refcmpd{glu-lactam} strukturą prostego \textdelta-laktamu.
    Labilne grupy benzoksylowe, obecne w~oryginalnym modelu, znacznie utrudniają skuteczną symulację
      układu, co dotkliwie odczułem podczas wspomaganych modelowaniem komputerowym prób ustalenia
      konfiguracji absolutnej jednego z~wydzielonych produktów.
    Zmagania te opisuję w~kolejnych sekcjach.
  }, posługując się oprogramowaniem Gaussian~09\sidecite{gaussian09}.
Geometrię struktur zaangażowanych w~przebieg symulowanej reakcji optymalizowałem, przyjmując
  funkcjonał B3LYP i~stosując bazę LANL2DZ do~opisu atomów \ch{Zr}
  oraz 6-31G(d,p) do~opisu pozostałych atomów.
Użyłem również empirycznej poprawki dyspersyjnej GD3.
Zoptymalizowane struktury poddałem powtórnej analizie, używając większej bazy Def2TZVP
  z~zastosowaniem modelu solwatacyjnego PCM dla \gls{thf}, tak jak został on
  zaimplementowany w~programie Gaussian~09.
Raportowane wartości energii to względne wartości sum energii elektronowych i~punktu zerowego,
  dane w~\si{\kcalpm}.

\begin{scheme}
  \includesvg{calc-imine}
  \caption{
    Porównanie możliwych ścieżek samoistnego rozpadu cyrkonowego kompleksu \refcmpd{int-1-a}
      do~iminy \refcmpd{int-3} w~uproszczonym układzie reakcyjnym. 
    Prezentowane wartości energii to względne wartości sum energii elektronowych i~punktu zerowego,
      dane w~\si{\kcalpm}.
    Bariera energetyczna obydwu ścieżek to \SI{22.6}{\kcalpm} w~przypadku zaznaczonej
      na~zielono ścieżki biegnącej przez pięcioczłonowy stan przejściowy \refcmpd{ts-1-a}
      oraz \SI{63.0}{\kcalpm} w~przypadku zaznaczonej na~różowo ścieżki biegnącej przez
      czteroczłonowy stan przejściowy \refcmpd{ts-1-b}.
  }\label{sch:calc-imine}
\end{scheme}

\Cref{sch:calc-imine} porównuje dwie możliwe ścieżki spontanicznego rozkładu cyrkonowego kompleksu
  \refcmpd{int-1-a} do~wolnej iminy \refcmpd{int-3}.
W~obliczu nieobecności kwasu Lewisa, który mógłby katalizować tę przemianę,
  zakładam po~prostu odłączenie \ch{Cp2Zr(OH)Cl} od~początkowej struktury,
  którą traktuję jako punkt odniesienia.
Zaznaczona zielonym kolorem ścieżka, biegnąca przez pięcioczłonowy cykliczny stan przejściowy
  \refcmpd{ts-1-a}, jest zdecydowanie bardziej prawdopodobna niż ścieżka alternatywna.
Wyraźnie wskazuje na~to bardzo duża różnica w~energii aktywacji, wynosząca ponad \SI{40}{\kcalpm}.
Bariera energetyczna preferowanej ścieżki, równa \SI{22.6}{\kcalpm}, wydaje się realistyczna.
Nie jest ona wiele większa niż raportowana przez Wanga~i~in. bariera hydrolizy kompleksu
  wywiedzionego z~amidu trzeciorzędowego, prowadząca do~utworzenia kationu iminiowego,
  która jest równa \SI{19.8}{\kcalpm}\sidecite{wang10}.

\subsection{Formowanie się tetrazolu}

Przeprowadziłem również wstępną symulację przebiegu etapu formowania się ugrupowania tetrazolu,
  korzystając z~takiej samej metodologii, jaką opisałem w~poprzednich akapitach.
Jej wyniki prezentuję na~\cref{sch:calc-tetrazole} \--- znajdujące się na~nim wartości energii
  poszczególnych etapów odnoszą się do~\refcmpd{int-1-a}\sidenote{\See{sch:calc-imine}.}.
Inaczej niż w~przykładzie modelowym, stosuję tu izocyjanek \iupac{4-metoksyfenylu}
  (\ch{\acrshort{pmp}NC}) zamiast izocyjanku cykloheksylu, celem dalszego uproszczenia układu,
  a~tym samym minimalizacji kosztu obliczeń.

\begin{scheme}
  \includesvg{calc-tetrazole}
  \caption{
    Relatywne wartości energii (dane w~\si{\kcalpm} względem \refcmpd{int-1-a}),
      obliczone dla poszczególnych kroków w~proponowanym mechanizmie tworzenia tetrazolu.
  }\label{sch:calc-tetrazole}
  \vspace{-8em}
\end{scheme}

Kluczowym założeniem jest, że dysocjacja \ch{TMSN3},
  która następuje w~początkowym etapie przemiany,
  jest wspomagana wolną parą elektronową z~atomu azotu iminy,
  jak to przedstawia stan przejściowy \refcmpd{ts-3}.
Powstaje w~jej wyniku anion azydkowy \ch{N3-} oraz sililowany kation iminiowy \refcmpd{int-4}.
Dalsza część tego wariantu reakcji Ugiego przebiega analogicznie do~opisanego wcześniej,
  ogólnie przyjętego mechanizmu reakcji azydo-Ugiego\sidenote{\See{sch:mechanism-classic}.}.
Aktywowana imina \refcmpd{int-4} ulega addycji izocyjanku poprzez stan przejściowy \refcmpd{ts-4},
  a~następnie przyłączeniu zdysocjowanego wcześniej \ch{N3-}.
Powstaje obojętna elektronowo struktura pośrednia \refcmpd{int-6}, ulegająca następnie
  cyklizacji do~tetrazolu \refcmpd{int-7}.

% compare with sharpless02 DFT calculations
% unsuccessful attempts at Glu-N-TMS synthesis
