\section{Przebieg klasycznej reakcji azydo-Ugiego}\label{numeric:classical}
W~sekcji \secref{synthesis:sugars:ugi} cytowałem pracę Ugiego i~Steinbr{\"u}cknera\sidecite{ugi61},
  którzy po~raz pierwszy zaprezentowali reakcję syntezy tetrazoli, bazującą
  na~wieloskładnikowej reakcji Ugiego.
Autorzy użyli do~jej przeprowadzenia, zależnie od~przykładu, azotowodoru (\ch{HN3}) w~benzenie
  z~metanolem albo azydku sodu w~acetonie z~dodatkiem wody i~\ch{HCl}, generując azotowodór \insitu.
Późniejsze prace często używają azydku trimetylosililu (\ch{\acrshort{tms}N3}) jako źródła
  anionu azydkowego, zazwyczaj w~obecności metanolu jako rozpuszczalnika,
  promującego jego hydrolizę.

% mechanism Maleki 2015 sch. 41

\section{Rozważania mechanistyczne}\label{numeric:mechanism}

% our mechanism proposition
% HNMR of reduction (MMW-387-A) with 7.47 ppm imine signal
% my DFT of spontaneous imine formation
% ? compare with calculations of iminium cation from Schwartz complex formation ?
% unsuccessful attempts at Glu-N-TMS synthesis
% my DFT of tetrazole formation
% compare with sharpless02 DFT calculationss
