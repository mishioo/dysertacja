\section{Przebieg klasycznej reakcji azydo-Ugiego}\label{numeric:classical}
W~sekcji \secref{synthesis:sugars:ugi} cytowałem pracę Ugiego i~Steinbr{\"u}cknera\sidecite{ugi61},
  którzy po~raz pierwszy zaprezentowali reakcję syntezy tetrazoli, bazującą
  na~wieloskładnikowej reakcji Ugiego.
Autorzy użyli do~jej przeprowadzenia, zależnie od~przykładu, azotowodoru (\ch{HN3}) w~benzenie
  z~metanolem albo azydku sodu w~acetonie z~dodatkiem wody i~\ch{HCl}, generując azotowodór \insitu.
Późniejsze prace często używają azydku trimetylosililu (\ch{\acrshort{tms}N3}) jako źródła
  anionu azydkowego, zazwyczaj w~obecności metanolu jako rozpuszczalnika.

\begin{scheme}
  \includesvg{mechanism-classic}
  \caption{
    Ogólnie przyjęty mechanizm reakcji azydo-Ugiego, wykorzystujący \ch{\acrshort{tms}N3}.
    Metanol (lub inne źródło protonu) jest niezbędny zarówno do~aktywacji
      iminy~\refcmpd{w:imine} na~atak izocyjanku, jak i~do~wytworzenia anionu azydkowego.
    W~takich warunkach reakcji może powstawać też produkt uboczny \refcmpd{n-tetrazole}
      w~wyniku cyklizacji samego izocyjanku z~anionem azydkowym.
  }\label{sch:mechanism-classic}
\end{scheme}

Badacze zdają się być zgodni co~do~konieczności użycia źródła protonu do~przeprowadzenia tej reakcji
  \--- czy to w~postaci protonowego medium reakcyjnego, czy dodatku kwasu\sidecite{neochoritis19}.
Owo źródło protonu przyczynia się zarówno do~aktywacji iminy na~atak izocyjanku, jak i~do~hydrolizy
  \ch{\acrshort{tms}N3} do~anionu azydkowego, który to jest faktycznym reagentem.
\Cref{sch:mechanism-classic} prezentuje ogólnie przyjęty mechanizm powstawania tetrazolu
  w~klasycznej wersji reakcji azydo-Ugiego, wykorzystującej kondensację pierwszorzędowej
  aminy i~aldehydu do~utworzenia iminy~\refcmpd{w:imine}.
Kation iminiowy \refcmpd{w:iminium-h} łatwo ulega addycji izocyjanku, prowadząc do~powstania
  kationu nitryliowego, który to ulega addycji anionu azydkowego.
Cyklizacja do~\refcmpd{aminotetrazole-sec} następuje najpewniej poprzez stan
  przejściowy~\refcmpd{amino-imino-azide} niż w~jednym etapie
  z~\refcmpd{nitrilium-sec}\sidecite{sharpless02}.

\section{Rozważania mechanistyczne}\label{numeric:mechanism}

% our mechanism proposition
% HNMR of reduction (MMW-387-A) with 7.47 ppm imine signal
% my DFT of spontaneous imine formation
% ? compare with calculations of iminium cation from Schwartz complex formation ?
% unsuccessful attempts at Glu-N-TMS synthesis
% my DFT of tetrazole formation
% compare with sharpless02 DFT calculationss
