\section{Badanie mechanizmu reakcji metodami obliczeniowymi}\label{experimental:mech}

% void
\begin{table*}[b!]
  \caption[][-1em]{
    Podane w~jednostkach Hartree wartości sum energii elektronowych i~oznaczonych
      w~nagłówku kolumny, obliczone dla struktur zaangażowanych w~przebieg badanej reakcji,
      przy użyciu teorii na~poziomie B3LYP/Def2TZVP oraz z~uwzględnieniem empirycznej poprawki
      dyspersyjnej GD3.
  }\label{tab:en-void}
  \setfloatalignment{b}
  \begin{tabular}{ c l S S S S }
    \toprule
    \textnumero & Struktura & {Energia punktu zerowego} & {Energia cieplna} & {Entalpia} & {Energia Gibbsa} \\
    \midrule
    \rownumber & \refcmpd{int-1-a} & -1221.100727 & -1221.082335 & -1221.081391 & -1221.147049 \\
    \rownumber & \refcmpd{ts-1-a} & -1221.056261 & -1221.037345 & -1221.036401 & -1221.103220 \\
    \rownumber & \refcmpd{int-1-b} & -1221.097576 & -1221.079218 & -1221.078274 & -1221.143662 \\
    \rownumber & \refcmpd{ts-1-b} & -1220.997720 & -1220.979655 & -1220.978711 & -1221.043114 \\
    \rownumber & \refcmpd{int-2-a} & -1221.095618 & -1221.076152 & -1221.075208 & -1221.143571 \\
    \rownumber & \refcmpd{int-2-b} & -1220.996618 & -1220.978030 & -1220.977086 & -1221.042461 \\
    \rownumber & \refcmpd{int-3} & -250.657212 & -250.651936 & -250.650992 & -250.685660 \\
    \rownumber & \refcmpd{ts-3} & -824.136544 & -824.120842 & -824.119898 & -824.178481 \\
    \rownumber & \refcmpd{int-4} & -659.723031 & -659.710379 & -659.709435 & -659.760382 \\
    \rownumber & \refcmpd{ts-4} & -1098.757589 & -1098.735845 & -1098.734901 & -1098.807223 \\
    \rownumber & \refcmpd{int-5} & -1098.761665 & -1098.739746 & -1098.738802 & -1098.811413 \\
    \rownumber & \refcmpd{ts-5} & -1263.191306 & -1263.165902 & -1263.164957 & -1263.246048 \\
    \rownumber & \refcmpd{int-6} & -1263.263938 & -1263.240007 & -1263.239063 & -1263.315595 \\
    \rownumber & \refcmpd{ts-6} & -1263.244227 & -1263.220917 & -1263.219973 & -1263.295234 \\
    \rownumber & \refcmpd{int-7} & -1263.282628 & -1263.259645 & -1263.258701 & -1263.333085 \\
    \rownumber & \refcmpd{int-8} & -854.597569 & -854.582008 & -854.581064 & -854.640248 \\
    \rownumber & \ch{PhCH2NC} & -439.032036 & -439.023634 & -439.022690 & -439.065131 \\
    \rownumber & \ch{TMSN3} & -573.532370 & -573.522633 & -573.521688 & -573.566263 \\
    \rownumber & \ch{Me3SiOH} & -485.145316 & -485.136801 & -485.135857 & -485.176639 \\
    \rownumber & \ch{Cp2Zr(OH)Cl} & -970.420570 & -970.407638 & -970.406693 & -970.459778 \\
    \rownumber & \ch{H2O} & -76.441875 & -76.439040 & -76.438096 & -76.459532 \\
    \rownumber & \ch{N3-} & -164.290536 & -164.287576 & -164.286632 & -164.307871 \\
    \bottomrule
  \end{tabular}
\end{table*}

Obliczenia związane z~symulacją mechanizmu badanego wariantu reakcji azydo-Ugiego przeprowadziłem
  korzystając z~oprogramowania Gaussian~09\sidecite[-1em]{gaussian09}.
Konstruując początkowe geometrie stanów stacjonarnych \refcmpd{int-1-a, int-1-b, int-2-a},
  a~także stanu przejściowego \refcmpd{ts-1-a}, wzorowałem się na~strukturach kompleksów
  cyrkonowych zaproponowanych przez Wanga i~in. w~ich badaniach \gls{dft} mechanizmu
  redukcji amidów trzeciorzędowych odczynnikiem Schwartza\sidecite[-6em]{wang10}.
W~przypadku struktur \refcmpd{ts-4, int-5, ts-5, int-6, ts-6} za~punkt odniesienia posłużyły
  mi geometrie zaprezentowane przez zespół Sharplessa w~ramach ich prac nad ustaleniem mechanizmu
  formowania się tetrazoli w~wyniku reakcji azydków z~nitrylami\sidecite[9em]{sharpless02}.

% solvent
\begin{table*}[t!]
  \caption[][1em]{
    Podane w~jednostkach Hartree wartości sum energii elektronowych i~oznaczonych
      w~nagłówku kolumny, obliczone dla struktur zaangażowanych w~przebieg badanej reakcji,
      przy użyciu teorii na~poziomie B3LYP/Def2TZVP oraz z~uwzględnieniem empirycznej poprawki
      dyspersyjnej GD3 i~z~zastosowaniem modelu solwatacyjnego PCM dla \gls{thf}.
  }\label{tab:en-solv}
  \begin{tabular}{ c l S S S S }
    \toprule
    \textnumero & Struktura & {Energia punktu zerowego} & {Energia cieplna} & {Entalpia} & {Energia Gibbsa} \\
    \midrule
    \rownumber & \refcmpd{int-1-a} & -1221.110627 & -1221.092041 & -1221.091096 & -1221.157622 \\
    \rownumber & \refcmpd{ts-1-a} & -1221.074579 & -1221.055690 & -1221.054746 & -1221.121681 \\
    \rownumber & \refcmpd{int-1-b} & -1221.106030 & -1221.087553 & -1221.086609 & -1221.152491 \\
    \rownumber & \refcmpd{ts-1-b} & -1221.010227 & -1220.991674 & -1220.990729 & -1221.057534 \\
    \rownumber & \refcmpd{int-2-a} & -1221.106694 & -1221.087804 & -1221.086860 & -1221.153766 \\
    \rownumber & \refcmpd{int-2-b} & -1221.009687 & -1220.990953 & -1220.990009 & -1221.056049 \\
    \rownumber & \refcmpd{int-3} & -250.661194 & -250.655917 & -250.654973 & -250.689640 \\
    \rownumber & \refcmpd{ts-3} & -824.174398 & -824.158746 & -824.157802 & -824.216465 \\
    \rownumber & \refcmpd{int-4} & -659.784677 & -659.772021 & -659.771077 & -659.821943 \\
    \rownumber & \refcmpd{ts-4} & -1098.809155 & -1098.787496 & -1098.786552 & -1098.858383 \\
    \rownumber & \refcmpd{int-5} & -1098.814285 & -1098.792355 & -1098.791411 & -1098.864088 \\
    \rownumber & \refcmpd{ts-5} & -1263.213899 & -1263.188581 & -1263.187637 & -1263.268579 \\
    \rownumber & \refcmpd{int-6} & -1263.270157 & -1263.246135 & -1263.245191 & -1263.322083 \\
    \rownumber & \refcmpd{ts-6} & -1263.250449 & -1263.227051 & -1263.226107 & -1263.301711 \\
    \rownumber & \refcmpd{int-7} & -1263.293985 & -1263.270959 & -1263.270015 & -1263.344561 \\
    \rownumber & \refcmpd{int-8} & -854.609153 & -854.593564 & -854.592619 & -854.651964 \\
    \rownumber & \ch{PhCH2NC} & -439.038580 & -439.030218 & -439.029274 & -439.071661 \\
    \rownumber & \ch{TMSN3} & -573.536386 & -573.526581 & -573.525636 & -573.570378 \\
    \rownumber & \ch{Me3SiOH} & -485.149910 & -485.141297 & -485.140353 & -485.181359 \\
    \rownumber & \ch{Cp2Zr(OH)Cl} & -970.431956 & -970.418867 & -970.417923 & -970.471550 \\
    \rownumber & \ch{H2O} & -76.448012 & -76.445177 & -76.444233 & -76.465669 \\
    \rownumber & \ch{N3-} & -164.379062 & -164.376100 & -164.375156 & -164.396397 \\
    \bottomrule
  \end{tabular}
\end{table*}

Wszystkie struktury zoptymalizowałem używając funkcjonału B3LYP i~stosując bazę LANL2DZ
  do~opisu atomów \ch{Zr} oraz 6-31G(d,p) do~opisu pozostałych atomów,
  z~uwzględnieniem empirycznej poprawki dyspersyjnej GD3.
Potwierdziłem, że proponowane stany przejściowe rzeczywiście łączą postulowane stany stacjonarne
  wyznaczając ścieżkę reakcji w~obydwu kierunkach\sidenote[4em]{
    W~języku angielskim nazywa się tego typu eksperyment
      \textit{following the intristic reaction coordinate (IRC)}. 
  }.
Geometrie zoptymalizowanych struktur dostępne są formie w~elektronicznej na~dołączonej
  do~dysertacji płycie CD oraz w~repozytorium\sidenote{\repourl} z~kodem źródłowym tej pracy
  w~katalogu \enquote{supplementary}.

Zoptymalizowane struktury poddałem powtórnej analizie, używając większej bazy Def2TZVP.
Porównałem wyniki symulacji w~próżni z~uzyskanymi przy~zastosowaniu modelu solwatacyjnego
  PCM dla \gls{thf}, tak jak został on zaimplementowany w~programie Gaussian~09.
W~pobliskich tabelach \cref{tab:en-void,tab:en-solv} prezentuję wartości energii poszczególnych
  struktur, uzyskanych w~obydwu przypadkach.

\section{Analiza widm chiralooptycznych}\label{experimental:spectra}

Widma \gls{uv} związków \refcmpd{gal-tet.cy,gal-epi-tet.cy} zostały zarejestrowane przy użyciu
  spektrofotometru Jasco~V\-/670 w~\ch{CH3CN}.
Widma te, wraz z~ich symulowanymi odpowiednikami, prezentuję na~\cref{fig:spectra-uv}.
Widma \gls{ecd} również zostały zarejestrowane w~\ch{CH3CN}, ale za~pomocą spektropolarymetru
  Jasco~J\-/815 w~zakresach \SIrange{195}{450}{\nano\meter} oraz \SIrange{225}{450}{\nano\meter}
  w~kuwetach kwarcowych o~drodze optycznej odpowiednio \SIrange{0.02}{2.0}{\centi\meter}.
Widma rejestrowane były przy stężeniu próbki \SI{0.00029}{\molar},
  z~szybkością \SI{100}{\nm\per\minute}, szerokością szczeliny \SI{1}{\nm}, liczbą akumulacji 5
  i~5~punktami na~nanometr.

Pierwszym krokiem do~symulacji widm było przeprowadzenie analizy konformacyjnej
  proponowanych struktur związków \refcmpd{gal-tet.cy,gal-epi-tet.cy}, którą wykonałem
  za~pomocą programu CONFLEX.
Otrzymane struktury konformerów zoptymalizowałem korzystając z~programu
  Gaussian~09\sidecite{gaussian09} na~poziomie teorii B3LYP/6-31G**.
Zoptymalizowane, najniżej energetyczne struktury (do~\SI{3}{\kcalpm}) wykorzystałem w~obliczeniach
  elektronowej aktywności optycznej w~programie Gaussian~16\sidecite{gaussian16}, stosując teorię
  na~poziomie B3LYP/TZVP z~zastosowaniem modelu rozpuszczalnikowego PCM dla \ch{CH3CN}.
Widma symulowane do~porównania z~eksperymentalnymi obliczyłem przy użyciu autorskiego
  programu komputerowego \texttt{tesliper}\sidenote{
    Program ten opisałem w~niniejszej dysertacji w~rozdziale \secref{chapter:tesliper}.}.
Prezentowane widma zostały uzyskane z~użyciem rozkładu Gaussa o~szerokości \SI{0.15}{\electronvolt}
  do~opisu kształtu pików i~skorygowane hipsochromowo o~\SI{15}{\nano\meter}.

\begin{figure}
  \begin{tikzpicture}
    \begin{axis}[
      ylabel={$\epsilon\ /\si{\deci\meter\cubed\per\mole\per\centi\meter}$},
      xlabel={$\lambda\ /\si{\nm}$},
      ymin=0, ymax=1.95,
      xmin=180, xmax=300,
      axis lines=left,
      axis line style={-},
      legend style={at={(0.97,0.9)}, anchor={north east}, draw=none},
      width=\textwidth,
      height=13em,
    ]
    
    \addplot[color=wongvermillion]
      table {chapter-3/stereochemistry/gal-uv-data.txt};
    \addlegendentry{\refcmpd{gal-tet.cy}}
      
    \addplot[color=wongblue]
      table {chapter-3/stereochemistry/epi-uv-data.txt};
    \addlegendentry{\refcmpd{gal-epi-tet.cy}}

    \addplot[color=wongpurple]
      table [x expr=\thisrow{x-s-uv}-15, y expr=\thisrow{y-s-uv}/10000]
      {chapter-3/stereochemistry/simulated-data.txt};
    \addlegendentry{Sym. \iupac{\cip{2S}}$\ /10^4$}
    
    \addplot[color=wonggreen]
      table [x expr=\thisrow{x-r-uv}-15, y expr=\thisrow{y-r-uv}/10000]
      {chapter-3/stereochemistry/simulated-data.txt};
    \addlegendentry{Sym. \iupac{\cip{2R}}$\ /10^4$}
      
    \end{axis}
  \end{tikzpicture}

  \caption{
    Zarejestrowane i~symulowane widma \gls{uv} badanych związków
      \refcmpd{gal-tet.cy, gal-epi-tet.cy}.
    Dla lepszej czytelności wartości intensywności poszczególnych widm zostały przeskalowane
      o~współczynnik podany w~legendzie.
  }
  \label{fig:spectra-uv}
\end{figure}