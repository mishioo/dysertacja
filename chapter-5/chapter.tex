\chapter{Detale techniczne}\label{chapter:experimental}

\section{Procedury}\label{experimental:procedures}
\subsection{%
  \refcmpd{cyclopropyl-dimethyl-malonate}:
  \iupac{1,1-cyclopropylodikarboksylan dimetylu}%
}\label{syn:cyclopropyl-dimethyl-malonate}
\marginnote{\includesvg{cyclopropyl-dimethyl-malonate}}
\marginnote{Analiza \ch{^1HNMR} zgodna z~literaturą.}
W kolbie umieściłem \SI{17.25}{\gram} \ch{K2CO3}, \SI{25}{\mL} \ch{PhMe},
  \SI{0.25}{\mL} \ch{H2O}, \SI{0.2}{\gram} \ch{Bn4N+ Br-}, \SI{5.8}{\mL}
  malonianu dimetylu oraz \SI{8.7}{\mL} \iupac{1,2-dibromoetanu}.
Mieszałem intensywnie przez \SI{4}{\day}, po~czym odsączyłem.
Osad przemyłem \SI[product-units = single]{3 x 25}{\mL} toluenu.
Zebrane frakcje organiczne odparowałem przy użyciu wyparki rotacyjnej.
Na~podstawie analizy \ch{^1HNMR} oszacowałem skład mieszaniny na~ok. $1:1$ substratu i~produktu.
Dodałem do~mieszaniny jeszcze \SI{9}{\gram} \ch{K2CO3}, \SI{25}{\mL} \ch{PhMe},
  \SI{2}{\mL} \ch{H2O}, \SI{0.1}{\gram} \ch{Bn4N+ Br-} oraz \SI{4.5}{\mL}
  \iupac{1,2-dibromoetanu}.
Mieszałem przez noc w~atmosferze argonu pod chłodnicą zwrotną w~temperaturze \SI{115}{\degC}.
Po~tym czasie odsączyłem i~destylowałem pod~zmniejszonym ciśnieniem (\SI{20}{\milli\bar}).
Otrzymałem \SI{3.14}{\gram} cieczy (\SI{40}{\percent} wydajności).

\subsection{%
  \refcmpd{cyclopropyl-monomethyl-malonate}:
  \iupac{kwas 1-(metoksykarbonylo)cyklopropanokarboksylowy}%
}\label{syn:cyclopropyl-monomethyl-malonate}
\marginnote{\includesvg{cyclopropyl-monomethyl-malonate}}
Do \SI{3}{\gram} związku \refcmpd{cyclopropyl-dimethyl-malonate} dodałem \SI{1066}{\milli\gram}
  \ch{KOH} rozpuszczonego w~\SI{15}{\mL} bezwodnego \ch{MeOH} w~atmosferze argonu.
Mieszałem we~wrzeniu przez \SI{2}{\hour}, po~czym usunąłem rozpuszczalnik
  przy użyciu wyparki rotacyjnej.
Otrzymany osad rozpuściłem w~\SI{10}{\mL} \ch{H2O}, zakwasiłem do~\pH$=7$ za~pomocą
  \SI{10}{\percent} \ch{HCl_{aq}}, nasyciłem \ch{NaCl} i~ekstrahowałem
  \SI[product-units = single]{4 x 10}{\mL} \ch{Et2O}.
Zebrane frakcje organiczne suszyłem \ch{MgSO4} i~usunąłem z~nich rozpuszczalnik
  przy użyciu wyparki rotacyjnej.
Otrzymałem \SI{2.22}{\gram} cieczy, będącej mieszaniną substratu i~produktu w~stosunku
  $1:4,44$ wg. analizy \ch{^1HNMR} (\SI{66}{\percent} wydajności).
Mieszaniny użyłem w~następnym etapie syntezy bez oczyszczania.

\subsection{%
  \refcmpd{cyclopropyl-methyl-malonate-chloride}:
  \iupac{chlorek kwasu 1-(metoksykarbonylo)cyklopropanokarboksylowego}%
}\label{syn:cyclopropyl-methyl-malonate-chloride}
\marginnote{\includesvg{cyclopropyl-methyl-malonate-chloride}}
Nieoczyszczony związek \refcmpd{cyclopropyl-monomethyl-malonate} rozpuściłem w~\SI{40}{\mL}
  bezwodnego \gls{dcm} w~atmosferze argonu. Dodałem \SI{2.2}{\mL} chlorku oksalilu
  i~\num{3} krople \gls{dmf}.
Mieszałem w~temperaturze pokojowej przez \SI{100}{\minute}, po~czym usunąłem rozpuszczalnik
  przy użyciu wyparki rotacyjnej.
Produktu użyłem w~następnym etapie syntezy bez oczyszczania.

\subsection{%
  \refcmpd{amidoester-cycloprop}:
  \iupac{1-[(4-metoksyfenylo)karbamylo]cyklopropanokarboksylan metylu}%
}\label{syn:amidoester-cycloprop}
\marginnote{\includesvg{amidoester-cycloprop}}
W~atmosferze argonu rozpuściłem surową mieszaninę poreakcyjną zawierającą związek
  \refcmpd{cyclopropyl-methyl-malonate-chloride} w~\SI{20}{\mL} acetonu.
Dodałem \SI{1550}{\milli\gram} \iupac{p-anizydyny} oraz \SI{1.75}{\mL} \ch{Et3N}.
Mieszałem w~temperaturze pokojowej przez noc, po~czym usunąłem rozpuszczalnik przy użyciu
  wyparki rotacyjnej.
Otrzymany osad zawiesiłem w~\SI{25}{\mL},
  ekstrahowałem \SI[product-units = single]{4 x 15}{\mL},
  suszyłem \ch{MgSO4}, odparowałem rozpuszczalnik za~pomocą wyparki rotacyjnej.
Oczyszczałem chromatograficznie na~żelu krzemionkowym w~eluencie \SI{20}{\percent} octanu
  etylu w~heksanie.
Otrzymałem \SI{2.5}{\gram} produktu (\SI{80}{\percent} wydajności).

% \subsection{%
%   \refcmpd{}:
%   \iupac{}%
% }\label{syn:}

\section{Analizy}\label{experimental:analyses}

