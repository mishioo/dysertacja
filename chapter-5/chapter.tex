\chapter{Podsumowanie}\label{chapter:conclusions}
Podczas prac powadzonych w~ramach niniejszej dysertacji w~pełni zrealizowałem wszystkie
  przedsięwzięte cele badawcze.
Poddałem próbie współcześnie dostępne metody reduktywnej aktywacji amidów,
  używając ich w~złożonych układach reakcyjnych \--- stosując niebanalne substraty i~przemiany.
Skuteczne przekształcenie wybranych amidów w~oczekiwane funkcjonalizowane aminy okazało się
  być wyzwaniem znacznie większym, niż można by przypuszczać na~podstawie przeprowadzonych
  przeze mnie wnikliwych studiów literatury fachowej.

Analizie poddałem dwa typy układów reakcyjnych \--- proste przekształcenia amidoestrów
  wywiedzionych z~kwasu malonowego oraz przekształcenia laktamów wywiedzionych
  z~cukrów prostych w~wariancie wieloskładnikowej reakcji Ugiego.
Pokazałem, że w~obydwu tych przypadkach jedynie zastosowanie odczynnika Schwartza pozwala
  przeprowadzić reduktywną aktywację zgodnie z~oczekiwaniami.
Zarówno procedury wykorzystujące katalityczne ilości kompleksów irydu, jak i~lepiej poznana
  metoda oparta o~użycie bezwodnika triflowego okazały się nieskuteczne.

Pokazałem również, że nawet w~przypadku powodzenia w~zastosowaniu procedury reduktywnej aktywacji,
  skala sukcesu jest w~znacznym stopniu zależna od~dokładnej struktury substratu.
Nawet niewielka zmiana może znacząco wpłynąć na~wynik \--- najdobitniej świadczy o~tym chyba
  przykład zastąpienia pierścienia cyklopropylowego dwoma geminalnymi grupami metylowymi\sidenote{%
    Chodzi tu o~przykład związków \refcmpd{amidoester-cycloprop, amidoester-dime},
      \see{sch:amidoester-other} i~wcześniej.},
  prowadzące do~całkowitej degradacji reaktywności pobliskiej grupy amidowej.
Podobnie, zmiana rozmiaru pierścienia laktamu z~6- na~5-członowy nieoczekiwanie prowadziła
  do~powstawania jedynie śladów oczekiwanego produktu reakcji azydo-Ugiego.
Mimo tych trudności przygotowałem kilka przykładów funkcjonalizowanych \textbeta-aminoestrów
  oraz pochodnych iminocukrów.

