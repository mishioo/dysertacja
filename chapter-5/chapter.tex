\chapter{Detale techniczne}\label{chapter:experimental}

\section{Ogólne procedury aktywacji amidów}\label{experimental:amidoester-activation}
\subsubsection{Procedura wykorzystująca odczynnik Schwartza}
Do wygrzanego w~płomieniu palnika i~wypełnionego gazem obojętnym naczynia Schlenka odważyłem
\SI{0.2}{\mmol} odczynnika Schwartza oraz \SI{0.2}{\mmol} amidu\sidenote[][-2\baselineskip]{
  W~przypadku amidu będącego cieczą lub olejem dodawałem przygotowany osobno roztwór amidu
    do~zawiesiny odczynnika Schwartza.
},
  dodałem \SI{1.0}{\mL} suchego \gls{thf}\sidenote{
    W przypadku prowadzenia reakcji w~obniżonej temperaturze dodałem rozpuszczalnik do~samego
      odczynnik Schwartza, ochłodziłem w~łaźni izopropanol\--suchy lód, i~dopiero dodałem amid.
  }.
Mieszałem do~sklarowania roztworu.
W~przepływie argonu dodałem \SI{0.4}{\mmol} wybranego kwasu oraz \SI{0.2}{\mmol} nukleofila\sidenote{
  W przypadku allilotributylocyny użyłem \SI{0.6}{\mmol}.
}.
Metodę terminacji reakcji oraz oczyszczania produktów dobierałem do~każdego przypadku osobno.

\subsubsection{Procedura wykorzystująca kompleks Vaski}
Do wygrzanego w~płomieniu palnika i~wypełnionego gazem obojętnym naczynia Schlenka odważyłem
  \SI{0.002}{\mmol} \ch{IrCl(CO)(PPh3)2} oraz \SI{0.2}{\mmol} amidu.
Rozpuściłem w~\SI{2.0}{\ml} suchego toluenu.
Dodałem \SI{0.4}{\mmol} \gls{tmds} i~mieszałem przez noc.
Po tym czasie dodałem \SI{0.6}{\mmol} allilotributylocyny oraz \SI{0.4}{\mmol} \ch{BF3.OEt2}.
Mieszałem jeszcze przez \SI{3}{\day}, po~czym wylałem na~\SI{4}{\ml} \ch{NaHCO3_{(aq)}},
  ekstrahowałem \SI[product-units = single]{2 x 4}{\mL} \ch{Et2O}, zebrane frakcje organiczne
  przemyłem \SI[product-units = single]{3 x 5}{\mL} \SI{10}{\percent} \ch{NH4F_{(aq)}},
  suszyłem \ch{MgSO4}, po~czym odparowałem rozpuszczalnik za~pomocą wyparki rotacyjnej.

\subsubsection{Procedura wykorzystująca kompleks van der Enta}
Do wygrzanego w~płomieniu palnika i~wypełnionego gazem obojętnym naczynia Schlenka odważyłem
  \SI{0.002}{\mmol} \ch{[Ir(coe)2Cl]2} oraz \SI{0.2}{\mmol} amidu.
Rozpuściłem w~\SI{2.0}{\ml} \ch{CH2Cl2}.
Dodałem \SI{0.4}{\mmol} \ch{Et2SiH2} i~mieszałem przez noc.
Po tym czasie dodałem \SI{0.6}{\mmol} allilotributylocyny oraz \SI{0.4}{\mmol} \ch{BF3.OEt2}.
Mieszałem jeszcze przez \SI{3}{\day}, po~czym wylałem na~\SI{5}{\ml} \ch{NaHCO3_{(aq)}},
  ekstrahowałem \SI[product-units = single]{2 x 4}{\mL} \ch{Et2O}, zebrane frakcje organiczne
  przemyłem \SI[product-units = single]{3 x 5}{\mL} \SI{10}{\percent} \ch{NH4F_{(aq)}},
  suszyłem \ch{MgSO4}, po~czym odparowałem rozpuszczalnik za~pomocą wyparki rotacyjnej.

\subsubsection{Procedura wykorzystująca bezwodnik triflowy}
Do wygrzanego w~płomieniu palnika i~wypełnionego gazem obojętnym naczynia Schlenka odważyłem
  \SI{0.2}{\mmol} amidu i~rozpuściłem w~\SI{2.0}{\ml} \ch{CH2Cl2}.
Dodałem \SI{0.24}{\mmol} \iupac{2-fluoropirydyny}, po czym ochłodziłem do~\SI{0}{\degC}
  na~łaźni izopropanol-suchy lód i~dodałem \SI{0.22}{\mmol} \ch{Tf2O}.
Mieszałem przez \SI{20}{\min}, po~czym dodałem \SI{0.22}{\mmol} \ch{Et3SiH} w~\SI{0}{\degC},
  a~następnie pozwoliłem mieszaninie ogrzać się do temperatury pokojowej i~mieszałem przez noc.
Ponownie ochłodziłem mieszaninę do~\SI{0}{\degC}, dodałem \SI{0.3}{\mmol} \ch{BF3.Et2O},
  a~po~\SI{40}{\min} dodałem jeszcze \SI{0.6}{\mmol} allilotributylocyny.
Mieszałem przez noc, pozwalając by mieszanina ogrzała się do temperatury pokojowej.
Po~tym czasie wylałem na~\SI{5}{\ml} \ch{NaHCO3_{(aq)}},
  ekstrahowałem \SI[product-units = single]{3 x 4}{\mL} \ch{CH2Cl2}, zebrane frakcje organiczne
  przemyłem \SI[product-units = single]{3 x 5}{\mL} \SI{10}{\percent} \ch{NH4F_{(aq)}},
  suszyłem \ch{MgSO4}, po~czym odparowałem rozpuszczalnik za~pomocą wyparki rotacyjnej.

\section{Substraty do~badań nad~aktywacją amidoestrów}\label{experimental:amidoester-substrates}
\begin{scheme}
  \includesvg{amidoester-cycloprop-synthesis}
  \caption{
    Synteza związku modelowego~\refcmpd{amidoester-cycloprop} do~prób aktywacji i~reduktywnej
      funkcjonalizacji amidosetrów o~strukturze kwasu malonowego.
  }
  \label{sch:amidoester-cycloprop-synthesis}
\end{scheme}

\procedure{cyclopropyl-dimethyl-malonate}{\iupac{1,1-cyclopropylodikarboksylan dimetylu}}
\marginnote{Analiza \NMR*{} zgodna z~literaturą.}
W kolbie umieściłem \SI{17.25}{\gram} \ch{K2CO3}, \SI{25}{\mL} \ch{PhMe},
  \SI{0.25}{\mL} \ch{H2O}, \SI{0.2}{\gram} \ch{Bn4N+ Br-}, \SI{5.8}{\mL}
  malonianu dimetylu oraz \SI{8.7}{\mL} \iupac{1,2-dibromoetanu}.
Mieszałem intensywnie przez \SI{4}{\day}, po~czym odsączyłem.
Osad przemyłem \SI[product-units = single]{3 x 25}{\mL} toluenu.
Zebrane frakcje organiczne odparowałem przy użyciu wyparki rotacyjnej.
Na~podstawie analizy \NMR*{} oszacowałem skład mieszaniny na~ok. $1:1$ substratu i~produktu.
Dodałem do~mieszaniny jeszcze \SI{9}{\gram} \ch{K2CO3}, \SI{25}{\mL} \ch{PhMe},
  \SI{2}{\mL} \ch{H2O}, \SI{0.1}{\gram} \ch{Bn4N+ Br-} oraz \SI{4.5}{\mL}
  \iupac{1,2-dibromoetanu}.
Mieszałem przez noc w~atmosferze argonu pod chłodnicą zwrotną w~temperaturze \SI{115}{\degC}.
Po~tym czasie odsączyłem i~destylowałem pod~zmniejszonym ciśnieniem (\SI{20}{\milli\bar}).
Otrzymałem \SI{3.14}{\gram} cieczy (\SI{40}{\percent} wydajności).

\procedure{cyclopropyl-monomethyl-malonate}{\iupac{kwas 1-(metoksykarbonylo)cyklopropanokarboksylowy}}
Do \SI{3}{\gram} związku \refcmpd{cyclopropyl-dimethyl-malonate} dodałem \SI{1066}{\milli\gram}
  \ch{KOH} rozpuszczonego w~\SI{15}{\mL} bezwodnego \ch{MeOH} w~atmosferze argonu.
Mieszałem we~wrzeniu przez \SI{2}{\hour}, po~czym usunąłem rozpuszczalnik
  przy użyciu wyparki rotacyjnej.
Otrzymany osad rozpuściłem w~\SI{10}{\mL} \ch{H2O}, zakwasiłem do~$\pH{}=7$ za~pomocą
  \SI{10}{\percent} \ch{HCl_{aq}}, nasyciłem \ch{NaCl} i~ekstrahowałem
  \SI[product-units = single]{4 x 10}{\mL} \ch{Et2O}.
Zebrane frakcje organiczne suszyłem \ch{MgSO4} i~usunąłem z~nich rozpuszczalnik
  przy użyciu wyparki rotacyjnej.
Otrzymałem \SI{2.22}{\gram} cieczy, będącej mieszaniną substratu i~produktu w~stosunku
  $1:4,44$ wg. analizy {\NMR*} (\SI{66}{\percent} wydajności).
Mieszaniny użyłem w~następnym etapie syntezy bez oczyszczania.

\procedure{cyclopropyl-methyl-malonate-chloride}{\iupac{chlorek kwasu 1-(metoksykarbonylo)cyklopropanokarboksylowego}}
Nieoczyszczony związek \refcmpd{cyclopropyl-monomethyl-malonate} rozpuściłem w~\SI{40}{\mL}
  bezwodnego \gls{dcm} w~atmosferze argonu. Dodałem \SI{2.2}{\mL} chlorku oksalilu
  i~\num{3} krople \gls{dmf}.
Mieszałem w~temperaturze pokojowej przez \SI{100}{\minute}, po~czym usunąłem rozpuszczalnik
  przy użyciu wyparki rotacyjnej.
Produktu użyłem w~następnym etapie syntezy bez oczyszczania.

\procedure{amidoester-cycloprop}{\iupac{1-[(4-metoksyfenylo)karbamylo]cyklopropanokarboksylan metylu}}
W~atmosferze argonu rozpuściłem surową mieszaninę poreakcyjną zawierającą związek
  \refcmpd{cyclopropyl-methyl-malonate-chloride} w~\SI{20}{\mL} acetonu.
Dodałem \SI{1550}{\milli\gram} \iupac{p-anizydyny} oraz \SI{1.75}{\mL} \ch{Et3N}.
Mieszałem w~temperaturze pokojowej przez noc, po~czym usunąłem rozpuszczalnik przy użyciu
  wyparki rotacyjnej.
Otrzymany osad zawiesiłem w~\SI{25}{\mL},
  ekstrahowałem \SI[product-units = single]{4 x 15}{\mL},
  suszyłem \ch{MgSO4}, odparowałem rozpuszczalnik za~pomocą wyparki rotacyjnej.
Oczyszczałem chromatograficznie na~żelu krzemionkowym w~eluencie \SI{20}{\percent} octanu
  etylu w~heksanie.
Otrzymałem \SI{2.5}{\gram} produktu (\SI{80}{\percent} wydajności).

\begin{fullexp}
  \NMR(400)[CDCl3] \num{10.65} (s, \#{1}), \numrange{7.57}{7.37} (m, \#{2}), \numrange{6.93}{6.76} (m, \#{2}), \num{3.78} (d, \J{2.2}, \#{3}), \num{3.72} (d, \J{2.2}, \#{3}), \numrange{1.83}{1.75} (m, \#{2}), \numrange{1.67}{1.60} (m, \#{2})\par\noindent
  \NMR{13,C}(101)[CDCl3] \numlist{174.4; 166.5; 156.3; 131.4; 121.8; 114.1; 55.5; 52.4; 26.4; 20.5}
\end{fullexp}
  \todo[inline]{MS, IR}

\section{Funkcjonalizowane aminy wywiedzione z~amidoestrów}\label{experimental:amidoester-products}
\procedure{b-aminoester-cycloprop.allyl}{\iupac{1-(1-[(4-metoksyfenylo)amino]but-3-en-1-ylo)cyklopropanokarboksylan metylu}}
Do wygrzanego w~płomieniu palnika i~wypełnionego argonem naczynia Schlenka odważyłem
  \SI{51.5}{\mg} odczynnika Schwartza oraz \SI{50.0}{\mg} amidu~\refcmpd{amidoester-cycloprop},
  dodałem \SI{1.0}{\mL} \gls{thf}.
Mieszałem do~sklarowania roztworu.
W~przepływie argonu dodałem \SI{249}{\mg} \ch{Yt(OTf)3} oraz \SI{186}{\uL} allilotributylocyny.
Mieszałem jeszcze przez noc, po czym wylałem na~\SI{5}{\mL} \ch{NaHCO3_{aq}},
  przemyłem \SI[product-units = single]{3 x 5}{\mL} \SI{10}{\percent} \ch{NH4F_{aq}}.
Zebrane frakcje organiczne suszyłem \ch{MgSO4}, po~czym odparowałem rozpuszczalnik za~pomocą
  wyparki rotacyjnej.
Oczyszczałem chromatograficznie na~żelu krzemionkowym w~eluencie \SI{15}{\percent} octanu
  etylu w~heksanie.
Otrzymałem \SI{16.9}{\mg} produktu w~postaci oleju (\SI{31}{\percent} wydajności).

\begin{fullexp}
  \NMR(600)[CDCL3] \numrange{6.77}{6.68} (m, \#{2}), \numrange{6.60}{6.51} (m, \#{2}), \numrange{5.88}{5.75} (m, \#{1}), \numrange{5.09}{5.04} (m, \#{1}), \numrange{5.04}{5.00} (m, \#{1}), \num{3.72} (s, \#{3}), \num{3.68} (s, \#{3}), \num{3.43} (dd, \J{8.3;5.5}, \#{1}), \numrange{2.62}{2.48} (m, \#{1}), \numrange{2.40}{2.25} (m, \#{1}), \numrange{1.20}{1.13} (m, \#{2}), \numrange{0.84}{0.76} (m, \#{2})\par\noindent
  \NMR{13,C}(151)[CDCL3] \numlist{174.8; 135.7; 117.0; 115.1; 114.8; 56.6; 55.7; 51.7; 39.0; 26.7; 14.8; 13.0}\par\noindent
  \data{HRMS} (ESI-TOF) m/z calcd for \ch{C16H22NO3}: \num{276.1600} found: \num{276.1597}
\end{fullexp}
  \todo[inline]{IR}

\procedure{b-aminoester-cycloprop.cn}{\iupac{1-(cyjano[(4-metoksyfenylo)amino]metylo)cyklopropanokarboksylan metylu}}
Synteza wg procedury \nameref{syn:b-aminoester-cycloprop.allyl}.
Użyłem \SI{80}{\uL} \ch{\acrshort{tms}CN} jako nukleofila do~funkcjonalizacji.
Oczyszczałem chromatograficznie na~żelu krzemionkowym w~eluencie \SI{20}{\percent} octanu
  etylu w~heksanie.
Otrzymałem \SI{20.4}{\mg} produktu w~postaci oleju (\SI{39}{\percent} wydajności).
  \todo[inline]{HNMR, CNMR, MS, IR}

\procedure{aminoester-plain-allyl}{\iupac{3-[(4-metoksyfenylo)amino]hex-5-enolan etylu}}
Synteza wg procedury \nameref{syn:b-aminoester-cycloprop.allyl}.
Użyłem \SI{50.0}{\mg} (\SI{0.21}{\milli\mole}) \refcmpd{amidoester-plain} jako substratu
  i~odpowiednią ilość pozostałych reagentów.
Oczyszczałem chromatograficznie na~żelu krzemionkowym w~eluencie \SI{10}{\percent} octanu
  etylu w~heksanie.
Otrzymałem \SI{23.0}{\mg} produktu w~postaci oleju (\SI{42}{\percent} wydajności).

\begin{fullexp}
  \NMR(400)[CDCl3] \numrange{6.82}{6.73} (m, \#{2}), \numrange{6.67}{6.55} (m, \#{2}), \numrange{5.88}{5.75} (m, \#{1}), \numrange{5.15}{5.11} (m, \#{1}), \numrange{5.11}{5.06} (m, \#{1}), \num{4.12} (q, \J{7.1}, \#{2}), \numrange{3.84}{3.76} (m, \#{1}), \num{3.74} (s, \#{3}), \numrange{3.65}{3.47} (m, \#{1}), \numrange{2.59}{2.43} (m, \#{2}), \numrange{2.41}{2.28} (m, \#{2}), \num{1.24} (t, \J{7.2}, \#{3})\par\noindent
  \NMR{13,C}(101)[CDCl3] \numlist{171.9; 152.5; 141.0; 134.3; 118.2; 115.5; 115.0; 60.4; 55.8; 51.2; 38.6; 38.6; 14.2}
\end{fullexp}
\todo[inline]{MS, IR}


\procedure{aminoester-bn-allyl}{\iupac{2-benzylo-3-[(4-metoksyfenylo)amino]hex-5-enolan etylu}}
Synteza wg procedury \nameref{syn:b-aminoester-cycloprop.allyl}.
Użyłem \SI{100}{\mg} (\SI{0.306}{\milli\mole}) \refcmpd{amidoester-bn} jako substratu
  i~odpowiednią ilość pozostałych reagentów.
Oczyszczałem chromatograficznie na~żelu krzemionkowym w~eluencie \SI{10}{\percent} octanu
  etylu w~heksanie.
Otrzymałem \SI{24.0}{\mg} produktu w~postaci oleju (\SI{22}{\percent} wydajności).

\todo[inline]{HNMR, CNMR, MS, IR}


