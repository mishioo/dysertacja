\section{Funkcjonalizowane iminocukry}\label{experimental:iminosugars}
\subsubsection{Procedura syntezy tetrazolowych pochodnych iminocukrów}\label{experimental:sugars:schwartz}
Poniższa procedura jest zmodyfikowaną wersją \hyperref[experimental:activation:schwartz]{%
	ogólnej procedury wykorzystującej odczynnik Schwartza}.

Do wygrzanego w~płomieniu palnika i~wypełnionego gazem obojętnym naczynia Schlenka odważyłem
\SI{0.36}{\mmol} odczynnika Schwartza\sidenote[][-5\baselineskip]{%
	Według doświadczeń z~badań prowadzonych wcześniej w~zespole Furmana \SI{1.6}{\equiv}
		jest optymalną ilością odczynnika Schwartza do~aktywacji laktamów
		wywiedzionych z~cukrów prostych: \colorcite{furman14}.
} oraz \SI{0.2}{\mmol} laktamu i~dodałem \SI{4.0}{\mL} suchego \gls{thf}\sidenote{
  W~przypadku laktamu nie będącego ciałem stałym dodawałem \SI{2.0}{\ml} jego roztworu
    w~\gls{thf} do~zawiesiny odczynnika Schwartza w~\SI{2.0}{\ml} \gls{thf}.
}.
Mieszałem do~sklarowania roztworu, zwykle około \SI{2}{\hour}.
W~przepływie argonu dodałem \SI{0.22}{\mmol} wybranego izocyjanku\sidenote{%
		Rozpuszczonego w~\SI{0.5}{\milli\liter} suchego \ch{THF}, jeśli nie była to ciecz.
	} oraz \SI{0.22}{\mmol} (\SI{29.1}{\micro\liter}) \ch{TMSN3}.
Mieszałem przez noc, po~czym odparowałem rozpuszczalnik przy użyciu wyparki rotacyjnej
	i~oczyszczałem chromatograficznie.

% TODO: translate names
\procedure{glu-tet.cy}{\iupac{\cip{2R,3S,4R,5R,6R}-3,4,5-tris(benzyloxy)-6-((benzyloxy)methyl)-2-(1-cyclohexyl-1\H-tetrazol-5-yl)piperidine}}  % MMW-221-A 
Otrzymałem zgodnie z~\hyperref[experimental:sugars:schwartz]{procedurą syntezy tetrazolowych
	pochodnych iminocukrów}, wychodząc z~wywiedzionego z~glukozy laktamu~\refcmpd{glu-lactam}
	i~używając \SI{26.7}{\micro\liter} (\SI{0.22}{\milli\mol}) izocyjanku cykloheksylu.
Oczyszczałem chromatograficznie na~Florisilu\textsuperscript{\textregistered},
	używając jako eluentu octanu etylu w~heksanie w~gradiencie \SIrange{25}{40}{\percent}.
Otrzymałem produkt w~postaci białych igieł (\SI{73}{\percent} wydajności).
Rekrystalizowałem z~mieszaniny eteru dietylowego i~heksanu aby uzyskać kryształy
	do~analizy rentgenostrukturalnej.

\begin{fullexp}
	\NMR(500)[CDCl3] \numrange{7.58}{6.95} (m, \#{20}), \numrange{4.97}{4.87} (m, \#{2}), \numlist{4.90;4.55} (ABq, \J{11.4}, \#{2}), \numlist{4.84;4.58} (ABq, \J{12.1}, \#{2}), \num{4.67} (t, \J{9.1}, \#{1}), \numlist{4.39;4.34} (ABq, \J{11.8}, \#{2}), \num{4.32} (d, \J{6.2}, \#{1}), \numrange{4.02}{3.92} (m, \#{1}), \num{3.84} (dd, \J{9.4;6.2}, \#{1}), \numrange{3.61}{3.54} (m, \#{1}), \num{3.51} (t, \J{9.5}, \#{1}), \numrange{3.51}{3.45} (m, \#{1}), \numrange{3.43}{3.34} (m, \#{1}), \numrange{2.04}{1.64} (m, \#{7}), \numrange{1.31}{1.10} (m, \#{3});
	\NMR{13,C}(126)[CDCl3] \numlist{152.3; 138.7; 138.7; 138.3; 137.9; 128.5; 128.4; 128.4; 128.3; 128.0; 128.0; 127.9; 127.8; 127.8; 127.6; 127.6; 127.5; 83.2; 80.9; 80.4; 75.6; 74.6; 74.5; 73.2; 69.6; 57.6; 53.9; 49.8; 33.3; 32.5; 25.3; 25.3; 24.8};
	\data{IR}[film] \numlist{3334; 3087; 3062; 3030; 2934; 2861; 1953; 1874; 1811; 1604; 1586; 1496; 1453; 1362; 1292; 1247; 1209; 1095; 1068; 1028; 1002};
	\data{HRMS} (ESI-TOF) m/z obliczone dla \ch{C41H47N5O4Na}: \num{696.3526}, zarejestrowane: \num{696.3503};
	\data{[$\alpha^{23}_D$]~$=$} \num{27.3} ($c = 2.10$, \ch{DCM});
	\data{t. topnienia} \numrange{166}{167}\si{\celsius}
\end{fullexp}

\procedure{glu-epi-tet.cy}{\iupac{\cip{2S,3S,4R,5R,6R}-3,4,5-tris(benzyloxy)-6-((benzyloxy)methyl)-2-(1-cyclohexyl-1\H-tetrazol-5-yl)piperidine}}  % MMW-221-B 
Otrzymałem zgodnie z~\hyperref[experimental:sugars:schwartz]{procedurą syntezy tetrazolowych
	pochodnych iminocukrów}, ale dodając przed etapem addycji izocyjanku
	\SI{0.5}{\micro\liter} suchego \ch{MeOH} do~mieszaniny.
Wyszedłem z~wywiedzionego z~glukozy laktamu~\refcmpd{glu-lactam}
	i~użyłem \SI{26.7}{\micro\liter} (\SI{0.22}{\milli\mol}) izocyjanku cykloheksylu.
Oczyszczałem chromatograficznie na~żelu krzemionkowym,
	używając jako eluentu octanu etylu w~heksanie w~gradiencie \SIrange{30}{40}{\percent}.
Otrzymałem produkt w~postaci białego ciała stałego (\SI{37}{\percent} wydajności).

\begin{fullexp}
	\NMR(600)[CDCl3] \numrange{7.62}{7.12} (m, \#{20}), \numlist{4.83;4.74} (ABq, \J{11.1}, \#{2}), \numlist{4.78;4.68} (ABq, \J{11.1}, \#{2}), \numlist{4.76;4.46} (ABq, \J{11.1}, \#{2}), \numlist{4.58;4.44} (ABq, \J{12.1}, \#{2}), \numrange{4.05}{4.00} (m, \#{1}), \numrange{3.83}{3.76} (m, \#{1}), \numrange{3.70}{3.59} (m, \#{3}), \numrange{3.53}{3.47} (m, \#{1}), \numrange{3.44}{3.37} (m, \#{1}), \numrange{3.03}{2.98} (m, \#{1}), \numrange{1.85}{1.75} (m, \#{2}), \numrange{1.62}{1.50} (m, \#{3}), \numrange{1.37}{1.27} (m, \#{3}), \numrange{1.19}{1.01} (m, \#{1});
	\NMR{13,C}(151)[CDCl3] \numlist{168.5; 137.3; 137.2; 137.1; 136.3; 127.5; 127.4; 127.3; 127.3; 127.2; 127.1; 127.0; 126.9; 126.9; 126.7; 126.7; 126.6; 81.7; 78.8; 78.6; 73.9; 73.6; 73.1; 71.9; 68.9; 55.1; 54.0; 46.6; 31.9; 24.5; 23.4};
	\data{IR}[film] \numlist{3340; 3087; 3062; 3030; 2928; 2854; 1951; 1874; 1809; 1666; 1529; 1496; 1453; 1363; 1311; 1252; 1208; 1070; 1028};
	\data{HRMS} (ESI-TOF) m/z obliczone dla \ch{C41H48N5O4}: \num{674.3706}, zarejestrowane: \num{674.3685};
	\data{[$\alpha^{23}_D$]~$=$} \num{21.5} ($c = 1.00$, \ch{DCM});
	\data{t. topnienia} \numrange{153}{154}\si{\celsius}
\end{fullexp}

\procedure{glu-tet.est}{\iupac{Ethyl (5-(\cip{2R,3S,4R,5R,6R}-3,4,5-tris(benzyloxy)-6-((benzyloxy)methyl)piperidin-2-yl)-1\H-tetrazol-1-yl)acetate}}  % MMW-289-B 
Otrzymałem zgodnie z~\hyperref[experimental:sugars:schwartz]{procedurą syntezy tetrazolowych
	pochodnych iminocukrów}, wychodząc z~wywiedzionego z~glukozy laktamu~\refcmpd{glu-lactam}
	i~używając \SI{24.0}{\micro\liter} (\SI{0.22}{\milli\mol}) izocyjanooctanu etylu.
Oczyszczałem chromatograficznie na~żelu krzemionkowym,
	używając jako eluentu octanu etylu w~\ch{DCM} w~gradiencie \SIrange{3}{7}{\percent}.
Otrzymałem produkt w~postaci białego ciała stałego (\SI{49}{\percent} wydajności).

\begin{fullexp}
	\NMR(600)[CDCl3] \numrange{7.35}{7.03} (m, \#{20}), \numlist{5.12;4.61} (ABq, \J{17.7}, \#{2}), \numlist{4.85;4.78} (ABq, \J{10.9}, \#{2}), \numlist{4.79;4.45} (ABq, \J{11.2}, \#{2}), \numlist{4.71;4.49} (ABq, \J{12.1}, \#{2}), \numrange{4.46}{4.42} (m, \#{1}), \numlist{4.33;4.27} (ABq, \J{11.9}, \#{2}), \num{4.29} (d, \J{5.7}, \#{1}), \num{4.06} (q, \J{7.1}, \#{2}), \num{3.81} (dd, \J{8.9;5.7}, \#{1}), \num{3.48} (dd, \J{9.4;4.9}, \#{1}), \num{3.43} (dd, \J{9.7;8.4}, \#{1}), \num{3.40} (dd, \J{9.4;2.8}, \#{1}), \num{3.15} (ddd, \J{9.7;4.9;2.8}, \#{1}), \num{1.14} (t, \J{7.1}, \#{3});
	\NMR{13,C}(151)[CDCl3] \numlist{164.8; 153.1; 137.6; 137.5; 137.2; 136.8; 127.6; 127.4; 127.3; 127.3; 127.0; 127.0; 127.0; 126.8; 126.7; 126.7; 126.6; 126.5; 81.5; 79.5; 78.7; 74.4; 73.5; 73.1; 72.1; 68.4; 61.5; 52.9; 48.9; 47.1; 13.0};
	\data{IR}[film] \numlist{3337; 3087; 3062; 3030; 2982; 2908; 2868; 1955; 1877; 1811; 1751; 1604; 1496; 1453; 1396; 1373; 1308; 1211; 1094; 1069; 1027};
	\data{HRMS} (ESI-TOF) m/z obliczone dla \ch{C39H43N5O6}: \num{678.3292}, zarejestrowane: \num{678.3286};
	\data{[$\alpha^{23}_D$]~$=$} \num{23.0} ($c = 1.00$, \ch{DCM});
	\data{t. topnienia} \numrange{94}{95}\si{\celsius}
\end{fullexp}

\procedure{glu-tet.benz}{\iupac{\cip{2R,3S,4R,5R,6R}- 3,4,5-tris(benzyloxy)-6-[(benzyloxy)methyl]-2-(1-benzyl-1\H-tetrazol-5-yl)piperidine}}  % MMW-281 
Otrzymałem zgodnie z~\hyperref[experimental:sugars:schwartz]{procedurą syntezy tetrazolowych
	pochodnych iminocukrów}, wychodząc z~wywiedzionego z~glukozy laktamu~\refcmpd{glu-lactam}
	i~używając \SI{26.8}{\micro\liter} (\SI{0.22}{\milli\mol}) izocyjanku benzylu.
Oczyszczałem chromatograficznie na~żelu krzemionkowym,
	używając jako eluentu octanu etylu w~heksanie w~gradiencie \SIrange{10}{40}{\percent}.
Otrzymałem produkt w~postaci białego ciała stałego (\SI{18}{\percent} wydajności).

\begin{fullexp}
	\NMR(600)[CDCl3] \numrange{7.30}{6.89} (m, \#{25}), \numlist{5.44;5.06} (ABq, \J{15.6}, \#{2}), \numlist{4.86;4.81} (ABq, \J{10.8}, \#{2}), \numlist{4.80;4.44} (ABq, \J{11.3}, \#{2}), \numlist{4.61;4.35} (ABq, \J{12.3}, \#{2}), \num{4.59} (t, \J{9.2}, \#{1}), \numlist{4.26;4.21} (ABq, \J{11.8}, \#{2}), \num{4.18} (d, \J{6.1}, \#{1}), \num{3.70} (dd, \J{9.4;6.1}, \#{1}), \numrange{3.36}{3.29} (m, \#{3}), \num{3.12} (ddd, \J{9.7;5.1;2.8}, \#{1});
	\NMR{13,C}(151)[cdcl3] \numlist{153.1; 138.7; 138.5; 138.3; 137.8; 133.7; 129.2; 128.7; 128.6; 128.4; 128.3; 128.3; 128.0; 127.9; 127.9; 127.8; 127.8; 127.6; 127.5; 127.5; 127.1; 83.0; 80.6; 80.1; 75.7; 74.6; 73.9; 73.1; 69.6; 53.6; 50.7; 49.6};
	\data{IR}[film] \numlist{3334; 3087; 3062; 3031; 2923; 2858; 1954; 1875; 1811; 1731; 1680; 1604; 1496; 1453; 1361; 1313; 1259; 1208; 1069; 1028; 1002};
	\data{HRMS} (ESI-TOF) m/z obliczone dla \ch{C42H44N5O4}: \num{682.3393}, zarejestrowane: \num{682.3383};
	\data{[$\alpha^{23}_D$]~$=$} \num{19.4} ($c = 1.07$, \ch{DCM});
	\data{t. topnienia} \numrange{160}{161}\si{\celsius}
\end{fullexp}

\procedure{glu-tet.pmp}{\iupac{\cip{2R,3S,4R,5R,6R}-3,4,5-tris(benzyloxy)-6-((benzyloxy)methyl)-2-(1-(4-methoxyphenyl)-1\H-tetrazol-5-yl)piperidine}}  % MMW-303-C 
Otrzymałem zgodnie z~\hyperref[experimental:sugars:schwartz]{procedurą syntezy tetrazolowych
	pochodnych iminocukrów}, wychodząc z~wywiedzionego z~glukozy laktamu~\refcmpd{glu-lactam}
	i~używając \SI{29.3}{\milli\gram} (\SI{0.22}{\milli\mol}) izocyjanku \iupac{4-metoksyfenylu}
	rozpuszczonego w~\SI{0.5}{\milli\liter} suchego \ch{THF}.
Oczyszczałem chromatograficznie na~żelu krzemionkowym,
	używając jako eluentu \SI{25}{\percent} octanu etylu w~heksanie.
Otrzymałem produkt w~postaci białego ciała stałego (\SI{23}{\percent} wydajności).

\begin{fullexp}
	\NMR(600)[CDCl3] \numrange{7.35}{7.14} (m, \#{20}), \numrange{7.03}{6.99} (m, \#{2}), \numrange{6.91}{6.86} (m, \#{2}), \num{4.94} (ABq, \J{11.1}, \#{2}), \numlist{4.92;4.56} (ABq, \J{11.4}, \#{2}), \numrange{4.85}{4.79} (m, \#{1}), \numlist{4.69;4.43} (ABq, \J{12.2}, \#{2}), \numlist{4.40;4.34} (ABq, \J{11.7}, \#{2}), \num{4.37} (d, \J{6.3}, \#{1}), \num{3.85} (s, \#{3}), \num{3.72} (dd, \J{9.5;6.3}, \#{1}), \numrange{3.60}{3.51} (m, \#{2}), \num{3.47} (d, \J{5.6}, \#{2});
	\NMR{13,C}(151)[CDCl3] \numlist{160.9; 153.5; 138.8; 138.7; 138.0; 137.9; 128.4; 128.4; 128.4; 128.3; 128.0; 127.8; 127.8; 127.7; 127.6; 127.5; 127.5; 127.5; 127.1; 126.3; 114.7; 83.2; 80.8; 80.4; 75.7; 74.7; 73.8; 73.2; 69.7; 55.7; 53.7; 49.2};
	\data{IR}[film] \numlist{3652; 3329; 3062; 3030; 2909; 2867; 2053; 1955; 1879; 1813; 1607; 1589; 1517; 1454; 1362; 1306; 1255; 1209; 1098; 1068; 1027};
	\data{HRMS} (ESI-TOF) m/z obliczone dla \ch{C42H44N5O5}: \num{698.3342}, zarejestrowane: \num{698.3341};
	\data{[$\alpha^{23}_D$]~$=$} \num{54.5} ($c = 0.99$, \ch{DCM});
	\data{t. topnienia} \numrange{156}{157}\si{\celsius}
\end{fullexp}

\procedure{glu-epi-tet.pmp}{\iupac{\cip{2S,3S,4R,5R,6R}-3,4,5-tris(benzyloxy)-6-((benzyloxy)methyl)-2-(1-(4-methoxyphenyl)-1\H-tetrazol-5-yl)piperidine}}  % MMW-303-A 
Otrzymałem zgodnie z~\hyperref[experimental:sugars:schwartz]{procedurą syntezy tetrazolowych
	pochodnych iminocukrów}, wychodząc z~wywiedzionego z~glukozy laktamu~\refcmpd{glu-lactam}
	i~używając \SI{29.3}{\milli\gram} (\SI{0.22}{\milli\mol}) izocyjanku \iupac{4-metoksyfenylu}
	rozpuszczonego w~\SI{0.5}{\milli\liter} suchego \ch{THF}.
Oczyszczałem chromatograficznie na~żelu krzemionkowym,
	używając jako eluentu \SI{25}{\percent} octanu etylu w~heksanie.
Otrzymałem produkt w~postaci żółtego, woskowatego oleju (\SI{6}{\percent} wydajności).

\begin{fullexp}
	\NMR(600)[CDCl3] \numrange{7.63}{7.55} (m, \#{2}), \numrange{7.49}{7.28} (m, \#{20}), \numrange{7.10}{7.02} (m, \#{2}), \numlist{4.88;4.65} (ABq, \J{11.4}, \#{2}), \num{4.82} (s, \#{2}), \numlist{4.80;4.59} (ABq, \J{12.1}, \#{2}), \numrange{4.52}{4.48} (m, \#{1}), \num{4.43} (d, \J{5.1}, \#{1}), \numrange{4.41}{4.36} (m, \#{1}), \num{4.04} (t, \J{7.3}, \#{1}), \num{3.99} (s, \#{3}), \num{3.81} (dd, \J{9.3;5.0}, \#{1}), \num{3.57} (dd, \J{9.3;4.1}, \#{1}), \num{3.05} (bs, \#{1});
	\NMR{13,C}(151)[CDCl3] \numlist{160.8; 153.8; 138.5; 138.5; 138.3; 138.1; 128.4; 128.4; 128.3; 127.9; 127.8; 127.8; 127.8; 127.6; 127.6; 127.6; 127.5; 126.7; 126.6; 114.6; 79.0; 76.7; 76.0; 73.8; 73.1; 73.0; 72.8; 69.0; 55.6; 55.4; 48.9};
	\data{IR}[film] \numlist{3330; 3061; 3029; 2924; 2854; 1952; 1878; 1728; 1607; 1517; 1497; 1454; 1364; 1306; 1256; 1208; 1174; 1096; 1027};
	\data{HRMS} (ESI-TOF) m/z obliczone dla \ch{C42H44N5O5}: \num{698.3342}, zarejestrowane: \num{698.3348};
	\data{[$\alpha^{23}_D$]~$=$} \num{8.7} ($c = 0.27$, \ch{DCM})
\end{fullexp}

\procedure{glu-tet.pmb}{\iupac{\cip{2R,3R,4R,5S,6R}-3,4,5-tris(benzyloxy)-6-((benzyloxy)methyl)-2-(1-(4-methoxybenzyl)-1\H-tetrazol-5-yl)piperidine}}  % MMW-408 
Otrzymałem zgodnie z~\hyperref[experimental:sugars:schwartz]{procedurą syntezy tetrazolowych
	pochodnych iminocukrów}, wychodząc z~wywiedzionego z~glukozy laktamu~\refcmpd{glu-lactam}
	i~używając \SI{32.4}{\milli\gram} (\SI{0.22}{\milli\mol}) izocyjanku \iupac{4-metoksybenzylu}
	rozpuszczonego w~\SI{0.5}{\milli\liter} suchego \ch{THF}.
Oczyszczałem chromatograficznie na~żelu krzemionkowym,
	używając jako eluentu eteru \iupac{\tert-butylowo} metylowego w~\ch{DCM} w~gradiencie
	\SIrange{0}{3}{\percent}.
Otrzymałem produkt w~postaci białych igieł (\SI{42}{\percent} wydajności).

\begin{fullexp}
	\NMR(600)[CDCl3] \numrange{7.33}{7.18} (m, \#{18}), \numrange{7.13}{7.09} (m, \#{2}), \numrange{6.97}{6.94} (m, \#{2}), \numrange{6.75}{6.72} (m, \#{2}), \numlist{5.43;5.09} (ABq, \J{15.4}, \#{2}), \numlist{4.92;4.88} (ABq, \J{10.8}, \#{2}), \numlist{4.87;4.51} (ABq, \J{11.3}, \#{2}), \numlist{4.68;4.42} (ABq, \J{12.3}, \#{2}), \num{4.65} (t, \J{9.2}, \#{1}), \numlist{4.34;4.30} (ABq, \J{11.9}, \#{2}), \num{4.29} (d, \J{5.8}, \#{1}), \numrange{3.81}{3.76} (m, \#{1}), \num{3.70} (s, \#{3}), \numrange{3.45}{3.37} (m, \#{3}), \numrange{3.24}{3.18} (m, \#{1});
	\NMR{13,C}(151)[CDCl3] \numlist{159.8; 152.9; 138.8; 138.6; 138.3; 137.8; 128.8; 128.6; 128.4; 128.4; 128.3; 128.1; 127.9; 127.8; 127.8; 127.8; 127.7; 127.6; 127.5; 114.6; 83.1; 80.7; 80.0; 75.7; 74.6; 73.9; 73.1; 69.5; 55.2; 53.7; 50.4; 49.6};
	\data{IR}[film] \numlist{3335; 3087; 3062; 3030; 2908; 2866; 1954; 1877; 1812; 1613; 1586; 1515; 1496; 1454; 1398; 1361; 1306; 1293; 1252; 1208; 1178; 1093; 1069; 1029; 1002};
	\data{HRMS} (ESI-TOF) m/z obliczone dla \ch{C43H46N5O5}: \num{712.3499}, zarejestrowane: \num{712.3484};
	\data{[$\alpha^{23}_D$]~$=$} \num{22.1} ($c = 1.00$, \ch{DCM});
	\data{t. topnienia} \numrange{139}{142}\si{\celsius}
\end{fullexp}

\procedure{glu-tet.tbu}{\iupac{\cip{2R,3R,4R,5S,6R}-3,4,5-tris(benzyloxy)-6-((benzyloxy)methyl)-2-(1-(\tert-butyl)-1\H-tetrazol-5-yl)piperidine}}  % MMW-280 
Otrzymałem zgodnie z~\hyperref[experimental:sugars:schwartz]{procedurą syntezy tetrazolowych
	pochodnych iminocukrów}, wychodząc z~wywiedzionego z~glukozy laktamu~\refcmpd{glu-lactam}
	i~używając \SI{22.6}{\micro\liter} (\SI{0.22}{\milli\mol}) izocyjanku \iupac{\tert-butylu}.
Etap addycji prowadziłem przez \SI{12}{\day}.
Oczyszczałem chromatograficznie na~żelu krzemionkowym,
	używając jako eluentu octanu etylu w~heksanie w~gradiencie \SIrange{10}{40}{\percent}.
Otrzymałem produkt w~postaci białego ciała stałego (\SI{40}{\percent} wydajności).

\begin{fullexp}
	\NMR(600)[CDCl3] \numrange{7.32}{6.96} (m, \#{20}), \num{4.98} (t, \J{9.2}, \#{1}), \numlist{4.88;4.87} (ABq, \J{10.8}, \#{2}), \numlist{4.85;4.49} (ABq, \J{11.2}, \#{2}), \numlist{4.76;4.53} (ABq, \J{12.0}, \#{2}), \num{4.68} (d, \J{6.4}, \#{1}), \numlist{4.30;4.27} (ABq, \J{11.8}, \#{2}), \num{3.79} (dd, \J{9.5;6.4}, \#{1}), \numrange{3.52}{3.46} (m, \#{1}), \num{3.43} (t, \J{9.5}, \#{1}), \numrange{3.40}{3.37} (m, \#{1}), \num{3.09} (ddd, \J{10.0;4.7;2.7}, \#{1}), \num{1.54} (s, \#{9});
	\NMR{13,C}(151)[CDCl3] \numlist{152.4; 138.9; 138.7; 138.2; 137.9; 128.4; 128.4; 128.4; 128.3; 128.1; 127.7; 127.7; 127.7; 127.7; 127.6; 127.5; 127.4; 83.5; 81.4; 80.6; 75.7; 74.6; 74.3; 73.1; 69.7; 61.2; 53.2; 50.5; 30.2};
	\data{IR}[film] \numlist{3328; 3087; 3061; 3030; 2984; 2915; 2866; 1954; 1875; 1812; 1728; 1604; 1496; 1453; 1400; 1363; 1334; 1285; 1238; 1211; 1094; 1069; 1028};
	\data{HRMS} (ESI-TOF) m/z obliczone dla \ch{C39H46N5O4}: \num{648.3550}, zarejestrowane: \num{648.3542};
	\data{[$\alpha^{23}_D$]~$=$} \num{37.1} ($c = 0.54$, \ch{DCM});
	\data{t. topnienia} \numrange{164}{165}\si{\celsius}
\end{fullexp}

\procedure{glu-tet.oct}{\iupac{\cip{2R,3S,4R,5R,6R}-3,4,5-tris(benzyloxy)-6-((benzyloxy)methyl)-2-(1-(\tert-octyl)-1\H-tetrazol-5-yl)piperidine}}  % MMW-439-C 
Otrzymałem zgodnie z~\hyperref[experimental:sugars:schwartz]{procedurą syntezy tetrazolowych
	pochodnych iminocukrów}, wychodząc z~wywiedzionego z~glukozy laktamu~\refcmpd{glu-lactam}
	i~używając \SI{40.9}{\micro\liter} (\SI{0.22}{\milli\mol}) izocyjanku \iupac{\tert-oktylu}.
Etap addycji prowadziłem przez \SI{3}{\day}.
Oczyszczałem chromatograficznie na~żelu krzemionkowym,
	używając jako eluentu \SI{40}{\percent} octanu etylu w~heksanie.
Otrzymałem produkt w~postaci żółtego oleju (\SI{48}{\percent} wydajności).

\begin{fullexp}
	\NMR(600)[CDCl3] \numrange{7.34}{7.01} (m, \#{20}), \num{5.07} (t, \J{9.2}, \#{1}), \numlist{4.89;4.87} (ABq, \J{10.8}, \#{2}), \numlist{4.85;4.50} (d, \J{11.3}, \#{2}), \num{4.70} (d, \J{6.5}, \#{1}), \numlist{4.74;4.54} (ABq, \J{12.1}, \#{2}), \numlist{4.27;4.27} (ABq, \J{12.1}, \#{2}), \num{3.77} (dd, \J{9.5;6.4}, \#{1}), \num{3.49} (dd, \J{9.1;4.6}, \#{1}), \num{3.45} (dd, \J{10.1;8.9}, \#{1}), \num{3.35} (dd, \J{9.1;2.7}, \#{1}), \num{3.02} (ddd, \J{10.1;4.6;2.7}, \#{1}), \num{2.06} (s, \#{1}), \num{1.99} (d, \J{15.2}, \#{1}), \num{1.74} (s, \#{3}), \num{1.74} (d, \J{15.2}, \#{1}), \num{1.50} (s, \#{3}), \num{0.62} (s, \#{9});
	\NMR{13,C}(151)[CDCl3] \numlist{152.2; 138.9; 138.8; 138.2; 137.9; 128.5; 128.4; 128.3; 128.3; 128.1; 127.8; 127.7; 127.7; 127.6; 127.5; 127.4; 127.0; 83.6; 81.5; 80.7; 75.7; 74.6; 74.1; 73.2; 69.5; 65.1; 53.7; 53.0; 50.7; 31.6; 30.8; 30.5; 30.0};
	\data{IR}[film] \numlist{3327; 3087; 3062; 3030; 2952; 2906; 2868; 1952; 1874; 1810; 1604; 1586; 1496; 1453; 1394; 1362; 1334; 1286; 1247; 1209; 1139; 1098; 1068; 1028; 1001};
	\data{HRMS} (ESI-TOF) m/z obliczone dla \ch{C41H55N5O4Na}: \num{704.4152}, zarejestrowane: \num{704.4152};
	\data{[$\alpha^{23}_D$]~$=$} \num{24.2} ($c = 0.83$, \ch{DCM})
\end{fullexp}

\procedure{gal-tet.cy}{\iupac{\cip{2R,3S,4R,5S,6R}-3,4,5-tris(benzyloxy)-6-((benzyloxy)methyl)-2-(1-cyclohexyl-1\H-tetrazol-5-yl)piperidine}}  % MMW-292 
Otrzymałem zgodnie z~\hyperref[experimental:sugars:schwartz]{procedurą syntezy tetrazolowych
	pochodnych iminocukrów}, wychodząc z~wywiedzionego z~galaktozy laktamu~\refcmpd{gal-lactam}
	i~używając \SI{26.7}{\micro\liter} (\SI{0.22}{\milli\mol}) izocyjanku cykloheksylu.
Oczyszczałem chromatograficznie na~żelu krzemionkowym,
	używając jako eluentu mieszaniny \ch{Et3N}/\ch{AcOEt}/heksan w~stosunku objętościowym
	\num{2}:\num{28}:\num{70}.
Otrzymałem produkt w~postaci białego ciała stałego (\SI{33}{\percent} wydajności).

\begin{fullexp}
	\NMR(600)[CDCl3] \numrange{7.30}{7.12} (m, \#{18}), \numrange{6.97}{6.91} (m, \#{2}), \numlist{4.69;4.33} (ABq, \J{10.8}, \#{2}), \numlist{4.60;4.52} (ABq, \J{12.1}, \#{2}), \numrange{4.53}{4.50} (m, \#{2}), \numlist{4.42;4.41} (ABq, \J{12.2}, \#{2}), \num{4.39} (t, \J{8.3}, \#{1}), \numrange{4.35}{4.29} (m, \#{1}), \num{4.12} (d, \J{8.5}, \#{1}), \num{3.88} (dd, \J{3.7;2.9}, \#{1}), \num{3.84} (dd, \J{8.0;2.6}, \#{1}), \num{3.56} (dd, \J{9.6;5.1}, \#{1}), \num{3.50} (dd, \J{9.6;5.6}, \#{1}), \num{3.20} (q, \J{5.0}, \#{1}), \numrange{1.94}{1.49} (m, \#{7}), \numrange{1.21}{0.99} (m, \#{3});
	\NMR{13,C}(151)[CDCl3] \numlist{154.2; 138.6; 138.5; 138.4; 138.1; 128.8; 128.7; 128.7; 128.6; 128.2; 128.2; 128.2; 128.2; 128.1; 128.0; 128.0; 127.7; 79.2; 75.2; 75.0; 73.8; 72.4; 72.2; 70.3; 58.1; 55.2; 51.8; 33.4; 32.8; 25.5; 25.5; 25.1};
	\data{IR}[film] \numlist{3328; 3087; 3062; 3030; 2933; 2860; 1952; 1872; 1811; 1670; 1604; 1496; 1453; 1365; 1261; 1207; 1097; 1027};
	\data{HRMS} (ESI-TOF) m/z obliczone dla \ch{C41H48N5O4}: \num{674.3706}, zarejestrowane: \num{674.3694};
	\data{[$\alpha^{23}_D$]~$=$} \num{33.4} ($c = 1.01$, \ch{DCM});
	\data{t. topnienia} \numrange{158}{159}\si{\celsius}
\end{fullexp}

\procedure{gal-epi-tet.cy}{\iupac{\cip{2R,3S,4R,5S,6R}-3,4,5-tris(benzyloxy)-6-((benzyloxy)methyl)-2-(1-cyclohexyl-1\H-tetrazol-5-yl)piperidine}}  % MMW-457-3 
Otrzymałem zgodnie z~\hyperref[experimental:sugars:schwartz]{procedurą syntezy tetrazolowych
	pochodnych iminocukrów}, ale dodając przed etapem addycji izocyjanku
	\SI{0.5}{\micro\liter} suchego \ch{MeOH} do~mieszaniny.
Wyszedłem z~wywiedzionego z~galaktozy laktamu~\refcmpd{gal-lactam}
	i~użyłem \SI{26.7}{\micro\liter} (\SI{0.22}{\milli\mol}) izocyjanku cykloheksylu.
Oczyszczałem metodą preparatywnej chromatografii wysokociśnieniowej,
	używając jako eluentu \SI{10}{\percent} acetonu w~toluenie.
Otrzymałem produkt w~postaci białego woskowatego oleju (\SI{3}{\percent} wydajności).

\begin{fullexp}
	\NMR(600)[CDCl3] \numrange{7.37}{7.23} (m, \#{18}), \numrange{7.16}{7.12} (m, \#{2}), \numlist{4.91;4.56} (ABq, \J{11.3}, \#{2}), \numlist{4.79;4.75} (ABq, \J{11.7}, \#{2}), \numlist{4.78;4.59} (ABq, \J{11.7}, \#{2}), \num{4.46} (d, \J{6.2}, \#{1}), \numlist{4.43;4.40} (ABq, \J{11.9}, \#{1}), \numrange{4.32}{4.25} (m, \#{1}), \numrange{4.17}{4.07} (bs, \#{1}), \num{4.09} (t, \J{2.4}, \#{1}), \num{3.72} (td, \J{6.8;2.1}, \#{1}), \numrange{3.52}{3.44} (m, \#{1}), \numrange{3.43}{3.35} (m, \#{1}), \numrange{2.00}{1.91} (m, \#{1}), \numrange{1.91}{1.86} (m, \#{1}), \numrange{1.86}{1.74} (m, \#{3}), \numrange{1.72}{1.63} (m, \#{1}), \numrange{1.25}{1.11} (m, \#{4});
	\NMR{13,C}(151)[CDCl3] \numlist{152.8; 138.8; 138.6; 138.3; 138.0; 128.4; 128.4; 128.3; 128.3; 128.1; 128.0; 127.8; 127.8; 127.7; 127.6; 127.5; 127.5; 76.6; 75.3; 74.4; 74.2; 73.2; 72.9; 69.5; 57.8; 53.2; 49.9; 33.2; 32.5; 25.3; 25.2; 24.8};
	\data{IR}[film] \numlist{3316; 3061; 3031; 2923; 2855; 1952; 1876; 1811; 1733; 1668; 1604; 1496; 1453; 1365; 1266; 1208; 1097; 1027};
	\data{HRMS} (ESI-TOF) m/z obliczone dla \ch{C41H48N5O4}: \num{674.3706}, zarejestrowane: \num{674.3707};
	\data{[$\alpha^{23}_D$]~$=$} \num{21.8} ($c = 0.17$, \ch{DCM})
\end{fullexp}

\procedure{gal-tet.est}{\iupac{Ethyl (5-(\cip{2R,3S,4R,5S,6R}-3,4,5-tris(benzyloxy)-6-((benzyloxy)methyl)piperidin-2-yl)-1\H-tetrazol-1-yl)acetate}}  % MMW-293-B 
Otrzymałem zgodnie z~\hyperref[experimental:sugars:schwartz]{procedurą syntezy tetrazolowych
	pochodnych iminocukrów}, wychodząc z~wywiedzionego z~galaktozy laktamu~\refcmpd{gal-lactam}
	i~używając \SI{24.0}{\micro\liter} (\SI{0.22}{\milli\mol}) izocyjanooctanu etylu.
Oczyszczałem chromatograficznie na~żelu krzemionkowym,
	używając jako eluentu \SI{30}{\percent} \ch{AcOEt} w~cykloheksanie.
Otrzymałem produkt w~postaci białego ciała stałego (\SI{30}{\percent} wydajności).

\begin{fullexp}
	\NMR(400)[CDCl3] \numrange{7.38}{7.13} (m, \#{20}), \numlist{5.29;5.26} (ABq, \J{17.3}, \#{2}), \num{4.74} (t, \#{1}), \num{4.62} (s, \#{2}), \numlist{4.53;4.44} (ABq, \J{12.4}, \#{2}), \numlist{4.50;4.46} (ABq, \J{12.1}, \#{2}), \numlist{4.40;4.32} (ABq, \J{11.8}, \#{2}), \numrange{4.36}{4.32} (m, \#{1}), \num{4.14} (q, \J{7.1}, \#{2}), \numrange{3.91}{3.85} (m, \#{2}), \num{3.71} (dd, \J{9.3;4.2}, \#{1}), \num{3.53} (dd, \J{9.3;3.8}, \#{1}), \numrange{3.16}{2.98} (m, \#{1}), \num{2.41} (s, \#{1}), \num{1.20} (t, \J{7.1}, \#{3});
	\NMR{13,C}(126)[CDCl3] \numlist{165.1; 153.8; 137.1; 136.9; 136.9; 136.8; 127.5; 127.4; 127.3; 127.3; 127.1; 126.9; 126.8; 126.8; 126.7; 126.6; 126.6; 126.5; 75.1; 72.4; 72.2; 72.1; 70.5; 70.1; 68.2; 61.1; 51.1; 50.5; 48.5; 13.1};
	\data{IR}[film] \numlist{3341; 3061; 3030; 2924; 2866; 1955; 1879; 1813; 1751; 1604; 1496; 1453; 1396; 1373; 1308; 1260; 1211; 1100; 1026};
	\data{HRMS} (ESI-TOF) m/z obliczone dla \ch{C39H43N5O6}: \num{678.3292}, zarejestrowane: \num{678.3298};
	\data{[$\alpha^{23}_D$]~$=$} \num{-48.9} ($c = 0.43$, \ch{DCM});
	\data{t. topnienia} \numrange{144}{145}\si{\celsius}
\end{fullexp}

\procedure{glu-tet-cyclo-lactam}{\iupac{\cip{8R,9R,10R,11S,11aR}-9,10,11-tris(benzyloxy)-8-((benzyloxy)methyl)-9,10,11,11a-tetrahydro-8\H-pyrido[1,6-a]tetrazolo[5,1-c]pyrazin-6(5\H)-one}}  % MMW-382 (i MMW-388) 
The compound was synthesized by the following procedure:
\SI{70}{\milli\gram} (\SI{0.10}{\milli\mol}) of compound \refcmpd{glu-tet.est} and
\SI{14.7}{\milli\gram} (\SI{0.12}{\milli\mol}, \SI{1.2}{\equiv}) of benzoic acid
was dissolved in \SI{5.0}{\milli\liter} of toluene.
The mixture was heated to \SI{70}{\degreeCelsius} and stirred for \SI{16}{\hour}.
It was then let to cool to room temperature, quenched with \SI{10}{\milli\liter} saturated \ch{NaHCO3_{(aq)}} and extracted with \ch{AcOEt} ($2 \times \SI{10}{\milli\liter}$).
Organic phase was washed with water, dried over \ch{MgSO4} and evaporated.
The crude product was purified by flash column chromatography using \SI{40}{\percent} \iupac{\tert-butyl methyl ether} in hexanes as eluent.
(\SI{95}{\percent} wydajności) (light yellow waxy oil)

\begin{fullexp}
	\NMR(600)[CDCl3] \numrange{7.37}{7.13} (m, \#{18}), \numrange{6.80}{6.77} (m, \#{2}), \numrange{5.36}{5.33} (m, \#{1}), \numrange{5.29}{5.25} (m, \#{1}), \num{5.01} (dd, \J{17.3;1.0}, \#{1}), \num{4.83} (dd, \J{17.3;1.0}, \#{1}), \numlist{4.69;4.53} (ABq, \J{11.5}, \#{2}), \numlist{4.51;4.48} (ABq, \J{11.8}, \#{2}), \numlist{4.48;4.39} (ABq, \J{12.1}, \#{2}), \numlist{4.34;4.01} (ABq, \J{11.1}, \#{2}), \num{3.98} (t, \J{2.1}, \#{1}), \num{3.82} (t, \J{3.5}, \#{1}), \numrange{3.81}{3.78} (m, \#{2}), \num{3.68} (dd, \J{10.4;5.7}, \#{1});
	\NMR{13,C}(151)[CDCl3] \numlist{161.6; 148.4; 137.6; 137.5; 136.9; 136.5; 128.7; 128.5; 128.4; 128.4; 128.3; 128.2; 128.1; 128.0; 127.9; 127.8; 127.8; 127.6; 77.5; 74.9; 73.0; 73.0; 72.6; 72.3; 72.3; 66.2; 52.1; 49.5; 47.9};
	\data{IR}[film] \numlist{3087; 3062; 3031; 2923; 2868; 1956; 1879; 1813; 1725; 1671; 1604; 1585; 1567; 1496; 1454; 1421; 1393; 1367; 1347; 1256; 1207; 1177; 1092; 1028};
	\data{HRMS} (ESI-TOF) m/z obliczone dla \ch{C37H37N5O5Na}: \num{654.2692}, zarejestrowane: \num{654.2670};
	\data{[$\alpha^{23}_D$]~$=$} \num{-10.8} ($c = 1.00$, \ch{DCM})
\end{fullexp}

\procedure{glu-tet-cyclo}{\iupac{\cip{8R,9S,10R,11S,11aR}-9,10,11-tris(benzyloxy)-8-((benzyloxy)methyl)-5,6,9,10,11,11a-hexahydro-8\H-pyrido[1,6-a]tetrazolo[5,1-c]pyrazine}}  % MMW-427 
The compound was synthesized by the following procedure:
\SI{10}{\milli\gram} (\SI{0.016}{\milli\mol}) of compound \refcmpd{glu-tet-cyclo-lactam} was dissolved in \SI{0.3}{\milli\liter} of dry \ch{THF} and added to \SI{6.6}{\milli\gram} (\SI{0.026}{\milli\mol}, \SI{1.6}{\equiv}) of \ch{Cp2Zr(H)Cl}.
The mixture was stirred at room temperature until it cleared (for about \SI{4}{\hour}).
\SI{2.0}{\milli\gram} (\SI{0.048}{\milli\mol}, \SI{3.0}{\equiv}) of \ch{NaBH4} was added and mixture was stirred at RT overnight.
Then \SI{0.048}{\milli\liter} \SI{1.0}{\Molar} solution of \ch{BH3*THF} in \ch{THF} (\SI{0.048}{\milli\mol}, \SI{3.0}{\equiv}) was added.
After \SI{3}{\day} of stirring at RT the reaction was quenched by \SI{2}{\milli\liter} of water, extracted with \ch{AcOEt} ($3 \times \SI{5}{\milli\liter}$), dried with \ch{MgSO4} and evaporated.
The crude product was purified by flash column chromatography using \iupac{\tert-butyl methyl ether} in \ch{DCM} in \SI{10}{\percent} to \SI{40}{\percent} gradient as eluent.
(\SI{75}{\percent} wydajności) (yellow oil)

\begin{fullexp}
	\NMR(600)[CDCl3] \numrange{7.34}{7.24} (m, \#{12}), \numrange{7.24}{7.20} (m, \#{4}), \numrange{7.18}{7.14} (m, \#{2}), \numrange{7.13}{7.09} (m, \#{2}), \numlist{4.58;4.48} (ABq, \J{12.4}, \#{2}), \num{4.56} (d, \J{2.8}, \#{1}), \num{4.53} (ddd, \J{12.7;8.5;4.5}, \#{1}), \numrange{4.46}{4.38} (m, \#{4}), \numlist{4.40;4.27} (ABq, \J{11.7}, \#{2}), \num{4.36} (dt, \J{12.5;4.3}, \#{1}), \numrange{4.20}{4.16} (m, \#{1}), \num{3.80} (dd, \J{10.1;6.3}, \#{1}), \num{3.67} (dt, \J{12.9;4.4}, \#{1}), \num{3.61} (dt, \J{7.5;3.2}, \#{2}), \num{3.58} (dd, \J{10.1;5.5}, \#{1}), \numrange{3.41}{3.37} (m, \#{1}), \num{3.18} (dt, \J{8.4;4.4}, \#{1});
	\NMR{13,C}(151)[CDCl3] \numlist{150.8; 137.1; 136.8; 136.6; 136.4; 127.5; 127.5; 127.4; 127.4; 127.3; 126.9; 126.9; 126.8; 126.8; 126.7; 126.7; 126.6; 75.3; 73.0; 72.8; 72.6; 72.3; 71.5; 71.1; 66.1; 61.0; 50.6; 47.0; 44.2};
	\data{IR}[film] \numlist{3059; 3031; 2923; 2852; 1953; 1878; 1812; 1728; 1676; 1604; 1544; 1496; 1454; 1364; 1265; 1207; 1172; 1090; 1028};
	\data{HRMS} (ESI-TOF) m/z obliczone dla \ch{C37H39N5O4Na}: \num{640.2900}, zarejestrowane: \num{640.2892};
	\data{[$\alpha^{23}_D$]~$=$} \num{3.8} ($c = 0.41$, \ch{DCM})
\end{fullexp}

\procedure{glu-tet-pmb-rearr}{\iupac{\cip{2R,3S,4R,5R,6R}-3,4,5-tris(benzyloxy)-6-((benzyloxy)methyl)-2-(2-(4-methoxyphenyl)-2\H-tetrazol-5-yl)piperidine}}  % MMW-411 
The compound was synthesized by the following procedure:
\SI{20.0}{\milli\gram} (\SI{0.028}{\milli\mol}) of compound \refcmpd{glu-tet.pmb} was dissolved in \SI{0.5}{\milli\liter} of \ch{TFA} and was stirred at RT for \SI{24}{\hour}.
It was then evaporated, dissolved in \SI{20}{\milli\liter} of \ch{DCM},
washed with \SI{20}{\milli\liter} of \ch{NaHCO3_{(aq)}} and \SI{20}{\milli\liter} of brine and then dried with \ch{MgSO4} and evaporated.
The crude product was purified by flash column chromatography using \SI{5}{\percent} of \ch{AcOEt} in \ch{DCM} as eluent.
(\SI{95}{\percent} wydajności) (colourless oil)

\begin{fullexp}
	\NMR(600)[CDCl3] \numrange{7.32}{7.12} (m, \#{22}), \numrange{6.84}{6.79} (m, \#{2}), \num{5.68} (ABq, \J{14.5}, \#{2}), \num{4.91} (d, \J{5.9}, \#{1}), \numlist{4.88;4.52} (ABq, \J{11.0}, \#{2}), \numlist{4.87;4.77} (ABq, \J{10.8}, \#{2}), \numlist{4.66;4.57} (ABq, \J{11.5}, \#{2}), \num{4.36} (ABq, \J{11.7}, \#{2}), \numrange{4.38}{4.32} (m, \#{1}), \num{3.90} (dd, \J{9.2;6.5}, \#{1}), \num{3.74} (s, \#{3}), \numrange{3.63}{3.58} (m, \#{1}), \numrange{3.49}{3.41} (m, \#{3});
	\NMR{13,C}(151)[CDCl3] \numlist{165.2; 159.8; 138.9; 138.4; 137.9; 137.8; 129.7; 128.3; 128.2; 128.2; 128.2; 127.9; 127.9; 127.8; 127.8; 127.6; 127.5; 127.5; 127.4; 125.4; 114.2; 82.9; 80.4; 80.0; 75.5; 74.9; 73.0; 72.2; 70.2; 56.2; 55.2; 53.9; 50.9};
	\data{IR}[film] \numlist{3341; 3087; 3062; 3030; 3005; 2925; 2864; 2056; 1954; 1877; 1812; 1613; 1586; 1515; 1496; 1454; 1394; 1362; 1330; 1305; 1251; 1208; 1178; 1088; 1070; 1029};
	\data{HRMS} (ESI-TOF) m/z obliczone dla \ch{C43H46N5O5}: \num{712.3499}, zarejestrowane: \num{712.3477};
	\data{[$\alpha^{23}_D$]~$=$} \num{49.6} ($c = 0.87$, \ch{DCM})
\end{fullexp}

\procedure{glu-tet-free}{\iupac{\cip{2R,3S,4R,5R,6R}-3,4,5-tris(benzyloxy)-6-((benzyloxy)methyl)-2-(1\H-tetrazol-5-yl)piperidine}}  % MMW-448 
The compound was synthesized by the following procedure:
\SI{44.3}{\milli\gram} (\SI{0.063}{\milli\mol}) of compound \refcmpd{glu-tet.oct} was dissolved in \SI{6.0}{\milli\liter} of \SI{4.0}{\Molar} solution of \ch{HCl} in dioxane and the flask was closed tightly.
The mixture was stirred at \SI{90}{\degreeCelsius} for \SI{24}{\hour}.
It was then let to cool to room temperature, diluted with \SI{10}{\milli\liter} of \iupac{\tert-butyl methyl ether} and \SI{10}{\milli\liter} of water.
\ch{NaHCO3} was slowly added while stirring until $\pH = 7$ was reached, then the mixture was separated.
Aqueous phase was extracted with \iupac{\tert-butyl methyl ether} ($2 \times \SI{5}{\milli\liter}$).
Combined organic phases were dried with \ch{MgSO4} and evaporated.
The crude product was purified by flash column chromatography using \ch{NH3_{(aq)}}/\ch{MeOH}/\ch{DCM} in 0:10:90 to 1:20:80 gradient as eluent.
(\SI{75}{\percent} wydajności) (light grey solid)

\begin{fullexp}
	\NMR(600)[CDCl3] \numrange{7.34}{7.10} (m, \#{20}), \numlist{4.86;4.71} (ABq, \J{11.2}, \#{2}), \numlist{4.81;4.78} (ABq, \J{11.0}, \#{2}), \numlist{4.77;4.46} (ABq, \J{11.0}, \#{2}), \numrange{4.69}{4.66} (m, \#{1}), \numlist{4.55;4.41} (ABq, \J{11.9}, \#{2}), \num{4.03} (dd, \J{9.2;5.7}, \#{1}), \num{3.64} (dd, \J{9.8;2.7}, \#{1}), \num{3.54} (dd, \J{9.8;6.0}, \#{1}), \numrange{3.51}{3.42} (m, \#{2}), \numrange{3.16}{3.07} (m, \#{1});
	\NMR{13,C}(126)[CDCl3] \numlist{154.8; 138.1; 137.8; 137.6; 136.8; 128.9; 128.7; 128.5; 128.4; 128.4; 128.2; 128.1; 127.9; 127.9; 127.8; 127.8; 127.8; 83.1; 79.6; 79.4; 75.7; 75.1; 74.8; 73.1; 69.1; 54.6; 51.0};
	\data{IR}[film] \numlist{3316; 3087; 3062; 3031; 2955; 2925; 2855; 1951; 1875; 1810; 1737; 1667; 1590; 1554; 1496; 1454; 1399; 1364; 1332; 1312; 1277; 1248; 1208; 1187; 1085; 1028};
	\data{HRMS} (ESI-TOF) m/z obliczone dla \ch{C35H38N5O4}: \num{592.2924}, zarejestrowane: \num{592.2925};
	\data{[$\alpha^{23}_D$]~$=$} \num{57.2} ($c = 0.60$, \ch{DCM});
	\data{t. topnienia} \numrange{180}{183}\si{\celsius}
\end{fullexp}
