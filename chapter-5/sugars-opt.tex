\section{Detale optymalizacji syntezy tetrazolowych pochodnych iminocukrów}\label{experimental:sugars-opt}
Badania wpływu ilości użytych \ch{TMSN3} i~\ch{CyNC} prowadziłem stosując
  \hyperref[experimental:sugars:schwartz]{procedurę syntezy tetrazolowych pochodnych iminocukrów}
  z~następującymi modyfikacjami:
  \begin{enumerate*}
    \item do~naczynia Schlenka, oprócz odczynnika Schwartza i~amidu \refcmpd{glu-lactam},
      odważyłem \SI{1.0}{\equiv} trifenylometanu;
    \item po~etapie redukcji mieszaninę reakcyjną podzieliłem na~5 równych części;
    \item do~każdej części mieszaniny reakcyjnej dodałem inną ilość \ch{TMSN3} i~\ch{CyNC},
      wyszczególnioną w~\cref{tab:sugars-opt-amount};
    \item po~zakończeniu reakcji z~mieszaniny usunąłem rozpuszczalnik przy użyciu wyparki 
      rotacyjnej i~pompy próżniowej;
    \item surową mieszaninę rozpuściłem w~\ch{CDCl3} i~analizowałem techniką \NMR*{}.
  \end{enumerate*}

\begin{margintable}
  \begin{tabular}{rccccc}
    \toprule
                & \multicolumn{5}{c}{Ilość reagentów /\si{\equiv}} \\
    \textnumero & \num{1.0} & \num{1.3} & \num{1.6} & \num{1.9} & \num{2.2} \\ \midrule
    \rownumber  & \num{89}  & \num{83}  & \num{87}  & \num{91}  & \num{86}  \\
    \rownumber  & \num{79}  & \num{85}  & \num{74}  & \num{87}  & \num{83}  \\
    \rownumber  & \num{88}  & \num{85}  & \num{91}  & \num{94}  & \num{90}  \\
    \bottomrule
  \end{tabular}
  \caption{%
    Procentowa wydajność badanej reakcji w~zależności od ilości użytych reagentów:
      \ch{TMSN3} i~\ch{CyNC}.
    Wydajność zmierzona na~podstawie analizy widm \NMR*{} surowych mieszanin reakcyjnych
      z~wzorcem wewnętrznym.
  }\label{tab:sugars-opt-amount}
\end{margintable}

Wybrałem trifenylometan jako wzorzec wewnętrzny do~tych badań, ponieważ diagnostyczny
  sygnał protonu metylowego tego związku\sidenote{\SI{5.535}{\ppm} w~\ch{CDCl3}.}
  nie nakłada się z~sygnałami substratów ani produktu reakcji,
  a~sam wzorzec pozostaje bierny w~warunkach prowadzenia przemiany.
Wydajność reakcji wyznaczyłem porównując obszar pod powierzchnią piku protonu alifatycznego
  wzorca ze~średnią powierzchnią piku przypadającą na~jeden proton produktu.
Do~porównania wybierałem dobrze wykształcone piki, które nie nakładają się z~sygnałami
  pochodzącymi od~substratu ani pozostałych reagentów.
\Cref{fig:suagr-opt-hnmr} przedstawia wycinek nałożonych na~siebie widm
  substratu~\refcmpd{glu-lactam} oraz produktu~\refcmpd{glu-tet.cy} zaznaczonymi sygnałami,
  które brałem pod uwagę.

\begin{figure*}
  \includesvg{sugars-opt-hnmr}
  \caption{
    Diagnostyczny wycinek widm \NMR*{} produktu~\refcmpd{glu-tet.cy}
      oraz substratu~\refcmpd{glu-lactam}.
    Pod wybranymi sygnałami umieściłem względne wartości pola pod powierzchnią piku.
  }\label{fig:suagr-opt-hnmr}
\end{figure*}

Na pierwszy rzut oka można zaobserwować, że wszystkie zarejestrowane wartości wydajności
  są zbliżone do~wartości średniej \--- \SI{85}{\percent} z~odchyleniem standardowym
  $\sigma=\SI{5}{\percent}$, można więc pokusić się o~stwierdzenie, że wydajność nie zależy
  od~proporcji reagentów.
Najprostszym sposobem na~potwierdzenie lub obalenie tej tezy jest próba dopasowania danych
  do~modelu aproksymacji wielomianowej i~sprawdzenie czy wyznaczone parametry funkcji
  są statystycznie istotne.
W~opisywanym przypadku zwykła dwuparametrowa regresja liniowa w~postaci $y = a \cdot x + b$
  wydaje się być wystarczającym przybliżeniem.

Korzystając z~metody najmniejszych kwadratów wyznaczyłem parametry $a = \num{82.0}$
  oraz $b = \num{2.6}$.
Interesującą mnie istotność parametru $b$ można wyznaczyć korzystając z~testu t\-/Studenta.
Niezbędny jest do~tego jeszcze błąd standardowy parametru $b$, który wynosi $SE_b = \num{3.1}$ oraz
  liczba stopni swobody, równa liczbie niezależnych pomiarów pomniejszonej o~2,
  $d_f = n - 2 = 13$.
Na~potrzeby tego testu przyjmuję hipotezę zerową $H_0$: \emph{parametr $b$ jest tak naprawdę
  równy zero} oraz hipotezę alternatywną $H_a$: \emph{parametr $b$ nie jest równy zero}.
Obliczam wartość zmiennej losowej $t = b / SE_b = \num{0.846}$ i~porównuję z~rozkładem t-Studenta
  dla 13 stopni swobody, otrzymując parametr $P(t > \num{0.846}) = \num{0.206}$ oraz
  $P(t < \num{-0.846}) = \num{0.206}$.
Suma tych parametrów $P = \num{0.412}$ jest większa niż typowo przyjmowany przedział wartości
  $\alpha = \num{0.1}$, a~zatem nie ma podstaw do~odrzucenia hipotezy zerowej.
