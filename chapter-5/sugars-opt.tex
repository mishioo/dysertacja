\section{Detale optymalizacji syntezy tetrazolowych pochodnych iminocukrów}\label{experimental:sugars-opt}
Badania wpływu ilości użytych \ch{TMSN3} i~\ch{CyNC} prowadziłem stosując
  \hyperref[experimental:sugars:schwartz]{procedurę syntezy tetrazolowych pochodnych iminocukrów}
  z~następującymi modyfikacjami:
  \begin{enumerate*}
    \item do~naczynia Schlenka, oprócz odczynnika Schwartza i~amidu \refcmpd{glu-lactam},
      odważyłem \SI{1.0}{\equiv} trifenylometanu;
    \item po~etapie redukcji mieszaninę reakcyjną podzieliłem na~5 równych części;
    \item do~każdej części mieszaniny reakcyjnej dodałem inną ilość \ch{TMSN3} i~\ch{CyNC},
      wyszczególnioną w~\cref{tab:sugars-opt-amount};
    \item po~zakończeniu reakcji z~mieszaniny usunąłem rozpuszczalnik przy użyciu wyparki 
      rotacyjnej i~pompy próżniowej;
    \item surową mieszaninę rozpuściłem w~\ch{CDCl3} i~analizowałem techniką \NMR*{}.
  \end{enumerate*}

\begin{margintable}
  \begin{tabular}{rccccc}
    \toprule
                & \multicolumn{5}{c}{Ilość reagentów /\si{\equiv}} \\
    \textnumero & \num{1.0} & \num{1.3} & \num{1.6} & \num{1.9} & \num{2.2} \\ \midrule
    \rownumber  & \num{89}  & \num{83}  & \num{87}  & \num{91}  & \num{86}  \\
    \rownumber  & \num{79}  & \num{85}  & \num{74}  & \num{87}  & \num{83}  \\
    \rownumber  & \num{88}  & \num{85}  & \num{91}  & \num{94}  & \num{90}  \\
    \bottomrule
  \end{tabular}
  \caption{%
    Procentowa wydajność badanej reakcji w~zależności od ilości użytych reagentów:
      \ch{TMSN3} i~\ch{CyNC}.
    Wydajność zmierzona na~podstawie analizy widm \NMR*{} surowych mieszanin reakcyjnych
      z~wzorcem wewnętrznym.
  }\label{tab:sugars-opt-amount}
\end{margintable}

Wybrałem trifenylometan jako wzorzec wewnętrzny do~tych badań, ponieważ diagnostyczny
  sygnał protonu metylowego tego związku\sidenote{\SI{5.535}{\ppm} w~\ch{CDCl3}.}
  nie nakłada się z~sygnałami substratów ani produktu reakcji,
  a~sam wzorzec pozostaje bierny w~warunkach prowadzenia przemiany.
