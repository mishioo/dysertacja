\section{Substraty do~badań nad~aktywacją amidoestrów}\label{experimental:amidoester-substrates}
\begin{scheme}
  \includesvg{amidoester-cycloprop-synthesis}
  \caption{
    Synteza związku modelowego~\refcmpd{amidoester-cycloprop} do~prób aktywacji i~reduktywnej
      funkcjonalizacji amidosetrów o~strukturze kwasu malonowego.
  }
  \label{sch:amidoester-cycloprop-synthesis}
\end{scheme}

\procedure{cyclopropyl-dimethyl-malonate}{\iupac{1,1-cyclopropylodikarboksylan dimetylu}}
\marginnote{Analiza \NMR*{} zgodna z~literaturą.}
W kolbie umieściłem \SI{17.25}{\gram} \ch{K2CO3}, \SI{25}{\mL} \ch{PhMe},
  \SI{0.25}{\mL} \ch{H2O}, \SI{0.2}{\gram} \ch{Bn4N+ Br-}, \SI{5.8}{\mL}
  malonianu dimetylu oraz \SI{8.7}{\mL} \iupac{1,2-dibromoetanu}.
Mieszałem intensywnie przez \SI{4}{\day}, po~czym odsączyłem.
Osad przemyłem \SI[product-units = single]{3 x 25}{\mL} toluenu.
Zebrane frakcje organiczne odparowałem przy użyciu wyparki rotacyjnej.
Na~podstawie analizy \NMR*{} oszacowałem skład mieszaniny na~ok. $1:1$ substratu i~produktu.
Dodałem do~mieszaniny jeszcze \SI{9}{\gram} \ch{K2CO3}, \SI{25}{\mL} \ch{PhMe},
  \SI{2}{\mL} \ch{H2O}, \SI{0.1}{\gram} \ch{Bn4N+ Br-} oraz \SI{4.5}{\mL}
  \iupac{1,2-dibromoetanu}.
Mieszałem przez noc w~atmosferze argonu pod chłodnicą zwrotną w~temperaturze \SI{115}{\degC}.
Po~tym czasie odsączyłem i~destylowałem pod~zmniejszonym ciśnieniem (\SI{20}{\milli\bar}).
Otrzymałem \SI{3.14}{\gram} cieczy (\SI{40}{\percent} wydajności).

\procedure{cyclopropyl-monomethyl-malonate}{\iupac{kwas 1-(metoksykarbonylo)cyklopropanokarboksylowy}}
Do \SI{3}{\gram} związku \refcmpd{cyclopropyl-dimethyl-malonate} dodałem \SI{1066}{\milli\gram}
  \ch{KOH} rozpuszczonego w~\SI{15}{\mL} bezwodnego \ch{MeOH} w~atmosferze argonu.
Mieszałem we~wrzeniu przez \SI{2}{\hour}, po~czym usunąłem rozpuszczalnik
  przy użyciu wyparki rotacyjnej.
Otrzymany osad rozpuściłem w~\SI{10}{\mL} \ch{H2O}, zakwasiłem do~$\pH{}=7$ za~pomocą
  \SI{10}{\percent} \ch{HCl_{aq}}, nasyciłem \ch{NaCl} i~ekstrahowałem
  \SI[product-units = single]{4 x 10}{\mL} \ch{Et2O}.
Zebrane frakcje organiczne suszyłem \ch{MgSO4} i~usunąłem z~nich rozpuszczalnik
  przy użyciu wyparki rotacyjnej.
Otrzymałem \SI{2.22}{\gram} cieczy, będącej mieszaniną substratu i~produktu w~stosunku
  $1:4,44$ wg. analizy {\NMR*} (\SI{66}{\percent} wydajności).
Mieszaniny użyłem w~następnym etapie syntezy bez oczyszczania.

\procedure{cyclopropyl-methyl-malonate-chloride}{\iupac{chlorek kwasu 1-(metoksykarbonylo)cyklopropanokarboksylowego}}
Nieoczyszczony związek \refcmpd{cyclopropyl-monomethyl-malonate} rozpuściłem w~\SI{40}{\mL}
  bezwodnego \gls{dcm} w~atmosferze argonu. Dodałem \SI{2.2}{\mL} chlorku oksalilu
  i~\num{3} krople \gls{dmf}.
Mieszałem w~temperaturze pokojowej przez \SI{100}{\minute}, po~czym usunąłem rozpuszczalnik
  przy użyciu wyparki rotacyjnej.
Produktu użyłem w~następnym etapie syntezy bez oczyszczania.

\procedure{amidoester-cycloprop}{\iupac{1-[(4-metoksyfenylo)karbamylo]cyklopropanokarboksylan metylu}}
W~atmosferze argonu rozpuściłem surową mieszaninę poreakcyjną zawierającą związek
  \refcmpd{cyclopropyl-methyl-malonate-chloride} w~\SI{20}{\mL} acetonu.
Dodałem \SI{1550}{\milli\gram} \iupac{p-anizydyny} oraz \SI{1.75}{\mL} \ch{Et3N}.
Mieszałem w~temperaturze pokojowej przez noc, po~czym usunąłem rozpuszczalnik przy użyciu
  wyparki rotacyjnej.
Otrzymany osad zawiesiłem w~\SI{25}{\mL},
  ekstrahowałem \SI[product-units = single]{4 x 15}{\mL},
  suszyłem \ch{MgSO4}, odparowałem rozpuszczalnik za~pomocą wyparki rotacyjnej.
Oczyszczałem chromatograficznie na~żelu krzemionkowym w~eluencie \SI{20}{\percent} octanu
  etylu w~heksanie.
Otrzymałem \SI{2.5}{\gram} produktu (\SI{80}{\percent} wydajności).

\begin{fullexp}
  \NMR(400)[CDCl3] \num{10.65} (s, \#{1}), \numrange{7.57}{7.37} (m, \#{2}), \numrange{6.93}{6.76} (m, \#{2}), \num{3.78} (d, \J{2.2}, \#{3}), \num{3.72} (d, \J{2.2}, \#{3}), \numrange{1.83}{1.75} (m, \#{2}), \numrange{1.67}{1.60} (m, \#{2})\par\noindent
  \NMR{13,C}(101)[CDCl3] \numlist{174.4; 166.5; 156.3; 131.4; 121.8; 114.1; 55.5; 52.4; 26.4; 20.5}\par\noindent
  \data{IR}[film] \numlist{3277; 3234; 3136; 3078; 3029; 2961; 2841; 2045; 1889; 1769; 1700; 1658; 1613; 1598; 1544; 1512; 1444; 1409; 1354; 1322; 1305; 1280; 1248; 1202; 1150; 1113; 1087; 1027}\par\noindent
  \data{HRMS} (ESI-TOF) m/z calcd for \ch{C13H15NO4Na}: \num{272.0899} found: \num{272.0902}
\end{fullexp}

\procedure{amidoester-plain}{\iupac{3-[(4-metoksyfenylo)amino]-3-oksopropanian etylu}}
\marginnote{Analiza \NMR*{} zgodna z~literaturą.}
W kolbie umieściłem \SI{2.24}{\gram} anizydyny i~\SI{2.8}{\mL} trietyloaminy,
  które rozpuściłem w~\SI{70}{\mL} acetonu.
Następnie dodałem \SI{2.53}{\mL} chlorku etylomalonylu, powoli wkraplając.
Mieszałem przez \SI{1}{\hour} w~temperaturze pokojowej, po~czym usunąłem rozpuszczalnik
  za~pomocą wyparki obrotowej.
Dodałem \SI{50}{\mL} \SI{2}{\Molar} \ch{HCl_{(aq)}}, a~otrzymaną zawiesinę poddałem ekstrakcji
  używając \SI[product-units = single]{5 x 30}{\mL} \ch{Et2O}.
Zebrane frakcje organiczne suszyłem \ch{MgSO4}, po~czym odparowałem rozpuszczalnik za~pomocą
  wyparki rotacyjnej.
Odmyłem barwne zanieczyszczenie zimnym \ch{Et2O},
  a~pozostała ciało stałe suszyłem pod~zmniejszonym ciśnieniem.
Otrzymałem \SI{2.96}{\gram} produktu w~postaci białego ciała stałego (\SI{68}{\percent} wydajności).

\procedure{amidoester-bn}{\iupac{2-benzylo-3-[(4-metoksyfenylo)amino]-3-oksopropanian etylu}}
Do wygrzanej w~płomieniu palnika i~wypełnionej argonem kolby odważyłem \SI{1.66}{\gram}
  związku~\refcmpd{amidoester-plain}, dodałem \SI{40}{\mL} \gls{thf}
  i~ochłodziłem do~\SI{-10}{\degC} w~łaźni izopropanol-suchy lód.
Dodałem \SI{280}{\mg} \ch{NaH} w~postaci \SI{60}{\percent} zawiesiny w~oleju mineralnym
  i~mieszałem w~\SI{-10}{\degC} przez \SI{30}{\minute}.
Wyjąłem mieszaninę z~łaźni chłodzącej i~po~chwili dodałem \SI{2.49}{\mL} \ch{PhCH2Br}.
Mieszałem w~temperaturze pokojowej przez noc.
Po~tym czasie odsączyłem sole nieorganiczne na~celicie i~odparowałem rozpuszczalnik za~pomocą
  wyparki rotacyjnej.
Oczyszczałem za~pomocą chromatografii \textit{flash} używając jako eluenta octanu etylu w~heksanie
  w~gradiencie \SIrange{10}{40}{\percent}.
Otrzymałem \SI{1.25}{\gram} produktu w~postaci białego ciała stałego (\SI{55}{\percent} wydajności).
\begin{fullexp}
  \NMR(600)[CDCl3] \num{8.41} (s, \#{1}), \numrange{7.40}{7.32} (m, \#{2}), \numrange{7.29}{7.16} (m, \#{5}), \numrange{6.89}{6.77} (m, \#{2}), \num{4.11} (q, \J{7.1}, \#{2}), \num{3.76} (s, \#{3}), \num{3.61} (dd, \J{8.5;6.5}, \#{1}), \numrange{3.37}{3.30} (m, \#{1}), \numrange{3.29}{3.22} (m, \#{1}), \num{1.14} (t, \J{7.1}, \#{3})\par\noindent
  \NMR{13,C}(151)[CDCl3] \numlist{171.6; 165.7; 156.6; 137.6; 130.6; 128.9; 128.5; 126.9; 121.9; 114.1; 61.7; 55.4; 55.3; 37.0; 13.9}\par\noindent
  \data{IR}[film] \numlist{3302; 3138; 3063; 3030; 2980; 2935; 2836; 1738; 1658; 1604; 1545; 1512; 1455; 1442; 1414; 1369; 1298; 1245; 1171; 1111; 1033}\par\noindent
  \data{HRMS} (ESI-TOF) m/z calcd for \ch{C19H21NO4Na}: \num{350.1372} found: \num{350.1372}
\end{fullexp}

\procedure{dime-malonian-mono-et}{\iupac{2,2-dimetylomalonian monoetylu}}
\marginnote{Analiza \NMR*{} zgodna z~literaturą.}
W~\SI{100}{\ml} bezwodnego \ch{EtOH} w~atmosferze argonu roztworzyłem \SI{4.6}{\gram} metalicznego sodu,
  po czym powoli dodałem \SI{15.2}{\gram} malonianu diematylu, a~następnie \SI{15.4}{\gram} jodku metylu.
Grzałem pod chłodnicą zwrotną przez \SI{1}{\hour}.
Po ostygnięciu mieszaniny usunąłem rozpuszczalnik i~nadmiar \ch{MeI} za~pomocą wyparki rotacyjnej.
Otrzymaną zawiesinę wylałem na \SI{50}{\ml} \ch{H2O}, rozdzieliłem fazy i~ekstrahowałem fazę wodną
  \SI[product-units = single]{2 x 25}{\mL} \ch{Et2O}.
Połączone fazy organiczne przemyłem \SI{25}{\ml} solanki i~suszyłem \ch{MgSO4}.

Bez dalszego oczyszczania rozpuściłem \SI{7.0}{\gram} otrzymanego \iupac{2,2-dimetylomalonianu dietylu}
  w~\SI{30}{\ml} \gls{thf} i~dodałem \SI{300}{\ml} \ch{H2O}.
Ochłodziłem mieszaninę do~\SI{0}{\degC}, po~czym powoli dodałem \SI{1.67}{\gram} \ch{KOH}
  rozpuszczonego w~\SI{60}{\ml} \ch{H2O} i~mieszałem przez \SI{1}{\hour},
  utrzymując roztwór w~temperaturze \SI{0}{\degC}.
Następnie dodałem \SI{1}{\Molar} \ch{HCl_{(aq)}} do~uzyskania \pH w~zakresie \numrange{2}{3}.
Nasyciłem roztwór chlorkiem sodu, ekstrahowałem \SI[product-units = single]{4 x 100}{\mL} \ch{Et2O},
  suszyłem \ch{MgSO4}, po~czym odparowałem rozpuszczalnik za~pomocą wyparki rotacyjnej.
Oczyszczałem destylując pod zmniejszonym ciśnieniem, zbierałem destylat w~temperaturze
  \SIrange{120}{125}{\degC}, otrzymując \SI{2.76}{\gram} produktu w~postaci bezbarwnej cieczy
  (\SI{58}{\percent} wydajności.)

\procedure{amidoester-dime}{\iupac{2,2-dimetylo-3-[(4-metoksyfenylo)amino]-3-oksopropanian etylu}}
\SI{957}{\mg} związku~\refcmpd{dime-malonian-mono-et} rozpuściłem w~\SI{20}{\ml}
  suchego \ch{CHCl2} w~atomosferze argonu, dodałem \SI{1.0}{\ml} chlorku oksalilu
  i~\num{2} krople \gls{dmf}.
Po \SI{2}{\hour} mieszania w~temperaturze pokojowej, usunąłem z~mieszaniny lotne ciecze
  przy użyciu wyparki rotacyjnej.
Do~przygotowanego w~ten sposób chlorku kwasowego wkropliłem powoli przygotowany wcześniej roztwór
  \SI{741}{\mg} anizydyny i~\SI{840}{\ul} \ch{Et3N} w~\SI{10}{\ml} acetonitrylu.
Mieszałem w~temperaturze pokojowej przez \SI{1}{\hour}, po czym usunąłem rozpuszczalnik
  przy użyciu wyparki rotacyjnej.
Otrzymany surowy produkt zawiesiłem w~\SI{12}{\ml} \SI{2}{\Molar} \ch{HCl_{(aq)}},
  ekstrahowałem \SI[product-units = single]{4 x 8}{\mL} \ch{Et2O}, suszyłem \ch{MgSO4}
  i~ rozpuszczalnik za~pomocą wyparki rotacyjnej.
Oczyszczałem krystalizując z~mieszaniny octanu etylu z~heksanem, otrzymując \SI{1.22}{\gram}
  produktu w~postaci białego ciała stałego (\SI{77}{\percent} wydajności).
\begin{fullexp}
  \NMR(600)[CDCl3] \num{8.44} (s, \#{1}), \numrange{7.46}{7.33} (m, \#{1}), \numrange{6.89}{6.76} (m, \#{2}), \num{4.21} (q, \J{7.1}, \#{2}), \num{3.76} (s, \#{3}), \num{1.52} (s, \#{6}), \num{1.28} (t, \J{7.1}, \#{3})\par\noindent
  \NMR{13,C}(151)[CDCl3] \numlist{175.5; 169.6; 156.4; 130.9; 121.7; 114.1; 61.9; 55.5; 50.2; 23.9; 14.0}\par\noindent
  \data{IR}[film] \numlist{3291; 3131; 3062; 2982; 2940; 2838; 1730; 1653; 1602; 1535; 1513; 1469; 1443; 1414; 1388; 1364; 1316; 1301; 1237; 1176; 1140; 1112; 1030}\par\noindent
  \data{HRMS} (ESI-TOF) m/z calcd for \ch{C14H19NO4Na}: \num{288.1212} found: \num{288.1216}
\end{fullexp}
  

\procedure{amidoester-cyclo}{\iupac{2-okso-\N-fenylotetrahydrofuran-3-karboksyamid}}
  \marginnote{Analiza \NMR*{} zgodna z~literaturą.}
  \todo[inline]{Add procedure description.\par|\par|}


\section{Funkcjonalizowane aminy wywiedzione z~amidoestrów}\label{experimental:amidoester-products}
\procedure{b-aminoester-cycloprop.allyl}{\iupac{1-(1-[(4-metoksyfenylo)amino]but-3-en-1-ylo)cyklopropanokarboksylan metylu}}
Do wygrzanego w~płomieniu palnika i~wypełnionego argonem naczynia Schlenka odważyłem
  \SI{51.5}{\mg} odczynnika Schwartza oraz \SI{50.0}{\mg} amidu~\refcmpd{amidoester-cycloprop},
  dodałem \SI{1.0}{\mL} \gls{thf}.
Mieszałem do~sklarowania roztworu.
W~przepływie argonu dodałem \SI{249}{\mg} \ch{Yt(OTf)3} oraz \SI{186}{\uL} allilotributylocyny.
Mieszałem jeszcze przez noc, po czym wylałem na~\SI{5}{\mL} \ch{NaHCO3_{aq}},
  przemyłem \SI[product-units = single]{3 x 5}{\mL} \SI{10}{\percent} \ch{NH4F_{aq}}.
Zebrane frakcje organiczne suszyłem \ch{MgSO4}, po~czym odparowałem rozpuszczalnik za~pomocą
  wyparki rotacyjnej.
Oczyszczałem chromatograficznie na~żelu krzemionkowym w~eluencie \SI{15}{\percent} octanu
  etylu w~heksanie.
Otrzymałem \SI{16.9}{\mg} produktu w~postaci oleju (\SI{31}{\percent} wydajności).

\begin{fullexp}
  \NMR(600)[CDCL3] \numrange{6.77}{6.68} (m, \#{2}), \numrange{6.60}{6.51} (m, \#{2}), \numrange{5.88}{5.75} (m, \#{1}), \numrange{5.09}{5.04} (m, \#{1}), \numrange{5.04}{5.00} (m, \#{1}), \num{3.72} (s, \#{3}), \num{3.68} (s, \#{3}), \num{3.43} (dd, \J{8.3;5.5}, \#{1}), \numrange{2.62}{2.48} (m, \#{1}), \numrange{2.40}{2.25} (m, \#{1}), \numrange{1.20}{1.13} (m, \#{2}), \numrange{0.84}{0.76} (m, \#{2})\par\noindent
  \NMR{13,C}(151)[CDCL3] \numlist{174.8; 135.7; 117.0; 115.1; 114.8; 56.6; 55.7; 51.7; 39.0; 26.7; 14.8; 13.0}\par\noindent
  \data{HRMS} (ESI-TOF) m/z calcd for \ch{C16H22NO3}: \num{276.1600} found: \num{276.1597}
\end{fullexp}
  \todo[inline]{IR}

\procedure{b-aminoester-cycloprop.cn}{\iupac{1-(cyjano[(4-metoksyfenylo)amino]metylo)cyklopropanokarboksylan metylu}}
Synteza wg procedury \nameref{syn:b-aminoester-cycloprop.allyl}.
Użyłem \SI{80}{\uL} \ch{\acrshort{tms}CN} jako nukleofila do~funkcjonalizacji.
Oczyszczałem chromatograficznie na~żelu krzemionkowym w~eluencie \SI{20}{\percent} octanu
  etylu w~heksanie.
Otrzymałem \SI{20.4}{\mg} produktu w~postaci oleju (\SI{39}{\percent} wydajności).
\begin{fullexp}
  \NMR(400)[CDCL3] \numrange{6.85}{6.79} (m, \#{2}), \numrange{6.76}{6.68} (m, \#{2}), \num{3.76} (s, \#{3}), \num{3.76} (s, \#{4}), \num{3.74} (s, \#{0}), \numrange{1.58}{1.51} (m, \#{1}), \numrange{1.47}{1.39} (m, \#{1}), \numrange{1.21}{1.13} (m, \#{1}), \numrange{1.14}{1.06} (m, \#{1})\par\noindent
  \NMR{13,C}(101)[CDCl3] \numlist{172.3; 154.4; 138.7; 118.0; 117.1; 115.0; 55.6; 52.6; 51.9; 26.6; 15.9; 13.9}
\end{fullexp}
  \todo[inline]{MS, IR}

\procedure{aminoester-plain-allyl}{\iupac{3-[(4-metoksyfenylo)amino]hex-5-enolan etylu}}
Synteza wg procedury \nameref{syn:b-aminoester-cycloprop.allyl}.
Użyłem \SI{50.0}{\mg} (\SI{0.21}{\milli\mole}) \refcmpd{amidoester-plain} jako substratu
  i~odpowiednią ilość pozostałych reagentów.
Oczyszczałem chromatograficznie na~żelu krzemionkowym w~eluencie \SI{10}{\percent} octanu
  etylu w~heksanie.
Otrzymałem \SI{23.0}{\mg} produktu w~postaci oleju (\SI{42}{\percent} wydajności).

\begin{fullexp}
  \NMR(400)[CDCl3] \numrange{6.82}{6.73} (m, \#{2}), \numrange{6.67}{6.55} (m, \#{2}), \numrange{5.88}{5.75} (m, \#{1}), \numrange{5.15}{5.11} (m, \#{1}), \numrange{5.11}{5.06} (m, \#{1}), \num{4.12} (q, \J{7.1}, \#{2}), \numrange{3.84}{3.76} (m, \#{1}), \num{3.74} (s, \#{3}), \numrange{3.65}{3.47} (m, \#{1}), \numrange{2.59}{2.43} (m, \#{2}), \numrange{2.41}{2.28} (m, \#{2}), \num{1.24} (t, \J{7.2}, \#{3})\par\noindent
  \NMR{13,C}(101)[CDCl3] \numlist{171.9; 152.5; 141.0; 134.3; 118.2; 115.5; 115.0; 60.4; 55.8; 51.2; 38.6; 38.6; 14.2}
\end{fullexp}
\todo[inline]{MS, IR}


\procedure{aminoester-bn-allyl}{\iupac{2-benzylo-3-[(4-metoksyfenylo)amino]hex-5-enolan etylu}}
Synteza wg procedury \nameref{syn:b-aminoester-cycloprop.allyl}.
Użyłem \SI{100}{\mg} (\SI{0.306}{\milli\mole}) \refcmpd{amidoester-bn} jako substratu
  i~odpowiednią ilość pozostałych reagentów.
Oczyszczałem chromatograficznie na~żelu krzemionkowym w~eluencie \SI{10}{\percent} octanu
  etylu w~heksanie.
Otrzymałem \SI{24.0}{\mg} produktu w~postaci oleju (\SI{22}{\percent} wydajności).
\todo[inline]{HNMR, CNMR, MS, IR}
