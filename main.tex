\documentclass[
  a4paper,
  twoside,
  justified,
  nobib,
  nofonts,
  marginals=raggedright,
]{tufte-book}
\usepackage{preambule}

\begin{document}

\frontmatter
\maketitle

% acknowledgements
\cleardoublepage
\thispagestyle{empty}
~\vfill
\vfill
\begin{fullwidth}
\raggedleft\noindent\fontsize{16}{26}\selectfont\itshape
\nohyphenation
dziękuję \\
Magdalenie \\
bez wsparcia której \\
ta praca nie zostałaby ukończona
\end{fullwidth}
\vfill

\tableofcontents

\import{chapter-0}{chapter}

\mainmatter
% widths of text elements
% zwykły tekst \printinunitsof{cm}\prntlen{\textwidth}\\  % 10.69847 cm
% margines \printinunitsof{cm}\prntlen{\marginparwidth}\\ %  4.93929 cm
% przerwy \printinunitsof{cm}\prntlen{\marginparsep}      %  0.81987 cm
%                                                 SUMA:     16.45763 cm

% TODO: add missing in-text references to schemes
\import{chapter-1}{chapter}
\import{chapter-2}{chapter}
\import{chapter-3}{chapter}
\import{chapter-4}{chapter}
\import{chapter-5}{chapter}

\backmatter

% \printbibliography

\cleardoublepage%
\thispagestyle{empty}%
~\vfill%
\vfill%
{%
  \begin{fullwidth}%
    \raggedleft\noindent\fontsize{16}{26}\selectfont\itshape%
    itktp.\hspace*{0.2\textwidth}\par%
  \end{fullwidth}%
}%
\vfill%
% istnieje konwencja typograficzna, według której samodzielne części tekstu należy kończyć
% specjalnym znakiem, zwykle pustym prostokątem, tak zwanym nagrobkiem lub halmosem.
% Stosuje się ją przede wszystkim w czasopismach, gdzie sygnalizuje sie koniec artykułu
% oraz w matematyce, gdzie wskazuje koniec dowodu twierdzenia, zamiast tradycyjnych "co
% należało dowieść", "quod erat demonstrandum", czy podobnych zwrotów i ich skrótów.
% Nauczycielka matematyki, która uczyła mnie jeszcze w gimnazjum, proponowała żartobliwie,
% żeby stosować w tym celu "i tym kończę tę pieśń", lub w skrócie "itktp.". Wtedy wydawało
% mi się to jedynie zabawne, teraz jednak doceniam porównanie dowodów matematycznych do
% sztuki, gdyż rozumiem, że praca matematyka, jak i naukowca, wymaga kreatywności wcale
% nie mniej niż praca artysty. Kończę niniejszą dysertację tym zwrotem, by złożyć hołd
% powyższej myśli, a także wyrazić moją wdzięczność wszystkim, których wymienić tu nie
% sposób, a którzy ugruntowali moje zainteresowanie naukami ścisłymi, tak jak wspomniana
% nauczycielka.

\end{document}
