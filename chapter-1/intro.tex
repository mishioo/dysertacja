\section{Trwałość amidów}\label{literature:amide-stability}
\begin{marginfigure}[7\baselineskip]
  \includesvg{glutathione}
  \caption{
    Glutation \--- trójpeptyd o~właściwościach przeciwulteniających,
    z~wiązaniami amidowymi zanaczonymi na~zielono.
  }
  \label{fig:glutathione}
\end{marginfigure}
Wiązanie amidowe występuje w~naturze powszechnie.
Można nawet pokusić się o~stwierdzenie, że jest ono jednym z~budulców życia \---
  w~końcu peptydy, podstawowa struktura biochemiczna złożonych organizmów,
  to łańcuchy aminokwasów, połączonych wiązaniami amidowymi.
Za~przykład posłużyć może, przedstawiony na~\cref{fig:glutathione}, glutation \---
  trójpeptyd o~właściwościach przeciwulteniających,
  występujący powszechnie w~organizmach roślinnych i~zwierzęcych\sidecite{wu04}.
  
\begin{marginfigure}
  \includesvg{lidocaine}
  \caption{
    Lidokaina \--- przykład leku posiadającego ugrupowanie amidowe
    (zaznaczone na~zielono).
  }
  \label{fig:lidocaine}
\end{marginfigure}
Fragment amidowy można też znaleźć w~wielu związkach biologicznie czynnych.
Prostym przykładem jest lidokaina, przedstawiona na~\cref{fig:lidocaine},
  powszechnie stosowana jako środek miejscowo znieczulający.
Przykładów takich można by przytoczyć wiele, bo~jak pokazuje analiza produkcji farmaceutyków,
  \SI{66}{\percent} leków syntezuje się tworząc wiązanie amidowe\sidecite{carey06}.

W~latach 30. ubiegłego wieku firma DuPont wprowadziła poliamidy na~rynek tworzyw
  sztucznych pod nazwą handlową Nylon.
Ten bardzo trwały materiał szybko znalazł zastosowanie w~wielu gałęziach przemysłu.
Stosuje się go przede wszystkim do~wytwarzania syntetycznych włókien tekstylnych,
  ale też do~produkcji szczoteczek do~zębów, strun do~instrumentów,
  żyłek wędkarskich, czy opakowań żywności.

Tę powszechność \--- zarówno wśród produktów naturalnych, jak i~wytworów cywilizacji \---
  amidy zawdzięczają między innymi swojej wyjątkowo niskiej reaktywności w~porównaniu
  do~innych związków karbonylowych.
Dobrze obrazuje to liniowa skala reaktywności związków karbonylowych zaproponowana
  przez Mucsiego i~Chassa (\cref{fig:carbonyl-scale})\sidecite{mucsi08},
  nazwana przez nich skalą karbonylowości (ang. \textit{carbonylicity}).
Wiązanie amidowe ulega niewielu przemianom chemicznym, a~jeśli już,
  to~zwykle wymaga stosowania bardzo ostrych warunków prowadzenia reakcji.
Ta niezwykła odporność wynika z~bardzo efektywnego nakładania się orbitali 
  molekularnych atomu azotu oraz \textpi{} wiązania podwójnego \ch{C=O}.
Jak widać na~\cref{sch:resonance}, pozwala to na~wydajną delokalizację elektronów
  w~obrębie wiązania i~znaczny udział dwóch możliwych struktur dipolarnych.
\begin{marginscheme}
  \includesvg{resonance}
  \caption{
    Struktury rezonansowe wiązania amidowego, zapewniające mu~niezwykłą trwałość.
  }
  \label{sch:resonance}
\end{marginscheme}

\begin{figure}
  \centering
  \includesvg{carbonyl-scale-acc}
  \caption{
    Względna reaktywność róźnych grup karbonylowych
    wg skali karbonylowości Mucsiego i Chassa.
  }
  \label{fig:carbonyl-scale}
\end{figure}


\section{Prezkształcenia amidów}\label{literature:amide-transformations}
Już w~drugiej połowie XIX~w.\ chemicy wiedzieli, że pierwszorzędowe amidy mogą ulegać
  reakcji odwodnienia pod wpływem tlenku fosforu, dając nitryle.
Amidy drugo-~i trzeciorzędowe nie ulegają takiej przemianie, ale badania nad ich
  reaktywnością doprowadziły Wallacha do odkrycia innej interesującej reakcji.
W~roku \citeyear{wallach77} pokazał on działanie \ch{PCl5} w~podwyższonej temperaturze
  na~amidy drugorzędowe\sidecite{wallach77}.
To doniesienie jest nie tylko pierwszą, ale i~przełomową publikacją na~temat
  przekształceń amidów, jaką można znaleźć w~literaturze.
\citeauthor{wallach77} zauważył powstawanie \iupac{\a-dichloroamin}~\refcmpd{w:dichloro},
  które podczas ogrzewania łatwo ulegają przekształceniu
  w~\iupac{\a-chloriminy}~\refcmpd{w:chloroimine}.
Zauważył też, że związki te wykazują znaczną elektrofilowość,
  bowiem wchodzą w~reakcję z~aminami, dając imidyny~\refcmpd{w:imidine}.
Obserwacje te, przedstawione na~\cref{sch:wallach},
  były podwaliną kolejnych odkryć w~tej dziedzinie.
\begin{scheme}
  \centering
  \includesvg{wallach}
  \caption{Przełomowe odkrycia Wallacha w dziedzinie chemii amidów.}
  \label{sch:wallach}
\end{scheme}
\begin{marginscheme}[5\baselineskip]
  \includesvg{bichler}
  \caption{Ogólny schemat reakcji Bichlera-Napieralskiego.}
  \label{sch:bichler}
\end{marginscheme}
W~\citeyear{bischler93} \citeauthor{bischler93} pokazali, że działając \ch{POCl3}
  na~wywiedziony z~\iupac{2-fenyloetyloaminy} amid~\refcmpd{w:bichler-sub} można otrzymać
  pochodną dihydroizochinoliny~\refcmpd{w:bichler-prod}\sidecite{bischler93}.
Jakiś czas później wariację tej przemiany przedstawili \citeauthor{pictet10}.
Wychodząc z~\iupac{2-hydroksy-2-fenetyloamidu} otrzymali w~jednym etapie produkt
  już odwodniony \--- izochinolinę\sidecite{pictet10}.
W~roku \citeyear{vilsmeier27} \citeauthor{vilsmeier27} pokazali,
  że reakcję tę można prowadzić nie tylko wewnątrzcząsteczkowo.
Działając \ch{POCl3} na~\iupac{\N,\,\N-dimetyloamid}~\refcmpd{w:dimethylamide}
  wytworzyli kation chloroiminiowy~\refcmpd{w:vilsmeier}%
  \footnote{
    Związek ten nazywany jest reagentem Vilsmeiera.
  },
  który ulega addycji do bogatych w~elektrony pierścieni aromatycznych,
  tworząc \iupac{\a-chloro}aminę~\refcmpd{w:vilsm-add}.
Reakcji nie da się jednak zatrzymać na~tym etapie \---
  podczas przerobu następuje hydroliza adduktu, skutkując powstaniem odpowiedniego
  aldehydu (lub ketonu) arylowego~\refcmpd{w:vilsm-prod}\sidecite{vilsmeier27}.
Mechanizm tego przekształcenia prezentuję na \cref{sch:vilsmeier}.
\begin{scheme}
  \centering
  \includesvg{vilsmeier}
  \caption{Mechanizm reakcji Vismeiera-Haacka.}
  \label{sch:vilsmeier}
  \setfloatalignment{b}
\end{scheme}

W~roku \citeyear{hofmann81}, czyli niedługo po odkryciu Wallacha, \citeauthor{hofmann81}
  pokazał pierwszą metodę syntezy amin z amidów\sidecite{hofmann81}.
W~reakcji tej, w~wyniku działania wodorotlenkiem sodu i~bromem na~pierwszorzędowy amid,
  powstaje izocyjanian, który następnie hydrolizuje z~uwolnieniem cząsteczki \ch{CO2}.
Powstaje pierwszorzędowa amina, która ma łańcuch węglowy krótszy o~jeden atom.
Obecnie, ze względu na bezpieczeństwo oraz wygodę eksperymentatora, do przeprowadzenia
  tej reakcji stosuje się \gls{nbs}\footnote{%
    Właściwie \acrlong{nbs} to~nazwa zwyczajowa, wg nazewnictwa systematycznego
      powinno być: \iupac{\N-bromo} imid kwasu bursztynowego.
  }
  jako źródło bromu oraz \gls{dbu} jako zasadę.
Dopiero niemal 70~lat po~publikacji Hofmanna pojawiła się w~literaturze wzmianka
  o~bardziej uniwersalnej metodzie syntezy amin z~amidów.
\citeauthor{brown48} stwierdzili, że możliwa jest redukcja amidowej grupy karbonylowej
  za pomocą glinowodorku litu\sidecite{brown48}.

Redukcja przy użyciu \ch{LiAlH4} jest dzisiaj najbardziej sztandarowym przykładem reaktywności amidów.
Razem z~reakcją odwodnienia, hydrolizą, przegrupowaniem Hofmanna,
  oraz reakcjami Vilsmeiera-Haacka i~Bischlera-Napieralskiego,
  stanowi teraz podręcznikowy kanon.
Przez długi czas przemiany te były właściwie jedynymi dostępnymi chemikom metodami modyfikacji grupy amidowej.
Wszystkie wymagały użycia agresywnych warunków,
  które z~dużym prawdopodobieństwem byłyby niekompatybilne z~innymi grupami funkcyjnymi
  obecnymi w~przekształcanym związku.
Chemia amidów była więc raczej uboga, a~wykorzystanie wiązania amidowego w~syntezie
  często sprowadzało się do jego obecności w~produktach.

Przełomem w~tej materii była praca, którą opublikowali \citeauthor{ghosez69} w~\citeyear{ghosez69} roku.
Przedstawili oni metodę syntezy \iupac{\a-chloroenamin}~(\refcmpd{w:ghosez-chloro}) z~trzeciorzędowych amidów~(\refcmpd{w:ghosez-sub})
  przy użyciu fosgenu i~zasady (trietyloaminy lub pirydyny)\sidecite{ghosez69}.
Autorzy byli zaskoczeni łatwością z~jaką związki te ulegają nukleofilowej substytucji atomu chloru.
W~roli nukleofila przetestowali reagenty Grignarda, związki litoorganiczne, alkoholany, tiolany oraz amidki
  (oznaczone na \cref{sch:chloroenamine} ogólnie jako \ch{Nu-}),
  otrzymując produkty z~dobrymi wydajnościami (\SIrange{65}{90}{\percent}).
Publikacja ta rozpoczyna nową erę w~chemii amidów, dając początek idei \emph{aktywacji wiązania amidowego}.
\begin{scheme}
  \centering
  \includesvg{chloroenamine}
  \caption{Aktywacja amidu przez przekształcenie w~\iupac{\a-chloro}enaminę.}
  \label{sch:chloroenamine}
\end{scheme}
\begin{figure*}
  \centering
  \includesvg{timeline-twoside-simple}
  \caption{Istotne wydarzenia związane z~badaniami reaktywności i~właściwości amidów.}
  \label{fig:timeline}
\end{figure*}
