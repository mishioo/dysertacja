\section{Aktywatory triflowe}\label{literature:triflic}
Gdy metodologia zaprezentowana przez Ghoseza i~in. została dokładniej przetestowana, okazało się że nie jest wolna od wad \---
  wyzwaniem jest wykorzystanie soli keteniminiowych typu \enquote{aldo}%
  \footnote{%
    Typu \enquote{aldo}, czyli posiadające jeden podstawnik w~pozycji \textbeta{} względem atomu azotu.
    Sole keteniminiowe z~dwoma podstawnikami w~tej pozycji, jak na~\cref{sch:chloroenamine}, nazywane są typem \enquote{keto}.%
  }, takich jak~\refcmpd{w:bh-dime-cl-ketenimine}.
Ze względu na zwiększoną nukleofilowość ich prekursora \--- odpowiedniej \iupac{\a-chloroenaminy}~\refcmpd{w:bh-dime-chloro} \---
  istotnym problemem jest znaczna ilość produktów ubocznych, powstających w~wyniku reakcji tych dwóch cząsteczek.
Sposób na pokonanie tej przeszkody został zaproponowany w latach 80-tych, również przez chemików z~zespołu Ghoseza.
Używając bezwodnika triflowego (\ch{(CF3SO2)2O}) zamiast fosgenu,
  otrzymali triflową sól keteniminy poprzez \iupac{\a-triflo}enaminę~\refcmpd{w:bh-dime-triflic}\sidecite{ghosez81}.
Związek ten jest mniej nukleofilowy i~nie wstępuje w~reakcję z~powstającym dalej triflanem keteniminiowym \refcmpd{w:bh-dime-tf-ketenimine},
  co obrazuje \cref{sch:chloro-vs-triflic}.
\begin{scheme*}
  \centering
  \includesvg[triflic/]{chloro-vs-triflic}
  \caption{Różnica w reaktywności chlorowych i triflowych pochodnych enamin z solami ketenimin. \acrshort{TfO}: \acrlong{TfO}.}
  \label{sch:chloro-vs-triflic}
\end{scheme*}

Ówcześnie szczególną wartość dostrzeżono w~reakcji formalnej [2+2] cykloaddycji, której mogą ulegać sole ketenimin.  % TODO: add citation ?
Powstające w~jej wyniku kationy iminiowe łatwo hydrolizują, prowadząc do~powstania pochodnych cyklobutanonu\todo{dodać cytowanie}.
Metodologię tę uznano za~wartościową alternatywę cykloaddycji ketenów, zwłaszcza że sole ketenimin wykazują większą aktywność \---
  w~przeciwieństwie do tych pierwszych wstępują w~reakcje z~prostymi alkenami bez potrzeby stosowania podwyższonej temperatury\sidecite{maulide18}.
Reakcję tę udało się też przeprowadzić w~wariancie wewnątrzcząsteczkowym,
  otrzymując związki o~pierścieniach połączonych\sidecite{ghosez85}, jak~\refcmpd{w:prgl-cyclo}.
Układy takie mogą być dalej funkcjonalizowane np.~poprzez ekspansję czteroczłonowego pierścienia.
Przykładem zastosowania tego podejścia w~syntezie, zilustrowanym na~\cref{sch:prostaglandin}, jest modyfikacja podejścia Coreya do prostaglandyn\sidecite{chen91}.
Warto zwrócić uwagę na~stereoselektywność tego procesu \--- w~przeciwieństwie do ketenu,
  sól keteniminiowa~\refcmpd{w:prgl-ketenimine}, może przenosić informację stereochemiczną poprzez otaczające atom azotu ustalone centra chiralności.
\begin{scheme*}
  \centering
  \includesvg[triflic/]{prostaglandin-fullwidth}
  \caption{
    Stereokontrolowana synteza prostaglandyny z~wykorzystaniem aktywacji wiązania amidowego bezwodnikiem triflowym.
    \acrshort{tbdps}: \acrlong{tbdps}; \acrshort{mcpba}: \acrlong{mcpba}.
  }
  \label{sch:prostaglandin}
\end{scheme*}

Wnikliwy czytelnik mógł zauważyć, że niemal wszystkie dotąd cytowane prace na~ten temat pochodzą z~grupy Ghoseza.
Metodologia oparta na~wykorzystaniu aktywatorów triflowych zaczęła się cieszyć szerszym zainteresowaniem dopiero na początku XXI~w.
Punktem zwrotnym była bardzo wnikliwa analiza mechanizmu tych przemian, przeprowadzona przez Grenona i~Charette'a.
Pokazali oni, że bezwodnik triflowy szybciej reaguje z~pirydyną, użytą jako zasada, niż z~amidem.
Powstaje wtedy triflan \iupac{\N-(trifluorometanosulfonylo)piridyniowy}, który jest właściwym środkiem triflującym\sidecite{charette01}.
Dopiero on reaguje z~amidem, tworząc triflan \iupac{\O-triflyliminiowy} (\refcmpd{w:tf-iminium-sec, w:bh-tf-tert, w:tf-iminium-tert}),
  postulowany już wcześniej przez Ghoseza jako produkt pośredni.
\citeauthor{charette01} udowodnili jednak, że związek ten reaguje natychmiast z~obecną w~mieszaninie pirydyną,
  a~dokładny przebieg tych przekształceń jest zależny od struktury amidu.
Obrazują to ze~szczegółami \cref{sch:triflic-secondary,sch:triflic-tertiary-beta,sch:triflic-tertiary-no-beta}.
\begin{scheme}
  \centering
  \includesvg[triflic/]{triflic-secondary}
  \caption{Mechanizm aktywacji drugorzędowych amidów za~pomocą bezwodnika triflowego i~pirydyny.}
  \label{sch:triflic-secondary}
\end{scheme}
\begin{scheme}
  \centering
  \includesvg[triflic/]{triflic-tertiary-no-beta}
  \caption{Mechanizm aktywacji trzeciorzędowych amidów nie posiadających protonu \textalpha{} za~pomocą bezwodnika triflowego i~pirydyny.}
  \label{sch:triflic-tertiary-no-beta}
\end{scheme}
\begin{scheme*}
  \centering
  \includesvg[triflic/]{triflic-tertiary-beta-fullwidth}
  \caption{Mechanizm aktywacji trzeciorzędowych amidów posiadających proton \textalpha{} za~pomocą bezwodnika triflowego i~pirydyny.}
  \label{sch:triflic-tertiary-beta}
\end{scheme*}

W~przypadku amidów drugorzędowych tworzy się triflan \iupac{1-pirydyloimidoilu}~\refcmpd{w:piridinium-sec}, stabilny w~warunkach reakcji,
  natomiast z~amidów trzeciorzędowych powstaje dwutriflan \iupac{1-pirydyloiminiowy}~\refcmpd{w:piridinium-tert}.
W~tym drugim przypadku, jeśli w~związku występuje proton w~pozycji \textalpha{} względem grupy karbonylowej, jak w~przypadku \refcmpd{w:bh-amide-tert},
  dochodzi do przesunięcia wiązania podwójnego w~tę pozycję, skutkując powstaniem układu~\refcmpd{w:bh-piridinium-tert}.
Autorzy proponują też alternatywną ścieżkę ku wynikowemu związkowi \--- poprzez sól keteniminiową \refcmpd{w:bh-keteniminium-tert}.

Przejściowe związki pirydyniowe \refcmpd{w:bh-piridinium-tert, w:piridinium-sec, w:piridinium-tert}
  mogą wstępować w~dalsze reakcje z~nukleofilem, jeśli jakiś jest obecny w~środowisku reakcyjnym.
Badacze z~grupa Charette'a pokazali, że metodologia ta może być zastosowana do~przekształcenia amidu w~inną grupę funkcyjną.
Przedstawili oni przykłady bezpośredniej transformacji w~tioamid, imidynę, czy tiazolinę, syntezę amidów znakowanych izotopem \ch{^{18}O},
  a~także formalną aktywację wiązania \ch{C-N} poprzez przekształcenie w~ester i~ortoester\sidecite{charette01}.
Synteza tiazolin z~amidów została przez DeRoya i~Charette'a użyta w~praktyce jako jeden z~pierwszych etapów syntezy totalnej
  \iupac{($+$)-cystotiazolu~A}\sidecite{deroy03}.
% TODO: add small scheme

\subsection{Addycja \textpi-nukleofili}\label{literature:triflic:addition}
Chemicy na nowo zainteresowali się też możliwościami, jakie daje addycja \textpi-nukleofili do~aktywowanych amidów.
Okazuje się, że ulegają jej nie tylko alkiny w~eksplorowanej przez Ghozesa formalnej [2+2] cykloaddycji, ale także nukleofilowe alkeny.
\citeauthor{belanger05} pokazali to syntezując różnorodne 5- i~6-członowe azabicykliczne związki nienasycone, między innymi o~strukturze alkaloidów,
  % TODO: check if all was azabi-
  poprzez wewnątrzcząsteczkową reakcję addycji enamin, allilosilanów, oraz sililowych eterów enoli do~aktywowanych amidów\sidecite{belanger05,belanger06}.
\Cref{sch:tashiromine} prezentuje przykład takiej przemiany, prowadzącej ostatecznie do~alkaloidu indolizydynowego \refcmpd{w:tashiromine} \--- taszirominy.
Cyklizacja sililowego eteru enolu \refcmpd{w:tashiromine-sub} pod wpływem bezwodnika triflowego i~\acrshort{dtbmp} (\acrlong{dtbmp}) i~hydroliza
  skutkuje powstaniem \textalpha,\textbeta-nienasyconego \textbeta-aminoaldehydu \refcmpd{w:tashiromine-prod}, prekursora wspomnianego związku.
Autorzy zaprezentowali tę metodologię także jako narzędzie w~niezwykle eleganckiej syntezie trójpierścieniowych alkaloidów
  \refcmpd{w:tricyclo-alkaloid} ze~związków liniowych \refcmpd{w:tricyclo-alkaloid-sub}.
Przemiana ta, widoczna na~\cref{sch:tricyclo-alkaloid} była przeprowadzona w~kaskadowym procesie, biegnącym szybko i~z~wysoką wydajnością,
  doskonale obrazując świetną chemoselektywność omawianego procesu aktywacji wiązania amidowego\sidecite{belanger08}.
\begin{scheme}
  \centering
  \includesvg[triflic/]{triflic-tashiromine}
  \caption{
    Wewnątrzcząsteczkowa addycja wiązania podwójnego do amidu poprzez aktywację bezwodnikiem triflowym,
    przedstawiona na przykładzie syntezy naturalnego alkaloidu (ang. tashiromine, \refcmpd{w:tashiromine}).
  }
  \label{sch:tashiromine}
\end{scheme}
\begin{scheme*}
  \centering
  \includesvg[triflic/]{tricyclo-alkaloid}
  \caption[]{%[-3\baselineskip]{  % TODO: not compatible with draft yet
    Elegancka synteza trójcyklicznego alkaloidu, którą zaprezentowali \citeauthor{belanger06}.
    Przykład ten doskonale obrazuje chemoselektywność metody względem amidowej grupy karbonylowej.
  }
  \label{sch:tricyclo-alkaloid}
\end{scheme*}

Metodologię tę wykorzystali później chemicy z~grupy Dixona jako kluczowy, finalny etap w~syntezie totalnej \iupac{($-$)-nakadomarinu~A},
  alkaloidu wydzielonego z~gąbki z~rodzaju \textit{Amphimedon}\sidecite{dixon11}.
Początkowo chcieli oni zsyntezować ten związek kończąc ścieżkę syntetyczną zamknięciem największego, 15-członowego pierścienia
  poprzez metatezę alkinów (transformacja \refcmpd{w:nakadomarin-wrong} do \refcmpd{w:nakadomarin}), okazało się to jednak niemożliwe.
Ostatecznie krok ten przeprowadzili jako jeden z~wcześniejszych etapów,
  a~finalny produkt otrzymali prowadząc wewnątrzcząsteczkową addycję furanu do aktywowanego 5-członowego laktamu.
Redukcja powstałej soli iminiowej \refcmpd{w:nakadomarin-iminium} dała oczekiwany alkaloid \refcmpd{w:nakadomarin}.
\begin{scheme}
  \centering
  \includesvg[triflic/]{nakadomarin}
  \caption{
    Wykorzystanie aktywacji amidu bezwodnikiem triflowym w~syntezie totalnej nakadomarinu~A (\refcmpd{w:nakadomarin}),
      alkaloidu wydzielonego z~gąbki z~rodzaju \textit{Amphimedon}.
    \acrshort{dtbp}: \acrlong{dtbp}.
  }
  \label{sch:dixon-alkaloid}
\end{scheme}

\citeauthor{movassaghi06a} zaadaptowali metodę Ghoseza do syntezy pochodnych pirydyny z~\iupac{\N-winylowych}%
  \footnote{%
    Także \iupac{\N-arylowych}, z~których otrzymuje się układ pierścieni połączonych.%
  } drugorzędowych amidów.
Pierwsza zaproponowana przez nich procedura była dwuetapowa \--- najpierw funkcjonalizowali amid acetylenkiem miedzi,
  żeby następnie przeprowadzić cyklizację katalizowaną kompleksem rutenu\sidecite{movassaghi06a, movassaghi07syn}.
Wiemy już jednak, że bogate w~elektrony alkeny i~enole sililowe są wystarczająco nukleofilowe,
  by wchodzić w~reakcję z~triflanami \iupac{1-pirydyloimidoilu} \refcmpd{w:tf-iminium-sec},
  bez konieczności generowania związków miedzioorganicznych%
  \footnote{Możliwe, że \citeauthor{movassaghi07} odkryli to niezależnie, bowiem nie powołują się w~następnej cytowanej pracy na~przytoczone wyżej publikacje.}.
Niedługo później autorzy zaprezentowali opartą o~ten fakt bezpośrednią syntezę pochodnych pirydyny, przedstawioną na~\cref{sch:pyridine-synthesis}.
Addycja \textpi-nukleofili (jak np.~\refcmpd{w:allyl}) do~aktywowanych amidów prowadzi do~powstania wysoce reaktywnych intermediatów \refcmpd{w:pyridine-int},
  które ulegają spontanicznej cyklizacji i~aromatyzacji, tworząc od razu pochodne pirydyny~\refcmpd{w:pyridine} \--- w~jednym kroku syntetycznym\sidecite{movassaghi07}.
W~analogicznym procesie, wykorzystując nitryle w~roli nukleofila, można otrzymać pochodne pirymidyny\sidecite{movassaghi06b}.
W~każdym z~tych przypadków kluczowym jest użycie \iupac{2-chloropirydyny} jako zasady do~aktywacji z~bezwodnikiem triflowym.
Autorzy nie dyskutowali nad przyczyną konieczności stosowania słabszej zasady,
  ale pokazali, że stosowanie innych zasad prowadzi do~znacznie niższych wydajności\sidecite{movassaghi06b}.
\begin{scheme}
  \centering
  \includesvg[triflic/]{pyridine}
  \caption{
    Synteza pirydyn wykorzystująca addycję alkenów i~alkinów do aktywowanych amidów.
    Górna ścieżka przedstawia pierwszą, dwuetapową metodę; dolna \--- nowszą.
    \acrshort{sphos}: \acrlong{sphos}.
  }
  \label{sch:pyridine-synthesis}
\end{scheme}

Tematyka syntezy azotowych heterocykli tą metodą została jeszcze wzbogacona przez Wanga i~in.,
  którzy użyli diazooctanu etylu \refcmpd{w:diazoacetate-et} jako nukleofila w~reakcji z~\iupac{\N-arylowymi}
    amidami \refcmpd{w:n-aryl-amide} w~nowej syntezie pochodnych indolu\sidecite{wang08}.
Co ciekawe, do wydajnego przebiegu reakcji potrzebne było dodanie niewielkiej ilości (\SI{0.2}{\equiv}) \iupac{2,6-dichloropirydyny},
  która dodatkowo zwiększa reaktywność generowanej soli iminiowej poprzez tworzenie układu \refcmpd{w:n-aryl-diclpy} (\cref{sch:indole-synthesis}).
Po addycji diazooctanu następuje ekstruzja cząsteczki azotu, cyklizacja układu, jego rearomatyzacja
  i~w~końcu powstanie pochodnej indolu \refcmpd{w:indole} poprzez 1-3 przeniesienie protonu.
Niedawno opublikowane prace rozszerzają wachlarz podobnych przekształceń \iupac{\N-arylowych} amidów o~syntezę różnorodnych
  chinolin\sidecite[-1\baselineskip]{wezeman16,liang17} i~\iupac{2,3-dihydrochinolin}\sidecite{huang19} w~reakcji odpowiednio z~alkinami i~alkenami.
\begin{scheme}
  \centering
  \includesvg[triflic/]{indole}
  \caption{Synteza indoli wykorzystująca diazooctan etylu w~roli nukleofila z~dodatkową aktywacją za~pomocą \iupac{2,6-dichloropirydyny}.}
  \label{sch:indole-synthesis}
\end{scheme}

Zamiast pozwalać na~samoistne przekształcenia produktów pośrednich, jak w~przypadkach opisywanych powyżej,
  reakcje addycji \textpi-nukleofili do~aktywowanych amidów można również terminować reduktywnie, co prowadzi do~powstania funkcjonalizowanych amin.
\citeauthor{belanger15} wykorzystali tę technikę opracowując analogiczną do~reakcji Mannicha
  metodę syntezy \textbeta-aminoestrów \refcmpd{w:b-amino-esther}\sidecite{belanger15}.
Użyli w~niej aktywowanych amidów jako odpowiedników imin i~sililowanych acetali ketenów w~roli nukleofila.
Powstające produkty pośrednie \refcmpd{w:b-iminium-esther} poddali redukcji za~pomocą \ch{NaBH4},
  w~przypadku amidów innych niż formylowe (\ch{R^3} $\neq$ \ch{H} na \cref{sch:branched-beta-aminoesthers})
  zakwaszając środowisko kwasem octowym, aby uniknąć niepożądanej reakcji retro-Mannicha.
W~rezultacie otrzymali rozmaite \textbeta-aminoestry \refcmpd{w:b-amino-esther}, które można łatwo przekształcić w~nienaturalne aminokwasy\sidecite{belanger15}.
Należy zaznaczyć, że pochodne o~takim stopniu rozgałęzienia łańcucha niełatwo otrzymać w~reakcji Mannicha,
  czy za~pomocą innych klasycznych metod syntezy związków tego typu.
\begin{scheme*}
  \centering
  \includesvg[triflic/]{branched-aminoesthers}
  \caption{
    Synteza rozgałęzionych \textbeta-aminoestrów, niedostępnych przy użyciu reakcji Mannicha i~innych klasycznych metod.
    \acrshort{tbs}: \acrlong{tbs}.
  }
  \label{sch:branched-beta-aminoesthers}
\end{scheme*}

Podobna reakcja z~wykorzystaniem trzeciorzędowej enaminy \refcmpd{w:enamine-pyrr} jako nukleofila również jest możliwa do przeprowadzenia,
  ale przebiega w~dość zaskakujący sposób.
Redukcja powstającego intermediatu \refcmpd{w:tf-diiminium-pyrr} prowadzi nie do diaminy,
  jak można by się spodziewać, ale do alliloaminy \refcmpd{w:allyl-amine}\sidecite{huang18}.
Autorzy opisujący tę przemianię z~powodzeniem wykorzystują enaminy i~drugorzędowe amidy posiadające różne grupy funkcyjne \---
  karbonylową, estrową, czy allilową \--- kolejny raz dowodząc znacznej chemoselektywności procedury aktywacji opartej na~użyciu bezwodnika triflowego.
Nie dociekają jednak, czy możliwe jest użycie substratów o~innej rzędowości.
\begin{scheme*}
  \centering
  \includesvg[triflic/]{triflic-nu-enamine}
  \caption{
    Niespodziewany przebieg redukcji soli iminoiminiowej \refcmpd{w:tf-diiminium-pyrr} borowodorkiem sodu,
    prowadzący do powstania alliloaminy.
  }
  \label{sch:allyloamine-synthesis}
\end{scheme*}

\subsection{Inne C-nukleofile}\label{literature:triflic:c-nucleophiles}
% TODO: try to rephrase following paragraph
Starając się dowieść wyższości metody wykorzystującej bezwodnik triflowy nad przemianami biegnącymi
  poprzez chloroenaminę nie mogę pominąć kwestii różnorodności możliwych do zastosowania nukleofili.
\citeauthor{ghosez69} we~wspomnianej wcześniej przełomowej pracy\sidecite{ghosez69} użyli rozmaitych związków w~tej roli.
Analogi triflowe nie ustępują chlorowym pierwowzorom w~tej materii,
  chociaż chemikom przyszło się o~tym przekonać dopiero stosunkowo niedawno.
Dekadę przed powstaniem niniejszej dysertacji \citeauthor{xiao10} pokazali,
  że możliwa jest addycja odczynników Grignarda do aktywowanego za~pomocą \ch{Tf2O} wiązania amidowego\sidecite{xiao10}.
Z~amidów trzeciorzędowych \refcmpd{w:amide-tert} powstają w~tej sekwencji przemian kationy iminiowe
  \refcmpd{w:iminium-keto-tert} zdolne do przyłączenia kolejnej cząsteczki nukleofila,
  prowadząc finalnie do \textalpha,\textalpha-dwupodstawionych amin \refcmpd{w:amine-tert-same}.

Aktywność kationu iminiowego \refcmpd{w:iminium-keto-tert} jest jednak niższa niż
  \iupac{\O-triflyliminiowego} \refcmpd{w:tf-iminium-tert} i~autorzy wykorzystują ten fakt,
  by zróżnicować strukturę otrzymywanych produktów.
Dowodzą, że stosując równomolową ilość odczynnika Grignarda i~manipulując temperaturą prowadzenia reakcji można przeprowadzić
  ten proces dwuetapowo, dodając inny nukleofil w~drugim etapie i~otrzymując aminę z~dwojgiem różnych podstawników w~pozycji \textalpha.
Ów drugi nukleofil nie musi być już związkiem magnezoorganicznym \---
  \citeauthor{xiao10} użyli też związków litoorganicznych i~enolanów prezentując tę koncepcję\sidecite{xiao10}.
Co więcej, stwierdzili, że można też zaadaptować ideę reduktywnej terminacji reakcji, wspomnianą już wcześniej\sidecite{belanger15}.
Stosując borowodorek sodu lub glinowodorek litu jako drugi nukleofil można otrzymać aminy monopodstawione \refcmpd{w:amine-tert-mono}\sidecite{xiao10eurj}.
Autorzy zauważyli też, że powstające po addycji pierwszej cząsteczki nukleofila sole iminiowe~\refcmpd{w:iminium-keto-tert}
  można po prostu poddać kwasowej hydrolizie, co prowadzi do powstania odpowiednich ketonu i~drugorzędowej aminy\sidecite{huang15tet}.
  \begin{scheme}
    \centering
    \includesvg[triflic/]{huang-tert}
    \caption{
      Różne, pokazane przez zespół Huanga, możliwości funkcjonalizacji amidów trzeciorzędowych
      poprzez aktywację bezwodnikiem triflowym: wyczerpujące alkilowanie,
      sekwencyjna difunkcjonalizacja, reduktywne monoalkilowanie, hydroliza do~ketonu i~aminy.
    }
    \label{sch:huang-tert}
    \setfloatalignment{b}
  \end{scheme}

Te pierwsze eksperymenty badaczy z~grupy Huanga obejmowały aktywację jedynie amidów trzeciorzędowych,
  jadnak niebawem udowodnili oni, że i~drugorzędowe amidy \refcmpd{w:amide-sec} ulegają takim przemianom.
Konieczny do ich przeprowadzenia jest dodatek chlorku ceru~(III), w~jego obecności ze~związków lito- i~magnezoorganicznych
  powstają \latin{in situ} bardziej nukleofilowe związki ceroorganiczne.
Dopiero te ulegają addycji do aktywowanego amidu drugorzędowego,
  a~powstającą iminę \refcmpd{w:imine-keto} można zredukować lub poddać działaniu kolejnego czynnika nukleofilowego\sidecite{xiao12}.
Procedurę syntezy \textalpha-monopodstawionych drugorzędowych amin udało się później usprawnić odwracając kolejność działań.
Przeprowadzając kontrolowaną redukcję aktywowanego wiązania amidowego za~pomocą \ch{Et3SiH},
  otrzymuje się iminę \refcmpd{w:imine}, którą można poddawać dalszym przemianom\sidecite{huang15joc}.
Autorzy pokazali szerszy zakres kompatybilnych nukleofili, w~tym silanów, enolanów, i~związków cynoorganicznych,
  których użycie w~pierwotnej procedurze nie było możliwe.
Być może nawet ważniejszą poprawą jest zwiększenie chemoselektywności \--- wcześniej nie powiodła się funkcjonalizacja
  związków zawierających grupę nitrylową czy grupę estrową, a~dzięki wykorzystaniu usprawnionej metodologii
  było to możliwe\sidecite{huang15joc}.
W~końcu autorzy sugerują, że w~przypadku użycia chiralnych substratów metodologie te powinny być komplementarne
  w~kwestii stereochemii powstającego nowego wiązania \ch{C-C}, jednak jak dotąd nie potwierdzili tej tezy.
  \begin{scheme}
    \centering
    \includesvg[triflic/]{sec-mono}
    \caption{
      Dwie ścieżki monofunkcjonalizacji drugorzędowych amidów zaprezentowane przez zespół Huanga,
      przedstawione na~przykładzie związku Grignarda jako nukleofila.
    }
    \label{sch:huang-sec-mono}
  \end{scheme}
  
Ostatnie prace pochodzące grupy Huanga prezentują bardziej złożone przemiany amidów.
Pokazują na~przykład, że możliwe jest auto-sprzęganie kationów iminiowych~\refcmpd{w:iminium-h},
  prowadzące do~powstania \iupac{1,2-diamin}~\refcmpd{w:b-diamine}.
Związki przejściowe~\refcmpd{w:iminium-h} otrzymane są na~drodze
  aktywacji-redukcji drugorzędowych amidów \refcmpd{w:amide-sec} dzięki użyciu jodku samaru (II).
Zapewniając lekko zasadowe środowisko poprzez dodatek trietyloaminy można też przeprowadzić
  sprzęganie krzyżowe imin \refcmpd{w:imine} z~ketonami, co pozwala otrzymać
  i~\iupac{1,2-aminoalkohole} \refcmpd{w:b-aminohydroxyl}\sidecite{huang15comm}.
Tę samą metodę generowania imin wykorzystano do przeprowadzenia wieloskładnikowej reakcji Ugiego\sidecite{zheng15}.
W~standardowej wersji tej reakcji imina \refcmpd{w:imine} generowana jest również \latin{in situ},
  ale w~wyniku kondensacji aminy i~ketonu.
Reakcja Ugiego jest przemianą bardzo interesującą, choćby ze~względu na~wysoką ekonomię atomową\footnote{%
  Ekonomia atomowa to pojęcie wchodzące w~skład koncepcji zielonej chemii,
  rozumie się przez nie dążenie do maksymalizacji udziału substratów w~produkcie.
}.
Wrócę do niej jeszcze niejednokrotnie, gdyż istotna część pracy eksperymentalnej
  zrealizowanej w~ramach niniejszej dysertacji wykorzystuje jeden z~jej wariantów.
\begin{scheme*}
  \centering
  \includesvg[triflic/]{ugi-diamine}
  \caption{
    Zaprezentowane przez zespół Huanga przekształcenia drugorzędowych amidów biegnące poprzez iminę: sprzęganie i~reakcja Ugiego.
  }
  \label{sch:huang-ugi-diamine}
\end{scheme*}

Pośród wspomnianych najnowszych doniesień znajdują się też przykłady specjalizacji metod funkcjonalizacji.
W~tej kategorii warto wymienić dialkinowanie trzeciorzędowych amidów\sidecite{chen19joc},
  pozwalające otrzymać bloki budulcowe użyteczne na~przykład w~syntezie związków pochodzenia naturalnego,
  zawierające ugrupowanie \iupac{1,4-dienu}.
Innym typem wyspecjalizowanej podwójnej funkcjonalizacji jest sekwencyjne cyjanowanie-fosforylacja
  drugorzędowych amidów\sidecite{chen19}.
Otrzymywane w~jej wyniku pochodne kwasu \iupac{\a-aminofosfoniowego}, ze~względu na~ich aktywność biologiczną,
  mają liczne zastosowania w~biologii, chemii medycznej czy~rolnictwie, ale także w~syntezie.
Warto w~końcu zwrócić uwagę na~pracę poświęconą syntezie \iupac{\a-(trifluorometylo)amin}
  poprzez bezpośrednie wprowadzenie grupy \ch{-CF3}\sidecite{chen18}.
Funkcja ta, jako bioizoster\footnote{%
    Więcej o~bioizosteryzmie, choć w~kontekście tetrazoli, można znaleźć w~rozdziale \secref{literature:tetrazole:bioisosterizm}.%
  } grupy karbonylowej, znajduje się w~wielu nowoczesnych lekach,
  a~łatwe w~użyciu metody jej wprowadzania do~cząsteczki są niezwykle wartościowymi narzędziami
  dla chemików pracujących nad syntezą nowych związków biologicznie aktywnych.
Swoją cegiełkę do~wysiłków tworzenia takich narzędzi dołożyli również naukowcy z~grupy autora niniejszej dysertacji,
  o~czym można przeczytać w~rozdziale \secref{literature:schwartz:our}.

\subsection{Addycja anionu wodorkowego}\label{literature:triflic:reduction}
Nie można pominąć doniesienia naukowców z~grupy Huanga o~opracowaniu nowej metody redukcji amidów do amin.
Wysoka reaktywność amidów aktywowanych bezwodnikiem triflowym pozwala na~użycie mniej nukleofilowego donora anionu wodorowego oraz
przeprowadzenie przemiany w~łagodniejszych warunkach niż w~przypadku klasycznej redukcji glinowodorkiem litu.
W~pierwszym podejściu do tego zagadnienia autorzy wykorzystali \ch{NaBH4} jako reduktor.
Reakcja ta przebiega w~pokojowej temperaturze, ale
  kluczowym dla powodzenia procesu jest użycie do~etapu redukcji tetrahydrofuranu (\acrshort{thf}) jako rozpuszczalnika.
\citeauthor{xiang10} stwierdzili, że można w~ten sposób prowadzić redukcję amidów drugo- jak i trzeciorzędowych,
  ale selektywność metody nie jest idealna \--- grupy estrowe również ulegają redukcji przy jej zastosowaniu\sidecite[-10\baselineskip]{xiang10}.

Lepsze wyniki osiągnęli \citeauthor{huang16b} łącząc\sidecite[-9.5\baselineskip]{huang16b} opracowaną dotąd metodologię 
  z~podejściem wykorzystującym \gls{tmds} oraz katalityczną ilość \ch{B(C6F5)3}\sidecite[-8\baselineskip]{tan09, chadwick14, blondiaux14}.
Drugi z~wymienionych jest silnym kwasem Lewisa; postuluje się, że zwiększa on aktywność \gls{tmds},
  tworząc separowaną parę jonową\sidenote[][-3\baselineskip]{%
    Ang. \emph{frustrated Lewis pair}. To bardzo ciekawy temat,
    wykracza jednak poza zakres zagadnień niezbędnych do omówienia w~ramach niniejszej dysertacji.
    Zainteresowanym proponuję zapoznać się z~odnośną literaturą, np. \cite{stephan15}.%
  }.
Opracowana metoda jest niezwykle łagodna i~selektywna \--- pozwala na~redukcję wiązania amidowego
  w~obecności wiązań wielokrotnych, estrów, nitrylu, grupy nitrowej, czy eterów sililowych.
Na podstawie zakresu przykładów przedstawionych w~cytowanej pracy można jednak wnioskować,
  że metoda ta jest ograniczona do liniowych amidów drugorzędowych\footnote{%
    Warto wspomnieć, że w~czasie ukazania się omawianej publikacji znane były też już inne metody łagodnej redukcji amidów.
    Więcej o~nich opowiem w~dalszej części tego przeglądu literatury.
  }.
\begin{scheme}
  \centering
  \includesvg[triflic/]{huang-reduction}
  \caption[]{
    Porównanie opracowanych przez zespół Huanga metod redukcji amidowej grupy karbonylowej.
  }
  \label{sch:huang-reduction}
  \setfloatalignment{b}
\end{scheme}

Alternatywną, dwuetapową procedurę redukcji drugorzędowych amidów \refcmpd{w:amide-sec} zaproponował zespół Charette'a.
Polega ona na~hydrosililowaniu aktywowanego amidu za~pomocą \ch{Et3SiH}, co prowadzi do powstania iminy \refcmpd{w:imine}.
Dodając w~następnym kroku \gls{heh} otrzymuje się produkt dalszej redukcji \--- aminę \refcmpd{w:amine-sec}.
Metoda ta jest bardzo selektywna, autorzy pokazują przykłady jej wydajnego zastosowania w~obecności nawet szerszego wachlarza
  grup funkcyjnych niż w~przypadku procedury zaproponowanej przez Huanga.
Dodatkową zaletą jest możliwość zatrzymania reakcji na~etapie iminy albo przeprowadzenia hydrolizy do aldehydu.
W~takim wypadku tolerowana jest nawet obecność innej grupy aldehydowej w~cząsteczce\sidecite{charette10}.
Redukcja amidów trzeciorzędowych \refcmpd{w:amide-tert} jest nawet prostsza \--- po ich aktywacji za~pomocą \ch{Tf2O},
  \gls{heh} redukuje je bezpośrednio do~amin\sidecite{barbe08}.
\begin{scheme}
  \centering
  \includesvg[triflic/]{charette-reduction}
  \caption{
    Redukcja drugo- i~trzeciorzędowych amidów za~pomocą \gls{heh} według metody Charette'a.
  }
  \label{sch:charette-reduction}
  \setfloatalignment{b}
\end{scheme}

\subsection{Przegrupowania wywołane aktywatorami triflowymi}\label{literature:triflic:rearangements}

Naukowcy z~grupy Maulide odkryli, że~amidy trzeciorzędowe mogą być aktywowane bezwodnikiem triflowym
  w~kierunku reakcji innego typu niż opisywane dotąd \--- w~kierunku formalnej \ch{C-H} aktywacji w~pozycji \textalpha{}.
Nazwali tę metodologię \emph{umpolungiem amidów} \--- przez analogię do~reakcji przebiegunowania karbonyli.
Przykłady takich przekształceń i~szerszy opis tematu znajduje się w~rozdziale \secref{literature:other:umpolung},
  tutaj zostanę jeszcze przy reakcjach angażujących karbonylowy atom wiązania amidowego,
  ale odkrytych w~toku badań nad wspomnianym zagadnieniem.
Badacze zaobserwowali kilka przypadków niespodziewanego przegrupowania aktywowanej cząsteczki.
Pierwszym z~takich spostrzeżeń był przebieg cyklizacji związku \refcmpd{w:nuno-claisen-substrate},
  zawierającego fragment eteru allilowego, który pokazuję na~\cref{sch:maulide-claisen}\sidecite[-2\baselineskip]{madelaine10}.
Zamiast prowadzić do~powstania dwupierścieniowego związku \refcmpd{w:nuno-claisen-expected},
  cyklizacja biegnie przez kation oksoniowy \refcmpd{w:nuno-claisen-oxonium},
  który następnie ulega przegrupowaniu Claisena.
Przerób wodny mieszaniny reakcyjnej pozwala wydzielić lakton \refcmpd{w:nuno-claisen-got},
  którego różne pochodne badacze otrzymali też w~dwóch późniejszych pracach\sidecite[-3\baselineskip]{peng12,padmanaban15}.
\begin{scheme*}
  \centering
  \includesvg[triflic/]{maulide-claisen}
  \caption{
    Nieoczekiwany przebieg cyklizacji aktywowanego amidu \refcmpd{w:nuno-claisen-substrate}
      z~następczym przegrupowaniem Claisena.
  }
  \label{sch:maulide-claisen}
\end{scheme*}

Inny przypadek nieoczekiwanej reaktywności aktywowanego amidu naukowcy zauważyli badając
  przemiany pochodnych proliny.
Amidoester \refcmpd{w:nuno-5-substrate} poddany działaniu samego bezwodnika triflowego
  cyklizuje do~soli alkoksyoksazoliniowej \refcmpd{w:nuno-5-cyclo},
  ale obecność pochodnej pirydyny, zwykle ułatwiającej przebieg procesu aktywacji,
  uniemożliwia tę przemianę\sidecite{spinozzi19}.
Zrozumienie tego fenomenu jest łatwiejsze po przypomnieniu sobie mechanizmu
  aktywacji triflanami amidów trzeciorzędowych posiadających atom wodoru w~pozycji \textalpha{}:
  reakcja z~\ch{Tf2O} jest zawsze pierwszym etapem (powstaje \refcmpd{w:nuno-5-tf}),
  ale w~obecności pirydyny przekształcenie do~pochodnej \refcmpd{w:nuno-5-pyr}
  następuje natychmiast\footnote{%
    Patrz: \cref{sch:triflic-tertiary-beta} na~str.~\pageref{sch:triflic-tertiary-beta},
      tam będą to odpowiednio struktury \refcmpd{w:bh-tf-tert, w:bh-piridinium-tert}.
  }.
W~przeciwieństwie do~\refcmpd{w:nuno-5-tf}, związek \refcmpd{w:nuno-5-pyr} o~strukturze
  enaminy nie ma możliwości cyklizacji do~\refcmpd{w:nuno-5-cyclo}, a~jego hydroliza
  prowadzi głównie do~odzyskania substratu \refcmpd{w:nuno-5-substrate}.
\Cref{sch:maulide-five} prezentuje te przemiany razem z~przykładem zastosowania
  związku \refcmpd{w:nuno-5-cyclo} w~syntezie, jako substrat
  do~[3+2] cykloaddycji\sidecite{spinozzi19}.
Warto wspomnieć, że autorzy cytowanej pracy zwrócili uwagę, że w przypadku bardziej
  rozgałęzionego podstawnika, utrudniony dostęp do~protonu w~pozycji \textalpha{}
  również promuje powstawanie produktów typu \refcmpd{w:nuno-5-cyclo}.
\begin{scheme}
  \centering
  \includesvg[triflic/]{maulide-five}
  \caption{
    Zaproponowana przez Spinozzi, Bauera i Maulide metoda otrzymywania bicyklicznych soli alkoksyoksazoliniowych
      \refcmpd{w:nuno-5-cyclo} z~pochodnej proliny \refcmpd{w:nuno-5-substrate}
      wraz z~przykładem zastosowania otrzymanego związku w~syntezie.
    Nietypowo, standardowy w~procesie aktywacji dodatek zasady (pochodnej pirydyny)
      uniemożliwia pożądany przebieg reakcji (patrz: zw.~\refcmpd{w:nuno-5-pyr}).
  }
  \label{sch:maulide-five}
\end{scheme}

W~kolejnej pracy badacze z~grupy Maulide przedstawili przegrupowanie prowadzące
  do~powstania nieudokumentowanej wcześniej klasy 7\-/członowych związków
  heterocyklicznych o~ogólnej strukturze \refcmpd{w:nuno-seven-product}\sidecite{bauer20}.
Substratem w~tym procesie jest imidoamid \refcmpd{w:nuno-seven-substrate}, który,
  poddany aktywacji triflanami, cyklizuje, podobnie jak w~poprzednim przykładzie.
Produkt pośredni \refcmpd{w:nuno-seven-insertion} ulega ekspansji nowo utworzonego
  pierścienia poprzez insercję cząsteczki acetonitrylu, a~produkt otrzymuje się
  po~hydrolizie powstającej soli iminiowej \refcmpd{w:nuno-seven-expantion}.
Co ciekawe, w~przypadku gdy \ch{R} na~\cref{sch:maulide-seven} jest protonem lub fenylem,
  w~ogóle nie otrzymuje się oczekiwanego związku, a~złożoną mieszaninę,
  natomiast prosty \textalpha-sukcynoimidoamid nie ulega tej reakcji wcale.
Przyczyna występowania tych zaskakujących wyjątków pozostaje nieznana.
\begin{scheme*}
  \centering
  \includesvg[triflic/]{maulide-seven}
  \caption{
    Uproszczona ścieżka przekształcenia imidoamidu \refcmpd{w:nuno-seven-substrate}
      do~związku \refcmpd{w:nuno-seven-product} o~7\-/członowym pierścieniu
      w~obecności aktywatora triflowego i~acetonitrylu.
  }
  \label{sch:maulide-seven}
\end{scheme*}

