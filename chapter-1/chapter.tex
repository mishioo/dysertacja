\chapter{Stan wiedzy}


\section{Trwałość amidów}

\begin{marginfigure}[7\baselineskip]
  \includesvg[graphics/]{glutathione}
  \caption{
    Glutation \--- trójpeptyd o~właściwościach przeciwulteniających,
    z~wiązaniami amidowymi zanaczonymi na~zielono.
  }
  \label{fig:glutathione}
\end{marginfigure}
Wiązanie amidowe występuje w~naturze niezwykle powszechnie.
Można nawet pokusić się o~stwierdzenie, że jest ono jednym z~budulców życia \---
w~końcu peptydy, podstawowa struktura biochemiczna złożonych organizmów,
to łańcuchy aminokwasów, połączonych wiązaniami amidowymi.
Za~przykład posłużyć może glutation \--- trójpeptyd o~właściwościach przeciwulteniających,
występujący powszechnie w~organizmach roślinnych i~zwierzęcych\autocite{wu04},
przedstawiony na~\cref{fig:glutathione}.
  
\begin{marginfigure}
  \includesvg[graphics/]{lidocaine}
  \caption{
    Lidokaina \--- przykład leku posiadającego ugrupowanie amidowe
    (zaznaczone na~zielono).
  }
  \label{fig:lidocaine}
\end{marginfigure}
Ugrupowanie to~można też znaleźć w~wielu związkach biologicznie czynnych.
Prostym przykładem może być lidokaina, przedstawiona na~\cref{fig:lidocaine},
powszechnie stosowana jako środek miejscowo znieczulający.
Przykładów takich można by przytoczyć wiele, bo~jak pokazuje analiza produkcji farmaceutyków,
\SI{66}{\percent} leków syntezuje się tworząc wiązanie amidowe\autocite{carey06}.

W~latach 30. ubiegłego wieku firma DuPont wprowadziła poliamidy na~rynek tworzyw sztucznych pod nazwą handlową Nylon.
Ten bardzo trwały materiał szybko znalazł zastosowanie w~wielu gałęziach przemysłu.
Stosuje się go przede wszystkim do~wytwarzania syntetycznych włókien tekstylnych,
ale też do~produkcji szczoteczek do~zębów, strun do~instrumentów, żyłek wędkarskich, czy opakowań żywności.

Tę powszechność \--- zarówno wśród produktów naturalnych, jak i~wytworów cywilizacji \---
amidy zawdzięczają między innymi swojej wyjątkowo niskiej reaktywności w~porównaniu do~innych karbonyli.
Dobrze obrazuje to liniowa skala karbonylowości zaproponowana przez Mucsiego i~Chassa (\cref{fig:carbonyl-scale})\autocite{mucsi08}.
Wiązanie amidowe ulega niewielu przemianom chemicznym, a~jeśli już,
to~zwykle wymaga stosowania bardzo ostrych warunków prowadzenia reakcji.
Ta niezwykła odporność wynika z~bardzo efektywnego nakładania się orbitali 
molekularnych atomu azotu oraz \textpi wiązania podwójnego \ch{C=O}.
Pozwala to na~wydajną delokalizację elektronów w~obrębie wiązania i~znaczny 
udział dwóch możliwych struktur dipolarnych, jak widać na~\cref{sch:resonance}.
\begin{marginscheme}
  \includesvg[graphics/]{resonance}
  \caption{
    Struktury rezonansowe wiązania amidowego, zapewniające mu~niezwykłą trwałość.
  }
  \label{sch:resonance}
\end{marginscheme}

\begin{figure}
  \centering
  \includesvg[graphics/]{carbonyl-scale-acc}
  \caption{
    Względna reaktywność róźnych grup karbonylowych
    wg skali karbonylowości Mucsiego i Chassa.
  }
  \label{fig:carbonyl-scale}
\end{figure}


\section{Prezkształcenia amidów}
Już w~drugiej połowie XIX~w.\ chemicy wiedzieli, że pierwszorzędowe amidy mogą ulegać reakcji odwodnienia pod wpływem tlenku fosforu, dając nitryle.
Amidy drugo-~i trzeciorzędowe nie ulegają takiej przemianie, ale badania nad ich reaktywnością doprowadziły Wallacha do odkrycia innej interesującej reakcji.
W~roku \citeyear{wallach77} pokazał on działanie \ch{PCl5} w~podwyższonej temperaturze na~amidy drugorzędowe\autocite{wallach77}.
To doniesienie jest nie tylko pierwszą publikacją na temat przekształceń amidów, jaką można znaleźć w~literaturze, ale również przełomową.
\citeauthor{wallach77} zauważył powstawanie \iupac{\a-dichloroamin}~(\refcmpd{w:dichloro}),
  które podczas ogrzewania łatwo ulegają przekształceniu w~\iupac{\a-chloriminy}~(\refcmpd{w:chloroimine}).
Zauważył też, że związki te wykazują znaczną elektrofilowość, bowiem wchodzą w~reakcję z~aminami, dając imidyny~(\refcmpd{w:imidine}).
Obserwacje te, przedstawione na~\cref{sch:wallach}, były podwaliną do~kolejnych odkryć w~tej dziedzinie.
\begin{scheme}
  \centering
  \includesvg[graphics/]{wallach}
  \caption{Przełomowe odkrycia Wallacha w dziedzinie chemii amidów.}
  \label{sch:wallach}
\end{scheme}
\begin{marginscheme}
  \includesvg[graphics/]{bichler}
  \caption{Ogólny schemat reakcji Bichlera-Napieralskiego.}
  \label{sch:bichler}
\end{marginscheme}
W~\citeyear{bischler93} \citeauthor{bischler93} pokazali, że działając \ch{POCl3}
na~wywiedziony z~\iupac{2-fenyloetyloaminy} amid~\refcmpd{w:bichler-sub} można otrzymać pochodną dihydroizochinoliny~\refcmpd{w:bichler-prod}\autocite{bischler93}.
Jakiś czas później wariację tej przemiany przedstawili \citeauthor{pictet10}.
Wychodząc z~\iupac{2-hydroksy-2-fenetyloamidu} otrzymali w~jednym etapie produkt już odwodniony \--- izochinolinę\autocite{pictet10}.
W~roku \citeyear{vilsmeier27} \citeauthor{vilsmeier27} pokazali, że reakcję tę można prowadzić nie tylko wewnątrzcząsteczkowo.
Działając \ch{POCl3} na~\iupac{\N,\,\N-dimetyloamid}~(\refcmpd{w:dimethylamide}) wytworzyli kation chloroiminiowy~\refcmpd{w:vilsmeier}%
  \footnote{Związek ten nazywany jest reagentem Vilsmeiera.},
  który ulega addycji do bogatych w~elektrony pierścieni aromatycznych, tworząc \iupac{\a-chloro}aminę~\refcmpd{w:vilsm-add}.
Reakcji nie da się jednak zatrzymać na~tym etapie \---
  podczas przerobu następuje hydroliza adduktu, skutkując powstaniem odpowiedniego aldehydu (lub ketonu) arylowego~(\refcmpd{w:vilsm-prod})\autocite{vilsmeier27}.
Mechanizm tego przekształcenia prezentuję na \cref{sch:vilsmeier}.
\begin{scheme}
  \centering
  \includesvg[graphics/]{vilsmeier}
  \caption{Mechanizm reakcji Vismeiera-Haacka.}
  \label{sch:vilsmeier}
  \setfloatalignment{b}
\end{scheme}

W~roku \citeyear{hofmann81}, czyli niedługo po odkryciu Wallacha, \citeauthor{hofmann81} pokazał pierwszą metodę syntezy amin z amidów\autocite{hofmann81}.
W~reakcji tej, w~wyniku działania wodorotlenkiem sodu i bromem na pierwszorzędowy amid,
  powstaje izocyjanian, który następnie hydrolizuje z uwolnieniem cząsteczki \ch{CO2}.
Powstaje pierwszorzędowa amina, która ma o jeden atom krótszy łańcuch węglowy.
Obecnie, ze względu na bezpieczeństwo oraz wygodę eksperymentatora, do przeprowadzenia tej reakcji stosuje się \gls{nbs} jako źródło bromu oraz \gls{dbu} jako zasadę.
Dopiero niemal 70~lat później pojawiła się w~literaturze wzmianka o~bardziej uniwersalnej metodzie syntezy amin z~amidów.
\citeauthor{brown48} pokazali, że możliwa jest redukcja amidowej grupy karbonylowej za pomocą glinowodorku litu\autocite{brown48}.

Redukcja przy użyciu \ch{LiAlH4} jest dzisiaj chyba najbardziej sztandarowym przykładem reaktywności amidów.
Razem z~reakcją odwodnienia, hydrolizą, przegrupowaniem Hofmanna, oraz reakcjami Vilsmeiera-Haacka i~Bischlera-Napieralskiego,
  stanowi teraz podręcznikowy kanon.
Przez długi czas przemiany te były właściwie jedynymi dostępnymi chemikom metodami modyfikacji grupy amidowej.
Wszystkie wymagały użycia agresywnych warunków,
  które z~dużym prawdopodobieństwem byłyby niekompatybilne z~innymi grupami funkcyjnymi obecnymi w~związku.
Chemia amidów była więc raczej uboga, a~wykorzystanie wiązania amidowego w~syntezie często sprowadzało się do jego obecności w~produktach.

Przełomem w~tej materii była praca, którą opublikowali \citeauthor{ghosez69} w~\citeyear{ghosez69} roku.
Przedstawili oni metodę syntezy \iupac{\a-chloroenamin}~(\refcmpd{w:ghosez-chloro}) z~trzeciorzędowych amidów~(\refcmpd{w:ghosez-sub})
  przy użyciu fosgenu i~zasady (trietyloaminy lub pirydyny)\autocite{ghosez69}.
Autorzy byli zaskoczeni łatwością z~jaką związki te ulegają nukleofilowej substytucji atomu chloru.
W~roli nukleofila przetestowali reagenty Grignarda, związki litoorganiczne, alkoholany, tiolany oraz amidki
  (oznaczone na \cref{sch:chloroenamine} ogólnie jako \ch{Nu-}),
  otrzymując produkty z~dobrymi wydajnościami (\SIrange{65}{90}{\percent}).
Publikacja ta rozpoczyna nową erę w~chemii amidów, dając początek idei \emph{aktywacji wiązania amidowego}.
\begin{scheme}
  \centering
  \includesvg[graphics/]{chloroenamine}
  \caption{Aktywacja amidu przez przekształcenie w~\iupac{\a-chloro}enaminę.}
  \label{sch:chloroenamine}
\end{scheme}
\begin{figure*}
  \centering
  \includesvg[graphics/]{timeline-twoside-simple}
  \caption{Istotne wydarzenia związane z~rozwojem idei aktywacji wiązania amidowego.}
  \label{fig:timeline}
\end{figure*}


\section{Aktywatory triflowe}
Gdy metodologia zaprezentowana przez Ghoseza i~in. została dokładniej przetestowana, okazało się że nie jest wolna od wad \---
  wyzwaniem jest wykorzystanie soli keteniminiowych typu \enquote{aldo}%
  \footnote{%
    Typu \enquote{aldo}, czyli posiadające jeden podstawnik w~pozycji \textbeta względem atomu azotu.
    Sole keteniminiowe z~dwoma podstawnikami w~tej pozycji, jak na~\cref{sch:chloroenamine}, nazywane są typem \enquote{keto}.%
  }, takich jak~\refcmpd{w:bh-dime-cl-ketenimine}.
Ze względu na zwiększoną nukleofilowość ich prekursora \--- odpowiedniej \iupac{\a-chloroenaminy}~\refcmpd{w:bh-dime-chloro} \---
  istotnym problemem jest znaczna ilość produktów ubocznych, powstających w~wyniku reakcji tych dwóch cząsteczek.
Sposób na pokonanie tej przeszkody został zaproponowany w latach 80-tych, również przez zespół Ghoseza.
Używając bezwodnika triflowego (\ch{(CF3SO2)2O}) zamiast fosgenu,
  otrzymali triflową sól keteniminy poprzez \iupac{\a-triflo}enaminę~\refcmpd{w:bh-dime-cl-ketenimine}\autocite{ghosez81}.
Związek ten jest mniej nukleofilowy i~nie wstępuje w~reakcję z~powstającym dalej triflanem keteniminiowym \refcmpd{w:bh-dime-tf-ketenimine},
  co obrazuje \cref{sch:chloro-vs-triflic}.
\begin{scheme*}
  \centering
  \includesvg[graphics/]{chloro-vs-triflic}
  \caption{Różnica w reaktywności chlorowych i triflowych pochodnych enamin z solami ketenimin. \acrshort{TfO}: \acrlong{TfO}.}
  \label{sch:chloro-vs-triflic}
\end{scheme*}

Ówcześnie szczególną wartość dostrzeżono w~reakcji formalnej [2+2] cykloaddycji, której mogą ulegać sole ketenimin.
Powstające w~jej wyniku kationy iminiowe łatwo ulegają hydrolizie, prowadząc do~powstania pochodnych cyklobutanonu\todo{dodać cytowanie}.
Metodologię tę uznano za~wartościową alternatywę cykloaddycji ketenów, zwłaszcza że sole ketenimin wykazują większą aktywność \---
  w~przeciwieństwie do tych pierwszych wstępują w~reakcje z~prostymi alkenami bez potrzeby prowadzenia procesu w~podwyższonej temperaturze\autocite{maulide18}.
Reakcję tę udało się też przeprowadzić w~wariancie wewnątrzcząsteczkowym,
  co prowadzi do powstania związków o~pierścieniach połączonych\autocite{ghosez85}, jak~\refcmpd{w:prgl-cyclo}.
Układy takie mogą być dalej funkcjonalizowane np.~poprzez ekspansję czteroczłonowego pierścienia.
Przykładem zastosowania tego podejścia w~syntezie, zilustrowanym na~\cref{sch:prostaglandin}, jest modyfikacja podejścia Coreya do prostaglandyn\autocite{chen91}.
Warto zwrócić uwagę na~stereoselektywność tego procesu \--- w~przeciwieństwie do ketenu,
  sól keteniminiowa, taka jak związek~\refcmpd{w:prgl-ketenimine}, może przenosić informację stereochemiczną poprzez chiralne podstawiniki na~atomie azotu.
\begin{scheme*}
  \centering
  \includesvg[graphics/]{prostaglandin-fullwidth}
  \caption{
    Stereokontrolowana synteza prostaglandyny z~wykorzystaniem aktywacji wiązania amidowego bezwodnikiem triflowym.
    \acrshort{tbdps}: \acrlong{tbdps}; \acrshort{mcpba}: \acrlong{mcpba}.
  }
  \label{sch:prostaglandin}
\end{scheme*}

Wnikliwy czytelnik mógł zauważyć, że właściwie wszystkie prace dotąd cytowane w~tej materii pochodzą z~grupy Ghoseza.
Szerszym zainteresowaniem metodologia oparta na~wykorzystaniu aktywatorów triflowych zaczęła się cieszyć dopiero na początku XXI~w.
Punktem zwrotnym była bardzo wnikliwa analiza mechanizmu tych przemian, przeprowadzona przez Grenona i~Charette'a.
Pokazali oni, że bezwodnik triflowy szybciej reaguje z~pirydyną, użytą jako zasada, niż z~amidem.
Powstaje wtedy triflan \iupac{\N-(trifluorometanosulfonoylo)piridyniowy}, który jest właściwym środkiem triflującym\autocite{charette01}.
Dopiero on reaguje z~amidem, tworząc triflan \iupac{\O-triflyliminiowy} (\refcmpd{w:tf-iminium-sec, w:bh-tf-tert, w:tf-iminium-tert}),
  postulowany już wcześniej przez Ghoseza jako produkt pośredni.
\citeauthor{charette01} udowodnili jednak, że związek ten reaguje natychmiast z~obecną w~mieszaninie pirydyną,
  a~dokładny przebieg tych przekształceń jest zależny od struktury amidu.
Obrazują to ze~szczegółami \cref{sch:triflic-secondary,sch:triflic-tertiary-beta,sch:triflic-tertiary-no-beta}.
\begin{scheme}
  \centering
  \includesvg[graphics/]{triflic-secondary}
  \caption{Mechanizm aktywacji drugorzędowych amidów za~pomocą bezwodnika triflowego i~pirydyny.}
  \label{sch:triflic-secondary}
\end{scheme}
\begin{scheme}
  \centering
  \includesvg[graphics/]{triflic-tertiary-no-beta}
  \caption{Mechanizm aktywacji trzeciorzędowych amidów nie posiadających protonu \textbeta za~pomocą bezwodnika triflowego i~pirydyny.}
  \label{sch:triflic-tertiary-no-beta}
\end{scheme}
\begin{scheme*}
  \centering
  \includesvg[graphics/]{triflic-tertiary-beta-fullwidth}
  \caption{Mechanizm aktywacji trzeciorzędowych amidów posiadających proton \textbeta za~pomocą bezwodnika triflowego i~pirydyny.}
  \label{sch:triflic-tertiary-beta}
\end{scheme*}

W~przypadku amidów drugorzędowych tworzy się triflan \iupac{1-pirydyloimidoilu}~\refcmpd{w:piridinium-sec}, stabilny w~warunkach reakcji,
  natomiast z~amidów trzeciorzędowych powstaje dwutriflan \iupac{1-pirydyloiminiowy}~\refcmpd{w:piridinium-tert}.
W~tym drugim przypadku, jeśli w~związku występuje proton w~pozycji \textbeta względem atomu azotu, jak w~przypadku \refcmpd{w:bh-amide-tert},
  dochodzi do przesunięcia wiązania podwójnego w~tę pozycję, skutkując powstaniem układu~\refcmpd{w:bh-piridinium-tert}.
Autorzy proponują też alternatywną ścieżkę ku wynikowemu związkowi \--- poprzez sól keteniminiową \refcmpd{w:bh-keteniminium-tert}.

Intermediaty pirydyniowe \refcmpd{w:bh-piridinium-tert, w:piridinium-sec, w:piridinium-tert}
  mogą wstępować w~dalsze reakcje z~nukleofilem, jeśli jakiś jest obecny w~środowisku reakcyjnym.
Grupa Charette'a pokazała wcześniej, że metodologia ta może być zastosowana do przekształcenia amidu w~inną grupę funkcyjną.
Przedstawili oni przykłady bezpośredniej transformacji w~tioamid, imidynę, czy tiazolinę, syntezę amidów znakowanych izotopem \ch{^{18}O},
  a~także formalną aktywację wiązania \ch{C-N} poprzez przekształcenie w~ester i~ortoester\autocite{charette01}.
Synteza tiazolin z~amidów została przez DeRoya i~Charette'a użyta w~praktyce jako jeden z~pierwszych etapów syntezy totalnej
  \iupac{($+$)-cystotiazolu~A}\autocite{deroy03}.

\subsection{Addycja \textpi-nukleofili}
Chemicy na nowo zainteresowali się też możliwościami, jakie daje addycja \textpi-nukleofili do~aktywowanych amidów.
Okazuje się, że ulegają jej nie tylko alkiny w~eksplorowanej przez Ghozesa formalnej [2+2] cykloaddycji, ale także nukleofilowe alkeny.
\citeauthor{belanger05} pokazali to syntezując różnorodne 5- i~6-członowe cykliczne związki nienasycone, w~tym o~strukturze alkaloidów,
  poprzez wewnątrzcząsteczkową reakcję addycji enamin, allilosilanów, oraz sililowych eterów enoli do~aktywowanych amidów\autocite{belanger05,belanger06}.
\Cref{sch:tashiromine} prezentuje przykład takiej przemiany, prowadzącej ostatecznie do~alkaloidu indolizydynowego o~angielskiej nazwie tashiromine (\refcmpd{w:tashiromine}).
Cyklizacja sililowego eteru enolu \refcmpd{w:tashiromine-sub} pod wpływem bezwodnika triflowego i~\acrshort{dtbmp} (\acrlong{dtbmp}) i~hydroliza
  skutkuje powstaniem \textalpha,\textbeta-nienasyconego \textbeta-aminoaldehydu \refcmpd{w:tashiromine-prod}, prekursora wspomnianego związku.
Autorzy zaprezentowali tę metodologię także jako narzędzie w~niezwykle eleganckiej syntezie trójpierścieniowych alkaloidów
  (\refcmpd{w:tricyclo-alkaloid}) ze~związków liniowych (\refcmpd{w:tricyclo-alkaloid-sub}).
Przemiana ta, widoczna na~\cref{sch:tricyclo-alkaloid} była przeprowadzona w~kaskadowym procesie, biegnącym szybko i~z~wysoką wydajnością,
  doskonale obrazując świetną chemoselektywność omawianego procesu aktywacji wiązania amidowego\autocite{belanger08}.
\begin{scheme}
  \centering
  \includesvg[graphics/]{triflic-tashiromine}
  \caption{
    Wewnątrzcząsteczkowa addycja wiązania podwójnego do amidu poprzez aktywację bezwodnikiem triflowym,
    przedstawiona na przykładzie syntezy naturalnego alkaloidu (ang. tashiromine, \refcmpd{w:tashiromine}).
  }
  \label{sch:tashiromine}
\end{scheme}
\begin{scheme*}
  \centering
  \includesvg[graphics/]{tricyclo-alkaloid}
  \caption[]{%[-3\baselineskip]{  % not compatible with draft yet
    Elegancka synteza trójcyklicznego alkaloidu, którą zaprezentowali \citeauthor{belanger06}.
    Przykład ten doskonale obrazuje chemoselektywność metody względem amidowej grupy karbonylowej.
  }
  \label{sch:tricyclo-alkaloid}
\end{scheme*}

Metodologię tę wykorzystano później w~grupie Dixona jako kluczowy, finalny etap w~syntezie totalnej \iupac{($-$)-nakadomarinu~A},
  alkaloidu wydzielonego z~gąbki z~rodzaju \textit{Amphimedon}\autocite{dixon11}.
Początkowo chcieli oni zsyntezować ten związek kończąc ścieżkę syntetyczną zamknięciem największego, 15-członowego pierścienia
  poprzez metatezę alkinów (transformacja \refcmpd{w:nakadomarin-wrong} do \refcmpd{w:nakadomarin}), okazało się to jednak niemożliwe.
Ostatecznie krok ten przeprowadzili jako jeden z~wcześniejszych etapów,
  a~finalny produkt otrzymali prowadząc wewnątrzcząsteczkową addycję furanu do aktywowanego 5-członowego laktamu.
Redukcja powstałej soli iminiowej \refcmpd{w:nakadomarin-iminium} dała oczekiwany alkaloid \refcmpd{w:nakadomarin}.
\begin{scheme}
  \centering
  \includesvg[graphics/]{nakadomarin}
  \caption{
    Wykorzystanie aktywacji amidu bezwodnikiem tryflowym w~syntezie totalnej nakadomarinu~A (\refcmpd{w:nakadomarin}),
      alkaloidu wydzielonego z~gąbki z~rodzaju \textit{Amphimedon}.
    \acrshort{dtbp}: \acrlong{dtbp}.
  }
  \label{sch:dixon-alkaloid}
\end{scheme}

\citeauthor{movassaghi06a} zaadaptowali metodę Ghoseza do syntezy pochodnych pirydyny z~\iupac{\N-winylowych}%
  \footnote{%
    Także \iupac{\N-arylowych}, z~których otrzymuje się układ pierścieni połączonych.%
  } drugorzędowych amidów.
Pierwsza zaproponowana przez nich procedura była dwuetapowa \--- najpierw funkcjonalizowali amid acetylenkiem miedzi,
  żeby następnie przeprowadzić cyklizację katalizowaną kompleksem rutenu\autocite{movassaghi06a, movassaghi07syn}.
Wiemy już jednak, że bogate w~elektrony alkeny i~enole sililowe są wystarczająco nukleofilowe, by wchodzić w~reakcję z~triflanami \iupac{1-pirydyloimidoilu},
  bez konieczności generowania związków miedzioorganicznych.%
  \footnote{Możliwe, że \citeauthor{movassaghi07} odkryli to niezależnie, bowiem nie powołują się w~swojej kolejnej pracy na~przytoczone wyżej publikacje.}
Niedługo później autorzy zaprezentowali opartą o~ten fakt bezpośrednią syntezę pochodnych pirydyny, przedstawioną na~\cref{sch:pyridine-synthesis}.
Addycja \textpi-nukleofili (jak np.~\refcmpd{w:allyl}) do~aktywowanych amidów prowadzi do powstania wysoce reaktywnych intermediatów \refcmpd{w:pyridine-int},
  które ulegają samoczynnej cyklizacji i~aromatyzacji, tworząc od razu pochodne pirydyny \refcmpd{w:pyridine} \--- w~jednym kroku syntetycznym\autocite{movassaghi07}.
W~analogicznym procesie, wykorzystując nitryle w~roli nukleofila, można otrzymać pochodne pirymidyny\autocite{movassaghi06b}.
W~każdym z~tych przypadków kluczowym jest użycie \iupac{2-chloropirydyny} jako zasady do~aktywacji z~bezwodnikiem tryflowym.
Autorzy nie wdają się w~dyskusję nad przyczyną konieczności stosowania słabszej zasady,
  ale pokazali, że stosowanie innych zasad prowadzi do~znacznie niższych wydajności\autocite{movassaghi06b}.
\begin{scheme}
  \centering
  \includesvg[graphics/]{pyridine}
  \caption{
    Synteza pirydyn wykorzystująca addycję alkenów i~alkinów do aktywowanych amidów.
    Górna ścieżka przedstawia pierwszą, dwuetapową metodę; dolna \--- nowszą.
    \acrshort{sphos}: \acrlong{sphos}.
  }
  \label{sch:pyridine-synthesis}
\end{scheme}

Tematyka syntezy azotowych heterocykli tą metodą została jeszcze wzbogacona przez Wanga i~in.,
  którzy użyli diazooctanu etylu \refcmpd{w:diazoacetate-et} jako nukleofila w~reakcji z~\iupac{\N-arylowymi}
    amidami \refcmpd{w:n-aryl-amide} w~nowej syntezie pochodnych indolu\autocite{wang08}.
Co ciekawe, do wydajnego przebiegu reakcji potrzebne było dodanie niewielkiej ilości (\SI{0.2}{\equiv}) \iupac{2,6-dichloropirydyny},
  która dodatkowo zwiększa reaktywność generowanej soli iminiowej poprzez tworzenie układu \refcmpd{w:n-aryl-diclpy} (\cref{sch:indole-synthesis}).
Po addycji diazooctanu następuje ekstruzja cząsteczki azotu, cyklizacja układu, jego rearomatyzacja,
  i~w~końcu powstanie pochodnej indolu \refcmpd{w:indole} poprzez 1-3 przeniesienie protonu.
Niedawno opublikowane prace rozszerzają wachlarz podobnych przekształceń \iupac{\N-arylowych} amidów o~syntezę różnorodnych
  chinolin\autocite{wezeman16,liang17} i~\iupac{2,3-dihydrochinolin}\autocite{huang19} w~reakcji odpowiednio z~alkinami i~alkenami.
\begin{scheme}
  \centering
  \includesvg[graphics/]{indole}
  \caption{Synteza indoli wykorzystująca diazooctan etylu w~roli nukleofila z~dodatkową aktywacją za~pomocą \iupac{2,6-dichloropirydyny}.}
  \label{sch:indole-synthesis}
\end{scheme}

Zamiast pozwalać na~samoistne przekształcenia produktów pośrednich, jak w~przypadkach opisywanych powyżej,
  reakcje addycji \textpi-nukleofili do~aktywowanych amidów można również gasić reduktywnie, co prowadzi do~powstania funkcjonalizowanych amin.
\citeauthor{belanger15} wykorzystali tę technikę opracowując analogiczną do~reakcji Mannicha
  metodę syntezy \textbeta-aminoestrów \refcmpd{w:b-amino-esther}\autocite{belanger15}.
Używają w~niej aktywowanych amidów jako odpowiedników imin i~sililowanych acetali ketenów w~roli nukleofila.
Powstające produkty pośrednie \refcmpd{w:b-iminium-esther} poddają redukcji za~pomocą \ch{NaBH4},
  w~przypadku amidów innych niż formylowe (\ch{R^3} $\neq$ \ch{H} na \cref{sch:branched-beta-aminoesthers})
  zakwaszając środowisko kwasem octowym, aby uniknąć niepożądanej reakcji retro-Mannicha.
W~rezultacie otrzymują rozmaite \textbeta-aminoestry \refcmpd{w:b-amino-esther}, które można łatwo przekształcić w~nienaturalne aminokwasy\autocite{belanger15}.
Należy zaznaczyć, że pochodne o~takim stopniu rozgałęzienia łańcucha nie jest łatwo otrzymać w~reakcji Mannicha,
  czy za~pomocą innych klasycznych metod syntezy związków tego typu.
\begin{scheme*}
  \centering
  \includesvg[graphics/]{branched-aminoesthers}
  \caption{
    Synteza rozgałęzionych \textbeta-aminoestrów, niedostępnych przy użyciu reakcji Mannicha i~innych klasycznych metod.
    \acrshort{tbs}: \acrlong{tbs}.
  }
  \label{sch:branched-beta-aminoesthers}
\end{scheme*}

Podobna reakcja z~wykorzystaniem trzeciorzędowej enaminy \refcmpd{w:enamine-pyrr} jako nukleofila również jest możliwa do przeprowadzenia,
  ale przebiega w~dość zaskakujący sposób.
Redukcja powstającego intermediatu \refcmpd{w:tf-diiminium-pyrr} prowadzi nie do diaminy,
  jak można by się spodziewać, ale do alliloaminy \refcmpd{w:allyl-amine}\autocite{huang18}.
Autorzy opisujący tę przemianię z~powodzeniem wykorzystują enaminy i~drugorzędowe amidy posiadające różne grupy funkcyjne \---
  karbonylową, estrową, czy allilową \--- kolejny raz dowodząc znacznej chemoselektywności procedury aktywacji opartej na~użyciu bezwodnika triflowego.
Nie dyskutują jednak, czy możliwe jest użycie substratów o~innej rzędowości.
\begin{scheme*}
  \centering
  \includesvg[graphics/]{triflic-nu-enamine}
  \caption{
    Niespodziewany przebieg redukcji soli iminoiminiowej \refcmpd{w:tf-diiminium-pyrr} borowodorkiem sodu,
    prowadzący do powstania alliloaminy.
  }
  \label{sch:allyloamine-synthesis}
\end{scheme*}

\subsection{Inne C-nukleofile}
Starając się dowieść wyższości metody wykorzystującej bezwodnik triflowy nad przemianami biegnącymi
  poprzez chloroenaminę nie mogę pominąć kwestii różnorodności możliwych do zastosowania nukleofili.
\citeauthor{ghosez69} we~wspomnianej wcześniej przełomowej pracy\autocite{ghosez69} pokazali użycie rozmaitych związków w~tej roli.
Analogi triflowe nie ustępują chlorowym pierwowzorom w~tej materii,
  chociaż chemikom przyszło się o~tym przekonać dopiero stosunkowo niedawno.
Dekadę przed powstaniem niniejszej dysertacji \citeauthor{xiao10} pokazali,
  że możliwa jest addycja odczynnika Grignarda do aktywowanego za~pomocą \ch{Tf2O} wiązania amidowego\autocite{xiao10}.
Z~amidów trzeciorzędowych \refcmpd{w:amide-tert} powstają w~tej sekwencji przemian kationy iminiowe
  \refcmpd{w:iminium-keto-tert} zdolne do przyłączenia kolejnej cząsteczki nukleofila,
  prowadząc finalnie do \textalpha,\textalpha-dwupodstawionych amin \refcmpd{w:amine-tert-same}.

Aktywność kationu iminiowego \refcmpd{w:iminium-keto-tert} jest jednak niższa niż
  \iupac{\O-triflyliminiowego} \refcmpd{w:tf-iminium-tert} i~autorzy wykorzystują ten fakt,
  by zróżnicować strukturę otrzymywanych produktów.
Pokazują, że stosując równomolową ilość odczynnika Grignarda i~manipulując temperaturą prowadzenia reakcji można przeprowadzić
  ten proces dwuetapowo, dodając inny nukleofil w~drugim etapie i~otrzymując aminę z~dwojgiem różnych podstawników w~pozycji \textalpha.
Ów drugi nukleofil nie musi być już związkiem magnezoorganicznym \---
  \citeauthor{xiao10} użyli też związków litoorganicznych i~enolanów prezentując tę koncepcję\autocite{xiao10}.
Co więcej, pokazali, że można też zaadaptować ideę reduktywnego gaszenia reakcji, wspomnianą już wcześniej\autocite{belanger15}.
Stosując borowodorek sodu lub glinowodorek litu jako drugi nukleofil można otrzymać aminy monopodstawione \refcmpd{w:amine-tert-mono}\autocite{xiao10eurj}.
Później zauważyli też, że powstające po addycji pierwszej cząsteczki nukleofila sole iminiowe \refcmpd{w:iminium-keto-tert}
  można po prostu poddać kwasowej hydrolizie, co prowadzi do powstania odpowiednich ketonu i~drugorzędowej aminy\autocite{huang15tet}.
  \begin{scheme}
    \centering
    \includesvg[graphics/]{huang-tert}
    \caption{
      Różne, pokazane przez zespół Huanga, możliwości funkcjonalizacji amidów trzeciorzędowych
      poprzez aktywację bezwodnikiem triflowym: wyczerpujące alkilowanie,
      sekwencyjna difunkcjonalizacja, reduktywne monoalkilowanie, hydroliza do~ketonu i~aminy.
    }
    \label{sch:huang-tert}
    \setfloatalignment{b}
  \end{scheme}

Te pierwsze eksperymenty grupy Huanga obejmowały aktywację jedynie amidów trzeciorzędowych,
  jadnak niebawem udowodnili oni, że i~drugorzędowe amidy \refcmpd{w:amide-sec} ulegają takim przemianom.
Konieczny do ich przeprowadzenia jest dodatek chlorku cezu~(III), w~jego obecności ze~związków lito- i~magnezoorganicznych
  powstają \latin{in situ} związki cezoorganiczne.
Dopiero te ulegają addycji do aktywowanego amidu drugorzędowego,
  a~powstającą iminę \refcmpd{w:imine-keto} można zredukować lub poddać działaniu kolejnego nukleofila\autocite{xiao12}.
Procedurę syntezy \textalpha-monopodstawionych drugorzędowych amin udało się później usprawnić odwracając kolejność działań.
Przeprowadzając kontrolowaną redukcję aktywowanego wiązania amidowego za~pomocą \ch{Et3SiH},
  otrzymuje się iminę \refcmpd{w:imine}, którą można poddawać dalszym przemianom\autocite{huang15joc}.
Autorzy pokazali szerszy zakres kompatybilnych nukleofili, w~tym silanów, enolanów, i~związków cynoorganicznych,
  których użycie w~pierwotnej procedurze nie jest możliwe.
Być może nawet ważniejszą poprawą jest zwiększenie chemoselektywności \--- wcześniej nie powiodła się funkcjonalizacja
  związków zawierających grupę nitrylową oraz grupę estrową, a dzięki wykorzystaniu usprawnionej metodologii
  było to możliwe\autocite{huang15joc}.
W~końcu autorzy sugerują, że w~przypadku użycia chiralnych substratów metodologie te powinny być komplementarne
  w~kwestii stereochemii powstającego nowego wiązania \ch{C-C}, jednak jak dotąd nie potwierdzili tej tezy.
  \begin{scheme}
    \centering
    \includesvg[graphics/]{sec-mono}
    \caption{
      Dwie ścieżki monofunkcjonalizacji drugorzędowych amidów zaprezentowane przez zespół Huanga,
      przedstawione na~przykładzie związku Grignarda jako nukleofila.
    }
    \label{sch:huang-sec-mono}
  \end{scheme}

Ostatnie prace pochodzące grupy Huanga prezentują bardziej złożone przemiany amidów.
Pokazują na~przykład, że możliwe jest auto-sprzęganie kationów iminiowych \refcmpd{w:iminium-h} otrzymanych na~drodze
  aktywacji-redukcji drugorzędowych amidów \refcmpd{w:amide-sec} dzięki użyciu jodku samaru (II).
Zapewniając lekko zasadowe środowisko poprzez dodatek trietyloaminy można też przeprowadzić
  sprzęganie krzyżowe imin \refcmpd{w:imine} z~ketonami.
Przemiany te pozwalają otrzymać odpowiednio \iupac{1,2-diaminy} \refcmpd{w:b-diamine}
  i~\iupac{1,2-aminoalkohole} \refcmpd{w:b-aminohydroxyl}\autocite{huang15comm}.
Tę samą metodę generowania imin wykorzystano do przeprowadzenia wieloskładnikowej reakcji Ugiego\autocite{zheng15}.
W~standardowej wersji tej reakcji imina \refcmpd{w:imine} generowana jest również \latin{in situ},
  ale w~wyniku kondensacji aminy i~ketonu.
Reakcja Ugiego jest przemianą bardzo interesującą, choćby ze~względu na~wysoką ekonomię atomową\footnote{%
  Ekonomia atomowa to pojęcie wchodzące w~skład koncepcji zielonej chemii,
  rozumie się przez nie dążenie do maksymalizacji udziału substratów w~produkcie.
}.
Wrócę do niej jeszcze niejednokrotnie, gdyż istotna część pracy eksperymentalnej
  zrealizowanej w~ramach niniejszej dysertacji wykorzystuje jeden z~jej wariantów.

\begin{scheme*}
  \centering
  \includesvg[graphics/]{ugi-diamine}
  \caption{
    Zaprezentowane przez zespół Huanga przekształcenia drugorzędowych amidów biegnące poprzez iminę: sprzęganie i~reakcja Ugiego.
  }
  \label{sch:huang-ugi-diamine}
\end{scheme*}

\todo[inline]{
  czy dodać \\
  trifluorometylowanie \textbf{chen18}: Chen et.al., Org. Chem. Frontiers, 2018, 5, 943 \\
  dialkinowanie \textbf{chen19joc}: Chen et.al., J. Org. Chem., 2019, 84, 9270 \\
  cyanjanowanie-fosforylacja \textbf{chen19}: Chen et.al., Org. Lett., 2019, 21, 3808 \\
}

\subsection{Łagodna redukcja}
Nie można pominąć doniesienia grupy Huanga o~opracowaniu nowej metody redukcji amidów do amin.
Wysoka reaktywność amidów aktywowanych bezwodnikiem triflowym pozwala na~użycie mniej nukleofilowego donora anionu wodorowego oraz
przeprowadzenie przemiany w~łagodniejszych warunkach niż w~przypadku klasycznej redukcji glinowodorkiem litu.
W~pierwszym podejściu do tego zagadnienia opracowali wykorzystanie \ch{NaBH4} jako reduktora.
Redukcja ta przebiega w~pokojowej temperaturze, ale
  kluczowym dla powodzenia procesu jest użycie do~etapu redukcji \acrshort{thf} (tetrahydrofuranu) jako rozpuszczalnika.
\citeauthor{xiang10} pokazali, że można w~ten sposób prowadzić redukcję amidów drugo- jak i trzeciorzędowych,
  ale selektywność metody nie jest idealna \--- estry również są przy jej zastosowaniu redukowane\autocite{xiang10}.

Lepsze wyniki osiągnęli \citeauthor{huang16b} kiedy udało im się połączyć\autocite{huang16b} opracowaną dotąd metodologię 
  z~podejściem wykorzystującym \gls{tmds} oraz katalityczną ilość \ch{B(C6F5)3}\autocite{tan09, chadwick14, blondiaux14}.
Drugi z~wymienionych jest silnym kwasem Lewisa; postuluje się, że zwiększa on aktywność \gls{tmds},
  tworząc separowaną parę jonową\footnote{%
    Ang. \emph{frustrated Lewis pair}. To bardzo ciekawy temat,
    wykracza jednak poza zakres zagadnień niezbędnych do omówienia w~ramach niniejszej dysertacji.
    Zainteresowanym proponuję zapoznać się z~odnośną literaturą, np. \cite{stephan15}.%
  }.
Opracowana metoda jest niezwykle łagodna i~selektywna \--- pozwala na~redukcję wiązania amidowego
  w~obecności wiązań wielokrotnych, estrów, nitrylu, grupy nitrowej, czy eterów sililowych.
Na podstawie zakresu przykładów przedstawionych w~cytowanej pracy można jednak wnioskować,
  że metoda ta jest ograniczona do liniowych amidów drugorzędowych\footnote{%
    Warto wspomnieć, że w~czasie ukazania się omawianej publikacji znane były też już inne metody łagodnej redukcji amidów.
    Więcej o~nich opowiem w~dalszej części tego przeglądu literatury.
  }.
\begin{scheme}[b]
  \centering
  \includesvg[graphics/]{huang-reduction}
  \caption[][3\baselineskip]{
    Porównanie opracowanych przez zespół Huanga metod redukcji amidowej grupy karbonylowej.
  }
  \label{sch:huang-reduction}
  \setfloatalignment{b}
\end{scheme}

Alternatywną, dwuetapową procedurę redukcji drugorzędowych amidów \refcmpd{w:amide-sec} zaproponował zespół Charette'a.
Polega ona na~hydrosililowaniu aktywowanego amidu za~pomocą \ch{Et3SiH}, co prowadzi do powstania iminy \refcmpd{w:imine}.
Dodając w~następnym kroku \gls{heh} otrzymuje się produkt dalszej redukcji \--- aminę \refcmpd{w:amine-sec}.
Metoda ta jest bardzo selektywna, autorzy pokazują przykłady jej wydajnego zastosowania w~obecności nawet szerszego wachlarza
  grup funkcyjnych niż w~przypadku procedury zaproponowanej przez Huanga.
Dodatkową zaletą jest możliwość zatrzymania reakcji na~etapie iminy albo przeprowadzenia hydrolizy do aldehydu.
W~takim wypadku tolerowana jest nawet obecność innej grupy aldehydowej w~cząsteczce\autocite{charette10}.
Redukcja amidów trzeciorzędowych \refcmpd{w:amide-tert} jest nawet prostsza \--- po ich aktywacji za~pomocą \ch{Tf2O},
  \gls{heh} redukuje je bezpośrednio do~amin\autocite{barbe08}.
\begin{scheme}
  \centering
  \includesvg[graphics/]{charette-reduction}
  \caption{
    Redukcja drugo- i~trzeciorzędowych amidów za~pomocą \gls{heh} według metody Charette'a.
  }
  \label{sch:charette-reduction}
  \setfloatalignment{b}
\end{scheme}

\todo[inline]{
  może dodać przegrupowania od Maulide? \\
  \textbf{bauer20}: Bauer et.al., Eur. J. Org. Chem., 2020, 3971 \\
  \textbf{madelaine10}: Madelaine et.al., Angew. Chem., Int. Ed., 2010, 49, 1583 \\
  \textbf{padmanaban15}: Padmanaban et.al., Tetrahedron, 2015, 71, 5994 \\
  \textbf{peng12}: Peng et.al., Chem. - Eur. J., 2012, 18, 16292 \\
  \textbf{spinozzi19}: Spinozzi et.al., Eur. J. Org. Chem., 2019, 5230 \\
}

\section{Odczynnik Schwartza}
\subsection{Hyrdocyrkonowanie}
\subsection{Selektywność wobec amidów}
\subsection{Z własnego podwórka}

\section{Kalityczna aktywacja kompleksami metali}
\subsection{Kompleks Vaski}
\subsection{Kompleks van~der~Enta}
\subsection{Heksakarbonylek molibdenu}

\section{Struktura a~reaktywność}
\subsection{Amidy Weinreba}
\subsection{Rezonans zakłócony geometrią}

\section{Nowe perspektywy}
\subsection{Redukcja izopropoksytytanem}
\subsection{Redukcja wodorkiem sodu}

\section{Aktywacja amidów alternatywą względem imin}

\section{Więcej niż deoksygenacja}
\subsection{Aktywacja wiązania \ch{C-N}}
\subsection{Umpolung amidów}

\section{Tetrazole}
\subsection{Bioizosteryzm}
\subsection{Metody syntezy tetrazoli}
