\chapter{Stan wiedzy}\label{chapter:literature}

\subimport{./}{intro}
\subimport{./}{triflic}
\subimport{./}{schwartz}

\section{Kalityczna aktywacja kompleksami metali}\label{literature:catalitic}
\subsection{Kompleks Vaski}\label{literature:catalitic:vasca}
\subsection{Kompleks van~der~Enta}\label{literature:catalitic:van-der-ent}
\subsection{Heksakarbonylek molibdenu}\label{literature:catalitic:molydenium}

\section{Struktura a~reaktywność}\label{literature:structure}
\subsection{Amidy Weinreba}\label{literature:structure:weinreb}
\subsection{Rezonans zakłócony geometrią}\label{literature:structure:geometry}

\section{Nowe perspektywy}\label{literature:perspectives}
\subsection{Redukcja izopropoksytytanem}\label{literature:perspectives:titanium}
\subsection{Redukcja wodorkiem sodu}\label{literature:perspectives:sodium-hydride}

\section{Więcej niż deoksygenacja}\label{literature:other}
\subsection{Aktywacja wiązania \ch{C-N}}\label{literature:other:c-n}
\subsection{Umpolung amidów}\label{literature:other:umpolung}

\section{Tetrazole}\label{literature:tetrazole}
\todo[inline]{Przenieść do części eksperymentalnej}
\subsection{Bioizosteryzm}\label{literature:tetrazole:bioisosterizm}
\subsection{Metody syntezy tetrazoli}\label{literature:tetrazole:synthesis}
