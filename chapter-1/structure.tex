\section{Struktura a~reaktywność}\label{literature:structure}
We wstępie do~niniejszej dysertacji wspomniałem o~wyjątkowej trwałości wiązania amidowego.
Dziś wiadomo, że wynika ona z~wydajnego rezonansu w~obrębie wiązania, zobrazowanego
  na~\cref{sch:resonance} w~sekcji~\secref{literature:amide-stability}.
Takie wyjaśnienie proponował już Pauling w~\textit{Naturze Wiązania Chemicznego},
  pracy fundamentalnej dla dzisiejszego rozumienia chemii.
Na~przykładzie amidu wyjaśnia jak określić co może, a~co nie może być uznane za~strukturę
  rezonansową danego związku; oblicza też wartość energii rezonansu w~obrębie wiązania amidowego,
  wynoszącą \SI{21}{\kcalpm}\sidecite[-3.5\baselineskip][str.~\numrange{192}{197}]{pauling60}.
  
A~jednak ta sprawa przez długi czas była kwestią sporną wśród chemików.
Sporo kontrowersji wzbudziła praca Wiberga~i~in.\sidecite[-3\baselineskip]{wiberg87},
  podważająca słuszność modelu rezonansowego.
W~oparciu o~teorię atomów w~cząsteczkach\sidenote[][-2\baselineskip]{
    Ang. \textit{atoms in molecules}, jeden z~modeli pozwalających (między innymi) na~obliczenie
      wartości cząstkowych ładunków na~poszczególnych atomach w~cząsteczce, \colorcite{bader91}.
  }
  Badera badacze ci zasugerowali, że rozkład elektronów w~obrębie wiązania amidowego jest inny,
  niż wynikałoby z~dużego udziału dwujonowej struktury rezonansowej.
Inne metody analizy ładunku nie potwierdziły tego twierdzenia\sidecite{wiberg92a}.

Naukowcy zgłębiali kwestię występowania bariery rotacji wokół wiązania \ch{C-N} w~ugrupowaniu
  amidowym, która wynosi ok. \SI{18}{\kcalpm} w~przypadku formamidu\sidecite{schneider60}.
Istotnym zarzutem wobec teorii o~rezonansie była niejednakowa zmiana długości wiązań \ch{C=O}
  i~\ch{C-N} podczas tej rotacji, przedstawionej na~\cref{sch:rotation}.
W~stanie przejściowym \textbf{TS} wiązanie \ch{C-N} jest o~około \SI{0.08}{\angstrom} dłuższe niż
  w~stanie stacjonarnym \textbf{GS}, a~wiązanie \ch{C=O} tylko o~około \SI{0.01}{\angstrom} krótsze.
Podobnej dysproporcji podlega różnica wartości ładunków na~atomie azotu i~tlenu.
Alternatywną teorią, tłumaczącą te rozbieżności, był model według którego jedynym wkładem
  atomu tlenu w~utrudnioną rotację jest polaryzacja wiązania \ch{C=O},
  a~stabilizacja wiązania \ch{C-N} następuje poprzez przesunięcie gęstości wolnej pary elektronowej
  z~atomu azotu w~kierunku dodatnio naładowanego atomu węgla\sidecite[1\baselineskip]{wiberg92b}.
\begin{marginscheme}[-24\baselineskip]
  \includesvg{rotation}
  \caption{
    Dysproporcja zmian długości wiązań \ch{C=O} i~\ch{C-N} między stanem stacjonarnym
      a~stanem przejściowym podczas rotacji wokół wiązania \ch{C-N} w~formamidzie.
  }
  \label{sch:rotation}
\end{marginscheme}

Model polaryzacyjny spotkał się z~bardzo mieszanym odbiorem i~wywołał lawinę publikacji
  poświęconych temu tematowi \--- niektórzy naukowcy przychylali się do~nowej hipotezy,
  inni bronili teorii o~rezonansie, albo szukali pośrednich możliwości.
Następne prace, uwzględniające zaktualizowaną teorię rezonansu chemicznego\sidecite{lauvergnat97},
  teorię wiązań walencyjnych\sidecite{glendening97}, a~także analizę ładunków atomowych
  Mulliklena\sidecite{geza97} przekonały społeczność naukowców, że to rezonans chemiczny
  odpowiada za~badane zjawisko.
Dopiero dekadę później \citeauthor{kemnitz07} w~pełni rozwiali wątpliwości przytoczone
  w~poprzednim akapicie.
Porównali oni długości wiązań w~strukturach analogicznych do~postulowanych form rezonansowych
  amidu i~były one w~pełni spójne z~przyjętym modelem\sidecite{kemnitz07}.
Zjawisko, w~którym podczas rotacji wiązanie \ch{C-N} wydłuża się znacznie bardziej niż
  wiązanie \ch{C=O} się skraca, choć nieintuicyjne, okazuje się oczekiwane. 

\subsection{Rezonans zakłócony geometrią}\label{literature:structure:geometry}
Wiązanie amidowe stabilizowane rezonansem jest płaskie \--- wszystkie wiązania w~jego obrębie
  leżą w~jednej płaszczyźnie.
W~niektórych przypadkach struktura pozostałej części cząsteczki uniemożliwia przyjęcie takiej
  konformacji.
Bezpośrednich źródeł tego zniekształcenia, jak obrazuje \cref{fig:non-planar} może być kilka:
  naprężenia w~układach mostkowych, zawada przestrzenna, efekty konformacyjne,
  czy efekty elektronowe podstawników.
Niezależnie od~przyczyny, dochodzi wtedy do~zaburzenia rezonansu w~obrębie wiązania amidowego,
  a~tym samym do~jego gorszej stabilizacji.
\begin{figure}
  \includesvg{non-planar}
  \caption{
    Możliwe przyczyny wymuszenia niepłaskiej konformacji wiązania amidowego,
      zaburzające rezonans elektronowy w~jego obrębie.
  }
  \label{fig:non-planar}
\end{figure}

Takie niepłaskie amidy, nazywane często amidami skręconymi, wykazują reaktywność bardziej
  zbliżoną do~typowych estrów lub ketonów.
Ulegają redukcji, addycji nukleofila do~wiązania \ch{C=O}, a~także \iupac{\N-metylowaniu}.
Wiązanie \ch{C-N} może zostać łatwo rozerwane w~reakcji hydrolizy, transamidowania,
  czy w~różnych reakcjach sprzęgania krzyżowego.
Szostak~i~in. opublikowali kilka artykułów przeglądowych\sidecite[-6\baselineskip]{szostak13, liu17, li20},
  wyczerpująco omawiających chemię tych związków.
\Cref{sch:twisted} przedstawia przykłady niektórych ze wspomnianych przemian\sidecite{kirby01}.
\begin{scheme}
  \includesvg{twisted}
  \caption{
    Przykładowe przemiany jednego z~najbardziej zniekształconych amidów \---
      \iupac{1-azaadamantan-2-onu}~\refcmpd{adamantan-amide}.
    Związek ten ulega przemianom typowym dla karbonyli \--- tworzeniu
      acetalu~\refcmpd{adamantan-acetal} czy reakcji Wittiga,
      dając enaminę~\refcmpd{adamantan-enamine}.
    Ponadto łatwo hydrolizuje do~dwujonowej formy \refcmpd{adamantan-zwiterion}
      i~ulega metylowaniu do~czwartorzędowej aminy~\refcmpd{adamantan-methylated}.
    \acrshort{ts}:~\acrlong{ts}; \acrshort{sm}:~\acrlong{sm}.
  }
  \label{sch:twisted}
  \setfloatalignment{b}
\end{scheme}

Niezwykle ciekawą pracę opublikowali niedawno \citeauthor{takezawa20}.
Zamykając prosty amid w~samoorganizującej się klatce koordynacyjnej, zbyt ciasnej by amid przyjął
  najbardziej korzystną konformację, wymusili jego skręcenie w~osi wiązania
  \ch{C-N}\sidecite{takezawa20}.
Tak aktywowany amid z~łatwością poddano hydrolizie w~łagodnych warunkach.
Jest to pierwszy przykład \textit{stricte} mechanicznego zakłócenia rezonansu w~obrębie wiązania
  amidowego, choć, jak zauważają sami autorzy, wcześniej udało się osiągnąć podobny efekt z~pomocą
  oddziaływań koordynacyjnych\sidecite{takezawa20}.
Cytowana publikacja zawiera również bardzo obrazowe wizualizacje i~objaśnienia tego procesu,
  zainteresowanego czytelnika odsyłam do~źródła.

Idea zakłócenia rezonansu geometrią wiązania amidowego ma duże znaczenie w~chemii medycznej.
Już w~roku \citeyear{woodward80} \citeauthor{woodward80} zauważył zależność między konformacją
  pierścienia \textbeta{}-laktamowego a~aktywnością biologiczną karbapenemów.
Jak pisze w~swoim eseju\sidecite{woodward80}, wydawać się może, że struktura podstawników
  nie ma wielkiego wpływu na~ten parametr, ale grupa badaczy z~firmy Merck doniosła o~drastycznej
  różnicy w~aktywności tienamycyny~\refcmpd{thienamycin} i~jej
  diastereomeru~\refcmpd{thienamycin-S}, widocznych na~\cref{fig:penems}.
Podobną zależność \citeauthor{woodward80} dostrzegł w~przypadku analogicznych
  penemów~\refcmpd{penem-r, penem-s}.
\begin{figure}
  \includesvg{penems}
  \caption{
    Pary diastereomerycznych penemów, na~które uwagę zwrócił \citeauthor{woodward80}.
    W~każdej parze jeden ze~związków (\refcmpd{thienamycin-S, penem-s})
      wykazuje diametralnie niższą aktywność biologiczną.
  }
  \label{fig:penems}
\end{figure}

\begin{marginfigure}[15\baselineskip]
  \includesvg{pyramidization}
  \caption{
    Wartość \enquote{piramidyzacji} atomu azotu w~wiązaniu amidowym określa się poprzez wysokość $h$
      atomu azotu nad płaszczyzną wyznaczoną przez związanie z~nim atomy.
  }
  \label{fig:pyramidization}
\end{marginfigure}
W~wyniku studiów struktury przestrzennej tych związków zwrócił uwagę, że w~tych aktywniejszych
  atom azotu i~związane z~nim podstawniki przyjmują bardziej kształt piramidy niż płaskiego
  trójkąta.
Im wiązanie amidowe przyjmuje mniej płaską, a~bardziej piramidalną konformację, tym większa
  jest też jego aktywność chemiczna wiązania amidowego i~łatwość otwierania pierścienia
  \textbeta{}-laktamu.
Woodward zaproponował, że aktywność biologiczna jest również związana z~\enquote{piramidyzacją} \---
  przy zbyt dobrej stabilizacji bioaktywność nie występuje w~ogóle, później rośnie razem
  z~aktywnością chemiczną, a~gdy wiązanie amidowe staje się zbyt labilne, związek staje się
  zbyt nietrwały, by wykorzystać go w~farmakologii.
Przeprowadzona później analiza znanych ówcześnie penemów dostępnych w~\gls{csd} pokazała, że
  znaczna większość struktur wykazujących aktywność biologiczną ma wartość parametru 
  $h$~(patrz~\cref{fig:pyramidization}) w~zakresie
  \SIrange{0.35}{0.45}{\angstrom}\sidecite{nangia96}, co może potwierdzać tę teorię.

\subsection{Amidy Weinreba}\label{literature:structure:weinreb}
Silnie elektronodonorowy lub elektronoakceptorowy charakter podstawników na~atomie azotu
  również przyczynia się do~zaburzenia stabilizacji wiązania amidowego.
Najbardziej znanym przykładem zastosowania takich amidów jest prawdopodobnie synteza ketonów
  zaproponowana przez Weinreba i~Nahma\sidecite{nahm81}, zilustrowana przeze mnie
  na~\cref{sch:weinreb}.
\iupac{\N-metoksy-\N-metylo} amid~\refcmpd{weinreb-amide}, nazywany zwyczajowo amidem Weinreba,
  łatwo ulega addycji odczynnika metaloorganicznego, tworząc kompleks~\refcmpd{weinreb-complex},
  który z~kolei hydrolizuje do~ketonu~\refcmpd{ketone}.
Hydroliza następuje dopiero podczas terminacji reakcji, nie ma zatem niebezpieczeństwa przyłączenia
  kolejnej cząsteczki nukleofila, jak ma to miejsce w~przypadku estrów czy chlorku
  kwasowego~\refcmpd{acid-chloride}.
\begin{scheme}
  \includesvg{weinreb}
  \caption{
    Synteza ketonów zaproponowana przez Weinreba i~Nahma, wykorzystująca destablilzację wiązania
      amidowego pod wpływem elektronodonorowej grupy \ch{-OMe}.
  }
  \label{sch:weinreb}
\end{scheme}

Amidy Weinreba szybko zostały dostrzeżone przez chemików jako wartościowe narzędzie
  syntetyczne\sidecite{mentzel97}.
Znalazły one zastosowanie w~chemii związków heterocyklicznych, syntezie totalnej, a~nawet
  z~powodzeniem zostały zaadaptowane w~wielkoskalowych syntezach przemysłowych\sidecite{aidhen08}.
Powstały liczne wariacje oryginalnej procedury, na~przykład wykorzystujące
  odczynniki Wittiga zamiast związków metaloorganicznych\sidecite{hisler06}.
Synteza ketonów i~aldehydów pozostaje główną sferą użycia związków tego typu,
  jednak naukowcy odkrywają też inne sposoby ich wykorzystania.
\citeauthor{baker13} pokazali na~przykład, że są one dogodnym zamiennikiem ketonów
  winylowych w~reakcjach sprzęgania Hecka\sidecite{baker13}.
