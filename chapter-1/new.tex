\section{Nowe perspektywy}\label{literature:new}
Zarówno aktywacja amidów bezwodnikiem triflowym jak i~ich częściowa redukcja odczynnikiem
  Schwartza znajdują się w~arsenale chemików-syntetyków już od dłuższego czasu.
Są sprawdzonymi i~uznanymi metodami, które z~powodzeniem znalazły zastosowanie
  we~współczesnej syntezie organicznej.
Zainteresowanie badaczy dziedziną aktywacji amidów jednak wciąż nie słabnie i~kilka
  ostatnich lat przyniosło nowe odkrycia.
Prawie wszystkie opisane w~tej części doniesienia zostały opublikowane już podczas
  realizacji niniejszej pracy doktorskiej.
Siłą rzeczy, ich zastosowanie w~przeprowadzonych przeze mnie eksperymentach, w~większości
  przypadków, nie było możliwe.
Mimo tego, te wybitne odkrycia zasługują, aby przynajmniej o~nich wspomnieć.

\subsection{Kompleks Vaski}\label{literature:new:vasca}
Grupa badaczy pod kierunkiem Nagashimy dowiodła w~\citeyear{motoyama09}, że \vaska{}, nazywany
  potocznie kompleksem Vaski, w~tandemie z~\gls{tmds} redukuje do~enamin \refcmpd{enamine-tert}
  trzeciorzędowe amidy\refcmpd{w:bh-amide-tert}, posiadające proton w~pozycji
  \textalpha{}\sidecite{motoyama09}.
Zobrazowana na~\cref{sch:nagashima} reakcja biegnie bardzo wydajnie, wymaga użycia jedynie
  \SI{0.5}{\mole\percent} \vaska{} jako katalizatora i~toleruje obecność innych grup
  karbonylowych w~cząsteczce amidu.
Jakiś czas później stała się ona inspiracją dla zespołów badawczych Dixona i~Huanga, które,
  pracując niezależnie, opracowały na~jej podstawie pierwszy katalityczny protokół aktywacji
  amidów przez częściową redukcję.
\begin{marginscheme}
  \includesvg{nagashima}
  \caption{Redukcja trzeciorzędowych amidów do~enamin katalizowana kompleksem Vaski.}
  \label{sch:nagashima}
\end{marginscheme}

W roku \citeyear{gregory15} badacze z~grupy prowadzonej przez Dixona przedstawili reduktywną
  cyklizację trzeciorzędowych amidów, posiadających grupę nitrową\sidecite{gregory15},
  pokazaną na~\cref{sch:ir-mannich}.
Bazując na~procedurze zaproponowanej przez Nagashimę, przeprowadzili częściową redukcję
  laktamu~\refcmpd{lactam-nitro}.
Powstająca enamina~\refcmpd{enamine-nitro}, podczas terminacji reakcji \SI{1}{\Molar} kwasem solnym,
  ulega przekształceniu w~kation iminiowy \refcmpd{iminium-nitro}, a~następnie wewnątrzcząsteczkowej
  reakcji nitro-Manicha\sidecite{gregory15}.
Procedura pozwala otrzymać bicykliczne związki typu \refcmpd{bicyclo-nitro} diastereoselektywnie,
  z~wydajnością umiarkowaną do~dobrej.
\begin{scheme*}
  \includesvg{ir-mannich}
  \caption{
    Pierwszy przykład reduktywnej funkcjonalizacji amidu w~układzie katalitycznym \---
      wewnątrzcząsteczkowa cyklizacja poprzez wariant reakcji Mannicha.
  }
  \label{sch:ir-mannich}
\end{scheme*}

Rok później \citeauthor{huang16c} zaproponowali reduktywną funkcjonalizację amidów poprzez
  sekwencję dwóch katalitycznych przemian: częściowej redukcji za~pomocą \vaska{} i~addycji
  acetylenu, promowanej solami miedzi \ch{(I)}\sidecite{huang16c}.
Co istotne, transformacji tej ulegają również amidy nie posiadające protonu w~pozycji \textalpha{},
  a~zatem takie, które nie mogą utworzyć enaminy.
Kluczowym dla przeprowadzenia funkcjonalizacji jest zatem redukcja do~hemiaminalu eteru sililowego
  i~jego przekształcenie do~kationu iminiowego, a~nie powstawanie enaminy.
Warto dodać, że przemiana biegnie selektywnie w~obecności innych grup karbonylowych i~grupy
  nitrowej, a~także nitrylu.

W kolejnych latach naukowcy z~grupy Dixona znacznie poszerzyli wachlarz dostępnych przemian
  możliwych do~zrealizowania tą metodą.
Przedstawili syntezę \iupac{\a-cyjanoamin} w~katalizowanym kompleksem Vaski wariancie reakcji
  Streckera\sidecite{fuentes17},
  reduktywną addycję odczynników Grignarda do~karbonylowej grupy amidowej\sidecite{xie17},
  użycie jej w~tandemie z~reakcjami wieloskładnikowymi typu reakcji Ugiego\sidecite{dixon18},
  czy syntezę związków spiro z~indoli\sidecite{gabriel19}.
Bardziej szczegółowy opis tych przemian, ich przykłady zastosowania w~syntezie związków pochodzenia
  naturalnego, a~nawet rozważania na~temat mechanizmu katalizowanej \vaska{} aktywacji amidów
  zainteresowany czytelnik może znaleźć dogodnie zebrane w~wydanej niedawno przeglądowej
  publikacji autorstwa Dixona i~in.\sidecite{dixon20rev}.

\subsection{Kompleks van~der~Enta}\label{literature:new:van-der-ent}
Istotnym ograniczeniem procedury opartej na kompleksie Vaski jest jej niekompatybilność z~amidami
  drugorzędowymi.
Pierwszy krok do~zaproponowania komplementarnej metody ich aktywacji zrobili \citeauthor{cheng12},
  prezentując redukcję dietylosilanem katalizowaną innym kompleksem irydu \---
  \ch{[Ir(coe)2Cl]2}\sidecite{cheng12}.
Autorzy pokazali, że redukcja może być zatrzymana na~etapie iminy lub aminy, w~zależności
  od~ilości użytego reduktora.
Zaproponowali też mechanizm tej redukcji i~dowiedli dobrej chemoselektywności kompleksu
  van~der~Enta\sidenote{%
    W~literaturze rzadko występuje pod nazwiskiem odkrywcy.
    Zwykle wspominany jest przy użyciu wzoru sumarycznego.
  } względem amidów.

Naukowcom z~grupy badawczej Huanga udało się zbudować na~tej podstawie protokół reduktywnej
  aktywacji prowadzonej w~jednym naczyniu reakcyjnym.
Aby poddać funkcjonalizacji powstającą iminę potrzebny jest dodatek kwasu Lewisa \---
  \citeauthor{ou18} z~powodzeniem użyli do~tego celu \ch{BF3.OEt2}\sidecite{ou18}.
Otrzymali szeroki wachlarz produktów, prowadząc nukleofilową addycję różnych związków
  metaloorganicznych, anionu cyjankowego, allilotributylocyny, czy \iupac{\H-fosfonianu}.
Nie udało im się jednak łatwo przeprowadzić trifluorometylowania, reakcji Ugiego, laktamizacji,
  ani reakcji imino-Dielsa-Aldera w~tym reduktywnym systemie katalitycznym.
Niezbędne było opracowanie warunków reakcji indywidualnie w~każdym z~przypadków.

Chiba i~Sato niezależnie prowadzili badania na~tym samym polu i~zaproponowali bardzo podobne
  warunki, z~tym, że używając~\ch{Yt(OTf)3} w~roli kwasu Lewisa\sidecite{takahashi18}.
Przedstawili porównanie różnych dostępnych metod aktywacji amidów drugorzędowych,
  pokazując wprost przewagę kompleksu van~der~Enta nad kompleksem Vaski w~tej materii
  \--- zastosowanie tego drugiego co prawda prowadzi do~powstania oczekiwanego produktu,
  ale z~bardzo niską wydajnością.
Zaskoczeniem dla autorów było zachowanie się \textgamma{}-laktamu w~badanych warunkach \---
  nie ulegał on w~ogóle reakcji wobec \ch{[Ir(coe)2Cl]2}, ale redukował się wobec \vaska{},
  choć niezbyt efektywnie.
Najlepsze wyniki dało zastosowanie obydwu systemów katalitycznej redukcji po~kolei,
  zaczynając od~kompleksu Vaski.
Użycie katalizatorów w~odwrotnej kolejności prowadziło jedynie do~odzyskania substratu.

\subsection{Heksakarbonylek molibdenu}\label{literature:new:molydenium}
\subsection{Redukcja izopropoksytytanem}\label{literature:new:titanium}
\subsection{Redukcja wodorkiem sodu}\label{literature:new:sodium-hydride}
