\section{Nowe perspektywy}\label{literature:new}
Zarówno aktywacja amidów bezwodnikiem triflowym jak i~ich częściowa redukcja odczynnikiem
  Schwartza znajdują się w~arsenale chemików-syntetyków już od dłuższego czasu.
Są sprawdzonymi i~uznanymi metodami, które z~powodzeniem znalazły zastosowanie
  we~współczesnej syntezie organicznej.
Zainteresowanie badaczy dziedziną aktywacji amidów jednak wciąż nie słabnie i~kilka
  ostatnich lat przyniosło nowe odkrycia.
Prawie wszystkie opisane w~tej części doniesienia zostały opublikowane już podczas
  realizacji niniejszej pracy doktorskiej.
Siłą rzeczy, ich zastosowanie w~przeprowadzonych przeze mnie eksperymentach, w~większości
  przypadków, nie było możliwe.
Mimo tego, te wybitne odkrycia zasługują, aby przynajmniej o~nich wspomnieć.

\subsection{Kompleks Vaski}\label{literature:new:vasca}
Grupa badaczy pod kierunkiem Nagashimy dowiodła w~\citeyear{motoyama09}, że \vaska{}, nazywany
  potocznie kompleksem Vaski, w~tandemie z~\gls{tmds} redukuje do~enamin trzeciorzędowe
  laktamy\refcmpd{lactam-tert-ah}, posiadające proton w~pozycji \textalpha{}\sidecite{motoyama09}.
Reakcja biegnie bardzo wydajnie, wymaga użycia jedynie \SI{0.5}{\mole\percent} \vaska{} jako
  katalizatora i~toleruje obecność innych grup karbonylowych w~cząsteczce amidu.
Jakiś czas później stała się ona inspiracją dla zespołów badawczych Dixona i~Huanga, które,
  pracując niezależnie, opracowały na~jej podstawie pierwszy katalityczny protokół aktywacji
  amidów przez częściową redukcję.

W roku \citeyear{gregory15} badacze z~grupy prowadzonej przez Dixona przedstawili reduktywną
  cyklizację trzeciorzędowych amidów, posiadających grupę nitrową\sidecite{gregory15}.
Bazując na~procedurze zaproponowanej przez Nagashimę, przeprowadzili częściową redukcję
  laktamu~\refcmpd{lactam-nitro}.
Powstająca enamina, podczas terminacji reakcji \SI{1}{\Molar} kwasem solnym, ulega przekształceniu
  w~kation iminiowy, a~następnie wewnątrzcząsteczkowej reakcji nitro-Manicha\sidecite{gregory15}.
Procedura pozwala otrzymać bicykliczne związki typu \refcmpd{bicycle-nitro} diastereoselektywnie,
  z~wydajnością umiarkowaną do~dobrej.

Rok później \citeauthor{huang16c} zaproponowali reduktywną funkcjonalizację amidów poprzez
  sekwencję dwóch katalitycznych przemian: częściowej redukcji za~pomocą \vaska{} i~addycji
  acetylenu, promowanej solami miedzi \ch{(I)}\sidecite{huang16c}.
Co istotne, transformacji tej ulegają również amidy nie posiadające protonu w~pozycji \textalpha{},
  a~zatem takie, które nie mogą utworzyć enaminy.
Kluczowym dla przeprowadzenia funkcjonalizacji jest zatem redukcja do~hemiaminalu eteru sililowego
  i~jego przekształcenie do~kationu iminiowego, a~nie powstawanie enaminy.
Warto dodać, że przemiana biegnie selektywnie w~obecności innych grup karbonylowych i~grupy
  nitrowej, a~także nitrylu.

W kolejnych latach naukowcy z~grupy Dixona znacznie poszerzyli wachlarz dostępnych przemian
  możliwych do~zrealizowania tą metodą.
Przedstawili syntezę \iupac{\a-cyjanoamin} w~katalizowanym kompleksem Vaski wariancie reakcji
  Streckera\sidecite{fuentes17},
  reduktywną addycję odczynników Grignarda do~karbonylowej grupy amidowej\sidecite{xie17},
  użycie jej w~tandemie z~reakcjami wieloskładnikowymi typu reakcji Ugiego\sidecite{dixon18},
  czy syntezę związków spiro z~indoli\sidecite{gabriel19}.
Bardziej szczegółowy opis tych przemian, ich przykłady zastosowania w~syntezie związków pochodzenia
  naturalnego, a~nawet rozważania na~temat mechanizmu katalizowanej \vaska{} aktywacji amidów
  zainteresowany czytelnik może znaleźć dogodnie zebrane w~wydanej niedawno przeglądowej
  publikacji autorstwa Dixona i~in.\sidecite{dixon20rev}.

\subsection{Kompleks van~der~Enta}\label{literature:new:van-der-ent}
\subsection{Heksakarbonylek molibdenu}\label{literature:new:molydenium}
\subsection{Redukcja izopropoksytytanem}\label{literature:new:titanium}
\subsection{Redukcja wodorkiem sodu}\label{literature:new:sodium-hydride}
