\section{Nowe perspektywy}\label{literature:new}
Zarówno aktywacja amidów bezwodnikiem triflowym jak i~ich częściowa redukcja odczynnikiem
  Schwartza znajdują się w~arsenale chemików-syntetyków już od kilkudziesięciu lat.
Są sprawdzonymi i~uznanymi metodami, które z~powodzeniem znalazły zastosowanie
  we~współczesnej syntezie organicznej.
Zainteresowanie badaczy dziedziną aktywacji amidów jednak wciąż nie słabnie i~kilka
  ostatnich lat przyniosło nowe odkrycia.
Prawie wszystkie opisane w~tej części doniesienia zostały opublikowane już podczas
  realizacji niniejszej pracy doktorskiej.
Siłą rzeczy, ich zastosowanie w~przeprowadzonych przeze mnie eksperymentach, w~większości
  przypadków, nie było możliwe.
Mimo tego, te wybitne odkrycia zasługują, aby przynajmniej o~nich wspomnieć.

\subsection{Kompleks Vaski}\label{literature:new:vasca}
W roku \citeyear{gregory15} badacze z~grupy prowadzonej przez Dixona przedstawili pierwszy
  katalityczny protokół aktywacji amidów przez częściową redukcję.
Pokazali, że \vaska{}, nazywany potocznie kompleksem Vaski, w~tandemie z~\gsl{tmds}
  redukuje trzeciorzędowy laktam \refcmpd{lactam-nitro} do~enaminy.
Pod wpływem terminacji reakcji \SI{1}{\Molar} kwasem solnym, ulega ona przekształceniu
  w~kation iminiowy, a~następnie wewnątrzcząsteczkowej reakcji nitro-Manicha\sidecite{gregory15}.
Procedura wymaga użycia jedynie \SI{0.5}{\mole\percent} katalizatora i~pozwala otrzymać
  bicykliczne związki typu \refcmpd{bicycle-nitro} diastereoselektywnie, z~wydajnością
  umiarkowaną do~dobrej.

\subsection{Kompleks van~der~Enta}\label{literature:new:van-der-ent}
\subsection{Heksakarbonylek molibdenu}\label{literature:new:molydenium}
\subsection{Redukcja izopropoksytytanem}\label{literature:new:titanium}
\subsection{Redukcja wodorkiem sodu}\label{literature:new:sodium-hydride}
