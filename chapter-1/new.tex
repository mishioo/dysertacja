\section{Nowe perspektywy}\label{literature:new}
Zarówno aktywacja amidów bezwodnikiem triflowym jak i~ich częściowa redukcja odczynnikiem
  Schwartza znajdują się w~arsenale chemików-syntetyków już od dłuższego czasu.
Są sprawdzonymi i~uznanymi metodami, które z~powodzeniem znalazły zastosowanie
  we~współczesnej syntezie organicznej.
Zainteresowanie badaczy dziedziną aktywacji amidów jednak wciąż nie słabnie i~kilka
  ostatnich lat przyniosło nowe odkrycia.
Prawie wszystkie opisane w~tej części doniesienia zostały opublikowane już podczas
  realizacji niniejszej pracy doktorskiej.
Siłą rzeczy, ich zastosowanie w~przeprowadzonych przeze mnie eksperymentach, w~większości
  przypadków, nie było możliwe.
Mimo tego, te wybitne odkrycia zasługują, aby przynajmniej o~nich wspomnieć.

\subsection{Kompleks Vaski}\label{literature:new:vasca}
Grupa badaczy pod kierunkiem Nagashimy dowiodła w~\citeyear{motoyama09}, że \vaska{}, nazywany
  potocznie kompleksem Vaski, w~tandemie z~\gls{tmds} redukuje do~enamin \refcmpd{enamine-tert}
  trzeciorzędowe amidy\refcmpd{w:bh-amide-tert}, posiadające proton w~pozycji
  \textalpha{}\sidecite{motoyama09}.
Zobrazowana na~\cref{sch:nagashima} reakcja biegnie bardzo wydajnie, wymaga użycia jedynie
  \SI{0.5}{\mole\percent} \vaska{} jako katalizatora i~toleruje obecność innych grup
  karbonylowych w~cząsteczce amidu.
Jakiś czas później stała się ona inspiracją dla zespołów badawczych Dixona i~Huanga, które,
  pracując niezależnie, opracowały na~jej podstawie pierwszy katalityczny protokół aktywacji
  amidów przez częściową redukcję.
\begin{marginscheme}
  \includesvg{nagashima}
  \caption{Redukcja trzeciorzędowych amidów do~enamin katalizowana kompleksem Vaski.}
  \label{sch:nagashima}
\end{marginscheme}

W roku \citeyear{gregory15} badacze z~grupy prowadzonej przez Dixona przedstawili reduktywną
  cyklizację trzeciorzędowych amidów, posiadających grupę nitrową w~łańcuchu
  bocznym\sidecite{gregory15}, pokazaną na~\cref{sch:ir-mannich}.
Bazując na~procedurze zaproponowanej przez Nagashimę, przeprowadzili częściową redukcję
  laktamu~\refcmpd{lactam-nitro}.
Powstająca enamina~\refcmpd{enamine-nitro}, podczas terminacji reakcji \SI{1}{\Molar} kwasem solnym,
  ulega przekształceniu w~kation iminiowy \refcmpd{iminium-nitro}, a~następnie wewnątrzcząsteczkowej
  reakcji nitro-Manicha\sidecite{gregory15}.
Procedura pozwala otrzymać bicykliczne związki typu \refcmpd{bicyclo-nitro} diastereoselektywnie,
  z~wydajnością umiarkowaną do~dobrej.
\begin{scheme*}
  \includesvg{ir-mannich}
  \caption{
    Pierwszy przykład reduktywnej funkcjonalizacji amidu w~układzie katalitycznym \---
      wewnątrzcząsteczkowa cyklizacja poprzez wariant reakcji Mannicha.
  }
  \label{sch:ir-mannich}
\end{scheme*}

Rok później \citeauthor{huang16c} zaproponowali reduktywną funkcjonalizację amidów poprzez
  sekwencję dwóch katalitycznych przemian: częściowej redukcji katalizowanej \vaska{} i~addycji
  acetylenu, promowanej solami miedzi \ch{(I)}\sidecite{huang16c}.
Co istotne, transformacji tej ulegają również amidy \refcmpd{w:amide-tert}, nie posiadające
  protonu w~pozycji \textalpha{}, a~zatem takie, które nie mogą utworzyć enaminy.
Kluczowym dla przeprowadzenia funkcjonalizacji jest zatem redukcja do~hemiaminalu eteru
  sililowego~\refcmpd{silyl-tert} i~jego przekształcenie do~kationu iminiowego~\refcmpd{iminium},
  a~nie powstawanie enaminy.
Warto dodać, że przemiana biegnie selektywnie w~obecności innych grup karbonylowych, grupy
  nitrowej, a~także nitrylu.
\begin{scheme*}
  \includesvg{ir-cu-activation}
  \caption[][-1\baselineskip]{
    Prace Huanga~i~in. pokazały, że trzeciorzędowe amidy, które nie mogą utworzyć enaminy
      również ulegają reduktywnej funkcjonalizacji katalizowanej kompleksem Vaski.
    Schemat przedstawia ten proces, zwieńczony addycją acetylenu promowaną solami miedzi \ch{(I)}.
  }
  \label{sch:ir-cu-activation}
  \setfloatalignment{b}
\end{scheme*}

W kolejnych latach naukowcy z~grupy Dixona znacznie poszerzyli wachlarz dostępnych przemian
  możliwych do~zrealizowania tą metodą.
Przedstawili syntezę \iupac{\a-cyjanoamin} w~katalizowanym kompleksem Vaski wariancie reakcji
  Streckera\sidecite[-3\baselineskip]{fuentes17},
  reduktywną addycję odczynników Grignarda do~karbonylowej grupy amidowej\sidecite{xie17},
  użycie jej w~tandemie z~reakcjami wieloskładnikowymi typu reakcji Ugiego\sidecite{dixon18},
  czy syntezę związków spiro z~indoli\sidecite{gabriel19}.
Bardziej szczegółowy opis tych przemian, ich przykłady zastosowania w~syntezie związków pochodzenia
  naturalnego, a~nawet rozważania na~temat mechanizmu katalizowanej \vaska{} aktywacji amidów
  zainteresowany czytelnik może znaleźć dogodnie zebrane w~wydanej niedawno przeglądowej
  publikacji autorstwa Dixona i~in.\sidecite{dixon20rev}.

\subsection{Kompleks van~der~Enta}\label{literature:new:van-der-ent}
Istotnym ograniczeniem procedury opartej na kompleksie Vaski jest jej niekompatybilność z~amidami
  drugorzędowymi.
Pierwszy krok do~zaproponowania komplementarnej metody ich aktywacji zrobili \citeauthor{cheng12},
  prezentując redukcję dietylosilanem katalizowaną innym kompleksem irydu \---
  \ch{[Ir(coe)2Cl]2}\sidecite{cheng12}.
Autorzy pokazali, że redukcja może być zatrzymana na~etapie iminy lub aminy, w~zależności
  od~ilości użytego reduktora.
Zaproponowali też mechanizm tej redukcji i~dowiedli dobrej chemoselektywności kompleksu
  van~der~Enta\sidenote{%
    W~literaturze rzadko występuje pod nazwiskiem odkrywcy.
    Zwykle wspominany jest przy użyciu wzoru sumarycznego.
  } względem amidów.

Naukowcom z~grupy badawczej Huanga udało się zbudować na~tej podstawie protokół reduktywnej
  aktywacji prowadzonej w~jednym naczyniu reakcyjnym.
Aby poddać funkcjonalizacji powstającą iminę potrzebny jest dodatek kwasu Lewisa \---
  \citeauthor{ou18} z~powodzeniem użyli do~tego celu \ch{BF3.OEt2}\sidecite{ou18}.
Otrzymali szeroki wachlarz produktów, prowadząc nukleofilową addycję różnych związków
  metaloorganicznych, anionu cyjankowego, allilotributylocyny, czy \iupac{\H-fosfonianu}.
Nie udało im się jednak łatwo przeprowadzić trifluorometylowania, reakcji Ugiego, laktamizacji,
  ani reakcji imino-Dielsa-Aldera w~tym reduktywnym systemie katalitycznym.
Niezbędne było opracowanie warunków reakcji indywidualnie w~każdym z~przypadków.

Chida i~Sato niezależnie prowadzili badania na~tym samym polu i~zaproponowali bardzo podobne
  warunki, z~tym, że używając~\ch{Yt(OTf)3} w~roli kwasu Lewisa\sidecite{takahashi18}.
Przedstawili porównanie różnych dostępnych metod aktywacji amidów drugorzędowych,
  pokazując wprost przewagę kompleksu van~der~Enta nad kompleksem Vaski w~tej materii.
Jak widać na~\cref{sch:sec-methods}, zastosowanie tego drugiego co prawda prowadzi
  do~powstania oczekiwanego produktu, ale z~bardzo niską wydajnością.
Sprawdzili też zachowanie odczynnika Schwartza w~tej reakcji, jego zastosowanie przyniosło
  rezultat niewiele gorszy niż nowo opracowana procedura katalityczna.
\begin{scheme*}
  \includesvg{sec-methods}
  \caption{
    Porównanie metod reduktywnej aktywacji drugorzędowego amidu dokonane przez Chidę, Sato i~in.
    Wykres po~prawej stronie przedstawia wydajności otrzymywania produktu
      funkcjonalizacji~\refcmpd{ph-bn-cyanoamine} w~zależności od~użytej metody aktywacji.
  }
  \label{sch:sec-methods}
\end{scheme*}

Zaskoczeniem dla autorów było zachowanie się \textgamma{}-laktamu~\refcmpd{ph-y-lactam}
  w~badanych warunkach \--- nie ulegał on w~ogóle reakcji wobec \ch{[Ir(coe)2Cl]2},
  ale redukował się wobec \vaska{}, choć niezbyt efektywnie.
Najlepsze wyniki dało zastosowanie obydwu systemów katalitycznej redukcji po~kolei,
  zaczynając od~kompleksu Vaski, jak widać na~\cref{sch:ir-dual}.
Użycie katalizatorów w~odwrotnej kolejności prowadziło jedynie do~odzyskania substratu.
Przyczyny tego fenomenu autorzy dopatrują się w~szczególnym mechanizmie tego typu
  reduktywnej aktywacji amidów drugorzędowych.
Uważają, że następuje ona poprzez \iupac{\N-sililowanie} substratu, a~następnie redukcję do~iminy,
  zamiast przez bezpośrednie hydrosililowanie grupy karbonylowej.
Dlaczego kompleks van der Enta katalizuje tworzenie \iupac{\N-sililowej} pochodnej amidów liniowych,
  ale nie laktamów nie zostało dotąd wyjaśnione.
\begin{scheme*}
  \includesvg{ir-dual}
  \caption{
    Wydajna funkcjonalizacja \textgamma{}-laktamu~\refcmpd{ph-y-lactam} wymagała zastosowania
      obydwu katalitycznych protokołów aktywacji w~tandemie.
  }
  \label{sch:ir-dual}
\end{scheme*}

\subsection{Heksakarbonylek molibdenu}\label{literature:new:molydenium}
Wszystkie opisane dotąd reduktywne metody aktywacji amidów pozwalają na~przeprowadzenie
  wyczerpującej redukcji, prowadzącej do~otrzymania amin.
Wymaga to zastosowania wodorku o~nukleofilowych właściwościach jako odczynnika do~funkcjonalizacji.
W~przypadku odczynnika Schwartza może to być na~przykład \ch{NaBH4}, a~w~protokołach
  wykorzystujących katalizatory irydowe \--- dodatkowa porcja silanu.
Jako, że reduktora do~aktywacji zwykle używa się w~nadmiarze względem amidowego substratu,
  niekiedy problemem może być następowanie niepożądanej redukcji do~aminy, zasygnalizowane
  w~wielu cytowanych dotąd pracach.

Niedawno Adolfsson i~in. zaproponowali nową metodę reduktywnej aktywacji, w~której kontrolę
  nad~procesem można sprawować temperaturą prowadzenia reakcji\sidecite{tinnis16}.
Katalizowana \ch{Mo(CO)6} redukcja amidu \refcmpd{amide-piperidine} za pomocą \gls{tmds}
  w~\SI{0}{\degreeCelsius} prowadzi do~powstania aktywnej pochodnej
  sililowej~\refcmpd{si-amide-piperidine}, ale podniesienie temperatury procesu
  do~\SI{80}{\degreeCelsius} skutkuje powstaniem wyłącznie aminy~\refcmpd{amine-piperidine},
  przy użyciu tej samej ilości odczynnika redukującego.
W~pierwszej pracy poświęconej temu tematowi autorzy wydzielili aldehyd~\refcmpd{benzaldehyde}
  powstający w~wyniku hydrolizy związku~\refcmpd{si-amide-piperidine}.
W~wyniku dalszych badań dowiedli, że można zamiast tego przeprowadzić jego funkcjonalizację,
  co~pokazali na~przykładzie otrzymywania produktu reakcji
  Streckera~\refcmpd{cyano-amine-piperidine}\sidecite{trillo18}.
\begin{scheme}
  \includesvg{molybdenum}
  \caption{
    Reduktywna aktywacja amidu katalizowana heksakarbonyliem molibdenu.
    Przebiegiem procesu można sterować zmieniając temperaturę prowadzenia reakcji.
  }
  \label{sch:molybdenum}
  \setfloatalignment{b}
\end{scheme}

Autorzy pokazali, że opracowany protokół pozwala aktywować amidy selektywnie wobec innych
  karbonyli \--- estry, ketony i~aldehydy nie są redukowane nawet w~przypadku syntezy~amin
  w~podwyższonej temperaturze.
Reduktywna funkcjonalizacja została przeprowadzona nawet wobec niezabezpieczonej grupy
  karboksylowej\sidecite{trillo18}, co jest nieosiągalne przy użyciu innych protokołów.
Niestety, metoda ta nie jest uniwersalna.
Podobnie jak w~przypadku kompleksu Vaski, amidy drugorzędowe ulegają \iupac{\N-sililowaniu}
  w~warunkach przemiany i~nie poddają się procesowi aktywacji.
Ponadto optymalna temperatura aktywacji zależy od~substratu, wśród przytoczonych przykładów
  waha się między \num{-5} a~\SI{65}{\degreeCelsius}.

\subsection{Redukcja wodorkiem sodu}\label{literature:new:sodium-hydride}
Dość zaskakująca, i~przez to niezwykle interesująca, jest najnowsza metoda aktywacji,
  zaproponowana przez Chibę, Dixona i~in., w~której zastosowali wodorek sodu jako reduktor.
\ch{NaH}, mimo obecności anionu wodorkowego, nie jest raczej wykorzystywany w~tej roli.
Można wymienić po~temu kilka przyczyn, począwszy od~jego nierozpuszczalności w~mediach
  organicznych, po~mały rozmiar orbitalu 1s, utrudniający nakładanie się z~orbitalem
  antywiążącym grupy karbonylowej\sidecite[-9\baselineskip]{furman2020}.
Niezwykle istotnym czynnikiem jest też zwięzłość i~regularność sieci krystalicznej \ch{NaH},
  która skutecznie ogranicza dostęp do~\ch{H-} potencjalnemu elektrofilowi.
Badaczom udało się wpłynąć na~ten ostatni aspekt dzięki dodatkowi \ch{NaI} \--- wprowadzenie
  atomów jodu do~struktury prowadzi do~jej zaburzenia i~pozwala na~\enquote{odsłonięcie}
  części anionów wodorkowych\sidecite[-11\baselineskip]{hong16}.
\Cref{fig:nah-nai-crystal} schematycznie przedstawia to zjawisko.
\begin{marginfigure}[-13\baselineskip]
  \includesvg{nah-nai-crystal}
  \caption{
    Zaburzenie regularności sieci krystalicznej \ch{NaH} poprzez wprowadzenie do~struktury
      atomów jodu pozwala na~\enquote{odsłonięcie} części anionów wodorkowych
      i~uwydatnienie nukleofilowych właściwości.
  }
  \label{fig:nah-nai-crystal}
\end{marginfigure}

\Cref{sch:nah-nai-activation} przedstawia przemiany amidów, jakich udało się dotąd dokonać
  stosując tę metodę.
Początkowo zaprezentowana została jedynie redukcja do~aminoalkoholanu~\refcmpd{amino-alcoholate}
  i~jego hydroliza do~aldehydu~\refcmpd{aldehyde}\sidecite{too16}.
Późniejsze badania pokazały, że terminacja reakcji za~pomocą \ch{TMSCl} owocuje powstaniem
  sililowego hemiaminalu~\refcmpd{sio-hemiaminal}, łatwo rozpadającego się do~soli
  iminiowej~\refcmpd{sio-iminium}.
Dzięki temu badacze mogli przeprowadzić funkcjonalizację do~\textalpha{}-rozgałęzionych
  trzeciorzędowych amin.
Zaprezentowane zostały reakcja z~odczynnikami Grignarda oraz cyjanowanie, prowadzące odpowiednio
  do~struktur \refcmpd{w:amine-tert-mono, amine-tert-cn}\sidecite[-2\baselineskip]{ong20}.
\begin{scheme*}
  \includesvg{nah-nai-activation}
  \caption{
    Najnowsza metoda reduktywnej aktywacji amidów, wykorzystująca jako reduktor wodorek sodu
      o~zaburzonej strukturze krystalicznej.
  }
  \label{sch:nah-nai-activation}
\end{scheme*}

Na podstawie przedstawionych dotąd przykładów trudno osądzić czy metodę można nazwać
  chemoselektywną, ale bez wątpienia ma inne istotne zalety.
Pozwala z~wysoką efektywnością przekształcać amidy alifatyczne, karbo- i~heteroaromatyczne,
  a~także laktamy, dobrze radzi sobie ze~strukturami silnie rozgałęzionymi.
Z~umiarkowanym sukcesem została zastosowana do~przeprowadzenia przemiany w~sposób
  diastereoselektywny.
Przede wszystkim natomiast, jest to pierwsza prawdziwie tania i~wygodna w~użyciu metoda
  reduktywnej aktywacji amidów, nie wymagająca nawet pracy w~atmosferze gazu obojętnego.
Z~jej zastosowaniem z~łatwością powinien poradzić sobie nie tylko chemik obeznany z~pracą
  z~odczynnikami metaloorganicznymi, ale~każdy zainteresowany.
