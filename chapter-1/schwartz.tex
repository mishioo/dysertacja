\section{Odczynnik Schwartza}\label{literature:schwartz}
Bezwodnik triflowy na dobre zagościł już w~warsztacie chemika-syntetyka
  jako narzędzie do aktywacji wiązania amidowego.
Jego użycie nie jest jednak jedyną dostępna metodą \---
  w~ciągu ostatnich 30 lat różni badacze zaproponowali kilka innych podejść do tego problemu,
  a~jednym z~nich jest wykorzystanie wodorku chlorocyrkonocenu (\schwartz{}, \refcmpd{w:schwartz}).
Związek ten został wprowadzony do~użytku w~chemii syntetycznej przez Jeffreya Schwartza\sidecite{schwartz74},
  stąd zwyczajowo nazywany jest od jego nazwiska \--- odczynnikiem Schwartza.

\begin{marginscheme}
  \includesvg[schwartz/]{synthesis-side}
  \caption{Standardowa metoda syntezy odczynnika Schwartza \refcmpd{w:schwartz}.}
  \label{sch:schwartz-synthesis}
\end{marginscheme}
Jak na związek metaloorganiczny jest dość stabilny \--- 
  zauważalna degradacja pod wpływem światła, tlenu, czy wilgoci
  następuje dopiero po kilkudniowej ekspozycji.
Przechowywanie w~atmosferze gazu obojętnego pozwala na~utrzymanie
  jego pierwotnej reaktywności nawet przez kilka miesięcy.
Zazwyczaj jest przygotowywany poprzez redukcję \ch{[Cp2ZrCl2]} (\refcmpd{w:zirconocene-dichloride})
  za~pomocą glinowodorku litu albo podobnego reduktora,
  oznaczonego ogólnie na~\cref{sch:schwartz-synthesis} jako jon wodorkowy.
Sposób ten jest prosty, choć nie pozbawiony mankamentów \---
  wymaga zastosowania beztlenowych i~bezwodnych warunków oraz ochrony przed światłem.
Co więcej, często dochodzi do powstania \ch{[Cp2ZrH2]} (\refcmpd{w:zirconocene-dihydride}),
  który jest mniej aktywnym reduktorem niż odczynnik Schwartza.
Prace Buchwalda i~in. zapewniły dogodne rozwiązanie tego ostatniego problemu:
  nadmiar diwodorku \refcmpd{w:zirconocene-dihydride} można przekształcić w~pożądany związek \refcmpd{w:schwartz},
  przemywając mieszaninę produktów chlorkiem metylenu\sidecite{buchwald87}.
Diwodorek \refcmpd{w:zirconocene-dihydride} reaguje z~\ch{CH2Cl2} o~wiele szybciej niż odczynnik Schwartza,
  nie ma więc niebezpieczeństwa otrzymania z~powrotem substratu \refcmpd{w:zirconocene-dichloride}.

Alternatywą jest generowanie odczynnika Schwartza \latin{in~situ},
  redukując dichlorek \refcmpd{w:zirconocene-dichloride} za~pomocą
  Red-Al\sidecite{gibson87}, \ch{LiEt3BH}\sidecite{lipshutz90},
  \ch{LiAlH(O "\textit{t-}" Bu)3}\sidecite{zhao14}, czy \gls{dibal}\sidecite{huang06}.
Ostatnia z~wymienionych metod wymaga dodatkowego komentarza \---
  według badań przeprowadzonych przez Negishi i~Huanga\sidecite{huang06}
  jako jedyna nie jest narażona na~powstawanie niepożądanego diwodorku~\refcmpd{w:zirconocene-dihydride}.
Należy jednak pamiętać, że w~wyniku jej zastosowania tworzy się równomolowa mieszanina
  \schwartz{} i~\ch{ "\textit{i-}" Bu2AlCl*THF}.
Nie jest ona dokładnym odpowiednikiem odczynnika Schwartza \--- obecność
  \ch{ "\textit{i-}" Bu2AlCl*THF} może mieć wpływ na~przebieg procesu hydrocyrkonowania.

Reakcje z~odczynnikiem Schwartza zwykle prowadzi się w~chlorku metylenu lub tetrahydrofuranie.
Sam \schwartz{} jest nierozpuszczalny w~tych, jak i~w~większości innych rozpuszczalników
  organicznych, ale metaloorganiczne pochodne powstające w~wyniku hydrocyrkonowania
  rozpuszczają się już bardzo dobrze we wspomnianych \acrshort{dcm} i~\acrshort{thf}.
Można dzięki temu łatwo śledzić postęp reakcji wraz z~klarowaniem się początkowo
  niehomogenicznej mieszaniny.
Rozpuszczalność, a~zarazem i~reaktywność związku~\refcmpd{w:schwartz} można zwiększyć,
  zastępując ligandy cyklopentadienylowe lub atom chloru innymi podstawnikami.
Przykładami takich modyfikacji mogą być pochodna z~metylowanym ligandem
  \ch{[(MeCp)2Zr(H)Cl]}\sidecite[-7\baselineskip]{erker89} czy pochodna triflowa
  \ch{[Cp2Zr(H)OTf]}\sidecite[-5\baselineskip]{husgen97,luinstra85}.
Związki takie mają pewne zalety, ale ich przygotowanie jest znacznie bardziej
  pracochłonne i~kosztowne niż w~przypadku odczynnika Schwartza,
  przez co nie cieszą się tak dużym zainteresowaniem chemików.

\begin{marginfigure}
  \includesvg[schwartz/]{structure}
  \caption{
    Rzeczywista, dimeryczna struktura odczynnika Schwartza~\refcmpd{w:schwartz},
    ustalona przy pomocy techniki MicroED.
    Atomy wodoru przy pierścieniach Cp zostały pominięte dla większej przejrzystości.
  }
  \label{fig:schwartz-structure}
\end{marginfigure}
  Długo spekulowano na~temat dokładnej struktury \schwartz{} \---
  mimo wielu lat jego powszechnego użytku w~syntezie, nikomu nie udało się dotąd
  przeprowadzić wiarygodnych badań rentgenograficznych.
\citeauthor{wailes70} postulowali polimeryczną strukturę tego związku\sidecite{wailes70},
  bazując na~jego słabej rozpuszczalności i~analizie spektroskopowej w~podczerwieni,
  sugerującej występowanie mostków \ch{Zr-H}.
Więcej informacji przyniosły dopiero wykonane 40~lat później pomiary \ch{^{35}Cl}~NMR
  w~ciele stałym, według których \refcmpd{w:schwartz} ma raczej budowę dimeryczną,
  podobnie do~\ch{[Cp2Zr(H)Me]}\sidecite{rossini09}.
Niedawno ostateczny tego dowód dostarczyli \citeauthor{jones19} \---
  wykorzystując technikę dyfrakcji elektronowej na~mikrokryształach (MicroED) uzyskali
  obraz rzeczywistej struktury wodorku chlorocyrkonocenu~\refcmpd{w:schwartz}\sidecite{jones19},
  pokazany na~\cref{fig:schwartz-structure}.


\subsection{Hydrocyrkonowanie}\label{literature:schwartz:hydrozirconation}
Pierwszym zastosowaniem odczynnika Schwartza było hydrocyrkonowanie alkenów i~alkinów
  i~w~tej roli wciąż jest on wykorzystywany najczęściej.
Duży rozmiar cząsteczki \refcmpd{w:schwartz} sprawia, że na~regioselektywność
  przyłączenia do~wiązania wielokrotnego wpływ mają przede wszystkim względy steryczne.
W~przypadku wewnętrznych alkenów, takich jak \refcmpd{w:alkane-inter} na~\cref{sch:zirconoorganics},
  często występuje zjawisko izomeryzacji w~kierunku terminalnego,
  najbardziej trwałego termodynamicznie izomeru.
Alkiny natomiast, w~obecności nadmiaru \schwartz{} mogą tworzyć dimeryczne kompleksy typu
  \refcmpd{w:zircono-alkane-dimer}\sidecite{erker83}.
Powstające związki cyrkonoorganiczne mają nukleofilowość zbliżoną do enamin,
  sililowych eterów enoli\sidecite{berionni16}, czy związków cynoorganicznych\sidecite{corral15},
  a~więc niższą niż najczęściej używane odczynniki metaloorganiczne.
Są zwykle wrażliwe na~wilgoć, światło i~tlen, więc podczas pracy z~nimi
  należy zachować odpowiednie środki ostrożności\sidecite{blackburn75}.
\begin{scheme}
  \centering
  \includesvg[schwartz/]{zirconoorganics}
  \caption{
    Reaktywność odczynnika Schwartza wobec alkanów i~alkenów.
    Kierunek addycji, która zawsze jest typu \textit{syn},
    dyktowany jest względami sterycznymi.
  }
  \label{sch:zirconoorganics}
  \setfloatalignment{b}
\end{scheme}

Schwartz i in. pokazali, że alkilowe pochodne cyrkonocenu \refcmpd{w:zircono-alkane-terminal}
  mogą zostać przekształcone w~alkohole\sidecite{blackburn75}, karbonyle\sidecite{bertelo75}, 
  czy halogenki alkilowe\sidecite{hart75}.
Od~czasu tych doniesień naukowcy zaproponowali wiele nowych metod wykorzystania
  związków cyrkonoorganicznych w~syntezie.
Można zastosować je jako reagenty w~krzyżowym sprzęganiu katalizowanym kompleksami metali%
  \sidenote{Zazwyczaj palladu lub niklu.},
  albo przeprowadzić z~ich użyciem substytucję \textit{tele} w~obrębie halogenku allilowego.
Związki te ulegają też addycji do~wiązań podwójnych węgiel-heteroatom oraz sprzężonych.
Co istotne, wszystkie te przemiany, zebrane na~\cref{sch:zr-transformations},
  zostały też zaprezentowane w~wersji asymetrycznej.
Tematyka ta jest bardzo rozległa, ale została wyczerpująco opisana w~wydanych ostatnio
  przeglądach\sidecite{nemethova21, pinheiro18}.
Dociekliwych czytelników odsyłam do~tych materiałów po szczegółowe wiadomości
  z~zakresu zastosowania alkilowych pochodnych cyrkonocenu
  \refcmpd{w:zircono-alkane-terminal,w:zircono-alkene-terminal}.
\begin{scheme}
  \centering
  \includesvg[schwartz/]{zirconoorganics-transformations}
  \caption{
    Najistotniejsze typy przemian związków cyrkonoorganicznych wywiedzionych z~alkanów.
    Wszystkie prezentowane przemiany mogą być prowadzone w~wariancie asymetrycznym.
  }
  \label{sch:zr-transformations}
\end{scheme}

Oprócz alkenów i alkinów, odczynnik Schwartza może redukować również
  wiązania wielokrotne węgiel\-/heteroatom.
Nie licząc addycji do amidów i~laktamów, o~których więcej w~kolejnych sekcjach tego rozdziału,
  są to raczej przemiany słabo zbadane, a~czasem tylko pojedyncze obserwacje.
Warto o~nich jednak wspomnieć, aby zapewnić pełniejszy obraz reaktywności \schwartz{},
  zwłaszcza w~kontekście chemoselektywności metod wykorzystujących ten związek.

\subsection{Redukcja nitryli}\label{literature:schwartz:nitriles}
Najszerzej opisaną z~tych przemian jest redukcja nitryli.
W~połowie lat 80-tych \citeauthor{erker84} zauważyli,
  że nitryle ławo reagują z~odczynnikiem Schwartza, tworząc metaloiminy,
  takie jak \refcmpd{w:metalloimine} na~\cref{sch:metalloimine}\sidecite{erker84}.
Obserwacja ta stała się podstawą badań nad hydrocyrkonowaniem cyjanofosfin
  i~geminalnych dinitryli, prowadzonych przez Majorala i in.\sidecite{%
    maraval01a, maraval01b, maraval03}
Powstające w~jego wyniku mono- i~dicyrkonowane kompleksy iminowe mogą reagować
  z~różnymi elektrofilowymi reagentami, na~przykład chlorofosfinami,
  chlorkami acylowymi czy solami iminiowymi.
Zastosowanie takiej ścieżki przemiany pozwoliło otrzymać szeroki wachlarz
  \iupac{\N-funkcjonalizowanych} amin oraz diamin.
\begin{marginscheme}[-36\baselineskip]
  \includesvg[schwartz/]{metalloimine}
  \caption{
    Hydrocyrkonowanie nitryli prowadzi do~powstania kompleksu \refcmpd{w:metalloimine},
    podatnego m.~in. na~atak czynnika elektrofilowego.
  }
  \label{sch:metalloimine}
\end{marginscheme}

Nitryl zredukowany za pomocą odczynnika Schwartza może być poddany hydrolizie,
  w~wyniku czego powstaje aldehyd.
Baran i~jego grupa użyli tej strategii w~jedenastoetapowej totalnej syntezie
  związku z~rodziny alkaloidów araiosaminowych\sidecite{tian16}.
Omawiany etap tej syntezy zaprezentowany jest na~\cref{sch:baran-aldehyde}.
Warto zwrócić uwagę, że redukcja ta biegnie w~obecności karbonylu z~grupy
  \iupac{\tert-butoksykarbonylowej} (\acrshort{Boc}), którą zabezpieczona została amina.
Proces biegnie wydajnie nawet w~dużej, jak na laboratoryjną, skali \---
  w~ramach cytowanej pracy autorzy otrzymali gram związku \refcmpd{w:baran-diindole-aldehyde}.
\begin{scheme}
  \includesvg[schwartz/]{baran-aldehyde}
  \caption{
    Redukcja nitrylu za pomocą \schwartz{} do~aldehydu.
    Przytoczony przykład jest jednym z~etapów syntezy totalnej,
    wykonanej na~skalę gramową.
  }
  \label{sch:baran-aldehyde}
  \setfloatalignment{b}
\end{scheme}

Floreancig i~in. zaproponowali reakcję wieloskładnikowej syntezy amidów,
  której elementem jest hydrocyrkonowanie nitryli\sidecite{wan07, debenedetto09}.
W~obecności chlorków acylowych kompleks \refcmpd{zr-mcr:metalloimine}
  przekształcany jest w~acyloiminę \refcmpd{zr-mcr:acylimine}, która następnie może
  zostać poddana reakcji z~nukleofilem.
Przykłady takich przekształceń, wykorzystujące \iupac{\a-etoksy} nitryl \refcmpd{zr-mcr:sub}
  oraz różne nukleofile, przedstawiam na~\cref{sch:nitrile-reduction-mcr}.
Autorzy metody donoszą, że jest ona w~pewnym stopniu diastereoselektywna,
  ale, ze względu na~konkurencję możliwych ścieżek przemiany,
  dużą rolę gra struktura użytych substratów.
Reakcja ta biegnie wydajnie również w~przypadku nitryli nierozgałęzione w~pozycji
  \textalpha{}, także aromatycznych.
Warto dodać, że została ona zastosowana w~syntezie totalnej cytotoksyn:
  pederyny i~psimberyny oraz ich pochodnych\sidecite{wu11, wan11}.
\begin{scheme}
  \includesvg[schwartz/]{nitrile-reduction-mcr}
  \caption{
    Wieloskładnikowa reakcja syntezy rozgałęzionych amidów i~\iupac{\N-acylowanych} hemiaminali,
      której jednym z~etapów jest redukcja nitryli odczynnikiem Schwartza.
    Proces ten jest diastereoselektywny, ale stereochemia produktów zależy od~struktury substratów.
  }
  \label{sch:nitrile-reduction-mcr}
  \setfloatalignment{b}
\end{scheme}

W obecności podstawnika arylowego w~nitrylowym substracie reakcję można przeprowadzić
  wewnątrzcząsteczkowo, dodając chlorku cynku (II) zamiast nukleofila\sidecite{zhang09, xiao08}.
Okazuje się, że dla powodzenia tak prowadzonej wariacji reakcji Friedla-Craftsa
  kluczowa jest obecność podstawnika w~pozycji \textalpha{} względem grupy \ch{-CN}.
Jak widać na~\cref{sch:nitrile-reduction-fc}, zastosowanie związku
  \refcmpd{zr-fc:sub-h} jako substratu w~tej przemianie owocuje tylko śladowymi
  ilościami oczekiwanego produktu, zupełnie inaczej niż w~przypadku pochodnej
  metylowej \refcmpd{zr-fc:sub-me}.
Autorzy zaproponowali alternatywną dwuetapową procedurę pozwalającą wydajnie otrzymać
  związki typu \refcmpd{zr-fc:prod-h}\sidecite{xiao08}.
Wykorzystuje ona opracowaną wcześniej metodologię, jak obrazuje \cref{sch:nitrile-cf-two-step}.
Związek \refcmpd{zr-fc:intermediate}, potraktowany \ch{TMSOTf},
  tworzy kation acyloiminiowy \refcmpd{zr-fc:acyloiminium},
  który łatwo ulega cyklizacji do~\refcmpd{zr-fc:prod-h}.
\begin{scheme}
  \includesvg[schwartz/]{nitrile-reduction-fc}
  \caption{
    Wewnątrzcząsteczkowy wariant reakcji zaproponowanej przez Floreanciga i~in.,
      biegnący poprzez reakcję alkilowania Friedla-Craftsa.
  }
  \label{sch:nitrile-reduction-fc}
\end{scheme}
\begin{scheme*}
  \includesvg[schwartz/]{nitrile-cf-two-step}
  \caption[][-\baselineskip]{
    Dwuetapowa alternatywa syntezy związków \refcmpd{zr-fc:prod-h},
      nieosiągalnych metodą prezentowaną powyżej.
  }
  \label{sch:nitrile-cf-two-step}
  \setfloatalignment{b}
\end{scheme*}

\subsection{Redukcja pierścieni heterocyklicznych.}\label{literature:schwartz:heterocycle}
\citeauthor{cenac94} zauważyli, że możliwe jest hydrocyrkonowanie nienasyconych
  pięcioczłonowych pierścieni heterocyklicznych zawierających atom fosforu,
  azotu lub tlenu\sidecite{cenac94}.
Addycja elektrofila do powstającego kompleksu, przebiegająca z~otwarciem pierścienia,
  prowadzi do powstania pochodnych odpowiednio fosfin, amin albo alkoholi.
Prosty przykład, w~którym tworzy się ester \refcmpd{w:zr-heterocycle-opened},
  prezentuję na~\cref{sch:zr-heterocycle-opening}.
Autorzy opisywanej pracy zastosowali w~roli elektrofila też inne reagenty,
  niż \ch{PhC(O)Cl}, na~przykład \ch{Ph2PCl}, \ch{[CH2=N+ Me2]Cl-}, czy \ch{TfOH}.
\begin{scheme}
  \centering
  \includesvg[schwartz/]{heterocycle-opening}
  \caption{
    Otwarcie \iupac{1,4-dihydrofuranu} pod wpływem odczynnika Schwartza i~elektrofila,
    w~przedstawionym przykładzie \--- chlorku benzoilu.
  }
  \label{sch:zr-heterocycle-opening}
\end{scheme}

Dalsze prace tych naukowców udowodniły, że możliwe jest też reduktywne otwarcie pierścieni
  laktonów oraz cyklicznych bezwodników\sidecite{cenac96}.
W~przypadku takich substratów hydrocyrkonowaniu ulega wiązanie \ch{C=O},
  a~kolejny ekwiwalent odczynnika Schwartza powoduje otwarcie pierścienia.
Tworzy się wtedy kompleks typu \refcmpd{w:di-oxo-zr}, w~którym podstawniki cyrkonowe
  na~atomach tlenu mogą zostać podstawione elektrofilem \---
  w~cytowanej publikacji autorzy użyli \ch{Ph2PCl}, otrzymując oligofosfininy.
Substratem w~przykładzie przywołanym na~\cref{sch:oligophospinities}
  jest nienasycony lakton \refcmpd{w:1-metine-y-lacton},
  a ugrupowanie karbonylowe redukowane jest selektywnie.
Kierunek reakcji z~\schwartz{} w~obecności więcej niż jednego typu wiązania wielokrotnego
  zazwyczaj można przewidzieć, ale więcej na~ten temat opowiem w~sekcji
  \secref{literature:schwartz:selecivity}.
\begin{scheme*}
  \centering
  \includesvg[schwartz/]{oligophospinities}
  \caption{
    Synteza oligofosfinin poprzez otwarcie pierścienia laktonu~\refcmpd{w:1-metine-y-lacton}
      za~pomocą odczynnika Schwartza.
    Wiązanie \ch{C=O} jest selektywnie redukowane w~obecności wiązania podwójnego
      podstawionego geminalnie.
  }
  \label{sch:oligophospinities}
  \setfloatalignment{b}
\end{scheme*}

\citeauthor{pinheiro17} zastosowali tę metodę do~redukcji nasyconych i~nienasyconych pochodnych
  azalaktonów\sidecite{pinheiro17}.
W~wyniku reduktywnego otwarcia pierścienia powstają \iupac{\a-amino-} i~\iupac{\a-amino-\a,\b-nienasycone}
  aldehydy, w~tym drugim przypadku z~zachowaniem konfiguracji wiązania podwójnego.
Zastosowanie nadmiaru odczynnika Schwartza (\SI{4}{\equiv}) w~przypadku tych związków prowadzi
  do~otrzymania odpowiedniego aminoalkoholu.
\Cref{sch:azalactone} obrazuje takie przekształcenie na~przykładzie nienasyconego azalaktonu
  \refcmpd{azalactone}.
\begin{marginscheme}
  \includesvg[schwartz/]{azalactones}
  \caption{Przykład reduktywnego otwarcia pierścienia nienasyconego azalaktonu.}
  \label{sch:azalactone}
\end{marginscheme}

\subsection{Redukcja innych grup funkcyjnych}\label{literature:schwartz:other}
Izocyjaniany łatwo ulegają addycji wszelkich czynników nukleofilowych ze~względu na~znaczną
  elektrofilowość ugrupowania \ch{-N=C=O}.
Silne reduktory wodorkowe, takie jak \ch{LiAlH4} redukują je wyczerpująco, czyli do~metyloamin,
  jak pokazuje przykład na~\cref{sch:isocyanide-reduction}.
Częściowa redukcja do~formamidu może zostać przeprowadzona dzięki wykorzystaniu łagodniejszego
  \ch{NaBH4}\sidecite[-1\baselineskip]{stecko16}, ale metoda ta nie toleruje wielu grup funkcyjnych.
\begin{marginscheme}
  \includesvg[schwartz/]{isocyanide-reduction}
  \caption{Redukcja izocyjanów za pomocą \ch{LiAlH4} przebiega wyczerpująco.}
  \label{sch:isocyanide-reduction}
\end{marginscheme}

\citeauthor{pace16} zaproponowali podejście oparte o~hydrocyrkonowanie odczynnikiem Schwartza
  generowanym \latin{in situ} za~pomocą \ch{LiAlH(O\textit{^t}Bu)3}\sidecite{pace16},
  pokazane na~\cref{sch:isocyanide-pace}.
Opracowana przez nich procedura jest bardzo wydajna i~niezwykle chemoselektywna.
Pokazali, że otrzymana pochodna formamidu \refcmpd{ph-formamide} może zostać bezpośrednio aktywowana
  kolejnym ekwiwalentem odczynnika Schwartza i~łatwo poddana dalszej funkcjonalizacji,
  czego wynikiem jest powstanie niesymetrycznej drugorzędowej aminy \refcmpd{ph-pentene-amine}.
\begin{scheme}
  \includesvg[schwartz/]{isocyanide-pace}
  \caption{
    Przykład częściowej redukcji izocyjanku, i~następującej po niej funkcjonalizacji,
      z~użyciem odczynnika Schwartza generowanego \latin{in situ} z~\ch{Cp2ZrHCl2} oraz
      \ch{LiAlH(O\textit{^t}Bu)3}.
  }
  \label{sch:isocyanide-pace}
\end{scheme}

Fosfiny są stosowane w~syntezie organicznej nie tylko jako reagenty, ale również organokatalizatory
  oraz ligandy w~katalizie matelami.
Zazwyczaj otrzymuje się je w~wyniku redukcji odpowiednich tlenków fosfin\sidecite{kolodiazhnyi16},
  ale znaczna siła wiązania \ch{P=O} wymusza użycie do~tego celu silnych reduktorów, głównie wodorkowych.
Czyni to niemożliwym zastosowanie tego prostego podejścia, jeśli cząsteczka zawiera
  wrażliwe grupy funkcyjne.
\citeauthor{zablocka97} udowodnili, że można przeprowadzić redukcję wiązania \ch{P=O}%
  \sidenote{A także wiązania \ch{P=S}.}
  selektywnie, używając odczynnika Schwartza\sidecite{zablocka97}.
Metoda ta toleruje wewnętrzne wiązania podwójne, choć w~niektórych przypadkach badacze
  zaobserwowali jego migrację.
Zwracają też uwagę, że terminalne wiązania podwójne oraz aldehydy reagują z~\schwartz{} szybciej.
% TODO: add scheme with phosphine oxides reductions

W~\citeyear{godfrey92} \citeauthor{godfrey92} opublikowali swoje badania nad reaktywnością
  \schwartz{} względem \textbeta-ketoestrów, opisując redukcję tych związków
  do~estrów \iupac{\a,\b-nienasyconych}\sidecite{godfrey92}.
Pierwszym etapem tej przemiany jest tworzenie enolanu litu pod wpływem \gls{lihmds},  % correct grammar
  który następnie ulega reakcji transmetalacji z~odczynnikiem Schwartza.
Mechanizm zachodzącej później eliminacji nie był dla autorów jasny, ale podejrzewali,
  że związek przejściowy \refcmpd{enolate-zr} ulega hydrocyrkonowaniu, po którym
  zachodzi \textbeta{}-eliminacja, prowadząca do obserwowanego produktu.
Transformacji tej ulegają zarówno \textbeta-ketoestry liniowe, jak i~cykliczne,
  z~podobną, umiarkowanie wysoką wydajnością.
% TODO: add a sentende about b-diketone reduction
% TODO: add scheme with b-ketoesters reduction

\subsection{Redukcja amidów odczynnikiem Schwartza}\label{literature:schwartz:amides}
% TODO: write intro using schedler93 in context of godfrey92

\subsection{Selektywność wobec amidów}\label{literature:schwartz:selecivity}
% Ph2P(=O)O-R > RC(=O)OR  \ zablocka97
% R2C=O > Ph2RP=O  \ zablocka97
% RHC=CH2 > Ph2RP=O  \ zablocka97
% Ph2RP=O > RHC=CHR  \ zablocka97
\subsection{Z własnego podwórka}\label{literature:schwartz:our}

