\section{Odczynnik Schwartza}\label{literature:schwartz}
Bezwodnik triflowy na dobre zagościł już w~warsztacie chemika-syntetyka
  jako narzędzie do aktywacji wiązania amidowego.
Jego użycie nie jest jednak jedyną dostępna metodą \---
  w~ciągu ostatnich 30 lat różni badacze zaproponowali kilka innych podejść do tego problemu,
  a~jednym z~nich jest wykorzystanie wodorku chlorocyrkonocenu (\schwartz{}, \refcmpd{w:schwartz}).
Związek ten został wprowadzony do~użytku w~chemii syntetycznej przez Jeffreya Schwartza\sidecite{schwartz74},
  stąd zwyczajowo nazywany jest od jego nazwiska \--- odczynnikiem Schwartza.

\begin{marginscheme}
  \centering
  \includesvg[schwartz/]{synthesis-side}
  \caption{Standardowa metoda syntezy odczynnika Schwartza \refcmpd{w:schwartz}.}
  \label{sch:schwartz-synthesis}
\end{marginscheme}
Jak na związek metaloorganiczny jest dość stabilny \--- 
  zauważalna degradacja pod wpływem światła, tlenu, czy wilgoci
  następuje dopiero po kilkudniowej ekspozycji.
Przechowywanie w~atmosferze gazu obojętnego pozwala na~utrzymanie
  jego pierwotnej reaktywności nawet przez kilka miesięcy.
Zazwyczaj jest przygotowywany poprzez redukcję \ch{[Cp2ZrCl2]} (\refcmpd{w:zirconocene-dichloride})
  za~pomocą glinowodorku litu albo podobnego reduktora,
  oznaczonego ogólnie na~\cref{sch:schwartz-synthesis} jako jon wodorkowy.
Sposób ten jest prosty, choć nie pozbawiony mankamentów \---
  wymaga zastosowania beztlenowych i~bezwodnych warunków oraz ochrony przed światłem.
Co więcej, często dochodzi do powstania \ch{[Cp2ZrH2]} (\refcmpd{w:zirconocene-dihydride}),
  który jest mniej aktywnym reduktorem niż odczynnik Schwartza.
Prace Buchwalda i~in. zapewniły dogodne rozwiązanie tego ostatniego problemu:
  nadmiar diwodorku \refcmpd{w:zirconocene-dihydride} można przekształcić w~pożądany związek \refcmpd{w:schwartz},
  przemywając mieszaninę produktów chlorkiem metylenu\sidecite{buchwald87}.
Diwodorek \refcmpd{w:zirconocene-dihydride} reaguje z~\ch{CH2Cl2} o~wiele szybciej niż odczynnik Schwartza,
  nie ma więc niebezpieczeństwa otrzymania z~powrotem substratu \refcmpd{w:zirconocene-dichloride}.

Alternatywą jest generowanie odczynnika Schwartza \latin{in~situ},
  redukując dichlorek \refcmpd{w:zirconocene-dichloride} za~pomocą
  Red-Al\sidecite{gibson87}, \ch{LiEt3BH}\sidecite{lipshutz90},
  \ch{LiAlH(O "\textit{t-}" Bu)3}\sidecite{zhao14}, czy \gls{dibal}\sidecite{huang06}.
Ostatnia z~wymienionych metod wymaga dodatkowego komentarza \---
  według badań przeprowadzonych przez Negishi i~Huanga\sidecite{huang06}
  jako jedyna nie jest narażona na~powstawanie niepożądanego diwodorku~\refcmpd{w:zirconocene-dihydride}.
Należy jednak pamiętać, że w~wyniku jej zastosowania tworzy się równomolowa mieszanina
  \schwartz{} i~\ch{ "\textit{i-}" Bu2AlCl*THF}.
Nie jest ona dokładnym odpowiednikiem odczynnika Schwartza \--- obecność
  \ch{ "\textit{i-}" Bu2AlCl*THF} może mieć wpływ na~przebieg procesu hydrocyrkonowania.

Reakcje z~odczynnikiem Schwartza zwykle prowadzi się w~chlorku metylenu lub tetrahydrofuranie.
Sam \schwartz{} jest nierozpuszczalny w~tych, jak i~w~większości innych rozpuszczalników
  organicznych, ale metaloorganiczne pochodne powstające w~wyniku hydrocyrkonowania
  rozpuszczają się już bardzo dobrze we wspomnianych \acrshort{dcm} i~\acrshort{thf}.
Można dzięki temu łatwo śledzić postęp reakcji wraz z~klarowaniem się początkowo
  niehomogenicznej mieszaniny.
Rozpuszczalność, a~zarazem i~reaktywność związku~\refcmpd{w:schwartz} można zwiększyć,
  zastępując ligandy cyklopentadienylowe lub atom chloru innymi podstawnikami.
Przykładami takich modyfikacji mogą być pochodna z~metylowanym ligandem
  \ch{[(MeCp)2Zr(H)Cl]}\sidecite[-7\baselineskip]{erker89} czy pochodna triflowa
  \ch{[Cp2Zr(H)OTf]}\sidecite[-5\baselineskip]{husgen97,luinstra85}.
Związki takie mają pewne zalety, ale ich przygotowanie jest znacznie bardziej
  pracochłonne i~kosztowne niż w~przypadku odczynnika Schwartza,
  przez co nie cieszą się tak dużym zastosowaniem chemików.

  \begin{marginfigure}
    \centering
    \includesvg[schwartz/]{structure}
    \caption{
      Rzeczywista, dimeryczna struktura odczynnika Schwartza~\refcmpd{w:schwartz},
      ustalona przy pomocy techniki MicroED.
      Atomy wodoru przy pierścieniach Cp zostały pominięte dla większej przejrzystości.
    }
    \label{fig:schwartz-structure}
  \end{marginfigure}
  Długo spekulowano na~temat dokładnej struktury \schwartz{} \---
  mimo wielu lat jego powszechnego użytku w~syntezie, nikomu nie udało się dotąd
  wyhodować kryształu, na~którym można by przeprowadzić badania rentgenograficzne.
\citeauthor{wailes70} postulowali polimeryczną strukturę tego związku\sidecite{wailes70},
  bazując na~jego słabej rozpuszczalności i~analizie spektroskopowej w~podczerwieni,
  sugerującej występowanie mostków \ch{Zr-H}.
Więcej informacji przyniosły dopiero wykonane 40~lat później pomiary \ch{^{35}Cl}~NMR
  w~ciele stałym, według których \refcmpd{w:schwartz} ma raczej budowę dimeryczną,
  podobnie do~\ch{[Cp2Zr(H)Me]}\sidecite{rossini09}.
Niedawno ostateczny tego dowód dostarczyli \citeauthor{jones19} \---
  wykorzystując technikę dyfrakcji elektronowej na~mikrokryształach (MicroED) uzyskali
  obraz rzeczywistej struktury wodorku chlorocyrkonocenu~\refcmpd{w:schwartz}\sidecite{jones19},
  pokazany na~\cref{fig:schwartz-structure}.


\subsection{Hyrdocyrkonowanie}\label{literature:schwartz:hydrozirconation}
Pierwszym zastosowaniem odczynnika Schwartza było hydrocyrkonowanie alkenów i~alkinów
  i~w~tej roli wciąż jest wykorzystywany najczęściej.
Duży rozmiar cząsteczki \refcmpd{w:schwartz} sprawia, że na~regioselektywność
  przyłączenia do~wiązania wielokrotnego wpływ mają przede wszystkim względy steryczne.
W~przypadku wewnętrznych alkenów często występuje zjawisko izomeryzacji
  w~kierunku terminalnego, najbardziej trwałego termodynamicznie izomeru.
Powstające związki cyrkonoorganiczne mają nukleofilowość zbliżoną do enamin,
  sililowych eterów enoli\sidecite{berionni16}, czy związków cynoorganicznych\sidecite{corral15},
  a~więc niższą niż najczęściej używane odczynniki metaloorganiczne.
  

\subsection{Selektywność wobec amidów}\label{literature:schwartz:selecivity}
\subsection{Z własnego podwórka}\label{literature:schwartz:our}

