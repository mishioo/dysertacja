\section{Odczynnik Schwartza}
Bezwodnik triflowy na dobre zagościł już w~warsztacie chemików-syntetyków
  jako aktywator wiązania amidowego, ale jego użycie nie jest jedyną dostępna metodą.
W~ciągu ostatnich 30 lat różni badacze zaproponowali kilka innych podejść do tego problemu,
  a~jednym z~nich jest wykorzystanie wodorku chlorocyrkonocenu (\ch{[Cp2Zr(H)Cl]}).
Związek ten został wprowadzony do~użytku w~chemii syntetycznej przez Jeffreya Schwartza\autocite{schwartz74},
  stąd zwyczajowo nazywany jest od jego nazwiska \--- odczynnikiem Schwartza.

Jak na związek metaloorganiczny jest dość stabilny \--- w~atmosferze gazu obojętnego
  można przechowywać go kilka miesięcy bez utraty reaktywności,
  jednak ulega degradacji przy wielodniowym wystawieniu na działanie światła,
  tlenu, lub wilgoci.
Przygotowywany jest zazwyczaj poprzez redukcję \ch{[Cp2ZrCl2]} za~pomocą glinowodorku litu,
  albo podobnego reduktora.
Sposób ten jest prosty, choć nie pozbawiony mankamentów \---
  wymaga zastosowania beztlenowych i~bezwodnych warunków oraz ochrony przed światłem.
Co więcej, często dochodzi do powstania przeredukowanego \ch{[Cp2ZrH2]},
  który jest mniej aktywnym reduktorem niż odczynnik Schwartza.
Jego nadmiar można przekształcić w~pożądany związek,
  przemywając mieszaninę produktów chlorkiem metylenu.

\subsection{Hyrdocyrkonowanie}
Początkowo odczynnik Schwartza stanowił narzędzie do~hydrocyrkonowania alkenów.

\subsection{Selektywność wobec amidów}
\subsection{Z własnego podwórka}

