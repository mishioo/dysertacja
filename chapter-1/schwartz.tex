\section{Odczynnik Schwartza}
Bezwodnik triflowy na dobre zagościł już w~warsztacie chemika-syntetyka
  jako narzędzie do aktywacji wiązania amidowego.
Jego użycie nie jest jednak jedyną dostępna metodą \---
  w~ciągu ostatnich 30 lat różni badacze zaproponowali kilka innych podejść do tego problemu,
  a~jednym z~nich jest wykorzystanie wodorku chlorocyrkonocenu (\ch{[Cp2Zr(H)Cl]}, \refcmpd{schwartz}).
Związek ten został wprowadzony do~użytku w~chemii syntetycznej przez Jeffreya Schwartza\sidecite{schwartz74},
  stąd zwyczajowo nazywany jest od jego nazwiska \--- odczynnikiem Schwartza.

Jak na związek metaloorganiczny jest dość stabilny \--- 
  zauważalna degradacja pod wpływem światła, tlenu, czy wilgoci
  następuje dopiero po kilkudniowej ekspozycji.
Przechowywanie w~atmosferze gazu obojętnego pozwala na utrzymanie
  jego pierwotnej reaktywności nawet przez kilka miesięcy.
Zazwyczaj jest przygotowywany poprzez redukcję \ch{[Cp2ZrCl2]}
  za~pomocą glinowodorku litu albo podobnego reduktora.
Sposób ten jest prosty, choć nie pozbawiony mankamentów \---
  wymaga zastosowania beztlenowych i~bezwodnych warunków oraz ochrony przed światłem.
Co więcej, często dochodzi do powstania \ch{[Cp2ZrH2]},
  który jest mniej aktywnym reduktorem niż odczynnik Schwartza.
Prace Buchwalda i~in. zapewniły dogodne rozwiązanie tego ostatniego problemu:
  nadmiar diwodorku \refcmpd{w:zirconocene-dihydride} można przekształcić w~pożądany związek \refcmpd{w:schwartz},
  przemywając mieszaninę produktów chlorkiem metylenu\sidecite{buchwald87}.
Diwodorek \refcmpd{w:zirconocene-dihydride} reaguje z~\ch{CH2Cl2} o~wiele szybciej niż odczynnik Schwartza,
  nie ma więc niebezpieczeństwa otrzymania z~powrotem substratu \refcmpd{w:zirconocene-dichloride}.

Alternatywą jest generowanie odczynnika Schwartza \latin{in~situ},
  redukując dichlorek \refcmpd{w:zirconocene-dichloride} za~pomocą
  Red-Al\sidecite{gibson87}, \ch{LiEt3BH}\sidecite{lipshutz90},
  \gls{dibal}\sidecite{huang06}, czy \ch{LiAlH(O "\textit{t-}" Bu)3}\sidecite{zhao14}.


\subsection{Hyrdocyrkonowanie}
Początkowo odczynnik Schwartza stanowił narzędzie do~hydrocyrkonowania alkenów.

\subsection{Selektywność wobec amidów}
\subsection{Z własnego podwórka}

