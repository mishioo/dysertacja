\section{Odczynnik Schwartza}\label{literature:schwartz}
Bezwodnik triflowy na dobre zagościł już w~warsztacie chemika-syntetyka
  jako narzędzie do aktywacji wiązania amidowego.
Jego użycie nie jest jednak jedyną dostępna metodą \---
  w~ciągu ostatnich 30 lat różni badacze zaproponowali kilka innych podejść do tego problemu,
  a~jednym z~nich jest wykorzystanie wodorku chlorocyrkonocenu (\schwartz{}, \refcmpd{w:schwartz}).
Związek ten został wprowadzony do~użytku w~chemii syntetycznej przez Jeffreya Schwartza\sidecite{schwartz74},
  stąd zwyczajowo nazywany jest od jego nazwiska \--- odczynnikiem Schwartza.

\begin{marginscheme}
  \centering
  \includesvg[schwartz/]{synthesis-side}
  \caption{Standardowa metoda syntezy odczynnika Schwartza \refcmpd{w:schwartz}.}
  \label{sch:schwartz-synthesis}
\end{marginscheme}
Jak na związek metaloorganiczny jest dość stabilny \--- 
  zauważalna degradacja pod wpływem światła, tlenu, czy wilgoci
  następuje dopiero po kilkudniowej ekspozycji.
Przechowywanie w~atmosferze gazu obojętnego pozwala na~utrzymanie
  jego pierwotnej reaktywności nawet przez kilka miesięcy.
Zazwyczaj jest przygotowywany poprzez redukcję \ch{[Cp2ZrCl2]} (\refcmpd{w:zirconocene-dichloride})
  za~pomocą glinowodorku litu albo podobnego reduktora,
  oznaczonego ogólnie na~\cref{sch:schwartz-synthesis} jako jon wodorkowy.
Sposób ten jest prosty, choć nie pozbawiony mankamentów \---
  wymaga zastosowania beztlenowych i~bezwodnych warunków oraz ochrony przed światłem.
Co więcej, często dochodzi do powstania \ch{[Cp2ZrH2]} (\refcmpd{w:zirconocene-dihydride}),
  który jest mniej aktywnym reduktorem niż odczynnik Schwartza.
Prace Buchwalda i~in. zapewniły dogodne rozwiązanie tego ostatniego problemu:
  nadmiar diwodorku \refcmpd{w:zirconocene-dihydride} można przekształcić w~pożądany związek \refcmpd{w:schwartz},
  przemywając mieszaninę produktów chlorkiem metylenu\sidecite{buchwald87}.
Diwodorek \refcmpd{w:zirconocene-dihydride} reaguje z~\ch{CH2Cl2} o~wiele szybciej niż odczynnik Schwartza,
  nie ma więc niebezpieczeństwa otrzymania z~powrotem substratu \refcmpd{w:zirconocene-dichloride}.

Alternatywą jest generowanie odczynnika Schwartza \latin{in~situ},
  redukując dichlorek \refcmpd{w:zirconocene-dichloride} za~pomocą
  Red-Al\sidecite{gibson87}, \ch{LiEt3BH}\sidecite{lipshutz90},
  \ch{LiAlH(O "\textit{t-}" Bu)3}\sidecite{zhao14}, czy \gls{dibal}\sidecite{huang06}.
Ostatnia z~wymienionych metod wymaga dodatkowego komentarza \---
  według badań przeprowadzonych przez Negishi i~Huanga\sidecite{huang06}
  jako jedyna nie jest narażona na~powstawanie niepożądanego diwodorku~\refcmpd{w:zirconocene-dihydride}.
Należy jednak pamiętać, że w~wyniku jej zastosowania tworzy się mieszanina
  \schwartz{} i~\ch{ "\textit{i-}" Bu2AlCl*THF}.
Nie jest ona dokładnym odpowiednikiem odczynnika Schwartza \--- obecność
  \ch{ "\textit{i-}" Bu2AlCl*THF} może mieć wpływ na~przebieg procesu hydrocyrkonowania.

Reakcje z~odczynnikiem Schwartza zwykle prowadzi się w~chlorku metylenu lub tetrahydrofuranie.
Sam \schwartz{} jest nierozpuszczalny w~tych, jak i~w~większości innych rozpuszczalników
  organicznych, ale metaloorganiczne pochodne powstające w~wyniku hydrocyrkonowania
  rozpuszczają się już bardzo dobrze \acrshort{dcm} i~\acrshort{thf}.
Można dzięki temu łatwo śledzić postęp reakcji wraz z~klarowaniem się początkowo
  niehomogenicznej mieszaniny.
Rozpuszczalność, a~zarazem i~reaktywność związku~\refcmpd{w:schwartz} można zwiększyć,
  zastępując ligandy cyklopentadienylowe\sidecite{annby90,barger84,erker89,yasuda84}
  lub atom chloru\sidecite{alvhaell88,husgen97,luinstra85,perrotin09} innymi podstawnikami.
Przykładami takich modyfikacji mogą być pochodna z~metylowanym ligandem \ch{[(MeCp)2Zr(H)Cl]}
  czy pochodna triflowa \ch{[Cp2Zr(H)OTf]}.
Związki takie mają pewne zalety, ale ich przygotowanie jest znacznie bardziej
  pracochłonne i~kosztowne niż w~przypadku odczynnika Schwartza, przez co nie znalazły
  szerszego zastosowania.

\subsection{Hyrdocyrkonowanie}\label{literature:schwartz:hydrozirconation}
Początkowo odczynnik Schwartza stanowił narzędzie do~hydrocyrkonowania alkenów.

\subsection{Selektywność wobec amidów}\label{literature:schwartz:selecivity}
\subsection{Z własnego podwórka}\label{literature:schwartz:our}

