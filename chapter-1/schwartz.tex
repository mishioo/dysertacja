\section{Odczynnik Schwartza}\label{literature:schwartz}
Bezwodnik triflowy na dobre zagościł już w~warsztacie chemika-syntetyka
  jako narzędzie do aktywacji wiązania amidowego.
Jego użycie nie jest jednak jedyną dostępna metodą \---
  w~ciągu ostatnich 30 lat różni badacze zaproponowali kilka innych podejść do tego problemu,
  a~jednym z~nich jest wykorzystanie wodorku chlorocyrkonocenu (\schwartz{}, \refcmpd{w:schwartz}).
Związek ten został wprowadzony do~użytku w~chemii syntetycznej przez Jeffreya Schwartza\sidecite{schwartz74},
  stąd zwyczajowo nazywany jest od jego nazwiska \--- odczynnikiem Schwartza.

\begin{marginscheme}
  \includesvg{synthesis-side}
  \caption{Standardowa metoda syntezy odczynnika Schwartza \refcmpd{w:schwartz}.}
  \label{sch:schwartz-synthesis}
\end{marginscheme}
Jak na związek metaloorganiczny jest dość stabilny \--- 
  zauważalna degradacja pod wpływem światła, tlenu, czy wilgoci
  następuje dopiero po kilkudniowej ekspozycji.
Przechowywanie w~atmosferze gazu obojętnego pozwala na~utrzymanie
  jego pierwotnej reaktywności nawet przez kilka miesięcy.
Zazwyczaj jest przygotowywany poprzez redukcję \ch{[Cp2ZrCl2]} (\refcmpd{w:zirconocene-dichloride})
  za~pomocą glinowodorku litu albo podobnego reduktora,
  oznaczonego ogólnie na~\cref{sch:schwartz-synthesis} jako jon wodorkowy.
Sposób ten jest prosty, choć nie pozbawiony mankamentów \---
  wymaga zastosowania beztlenowych i~bezwodnych warunków oraz ochrony przed światłem.
Co więcej, często dochodzi do powstania \ch{[Cp2ZrH2]} (\refcmpd{w:zirconocene-dihydride}),
  który jest mniej aktywnym reduktorem niż odczynnik Schwartza.
Prace Buchwalda i~in. zapewniły dogodne rozwiązanie tego ostatniego problemu:
  nadmiar diwodorku \refcmpd{w:zirconocene-dihydride} można przekształcić w~pożądany związek \refcmpd{w:schwartz},
  przemywając mieszaninę produktów chlorkiem metylenu\sidecite{buchwald87}.
Diwodorek \refcmpd{w:zirconocene-dihydride} reaguje z~\ch{CH2Cl2} o~wiele szybciej niż odczynnik Schwartza,
  nie ma więc niebezpieczeństwa otrzymania z~powrotem substratu \refcmpd{w:zirconocene-dichloride}.

Alternatywą jest generowanie odczynnika Schwartza \insitu,
  redukując dichlorek \refcmpd{w:zirconocene-dichloride} za~pomocą
  Red\-/Al\sidecite{gibson87}, \ch{LiEt3BH}\sidecite{lipshutz90},
  \ch{LiAlH(O\textit{^t}Bu)3}\sidecite{zhao14}, czy \gls{dibal}\sidecite{huang06}.
Ostatnia z~wymienionych metod wymaga dodatkowego komentarza \---
  według badań przeprowadzonych przez Negishi i~Huanga\sidecite{huang06}
  jako jedyna nie jest narażona na~powstawanie niepożądanego diwodorku~\refcmpd{w:zirconocene-dihydride}.
Należy jednak pamiętać, że w~wyniku jej zastosowania tworzy się równomolowa mieszanina
  \schwartz{} i~\ch{\textit{^i}Bu2AlCl*THF}.
Nie jest ona dokładnym odpowiednikiem odczynnika Schwartza \--- obecność
  \ch{\textit{^i}Bu2AlCl*THF} może mieć wpływ na~przebieg procesu hydrocyrkonowania.

Reakcje z~odczynnikiem Schwartza zwykle prowadzi się w~chlorku metylenu lub tetrahydrofuranie.
Sam \schwartz{} jest nierozpuszczalny w~tych, jak i~w~większości innych rozpuszczalników
  organicznych, ale metaloorganiczne pochodne powstające w~wyniku hydrocyrkonowania
  rozpuszczają się już bardzo dobrze we wspomnianych \acrshort{dcm} i~\acrshort{thf}.
Można dzięki temu łatwo śledzić postęp reakcji wraz z~klarowaniem się początkowo
  niehomogenicznej mieszaniny.
Rozpuszczalność, a~zarazem i~reaktywność związku~\refcmpd{w:schwartz} można zwiększyć,
  zastępując ligandy cyklopentadienylowe lub atom chloru innymi podstawnikami.
Przykładami takich modyfikacji mogą być pochodna z~metylowanym ligandem
  \ch{[(MeCp)2Zr(H)Cl]}\sidecite[-7\baselineskip]{erker89} czy pochodna triflowa
  \ch{[Cp2Zr(H)OTf]}\sidecite[-5\baselineskip]{husgen97,luinstra85}.
Związki takie mają pewne zalety, ale ich przygotowanie jest znacznie bardziej
  pracochłonne i~kosztowne niż w~przypadku odczynnika Schwartza,
  przez co nie cieszą się tak dużym zainteresowaniem chemików.

\begin{marginfigure}
  \includesvg{structure}
  \caption{
    Rzeczywista, dimeryczna struktura odczynnika Schwartza~\refcmpd{w:schwartz},
    ustalona przy pomocy techniki MicroED.
    Atomy wodoru przy pierścieniach Cp zostały pominięte dla większej przejrzystości.
  }
  \label{fig:schwartz-structure}
\end{marginfigure}
  Długo spekulowano na~temat dokładnej struktury \schwartz{} \---
  mimo wielu lat jego powszechnego użytku w~syntezie, nikomu nie udało się dotąd
  przeprowadzić wiarygodnych badań rentgenograficznych.
\citeauthor{wailes70} postulowali polimeryczną strukturę tego związku\sidecite{wailes70},
  bazując na~jego słabej rozpuszczalności i~analizie spektroskopowej w~podczerwieni,
  sugerującej występowanie mostków \ch{Zr-H}.
Więcej informacji przyniosły dopiero wykonane 40~lat później pomiary \ch{^{35}Cl}~NMR
  w~ciele stałym, według których \refcmpd{w:schwartz} ma raczej budowę dimeryczną,
  podobnie do~\ch{[Cp2Zr(H)Me]}\sidecite{rossini09}.
Niedawno ostateczny tego dowód dostarczyli \citeauthor{jones19} \---
  wykorzystując technikę dyfrakcji elektronowej na~mikrokryształach (MicroED) uzyskali
  obraz rzeczywistej struktury wodorku chlorocyrkonocenu~\refcmpd{w:schwartz}\sidecite{jones19},
  pokazany na~\cref{fig:schwartz-structure}.


\subsection{Hydrocyrkonowanie}\label{literature:schwartz:hydrozirconation}
Pierwszym zastosowaniem odczynnika Schwartza było hydrocyrkonowanie alkenów i~alkinów
  i~w~tej roli wciąż jest on wykorzystywany najczęściej.
Duży rozmiar cząsteczki \refcmpd{w:schwartz} sprawia, że na~regioselektywność
  przyłączenia do~wiązania wielokrotnego wpływ mają przede wszystkim względy steryczne.
W~przypadku wewnętrznych alkenów, takich jak \refcmpd{w:alkane-inter} na~\cref{sch:zirconoorganics},
  często występuje zjawisko izomeryzacji w~kierunku terminalnego,
  najbardziej trwałego termodynamicznie izomeru.
Alkiny natomiast, w~obecności nadmiaru \schwartz{} mogą tworzyć dimeryczne kompleksy typu
  \refcmpd{w:zircono-alkane-dimer}\sidecite{erker83}.
Powstające związki cyrkonoorganiczne mają nukleofilowość zbliżoną do enamin,
  sililowych eterów enoli\sidecite{berionni16}, czy związków cynoorganicznych\sidecite{corral15},
  a~więc niższą niż najczęściej używane odczynniki metaloorganiczne.
Są zwykle wrażliwe na~wilgoć, światło i~tlen, więc podczas pracy z~nimi
  należy zachować odpowiednie środki ostrożności\sidecite{blackburn75}.
\begin{scheme}
  \centering
  \includesvg{zirconoorganics}
  \caption{
    Reaktywność odczynnika Schwartza wobec alkanów i~alkenów.
    Kierunek addycji, która zawsze jest typu \textit{syn},
    dyktowany jest względami sterycznymi.
  }
  \label{sch:zirconoorganics}
  \setfloatalignment{b}
\end{scheme}

Schwartz i in. pokazali, że alkilowe pochodne cyrkonocenu \refcmpd{w:zircono-alkane-terminal}
  mogą zostać przekształcone w~alkohole\sidecite{blackburn75}, karbonyle\sidecite{bertelo75}, 
  czy halogenki alkilowe\sidecite{hart75}.
Od~czasu tych doniesień naukowcy zaproponowali wiele nowych metod wykorzystania
  związków cyrkonoorganicznych w~syntezie.
Można zastosować je jako reagenty w~krzyżowym sprzęganiu katalizowanym kompleksami metali%
  \sidenote{Zazwyczaj palladu lub niklu.},
  albo przeprowadzić z~ich użyciem substytucję \textit{tele} w~obrębie halogenku allilowego.
Związki te ulegają też addycji do~wiązań podwójnych węgiel-heteroatom oraz sprzężonych.
Co istotne, wszystkie te przemiany, zebrane na~\cref{sch:zr-transformations},
  zostały też zaprezentowane w~wersji asymetrycznej.
Tematyka ta jest bardzo rozległa, ale została wyczerpująco opisana w~wydanych ostatnio
  przeglądach\sidecite{nemethova21, pinheiro18}.
Dociekliwych czytelników odsyłam do~tych materiałów po szczegółowe wiadomości
  z~zakresu zastosowania alkilowych pochodnych cyrkonocenu
  \refcmpd{w:zircono-alkane-terminal,w:zircono-alkene-terminal}.
\begin{scheme}
  \centering
  \includesvg{zirconoorganics-transformations}
  \caption{
    Najistotniejsze typy przemian związków cyrkonoorganicznych wywiedzionych z~alkanów.
    Wszystkie prezentowane przemiany mogą być prowadzone w~wariancie asymetrycznym.
  }
  \label{sch:zr-transformations}
\end{scheme}

Oprócz alkenów i alkinów, odczynnik Schwartza może redukować również
  wiązania wielokrotne węgiel\-/heteroatom.
Nie licząc addycji do amidów i~laktamów, o~których więcej w~kolejnych sekcjach tego rozdziału,
  są to raczej przemiany słabo zbadane, a~czasem tylko pojedyncze obserwacje.
Warto o~nich jednak wspomnieć, aby zapewnić pełniejszy obraz reaktywności \schwartz{},
  zwłaszcza w~kontekście chemoselektywności metod wykorzystujących ten związek.

\subsection{Redukcja nitryli}\label{literature:schwartz:nitriles}
Najszerzej opisaną z~tych przemian jest redukcja nitryli.
W~połowie lat 80-tych \citeauthor{erker84} zauważyli,
  że nitryle ławo reagują z~odczynnikiem Schwartza, tworząc metaloiminy,
  takie jak \refcmpd{w:metalloimine} na~\cref{sch:metalloimine}\sidecite{erker84}.
Obserwacja ta stała się podstawą badań nad hydrocyrkonowaniem cyjanofosfin
  i~geminalnych dinitryli, prowadzonych przez Majorala i in.\sidecite{%
    maraval01a, maraval01b, maraval03}
Powstające w~jego wyniku mono- i~dicyrkonowane kompleksy iminowe mogą reagować
  z~różnymi elektrofilowymi reagentami, na~przykład chlorofosfinami,
  chlorkami acylowymi czy solami iminiowymi.
Zastosowanie takiej ścieżki przemiany pozwoliło otrzymać szeroki wachlarz
  \iupac{\N-funkcjonalizowanych} amin oraz diamin.
\begin{marginscheme}[-36\baselineskip]
  \includesvg{metalloimine}
  \caption{
    Hydrocyrkonowanie nitryli prowadzi do~powstania kompleksu \refcmpd{w:metalloimine},
    podatnego m.~in. na~atak czynnika elektrofilowego.
  }
  \label{sch:metalloimine}
\end{marginscheme}

Nitryl zredukowany za pomocą odczynnika Schwartza może być poddany hydrolizie,
  w~wyniku czego powstaje aldehyd.
Baran i~jego grupa użyli tej strategii w~jedenastoetapowej totalnej syntezie
  związku z~rodziny alkaloidów araiosaminowych\sidecite{tian16}.
Omawiany etap tej syntezy zaprezentowany jest na~\cref{sch:baran-aldehyde}.
Warto zwrócić uwagę, że redukcja ta biegnie w~obecności karbonylu z~grupy
  \iupac{\tert-butoksykarbonylowej} (\acrshort{Boc}), którą zabezpieczona została amina.
Proces biegnie wydajnie nawet w~dużej, jak na laboratoryjną, skali \---
  w~ramach cytowanej pracy autorzy otrzymali gram związku \refcmpd{w:baran-diindole-aldehyde}.
\begin{scheme}
  \includesvg{baran-aldehyde}
  \caption{
    Redukcja nitrylu za pomocą \schwartz{} do~aldehydu.
    Przytoczony przykład jest jednym z~etapów syntezy totalnej,
    wykonanej na~skalę gramową.
  }
  \label{sch:baran-aldehyde}
  \setfloatalignment{b}
\end{scheme}

Floreancig i~in. zaproponowali reakcję wieloskładnikowej syntezy amidów,
  której elementem jest hydrocyrkonowanie nitryli\sidecite{wan07, debenedetto09}.
W~obecności chlorków acylowych kompleks \refcmpd{zr-mcr:metalloimine}
  przekształcany jest w~acyloiminę \refcmpd{zr-mcr:acylimine}, która następnie może
  zostać poddana reakcji z~nukleofilem.
Przykłady takich przekształceń, wykorzystujące \iupac{\a-etoksy} nitryl \refcmpd{zr-mcr:sub}
  oraz różne nukleofile, przedstawiam na~\cref{sch:nitrile-reduction-mcr}.
Autorzy metody donoszą, że jest ona w~pewnym stopniu diastereoselektywna,
  ale, ze względu na~konkurencję możliwych ścieżek przemiany,
  dużą rolę gra struktura użytych substratów.
Reakcja ta biegnie wydajnie również w~przypadku nitryli nierozgałęzione w~pozycji
  \textalpha{}, także aromatycznych.
Warto dodać, że została ona zastosowana w~syntezie totalnej cytotoksyn:
  pederyny i~psimberyny oraz ich pochodnych\sidecite{wu11, wan11}.
\begin{scheme}
  \includesvg{nitrile-reduction-mcr}
  \caption{
    Wieloskładnikowa reakcja syntezy rozgałęzionych amidów i~\iupac{\N-acylowanych} hemiaminali,
      której jednym z~etapów jest redukcja nitryli odczynnikiem Schwartza.
    Proces ten jest diastereoselektywny, ale stereochemia produktów zależy od~struktury substratów.
  }
  \label{sch:nitrile-reduction-mcr}
  \setfloatalignment{b}
\end{scheme}

W obecności podstawnika arylowego w~nitrylowym substracie reakcję można przeprowadzić
  wewnątrzcząsteczkowo, dodając chlorku cynku (II) zamiast nukleofila\sidecite{zhang09, xiao08}.
Okazuje się, że dla powodzenia tak prowadzonej wariacji reakcji Friedla-Craftsa
  kluczowa jest obecność podstawnika w~pozycji \textalpha{} względem grupy \ch{-CN}.
Jak widać na~\cref{sch:nitrile-reduction-fc}, zastosowanie związku
  \refcmpd{zr-fc:sub-h} jako substratu w~tej przemianie owocuje tylko śladowymi
  ilościami oczekiwanego produktu, zupełnie inaczej niż w~przypadku pochodnej
  metylowej \refcmpd{zr-fc:sub-me}.
Autorzy zaproponowali alternatywną dwuetapową procedurę pozwalającą wydajnie otrzymać
  związki typu \refcmpd{zr-fc:prod-h}\sidecite{xiao08}.
Wykorzystuje ona opracowaną wcześniej metodologię, jak obrazuje \cref{sch:nitrile-cf-two-step}.
Związek \refcmpd{zr-fc:intermediate}, potraktowany \ch{TMSOTf},
  tworzy kation acyloiminiowy \refcmpd{zr-fc:acyloiminium},
  który łatwo ulega cyklizacji do~\refcmpd{zr-fc:prod-h}.
\begin{scheme}
  \includesvg{nitrile-reduction-fc}
  \caption{
    Wewnątrzcząsteczkowy wariant reakcji zaproponowanej przez Floreanciga i~in.,
      biegnący poprzez reakcję alkilowania Friedla-Craftsa.
  }
  \label{sch:nitrile-reduction-fc}
\end{scheme}
\begin{scheme*}
  \includesvg{nitrile-cf-two-step}
  \caption[][-\baselineskip]{
    Dwuetapowa alternatywa syntezy związków \refcmpd{zr-fc:prod-h},
      nieosiągalnych metodą prezentowaną powyżej.
  }
  \label{sch:nitrile-cf-two-step}
  \setfloatalignment{b}
\end{scheme*}
% make sure both schemes are on the same page

\subsection{Redukcja pierścieni heterocyklicznych.}\label{literature:schwartz:heterocycle}
\citeauthor{cenac94} zauważyli, że możliwe jest hydrocyrkonowanie nienasyconych
  pięcioczłonowych pierścieni heterocyklicznych zawierających atom fosforu,
  azotu lub tlenu\sidecite{cenac94}.
Addycja elektrofila do powstającego kompleksu, przebiegająca z~otwarciem pierścienia,
  prowadzi do powstania pochodnych odpowiednio fosfin, amin albo alkoholi.
Prosty przykład, w~którym tworzy się ester \refcmpd{w:zr-heterocycle-opened},
  prezentuję na~\cref{sch:zr-heterocycle-opening}.
Autorzy opisywanej pracy zastosowali w~roli elektrofila też inne reagenty,
  niż \ch{PhC(O)Cl}, na~przykład \ch{Ph2PCl}, \ch{[CH2=N+ Me2]Cl-}, czy \ch{TfOH}.
\begin{scheme}
  \centering
  \includesvg{heterocycle-opening}
  \caption{
    Otwarcie \iupac{1,4-dihydrofuranu} pod wpływem odczynnika Schwartza i~elektrofila,
    w~przedstawionym przykładzie \--- chlorku benzoilu.
  }
  \label{sch:zr-heterocycle-opening}
\end{scheme}

Dalsze prace tych naukowców udowodniły, że możliwe jest też reduktywne otwarcie pierścieni
  laktonów oraz cyklicznych bezwodników\sidecite{cenac96}.
W~przypadku takich substratów hydrocyrkonowaniu ulega wiązanie \ch{C=O},
  a~kolejny ekwiwalent odczynnika Schwartza powoduje otwarcie pierścienia.
Tworzy się wtedy kompleks typu \refcmpd{w:di-oxo-zr}, w~którym podstawniki cyrkonowe
  na~atomach tlenu mogą zostać podstawione elektrofilem \---
  w~cytowanej publikacji autorzy użyli \ch{Ph2PCl}, otrzymując oligofosfininy.
Substratem w~przykładzie przywołanym na~\cref{sch:oligophospinities}
  jest nienasycony lakton \refcmpd{w:1-metine-y-lacton},
  a ugrupowanie karbonylowe redukowane jest selektywnie.
Kierunek reakcji z~\schwartz{} w~obecności więcej niż jednego typu wiązania wielokrotnego
  zazwyczaj można przewidzieć, ale więcej na~ten temat opowiem w~sekcji
  \secref{literature:schwartz:selecivity}.
\begin{scheme*}
  \centering
  \includesvg{oligophospinities}
  \caption{
    Synteza oligofosfinin poprzez otwarcie pierścienia laktonu~\refcmpd{w:1-metine-y-lacton}
      za~pomocą odczynnika Schwartza.
    Wiązanie \ch{C=O} jest selektywnie redukowane w~obecności wiązania podwójnego
      podstawionego geminalnie.
  }
  \label{sch:oligophospinities}
  \setfloatalignment{b}
\end{scheme*}

\citeauthor{pinheiro17} zastosowali tę metodę do~redukcji nasyconych i~nienasyconych pochodnych
  azalaktonów\sidecite{pinheiro17}.
W~wyniku reduktywnego otwarcia pierścienia powstają \iupac{\a-amino-} i~\iupac{\a-amino-\a,\b-nienasycone}
  aldehydy, w~tym drugim przypadku z~zachowaniem konfiguracji wiązania podwójnego.
Zastosowanie nadmiaru odczynnika Schwartza (\SI{4}{\equiv}) w~przypadku tych związków prowadzi
  do~otrzymania odpowiedniego aminoalkoholu.
\Cref{sch:azalactone} obrazuje takie przekształcenie na~przykładzie nienasyconego azalaktonu
  \refcmpd{azalactone}.
\begin{marginscheme}
  \includesvg{azalactones}
  \caption{Przykład reduktywnego otwarcia pierścienia nienasyconego azalaktonu.}
  \label{sch:azalactone}
\end{marginscheme}

\subsection{Redukcja innych grup funkcyjnych}\label{literature:schwartz:other}
Izocyjaniany łatwo ulegają addycji wszelkich czynników nukleofilowych ze~względu na~znaczną
  elektrofilowość ugrupowania \ch{-N=C=O}.
Silne reduktory wodorkowe, takie jak \ch{LiAlH4} redukują je wyczerpująco, czyli do~metyloamin,
  jak pokazuje przykład na~\cref{sch:isocyanide-reduction}.
Częściowa redukcja do~formamidu może zostać przeprowadzona dzięki wykorzystaniu łagodniejszego
  \ch{NaBH4}\sidecite[-3\baselineskip]{stecko16}, ale metoda ta nie toleruje wielu grup funkcyjnych.
\begin{marginscheme}
  \includesvg{isocyanide-reduction}
  \caption{Redukcja izocyjanów za pomocą \ch{LiAlH4} przebiega wyczerpująco.}
  \label{sch:isocyanide-reduction}
\end{marginscheme}

\citeauthor{pace16} zaproponowali podejście oparte o~hydrocyrkonowanie odczynnikiem Schwartza
  generowanym \insitu{} za~pomocą \ch{LiAlH(O\textit{^t}Bu)3}\sidecite{pace16},
  pokazane na~\cref{sch:isocyanide-pace}.
Opracowana przez nich procedura jest bardzo wydajna i~niezwykle chemoselektywna.
Pokazali, że otrzymana pochodna formamidu \refcmpd{ph-formamide} może zostać bezpośrednio aktywowana
  kolejnym ekwiwalentem odczynnika Schwartza i~łatwo poddana dalszej funkcjonalizacji,
  czego wynikiem jest powstanie niesymetrycznej drugorzędowej aminy \refcmpd{ph-pentene-amine}.
Naukowcy z~tej samej grupy pokazali niedawno, że tioizocyjanki mogą być poddane analogicznej
  przemianie, co prowadzi do powstania pochodnych tioformamidu\sidecite[-2\baselineskip]{pace19}.
\begin{scheme}
  \includesvg{isocyanide-pace}
  \caption{
    Przykład częściowej redukcji izocyjanku, i~następującej po niej funkcjonalizacji,
      z~użyciem odczynnika Schwartza generowanego \insitu{} z~\ch{Cp2ZrHCl2} oraz
      \ch{LiAlH(O\textit{^t}Bu)3}.
  }
  \label{sch:isocyanide-pace}
\end{scheme}

Fosfiny są stosowane w~syntezie organicznej nie tylko jako reagenty, ale również organokatalizatory
  oraz ligandy w~katalizie metalami.
Zazwyczaj otrzymuje się je w~wyniku redukcji odpowiednich tlenków fosfin\sidecite{kolodiazhnyi16},
  ale znaczna siła wiązania \ch{P=O} wymusza użycie do~tego celu silnych reduktorów, głównie wodorkowych.
Czyni to niemożliwym zastosowanie tego prostego podejścia, jeśli cząsteczka zawiera
  wrażliwe grupy funkcyjne.
\citeauthor{zablocka97} udowodnili, że można przeprowadzić redukcję wiązania \ch{P=O}%
  \sidenote{A także wiązania \ch{P=S}.}
  selektywnie, używając odczynnika Schwartza\sidecite{zablocka97}.
Jak widać na~\cref{sch:phosphine-oxide} metoda ta toleruje wewnętrzne wiązania podwójne,
  choć w~niektórych przypadkach badacze zaobserwowali jego migrację.
Zwracają też uwagę, że terminalne wiązania podwójne oraz aldehydy reagują z~\schwartz{} szybciej.
\begin{marginscheme}
  \includesvg{phosphine-oxide}
  \caption{
    Przykłady redukcji tlenków fosfiny odczynnikiem Schwartza, pokazujące granicę
    selektywności metody wobec wiązań podwójnych \ch{C=C}.
  }
  \label{sch:phosphine-oxide}
\end{marginscheme}

W~\citeyear{godfrey92} \citeauthor{godfrey92} opublikowali wyniki badań nad reaktywnością
  \schwartz{} względem \textbeta-ketoestrów \refcmpd{b-ketoester}, opisując redukcję tych związków
  do~estrów \iupac{\a,\b-nienasyconych}\sidecite{godfrey92}.
Pierwszym etapem tej przemiany jest tworzenie enolanu litu \refcmpd{li-enol-ester}
  pod wpływem \gls{lihmds}, który następnie ulega reakcji transmetalacji z~odczynnikiem Schwartza.
Mechanizm zachodzącej później eliminacji nie był dla autorów jasny, ale podejrzewali,
  że związek przejściowy \refcmpd{zr-enol-ester} ulega hydrocyrkonowaniu, po którym
  zachodzi \textbeta{}-eliminacja, prowadząca do obserwowanego produktu \refcmpd{ab-unsat-ester},
  jak obrazuje \cref{sch:b-ketoester}.
Transformacji tej ulegają zarówno \textbeta-ketoestry liniowe, jak i~cykliczne,
  z~podobną, umiarkowanie wysoką wydajnością.
Ulegają jej także \iupac{\b-diketony}, ale już nie z~tak dobrym efektem \---
  autorzy wydzielili jedynie około \SI{30}{\percent} oczekiwanego produktu przemiany
  \iupac{cykloheksano-1,3-dionu}.
\begin{scheme}
  \includesvg{b-ketoester}
  \caption{
    Redukcja \iupac{\b-ketoestru} odczynnikiem Schwartza wraz z~mechanizmem tej przemiany,
    zaproponowanym przez jej autorów.
  }
  \label{sch:b-ketoester}
\end{scheme}

\subsection{Redukcja amidów odczynnikiem Schwartza}\label{literature:schwartz:amides}
W~\citeyear{schedler93} badawcze z~grupy Ganema, opierając się na~poprzednich
  odkryciach\sidecite{godfrey92} w~dziedzinie redukcji elektrofilowych układów \textpi{},
  przeprowadzili próby zastosowania odczynnika Schwartza do~redukcji amidów.
Sądzili, że niezbędne będzie wstępne hydrometalowanie wiązania \ch{C=O} wodorkiem potasu,
  podobnie jak w~przypadku ich wcześniejszych badań\sidenote{Są one opisane w~poprzednim akapicie.},
  ale przeprowadzone przez nich eksperymenty pokazały, że \schwartz{} może reagować z~amidami
  bezpośrednio, prowadząc do~powstania imin\sidecite{schedler93}.
Przedstawiona w~górnej części~\cref{sch:amide-reduction-zr}, procedura ta była pierwszą
  pozwalającą w~kontrolowany i~selektywny sposób przekształcić amid \refcmpd{w:amide-sec}
  w~odpowiednią \iupac{\N-}podstawioną iminę \refcmpd{w:imine}.
\begin{scheme}
  \includesvg{amide-reduction-zr}
  \caption{
    W~przeciwieństwie do~wywiedzionych z~amidów drugorzędowych kompleksów~\refcmpd{zr-amide-sec},
      wiązanie \ch{C-O[Zr]} w~\refcmpd{zr-amide-tert} nie ulega samoczynnemu rozpadowi.
    Być może właśnie z~tego powodu \citeauthor{schedler93} nie zaobserwowali
      redukcji amidów trzeciorzędowych~\refcmpd{w:amide-tert} odczynnikiem Schwartza.
  }
  \label{sch:amide-reduction-zr}
\end{scheme}

Badając zakres stosowalności nowej metody, \citeauthor{schedler93} uznali, że amidy trzeciorzędowe
  \refcmpd{w:amide-tert} nie ulegają redukcji wobec odczynnika Schwartza.
Na~samym początku obecnego wieku \citeauthor{white00} udowodnili, że nie jest to prawdą,
  pokazując redukcję tych związków do~aldehydów\sidecite[-2\baselineskip]{white00}.
Z~przeprowadzonych później wnikliwych badań mechanistycznych\sidenote{%
    Badania te obejmowały analizę mieszanin reakcyjnych przy użyciu technik \ch{^1H}\-/NMR,
      \ch{^{13}C}\-/NMR oraz IR, a~także eksperymenty z~użyciem związków znaczonych izotopami
      \ch{D} i~\ch{^{18}O}.
  } wynika, że przemiana ta biegnie nie przez kation iminiowy \refcmpd{zr-amide-tert.iminium},
  jak proponowali w~pierwszej pracy, ale przez 18-elektronowy kompleks cyrkonu
  \refcmpd{zr-amide-tert.complex}\sidecite{spletstoser07}.
Obydwie ścieżki przemiany prezentuję na~\cref{sch:georg-paths}.
\citeauthor{spletstoser07} dowiedli jak selektywna jest ta metoda, obserwując konkurencyjne
  hydrocyrkonowanie tylko w~przypadku terminalnego alkinu oraz ketonu \---
  toleruje ona estry, nitryle, grupy \ch{-NO2} i~\ch{-NHBoc}, czy wiązania podwójne%
  \sidenote{%
    Jak wpomniałem w~poprzedniej sekcji, wiązania podwójne i~nitryle również ulegają
    hydrocyrkonowaniu, choć z~różną szybkością.
    Więcej o~selektywności odczynnika Schwartza w~\secref{literature:schwartz:selecivity}.
  }.
\begin{scheme}
  \includesvg{georg-paths}
  \caption{
    Pierwotnie zaproponowana (dolna) i~faktyczna (górna) ścieżka przemiany trzeciorzędowego amidu
    w~odpowiedni aldehyd pod wpływem odczynnika Schwartza.
  }
  \label{sch:georg-paths}
  \setfloatalignment{b}
\end{scheme}

Dodatkowe światło na~sprawę mechanizmu tej przemiany rzucili \citeauthor{wang10}, publikując
  wyniki badań teoretycznych z~użyciem metod \gls{dft}.
Wykonane przez nich obliczenia sugerują, że pierwszym etapem reakcji jest atak \schwartz{}
  na~amid poprzez kompleks \refcmpd{mech-zr-1}, a~następnie przeniesienie anionu wodorkowego
  przez czteroczłonowy stan przejściowy \refcmpd{mech-zr-2}.
Powstały w~wyniku tych transformacji stabilny związek pośredni \refcmpd{mech-zr-3} hydrolizuje
  pod wpływem wody, a~kolejne etapy tego procesu obrazują struktury
  \refcmpd{mech-zr-4}\--\refcmpd{mech-zr-8} na~\cref{sch:schwartz-calc}\sidecite{wang10}.

W~środowisku bezwodnym rozpad wiązania \ch{C-O} w~\refcmpd{mech-zr-3} jest niekorzystny zarówno
  kinetycznie, jak i~termodynamicznie, ale wiązania wodorowe, powstające w~obecności wody,
  promują tę przemianę.
Tworzący się, wysoce reaktywny kation iminiowy \refcmpd{mech-zr-5} ulega nukleofilowemu atakowi
  cząsteczki \ch{H2O}, w~rezultacie hydrolizując do~aldehydu i~aminy.
Symulacje te sugerują, że karbonylowy atom tlenu oraz atom wodoru obecne w~produkcie pochodzą
  odpowiednio z~cząsteczki wody oraz odczynnika Schwartza, co jest zgodne z~obserwacjami dokonanymi
  wcześniej przez grupę badawczą pod kierunkiem Georg\sidecite{spletstoser07}.
\begin{scheme*}
  \includesvg{schwartz-calc}
  \caption{
    Mechanizm redukcji trzeciorzędowego amidu odczynnikiem Schwartza i~hydrolizy do~aldehydu,
    zaproponowany na~podstawie obliczeń~\gls{dft}.
  }
  \label{sch:schwartz-calc}
\end{scheme*}

Iminy są związkami dość mało stabilnymi, łatwo ulegającymi hydrolizie.
Ich bezpośrednia funkcjonalizacja poprzedzona generowaniem \insitu{} jest rozwiązaniem
  o~wiele dogodniejszym niż ich wydzielanie i~operowanie czystymi iminami.
Problemem może być stosunkowo niska elektrofilowość tych związków, jednak Chida, Sato i~in.
  dowiedli, że można tę trudność pokonać używając dodatku kwasu\sidecite{nakajima14}.
Przeprowadzili oni wnikliwe badania nad możliwościami zastosowania tej metody do~reduktywnej
  funkcjonalizacji, przede wszystkim allilowania, amidów drugo- i~trzeciorzędowych,
  a~także amidów Weinreba\sidenote{%
    Czyli \iupac{\N-metoksy}amidów, więcej na~ich temat w~\secref{literature:structure:weinreb}
  }.

Przekształcenia te prowadzili generując \schwartz{} \insitu{} za~pomocą Red\-/Al w~\gls{dcm},
  a~zredukowany amid aktywując za~pomocą \gls{tfa}.
Obecność kwasu\sidenote{Zarówno kwasu protonowego, jak i~kwasu Lewisa.} promuje rozpad
  kompleksu~\refcmpd{zr-amide-tert} powstającego w~wyniku redukcji odczynnikiem Schwartza.
Dzięki temu poddane działaniu \schwartz{} amidy trzeciorzędowe~\refcmpd{w:amide-tert} zostają
  przekształcone w~odpowiednie kationy iminiowe~\refcmpd{iminium}, podobnie jak w~początkowym
  etapie hydrolizy pod wpływem \ch{H2O}.
Katoiny iminiowe łatwo ulegające addycji nukleofila, prowadząc do~powstania rozgałęzionych,
  funkcjonalizowanych amin~\refcmpd{amine-tert-nu}.
\Cref{sch:amide-tert-zr-reduction} obrazuje ten ciąg przekształceń, razem z~finalną funkcjonalizacją
  nukleofilem, oznaczonym ogólnie jako \ch{Nu-}.
Warto może przypomnieć, że drugorzędowe amidy ulegają rozpadowi do~imin samoistnie\sidenote{%
  Choć niektóre źródła sugerują, że pod wpływem dodatkowej cząsteczki \schwartz{}.
    Należy do~nich, między innymi, opisywana właśnie praca.
  }, co obrazował \cref{sch:amide-reduction-zr}.
\begin{scheme*}
  \includesvg{amide-tert-zr-reduction}
  \caption{
    Przekształcenie trzeciorzędowego amidu w~kation iminiowy na~drodze redukcji i~aktywacji kwasem
      z~następującą później funkcjonalizacją nukleofilem.
  }
  \label{sch:amide-tert-zr-reduction}
\end{scheme*}

Amidy trzeciorzędowe ulegają tym przemianom wydajnie i~selektywnie, konkurencyjny proces
  hydrocyrkonowania przebiega jedynie wobec terminalnego wiązania potrójnego.
Większą przeszkodą są wolne grupy hydroksylowe, w~obecności których redukcja nie biegnie w~ogóle.
Warto zwrócić uwagę, że wrażliwe na kwaśne środowisko grupy zabezpieczające \--- \gls{Boc},
  \gls{thp}, \gls{mom} \--- pozostały nienaruszone mimo obecności \gls{tfa}.

Analogiczny proces prowadzony wobec amidów drugorzędowych okazał się być zdecydowanie
  mniej efektywny \--- konieczne było użycie aż \SI{2.8}{\equiv} odczynnika Schwartza,
  a~wydajność i~selektywność były niższe.
Jak widać na~\cref{sch:schwartz-bpin} wynik tej przemiany zależy od użytego czynnika allilującego,
  w~przypadku allilopinakoloboranu~\refcmpd{bpin-allyl} autorzy wydzielili wyłącznie
  alkohol~\refcmpd{ph-oh-allyl} zamiast oczekiwanej aminy \refcmpd{ph-sec-allyl}.
Wspomniane wcześniej grupy wrażliwe na~kwaśne środowisko nie przetrwały warunków reakcji,
  podobnie jak terminalne wiązanie potrójne.
Jako sposób na~ominięcie tych przeszkód autorzy zaproponowali wykorzystanie amidów Weinreba
  z~użyciem katalitycznej ilości \ch{Sc(OTf)3} jako kwaśnego aktywatora.
Ich reaktywność w~takich warunkach jest zbliżona do~wykazywanej przez amidy trzeciorzędowe,
  a~rozszczepienie wiązania \ch{N-OMe}, prowadzące do~amidów drugorzędowych,
  można przeprowadzić stosunkowo łatwo.
\begin{marginscheme}
  \includesvg{schwartz-bpin}
  \caption{
    Zmiana rezultatu reduktywnej funkcjonalizacji drugorzędowego amidu w~zależności
    od~użytego czynnika allilującego.
  }
  \label{sch:schwartz-bpin}
\end{marginscheme}

Wysoka chemoselektywność opracowanej procedury redukcji i~nukleofilowej addycji czyni
  z~niej wartościowe narzędzie syntetyczne.
Podejście to zostało zaadaptowane w~syntezie totalnej związków naturalnych, na~przykład
  $(\pm)$\-/gefirotoksyny, \refcmpd{gephyro} na~\cref{sch:gephyrotoxin}\sidecite{shirokane14}.
Ważnym jest, że metoda nie jest ograniczona jedynie do~allilowania aktywnego kompleksu
  typu \refcmpd{zr-amide-sec,zr-amide-tert}.
Możliwe jest zastosowanie różnych czynników nukleofilowych jako partnerów reakcyjnych w~etapie
  funkcjonalizacji \--- Chida, Sato~i~in. zademonstrowali użycie sililowych eterów enoli,
  cyjanku trimetylosililu, \iupac{\N-metyloindolu} oraz allenu tributylocynowego\sidecite{nakajima14}.
Inne prace, opisane w~kolejnych akapitach niniejszej dysertacji, pokazały jeszcze więcej 
  możliwości.
\begin{scheme}
  \includesvg{gephyrotoxin}
  \caption{
    Fragment syntezy totalnej $(\pm)$\-/gefirotoksyny~\refcmpd{gephyro},
    wykorzystującej reduktywną funkcjonalizację amidu~\refcmpd{gephyro-substrate}
  }
  \label{sch:gephyrotoxin}
  \setfloatalignment{b}
\end{scheme}

\citeauthor{gao13} zaproponowali ogólną metodę syntezy \iupac{\a-amino|fo|sfo|nia|nów} \---
  związków, które grają istotną rolę w~chemii medycznej\sidecite{gao13}.
Oparta o~hydrocyrkonowanie amidów i~następczą addycję fosfonianu, metoda pozwala na~otrzymanie
  pożądanych związków wydajnie i~łatwo, w~procesie prowadzonym w~jednym etapie.
Istotną różnicą względem opisanych dotąd przemian jest konieczność zastosowania podwyższonej
  temperatury \--- najlepiej \SI{60}{\degreeCelsius}, poniżej której wydajność wyraźnie spada.
Zarówno amidy drugo- jak i~trzeciorzędowe ulegają tej przemianie, jak obrazuje
  \cref{sch:phosphonation}.
\begin{scheme}
  \includesvg{phosphonation}
  \caption{
    Metoda syntezy \textalpha{}-aminofosfonianów z~amidów wymaga zastosowania podwyższonej
      temperatury prowadzenia procesu, ale może być łatwo przeprowadzona w~jednym etapie.
  }
  \label{sch:phosphonation}
\end{scheme}

\citeauthor{katahara17} odkryli, że trzeciorzędowe \iupac{\N-sililoksyamidy}
  \refcmpd{notbs-amide-tert} mogą być przekształcone w~nitrony \refcmpd{nitrone} działaniem
  \schwartz{} oraz \gls{tfa}, jak obrazuje \cref{sch:nitrones-zr}\sidecite{katahara17}.
Ta nowa metoda syntezy funkcjonalizowanych nitronów, dzięki wyjątkowej selektywności odczynnika
  Schwartza oraz prostocie procedury, jest godną uwagi alternatywą dla typowych sposobów
  ich otrzymywania.
Co więcej, pozwala na~syntezę nitronów cyklicznych i~makrocyklicznych, do~których dostęp metodami
  stosowanymi zazwyczaj jest znacznie trudniejszy.
\begin{marginscheme}
  \includesvg{nitrones-zr}
  \caption{
    Redukcja amidów z~podstawnikiem \ch{-O-\gls{tbs}} na~atomie azotu prowadzi
    do~otrzymania nitronu.
  }
  \label{sch:nitrones-zr}
\end{marginscheme}

Jeszcze inny sposób wykorzystania selektywnej redukcji amidów odczynnikiem Schwartza,
  widoczny na~cref{sch:deacylation}, zaproponowali \citeauthor{sultane14}.
W~oparciu o~wcześniejsze prace zespołów Ganema\sidecite{schedler96} i~Gundy\sidecite{white00}
  opracowali procedurę błyskawicznego, reduktywnego \iupac{\N-deacylowania} amidów.
Pokazali, że różne \iupac{\N-acylowane} amidy ulegają przekształceniu do~odpowiednich wolnych
  amin pod wpływem \schwartz{} i~nstępczej hydrolizy w~zaledwie \SIrange{2}{5}{\minute},
  w~pokojowej temperaturze\sidecite{sultane14}.
Zarówno amidy alifatyczne, aromatyczne, jak i~heteroaromatyczne mogą być poddane takiemu
  odbezpieczaniu bez trudności.
Metoda ta nadaje także się do~pracy z~chiralnymi \iupac{\N-acylowanymi} aminami \---
  jej zastosowanie nie wiąże się z~utratą czystości optycznej odbezpieczanego związku.
Autorzy tej pracy udowodnili, że \iupac{\N-acylowane} aminokwasy też ulegają temu procesowi
  bez racemizacji.
\begin{marginscheme}
  \includesvg{deacylation}
  \caption{Błyskawiczne deacetylowanie za~pomocą odczynnika Schwartza.}
  \label{sch:deacylation}
\end{marginscheme}

\citeauthor{ferrari15} wykorzystali to podejście do~deacetylowania serii nukleozydów i~nukleotydów,
  choć z~umiarkowanym powodzeniem \--- wydajność procesu wahała się w~granicach
  \SIrange{40}{70}{\percent}, a~konwersja substratu była niepełna\sidecite{ferrari15}.
Przykład takiego przekształcenia widoczny jest na~\cref{sch:deacylation-nucleoside}.
Próby dalszej optymalizacji nie przyniosły oczekiwanych rezultatów.
Wyraźną poprawą zaowocowało zastosowanie procedury \insitu{} zaproponowanej przez
  \citeauthor{zhao14}\sidecite{zhao14}, ale autorzy zastosowali ją w~tylko jednym przykładzie.
\begin{scheme}
  \includesvg{deacylation-nucleoside}
  \caption{Przykład selektywnego deacetylowania nukleozydu~\refcmpd{nucleoside-ac}.}
  \label{sch:deacylation-nucleoside}
\end{scheme}
% TODO: add phosphoramide tolerance (rev. scheme 39) ?
% TODO: full reduction ?

Fakt, że zredukowany odczynnikiem Schwartza trzeciorzędowy amid może zostać przekształcony
  w~kation iminiowy\sidenote{\See{sch:amide-tert-zr-reduction}.},
  został zmyślnie wykorzystany przez Ou i~Huanga.
Użyli oni częściowo zredukowanego amidu w~reakcji kondensacji Knoevenagla zamiast iminy, typowo
  generowanej \insitu{} z~aldehydu i~aminy\sidecite{ou20}.
Większość cytowanej publikacji opisuje wykorzystanie litoorganicznego wodorku glinu,
  ale autorzy dowiedli, że \schwartz{} również może być użyty w~roli reduktora.
Jeden z~dwóch przedstawionych przez nich przykładów cytuję na~\cref{sch:zr-knoevenagel}.
\begin{scheme*}
  \includesvg{zr-knoevenagel}
  \caption{
    Jeden z~dwóch pokazanych przez Ou i~Huanga przykładów wykorzystania redukcji amidów odczynnikiem
      Schwartza w~reakcji kondensacji typu Knoevenagla.
  }
  \label{sch:zr-knoevenagel}
\end{scheme*}

\subsection{Wiązania amidowe w~pierścieniach}\label{literature:schwartz:rings}
\Cref{sch:zr-ncac-add} obrazuje, że reakcja reduktywnej addycji izocyjanoacetatów
  \refcmpd{isocyanoacetate} do~amidów drugo- \refcmpd{w:amide-sec} i~trzeciorzędowych
  \refcmpd{w:amide-tert} skutkuje powstaniem pochodnych \iupac{5-metoksyoksoazolu}
  \refcmpd{ncac-amide}\sidecite[-2\baselineskip]{zheng17}.
Co ciekawe, \iupac{\N-niepodstawione} laktamy \refcmpd{5-lactam} w~tych samych warunkach ulegają
  przekształceniu do~bicyklicznych pochodnych imidazoliny \refcmpd{ncac-lactam}.
Proces biegnie wydajnie w~przypadku laktamów pięcioczłonowych, sześcioczłonowe również ulegają
  tej przemianie, ale z gorszymi wydajnościami.
Co prawda redukcja tych drugich przebiega bez problemu, ale addycja izocyjanku jest trudniejsza
  ze~względu na~niższą elektrofilowość sześcioczłonowych imin.
\citeauthor{zheng17} zaproponowali mechanizmy tych przemian, według których trietyloamina
  jest aktywatorem izocyjanoacetatu, a~chlorek cyknu, jako kwas Lewisa, promuje rozpad
  zredukowanego amidu do~iminy lub soli iminiowej.
Czytelnika zainteresowanego szczegółami odsyłam do~źródła\sidecite[6\baselineskip]{zheng17}.
\begin{marginscheme}[-27\baselineskip]
  \includesvg{zr-ncac-add}
  \caption{
    Reduktywna addycja izocyjanoacetatów prowadzi do~powstania różnych produktów,
      w~zależności od~charakteru wiązania amidowego w~substracie.
  }
  \label{sch:zr-ncac-add}
\end{marginscheme}

Tak różny przebieg tych reakcji jest jaskrawym przykładem, obrazującym nieco odmienną reaktywność
  laktamów i~amidów liniowych.
Inny tego dowód przedstawili \citeauthor{piperno11} przeprowadzając reakcję \iupac{\N-Boc}
  zabezpieczonych laktamów~\refcmpd{nboc-lactam} z~odczynnikiem Schwartza.
W~jej wyniku powstają hemiaminal~\refcmpd{nboc-hemiaminal} oraz inne produkty
  redukcji\sidecite{piperno11}, pokazane na~\cref{sch:nboc-lactams-reduction-zr}.
Skład mieszaniny poreakcyjnej zależny jest od rozmiaru pierścienia redukowanego laktamu \---
  pięcioczłonowy laktam redukuje się czysto do~hemiaminalu~\refcmpd{nboc-hemiaminal},
  sześcioczłonowy częściowo do~enaminy~\refcmpd{nboc-enamine},
  a~redukcja laktamów cztero- oraz siedmioczłonowego skutkuje powstaniem równomolowej mieszaniny
  hemiaminalu~\refcmpd{nboc-hemiaminal} i~aminoaldehydu~\refcmpd{nboc-aminoaldehyde}.
Kontrastuje to z~opisanym wcześniej przypadkiem liniowych amidów trzeciorzędowych, które,
  choć wchodzą w~reakcję z~\schwartz{}, to nie ulegają samoistnie dalszym przemianom.
Należy zaznaczyć, że przemiana ta jest specyficzna dla \iupac{\N-alkoksykarbonylowych} laktamów \---
  w~przypadku \iupac{\N-\H-} oraz \iupac{\N-metylo-\g-laktamu} nie zachodzi\sidenote{%
    Jednakże \iupac{\N-\H-laktamy} mogą być zredukowane do~cyklicznych amin,
      \see{sch:zr-deoxygenation}.%
    },
  a~\iupac{\N-benzoilo-\g-laktam} ulega w~jej warunkach debenzoilowaniu.
\begin{scheme}
  \includesvg{nboc-lactams-reduction-zr}
  \caption{
    Inaczej niż liniowe amidy trzeciorzędowe, \iupac{\N-Boc} zabezpieczone
      laktamy~\refcmpd{nboc-lactam} redukują się pod wpływem \schwartz{}
      do~hemiaminali~\refcmpd{nboc-hemiaminal}.
    Inne produkty redukcji, \refcmpd{nboc-enamine, nboc-aminoaldehyde}, mogą powstawać
      jako produkty uboczne.
  }
  \label{sch:nboc-lactams-reduction-zr}
\end{scheme}

Redukcja pochodnych \iupac{\N-\acrshort{Boc}-pirolidynonu} do~hemiaminali, a~także ich dalsze
  przekształcenia do funkcjonalizowanych liniowych amin były znane już wcześniej\sidecite{prince19}.
\citeauthor{prince19} jako pierwsi użyli odczynnika Schwartza w~tandemie z~reduktywnym aminowaniem
  do~realizacji tego typu syntezy, widocznej na~\cref{sch:lactams-reductive-amination}.
Przemianie tej poddali różne \iupac{\N-Arylowane} laktamy~\refcmpd{lactam-tert},
  w~drugim etapie używając amin zarówno pierwszo-, jak i~drugorzędowych oraz \ch{NaHB(OAc)3}
  jako drugiego reduktora\sidecite{prince19}.
Dowodząc wyższości zastosowania \schwartz{} nad innymi czynnikami redukującymi laktamy,
  autorzy pokazują przypadek substratu \refcmpd{co2et-lactam-tert}, zawierającego grupę estrową.
Laktam jest wobec niej redukowany selektywnie, a~powstający związek \refcpd{co2et-iminoamine}
  może zostać od~razu poddany cyklizacji do~związku o~większym pierścieniu \refcmpd{aminopiperidinone}.
\begin{scheme*}
  \includesvg{lactams-reductive-amination}
  \caption{
    Reduktywne aminowanie cyrkonowych kompleksów otrzymanych w~wyniku hydrocyrkonowania laktamów.
  }
  \label{sch:lactams-reductive-amination}
\end{scheme*}

\citeauthor{dandepally13} zademonstrowali redukcję
  \iupac{tiazolidyno-2,4-dionów}~\refcmpd{thiazodione} i~hydantoin \refcmpd{nh-hydantoin,hydantoin}
  przeprowadzoną przy użyciu odczynnika Schwartza\sidecite{dandepally13}.
Reakcja ta biegnie z~dobrymi wydajnościami i~skutkuje powstaniem tiazolonów~\refcmpd{thiazolone}
  i~imidazolonów~\refcmpd{nh-imidazolone, imidazolone}
  na~drodze redukcji i~następującej po niej eliminacji.
W~przeciwieństwie do~\iupac{\N-alkilohydantoin}~\refcmpd{nh-hydantoin}
  i~\iupac{\N,\N-dialkilohydantoin}~\refcmpd{hydantoin},
  jedynym produktem reakcji \iupac{\N-\gls{Boc}-hydantoiny}~\refcmpd{nboc-hydantoin} z~\schwartz{}
  jest związek hydroksylowy~\refcmpd{nboc-reduced-hydantoin}.
Wynik ten, zobrazowany na~\cref{sch:nboc-hydantoin-reduction-zr}, jest zgodny z~opisanymi
  powyżej doniesieniami o~redukcji \iupac{\N-Boc-}lakatamów~\refcmpd{nboc-lactam}.
\begin{marginscheme}
  \includesvg{hydantoine-reduction-zr}
  \caption{
    Redukcja \iupac{tiazolidyno-2,4-dionów}~\refcmpd{thiazodione}
    i~hydantoin~\refcmpd{nh-hydantoin,hydantoin}.
  }
  \label{sch:hydantoine-reduction-zr}
\end{marginscheme}
\begin{scheme}
  \includesvg{nboc-hydantoin-reduction-zr}
  \caption{
    Odmienny przebieg redukcji hydantoiny odczynnikiem Schwartza w~obecności grupy \iupac{\N-Boc}.
  }
  \label{sch:nboc-hydantoin-reduction-zr}
  \setfloatalignment{b}
\end{scheme}

\begin{marginscheme}
  \includesvg{isoxazolidinone-reduction-zr}
  \caption{
    Przebieg redukcji pochodnych izooksazolidynonu zależy od~charakteru podstawnika
      na~atomie azotu.
  }
  \label{sch:isoxazolidinone-reduction-zr}
\end{marginscheme}
Istotny wpływ podstawnika na~atomie azotu na~przebieg reakcji hydrocyrkonowania został
  dostrzeżony również przez innych badaczy.
\citeauthor{lanza13} zauważyli, że \iupac{\N-alkoksykarbonylowe} pochodne
  izooksazolidynonu~\refcmpd{ncoor-isoxazolidinone}
  selektywnie i~wydajnie redukują się do~hemiaminali~\refcmpd{ncoor-isoxazolidine-oh},
  ale \iupac{\N-benzoiloizooksazolidynon}~\refcmpd{ncoph-isoxazolidinone}
  ulega w~tych warunkach rozpadowi wiązania \ch{N-C(O)Bn}, dając laktam~\refcmpd{isoxazolidinone}.
Różnica w~reaktywności tych związków zaintrygowała ich na~tyle, że podjęli się wyjaśnienia
  jej metodami modelowania kwantowo-mechanicznego.
Przeprowadzone przez nich obliczenia \gls{dft} są zgodne z~wynikami wspomnianymi
  wcześniej\sidecite[1\baselineskip]{wang10} w~kwestii przebiegu redukcji, która następuje przez przeniesienie
  anionu wodorkowego\sidenote{\See{sch:schwartz-calc}.}.
Bariera aktywacji tego procesu w~znacznej mierze zależy od~charakteru podstawnika
  na~atomie azotu, co najpewniej jest przyczyną innego zachowania laktamów \iupac{\N-acylowych}
  i~\iupac{\N-karbamylowych}\sidecite{lanza13}.

Powyższe przykłady mogą sprawiać mylne wrażenie, że laktamy nie są w~tej metodologii
  użytecznymi substratami.
Liczne przykłady zastosowania ich w~syntezie związków pochodzenia naturalnego dowodzą,
  że w~większości przypadków laktamy reagują z~odczynnikiem Schwartza w~sposób przewidywalny.
Jednym z~takich przykładów jest wspomniana wcześniej procedura syntezy
  $(\pm)$\-/gefirotoksyny\sidenote[][-2\baselineskip]{\See{sch:gephyrotoxin}.}.
Innym \--- synteza kwasu kainowego \refcmpd{kainic-acid}, którą zaproponowali
  \citeauthor{xia01}\sidecite[-2\baselineskip]{xia01}.
Poddali oni pochodną \iupac{2-pirolidynonu} \refcmpd{kainic-amide} działaniu \schwartz{},
  i~\ch{Me3SiCN}, a~otrzymany nitryl \refcmpd{kainic-cn} epimeryzacji i~hydrolizie.
Ten tok przemian widoczny jest na~\cref{sch:kainic}.
\begin{scheme*}
  \includesvg{kainic}
  \caption[][-1\baselineskip]{
    Synteza kwasu \iupac{($-$)-\a-kainowego} \refcmpd{kainic-acid}, oparta o~redukcję
    i~funkcjonalizację laktamu \refcmpd{kainic-amide} z~użyciem odczynnika Schwartza.
  }
  \label{sch:kainic}
  \setfloatalignment{b}
\end{scheme*}

Niektóre z~prac cytowanych w~sekcji \textit{\nameref{literature:schwartz:amides}} prezentują też
  zastosowanie opisywanych metodologii do~przekształcenia laktamów.
Zazwyczaj są to pojedyncze, proste przykłady, służące weryfikacji koncepcji.
W~jednym z~pierwszych doniesień o~redukcji amidów odczynnikiem Schwartza \citeauthor{schedler93}
  przeprowadzili redukcję laurolaktamu~\refcmpd{laurolactam} do~cyklicznej
  iminy~\refcmpd{lauro-imine}\sidecite[4\baselineskip]{schedler93}, jak widać na~\cref{sch:laurolactam}.
\citeauthor{katahara17} zaprezentowali syntezę dwóch cyklicznych nitronów wraz z~przykładami ich
  dalszych przekształceń\sidecite{katahara17}, ujęte na~\cref{sch:nitrones-zr-cyclic}.
Cykliczne i~makrocykliczne nitrony typu \refcmpd{nitrone-cyclic, nitrone-macro} trudno
  otrzymać konwencjonalnymi metodami.
\begin{marginscheme}[-25\baselineskip]
  \includesvg{laurolactam}
  \caption{
    Redukcja laurolaktamu do~odpowiedniej cyklicznej iminy, zaprezentowana w~jednej z~pierwszych
      prac poświęconych redukcji amidów odczynnikiem Schwartza.
    }
    \label{sch:laurolactam}
\end{marginscheme}
\begin{scheme*}
  \includesvg{nitrones-zr-cyclic}
  \caption[][4\baselineskip]{
    Cykliczny \refcmpd{nitrone-cyclic} i~makrocykliczny~\refcmpd{nitrone-macro} laktam
      otrzymane w~wyniku redukcji odpowiednich laktamów \refcmpd{nitrone-lactam, nitrone-lactam-macro}.
  }
  \label{sch:nitrones-zr-cyclic}
  \setfloatalignment{b}
\end{scheme*}

Należy również wspomnieć, że laktamy mogą zostać zredukowane bezpośrednio do~amin przy użyciu
  odczynnika Schwartza oraz donora nukleofilowego anionu wodorkowego.
Dowód na to znajduje się w~pracy Huanga i~Ganga, poświęconej głównie przemianom promowanym
  bezwodnikiem triflowym\sidecite{huang15}.
Autorzy przedstawili przemianę dwóch \iupac{\g-laktamów} do~odpowiednich pochodnych pirolidyny,
  uzywając \schwartz{} oraz \ch{NaBH4}.
Jak widać na~\cref{sch:zr-deoxygenation}, metoda ta jest na~tyle selektywna, że nie redukuje
  karbonylu z~grupy estrowej, obecnej w~związkach \refcmpd{5-lactam-ester, pyrrolidine-ester}.
\begin{scheme}
  \includesvg{zr-deoxygenation}
  \caption{
    Redukcja wiązania amidowego do~aminy jest możliwa przy użyciu \ch{NaBH4} jako donora
      nukleofilowego anionu wodorkowego.
    Selektywności metody dowodzi obecność nienaruszonej grupy estrowej
      w~produkcie~\refcmpd{pyrrolidine-ester}.
  }
  \label{sch:zr-deoxygenation}
\end{scheme}

\subsection{Z własnego podwórka}\label{literature:schwartz:our}
Grupa badawcza działająca pod kierunkiem Furmana, w~której wykonana została praca opisana
  w~niniejszej dysertacji, ma znaczny wkład w~rozwój metodologii reduktywnej
  funkcjonalizacji amidów.
Czerpiąc inspirację z~prac Ganema, \citeauthor{furman14} dokonali redukcji
  laktamu~\refcmpd{lactam-glu}, wywiedzionego z~glukozy\sidecite{furman14}.
Rozważny dobór metody terminacji reakcji i~oczyszczania produktu pozwolił im na~wydzielenie
  i~scharakteryzowanie iminy~\refcmpd{imine-glu}, pokazanej na~\cref{sch:zr-glu},
  a~także innych pięcio- i~sześcioczłonowych imin, pochodnych cukrów prostych.
Podejście to zdaje się być bardziej atrakcyjne niż inne metody, jak na~przykład deoksygenacja
  nitronu czy \iupac{\N-chlorowanie}/eliminacja aminy.
Co więcej, taka przemiana nie jest możliwa przy zastosowaniu metodologii
  opartej o~aktywację bezwodnikiem triflowym\sidecite{furman14}.
\begin{scheme}
  \includesvg{zr-glu}
  \caption{
    Synteza \textit{gluko}-iminy z~odpowiedniego \textit{gluko}-laktamu jest możliwa przy
      użyciu odczynnika Schwartza, ale nie bezwodnika triflowego.
  }
  \label{sch:zr-glu}
\end{scheme}

Ponieważ mała stabilność cyklicznych imin utrudnia ich stosowanie w~praktyce laboratoryjnej,
  Furman i~in. zaproponowali prowadzony w~jednym naczyniu proces redukcji i~diastereoselektywnej
  funkcjonalizacji laktamów, prowadzący do~polihydroksylowanych \iupac{2-podstawionych}
  pirolidyn i~piperydyn.
Przedstawiona na~\cref{sch:glu-allyl} reakcja wykorzystuje allilotributylocynę jako reagent
  do~funkcjonalizacji, ale inne nukleofilowe związki również ulegają addycji do~wywiedzionych
  z~cukrów aktywowanych laktamów.
W~opisywanej pracy \citeauthor{furman14} pokazują przykłady użycia odczynników Grignarda,
  sililowych eterów enoli, czy \ch{Me3SiCN} w~tej roli.


\subsection{Selektywność wobec amidów}\label{literature:schwartz:selecivity}
% RC(=O)NR2 & RC(=O)NRH  \ nakajima14
% RC(=O)NR2 >  RCN, RNO2, RC(=O)OR  \ white00
% RC(+O)R > RC(=O)NR2  \ white00
% BocNHR < RC(=O)NR2  \ spletstoser07
% HC+CR ~ RC(=O)NR2  \ spletstoser07
% RC+CR ~< RC(=O)NR2  \ spletstoser07
% RHC=CHR < RC(=O)NR2  \ spletstoser07
% H2C=CHR < RC(=O)NR2  \ spletstoser07
% H2C=CHR > RC(=O)NRH  \ schedler93
% Ph2P(=O)O-R > RC(=O)OR  \ zablocka97
% R2C=O > Ph2RP=O  \ zablocka97
% RHC=CH2 > Ph2RP=O  \ zablocka97
% Ph2RP=O > RHC=CHR  \ zablocka97
% phosphoramide < RNHAc  \ ferrari15