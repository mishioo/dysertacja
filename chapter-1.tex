\chapter{In litterae}

\section{Trwałość amidów}
Wiązanie amidowe występuje w naturze niezwykle powszechnie.
Można nawet pokusić się o stwierdzenie, że jest ono jednym z budulców życia ---
w końcu peptydy, podstawowa struktura biocheniczna złożonych organizmów,
to łańcuchy aminowkasów, połączonych wiązaniami amidowymi.

{\color{wongpurple} [rysunek: peptyd]}  % TODO

Ugrupowanie to można też znaleźć w wielu związkach biologicznie czynnych.
Za prosty przykład niech posłuży lidokaina, powszechnie stosowana jako środek miejscowo znieczulający.

{\color{wongpurple} [rysunek: lidokaina]}  % TODO

Przykładów takich możnaby przytoczyć wiele, bo jak pokazuje analiza produkcji farmaceutyków,
\SI{66}{\percent} leków syntezuje się tworząc wiązanie amidowe\autocite{carey06}.

Tę powszechność amidy zawdzięczają między innymi wyjątkowo niskiej reaktywności.
Związki te ulegają niewielu przemianom chemicznym ze względu  % TODO

Związki te zawdzięczają tę niezwykłą trwałosć bardzo efentywnemu nakładaniu się orbitali molekularnych atomu azotu oraz $\pi$ wiązania podwójnego \ch{C=O}.
Pozwala to na wydajną delokalizację elektronów w obrębie wiązania i znaczny udział dwóch możliwych struktur zwiterionowych, jak widać na \autoref{sch:resonance}.

Ze względu na swoje właściwości amidy znalazły zastosowanie także w przemyśle.
{\color{wongpurple} [nylon, uretany]}  % TODO

\section{Prezkształcenia amidów}
Przez długi czas chemia amidów była raczej uboga ---
ograniczała się przede wszystkim do prostych reakcji, dziś uznanych za podręcznikowe.
W pierwszej kolejności można wymienić ich redukcję do amin oraz hydrolizę.

\section{Odczynnik Schwartza}
