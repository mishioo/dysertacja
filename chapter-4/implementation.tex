\section{Detale implementacji}\label{tesliper:implementation}
\subsection{Wybór języka programowania}\label{implementation:language}
\subsection{Ogólna architektura kodu}\label{implementation:architecture}
Program \tesliper ma strukturę obiektową, to znaczy wykorzystuje zwane obiektami konstrukcje,
  które zawierają zarówno dane, jak i~kod odpowiedzialny za~działania.
Ten styl programowania próbuje do~pewnego stopnia naśladować rzeczywisty świat \---
  obiekty mogą \enquote{posiadać} i~\enquote{robić}, reprezentują rzecz lub ideę.
Gdy obiekty \enquote{posiadają}, posiadane przez nich dane nazywa się atrybutami,
  możliwe działania obiektów są natomiast reprezentowane przez fragmenty kodu nazywane metodami.

Obiekty mogą posiadać inne obiekty \--- taka relacja nazywa się kompozycją.
Innym rodzajem relacji między obiektami jest dziedziczenie.
Mówi się o~nim, gdy obiekt implementuje swoją funkcjonalność na~bazie innego obiektu.
Po~obiektach będących w~relacji dziedziczenia względem tego samego \enquote{rodzica}
  można spodziewać się, że realizują podobne zadania.
Charakterystyczną cechą obiektów jest to, że mają pewien czas istnienia w~obrębie używającego
  ich programu komputerowego \--- są tworzone w~konkretnym celu i~przechowywane na~czas
  jego realizacji.
Programista, pisząc kod zorientowany obiektowo, przygotowuje szablony tworzenia obiektów,
  nazywane klasami.
Konkretny obiekt stworzony przez program na~podstawie takiego szablonu to egzemplarz klasy.

\begin{figure}
  \includesvg{class-diagram}
  \caption{
    Schematyczne przedstawienie architektury programu \tesliper{} oraz przepływu danych,
      począwszy od~plików wynikowych programu Gaussian do~zapisu przetworzonych danych
      na~dysk twardy.
    Legenda po~lewej stronie przybliża znaczenie poszczególnych elementów.
  }
  \label{fig:class-diagram}
\end{figure}
\subsection{Sposób przetwarzania danych}\label{implementation:parsing}
\subsection{Analiza rozkładem Boltzmanna}\label{implementation:boltzmann}
\subsection{Metoda porównania geometrii}\label{implementation:rmsd}
\subsection{Obliczanie symulowanego widma}\label{implementation:spectra}
