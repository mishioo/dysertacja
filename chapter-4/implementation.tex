\section{Detale implementacji}\label{tesliper:implementation}
Program \tesliper{} napisałem w~języku programowania Python~3.6\sidecite{python36}.
Korzystałem przy tym z~paczki NumPy\sidecite{numpy}, będącej standardem wśród naukowej społeczności
  użytkowników Pythona, zapewniającym wydają pracę w~zakresie obliczeń numerycznych.
Po~wyłożeniu motywów wyboru języka, przybliżam ogólną strukturę kodu \tesliper{}a
  i~sposób realizacji jego podstawowych zadań, ale bez wdawania się w~szczegóły.
Zainteresowanego czytelnika odsyłam raczej do~wspomnianej wcześniej instrukcji\sidenote{%
    \href{https://tesliper.readthedocs.io/}{www.tesliper.readthedocs.io}}.
Dokładniej opisuję jedynie wybrane, ważne lub po~prostu bardziej interesujące, zagadnienia
  \--- przede wszystkim stojącą za~realizacją niektórych zadań logikę i~matematykę.

\subsection{Wybór języka programowania}\label{implementation:language}
Wybrany przeze mnie Python jest interpretowanym językiem programowania wysokiego poziomu.
Interpretowanym, czyli nie wymagającym kompilacji kodu do~jego wykonania \--- program nazywany
  interpreterem odczytuje napisane przez użytkownika instrukcje i~bezpośrednio je wykonuje%
  \sidenote{W~przeciwieństwie do~języków kompilowanych, których kod źródłowy trzeba najpierw
    \enquote{przetłumaczyć} na~język zrozumiały przez komputer, tak zwany kod maszynowy.}.
Wysokiego poziomu, znaczy natomiast, że jest on \enquote{bardziej przyjazny} programiście
  niż komputerowi \--- niezależny od~platformy, łatwiejszy do~zrozumienia, bliższy językowi
  naturalnemu niż kodowi binarnemu.

Cechy te, razem z~filozofią Pythona, stawiającą na~prostotę i~czytelność, sprawiają,
  że popularność tego języka wśród programistów sukcesywnie rośnie od~kilku lat.
W~czasie pisania niniejszej dysertacji jest on najbardziej popularnym językiem programowania
  według indeksu TIOBE\sidecite{tiobe}, sprawdzającego na~ilu stronach internetowych dany
  język jest wspominany, jak i~indeksu \gls{pypl}]\sidecite{pypl}, mierzącego ilość
  wyszukiwań samouczków dla danego języka.
W~rankingu RedMonk\sidecite{redmonk}, biorącym pod uwagę ilość wpisów związanych z~danym
  językiem w~popularnych serwisach internetowych\sidenote{%
    Chodzi dokładnie o~serwis GitHub (\href{https://github.com}{www.github.com}),
      pozwalający na~upublicznianie kodu źródłowego oraz forum StackOverflow
      (\href{https://stackoverflow.com}{www.stackoverflow.com}),
      będące istotnym miejscem wymiany wiedzy między programistami.
  }, jest natomiast na~drugiej pozycji.

Ochota z~jaką programiści sięgają po to narzędzie wynika też z~jego dużej przystępności,
  łatwości pracy z~nim i~z~zaangażowania społeczności programistów Pythona.
Ze~względu na~nieskomplikowaną składnię i~dostępność wielu gotowych rozwiązań, Python jest
  zarówno łatwy do~nauczenia, jak i~zapewniający rozległe możliwości.
Istnienie dodatków takich jak NumPy\sidecite{numpy} \--- paczka służąca do~wydajnego prowadzenia
  obliczeń na~macierzach \--- pozwala przekroczyć część ograniczeń Pythona, wynikających z~jego
  wysokiego poziomu abstrakcji\sidenote{Chodzi tu o~ukrycie detali związanych z~implementacją
    różnych algorytmów, komunikacji z~systemem operacyjnym i~podzespołami komputera.}.
Stąd też został on szeroko zaadaptowany w~nauce \--- zwłaszcza w~naukach o~danych i~w~uczeniu
  maszynowym.
Reasumując \--- liczne zalety Pythona oraz jego popularność wśród naukowców przyczyniły się
  do~mojego wyboru.

\subsection{Ogólna architektura kodu}\label{implementation:architecture}
Program \tesliper{} ma strukturę obiektową\sidenote{%
    Dla ścisłości muszę tu zaznaczyć, że \tesliper{} nie jest w~pełni utrzymany w~stylu obiektowym.
    Python nie wymusza konkretnego paradygmatu programowania,
      pozwalając mieszać ze~sobą różne wzorce.
    Te jego części, które odpowiadają za~przeliczanie konkretnych danych w~inne \---
      bez uwzględniania kontekstu wcześniejszych działań \--- są napisane w~formie niezależnych
      funkcji, czerpiąc częściowo z~paradygmatu programowania funkcjonalnego.
    Takie podejście jest typowe dla~programów napisanych w~języku Python, i~można traktować
      je jako szczegół w~organizacji kodu.
    W~związku z~tym nie opisuję tej kwestii bardziej drobiazgowo.
  }, to znaczy wykorzystuje konstrukcje zwane obiektami,
  które zawierają zarówno dane, jak i~kod odpowiedzialny za~działania.
Ten styl programowania próbuje do~pewnego stopnia naśladować rzeczywisty świat \---
  obiekty mogą \enquote{posiadać} i~\enquote{robić}; reprezentują rzeczy lub idee.
Gdy obiekty \enquote{posiadają}, posiadane przez nich dane nazywa się atrybutami,
  natomiast możliwe działania obiektów są reprezentowane przez fragmenty kodu nazywane metodami.

Obiekty mogą posiadać inne obiekty \--- taka relacja nazywana jest kompozycją.
Innym rodzajem relacji między obiektami jest dziedziczenie.
Mówi się o~nim, gdy obiekt implementuje swoją funkcjonalność na~bazie innego obiektu.
Po~obiektach, będących w~relacji dziedziczenia względem tego samego \enquote{rodzica},
  można spodziewać się, że realizują podobne zadania i~\--- zależnie od~kontekstu \--- 
  można ich używać zamiennie.

Charakterystyczną cechą obiektów jest to, że mają pewien czas istnienia w~obrębie używającego ich
  programu komputerowego \--- są tworzone w~konkretnym celu i~przechowywane na~czas jego realizacji.
Programista, pisząc kod zorientowany obiektowo, przygotowuje szablony tworzenia obiektów,
  nazywane klasami.
Konkretny obiekt stworzony przez program na~podstawie takiego szablonu to egzemplarz klasy.

Ten skoncentrowany wstęp do~programowania obiektowego wystarczy, aby przybliżyć relacje między
  poszczególnymi konstruktami, z~których składa się kod \tesliper{}a oraz sposób jego działania.
\Cref{fig:class-diagram} przedstawia uogólniony diagram klas\sidenote{
    Nie w~scisłym znaczeniu diagramu \gls{uml}, jakie zwykle się za tym sformułowaniem kryje.
    Wspomniany \cref{fig:class-diagram} to raczej luźna graficzna reprezentacja.
  } wchodzących w~skład programu.
Ująłem na~nim również przepływ danych, aby unaocznić w~jaki sposób obiekty współpracują ze~sobą.

\begin{figure}
  \includesvg{class-diagram}
  \caption{
    Schematyczne przedstawienie architektury programu \tesliper{} oraz przepływu danych,
      począwszy od~plików wynikowych programu Gaussian do~zapisu przetworzonych danych
      na~dysk twardy.
    Legenda po~lewej stronie przybliża znaczenie poszczególnych elementów.
  }
  \label{fig:class-diagram}
\end{figure}

Obiekt \texttt{Tesliper} jest obiektem fasadowym, nazywanym tak przez nawiązanie do~architektury.
Tak jak fasada budynku, obiekt tego typu jest pierwszym, z~czym ma do~czynienia użytkownik.
Stanowi główny punkt dostępu do~funkcjonalności programu, ukrywając skomplikowane elementy
  systemu i~ułatwiając jego użycie.
\texttt{Tesliper} posiada obiekt \texttt{Conformers},
  który jest głównym magazynem danych \tesliper{}a.
W~jego obrębie wykonywana jest selekcja danych \--- posiada metody pozwalające
  na~zatrzymanie lub odrzucenie\sidenote{A~właściwie na~tymczasowe ignorowanie.}
  konformerów według zadanych parametrów.
Zapewnia również dostęp do~przechowywanych danych w~formie wyspecjalizowanych egzemplarzy
  klasy \texttt{DataArray}, a~dokładnie \--- klas po~niej dziedziczących.

Obiekty typu \texttt{DataArray} są centralnym elementem funkcjonalności \tesliper{}a.
Zapewniają dostęp do~metod odpowiedzialnych za~analizę danych dotyczących wybranych konformerów
  oraz przeliczanie tych danych na~finalne wartości.
Każdy rodzaj danych \--- wartości energii, geometria struktur, aktywności spektroskopowe, itd.
  \--- jest reprezentowany przez obiekt dedykowanej klasy, dziedziczącej po~\texttt{DataArray}.
Owa dedykowana klasa dostarcza adekwatnych metod transformacji danych, które do~niej należą.
Ponadto, podczas tworzenia tych obiektów, przeprowadzana jest dokładna walidacja danych \---
  sprawdzana jest ich spójność i~są one przekształcane w~formę tablic, na~których można
  wydajnie przeprowadzać operacje matematyczne.

\subsection{Odczyt i~zapis danych}\label{implementation:parsing}
Przetworzone dane na~dysk twardy \tesliper{} zapisuje za~pomocą wyspecjalizowanych na~format
  pliku docelowego obiektów typu \texttt{Writer}\sidenote{%
    Dziedziczących tak na~prawdę po~klasie \texttt{WriterBase},
      nieuwzględnionej na~\protect\cref{fig:class-diagram}.}.
Klasy te definiują metody zapisu i~sposób formatowania różnych rodzajów danych do~obsługiwanego
  przez siebie formatu plików.
Bardziej interesujący i~godny komentarza jest proces odczytu plików wynikowych programu Gaussian,
  zarządzany przez obiekt \texttt{Soxhlet}\sidenote{%
    Jest to, humorystyczne i~prawdopodobnie oczywiste, nawiązanie do~aparatu Soxhleta.
    Uznałem je za~adekwatne, bo w~końcu również służy do~ekstrakcji.
    Jeśli, czytelniku, uznasz, że jest ono nie na~miejscu, będę bronił się, cytując Phila Karltona,
      który twierdził, że \enquote{są tylko dwia trudne zagadnienia w~naukach komputerowych:
      unieważnienie pamięci podręcznej i~nazywanie rzeczy}.
  }.

\texttt{Soxhlet} posiada wyspecjalizowany obiekt typu \texttt{Parser}\sidenote{%
    Podobnie jak w~poprzednim przypadku, jest to obiekt dziedziczący tak na~prawdę po~klasie
      \texttt{ParserBase}, również nie uwzględnionej na~\protect\cref{fig:class-diagram}.
    W~przypadku opisywanych tutaj plików wynikowych programu Gaussian używany jest egzemplarz
      klasy \texttt{GaussianParser}.}
  który odpowiada za~odczyt danych tekstowych ze~wskazanego pliku i~przetworzenie ich
  na~dane liczbowe.
\texttt{Parser} odczytuje jeden plik na~raz.
Wyekstrahowane wartości przekazuje do~obiektu \texttt{Soxhlet}, który magazynuje je przed
  przekazaniem kompletu danych do~obiektu \texttt{Tesliper}.
\texttt{Soxhlet} jest też odpowiedzialny za~zawiadywanie pracą obiektu \texttt{Parser}.
Ustala, które pliki we~wskazanym folderze należy przetworzyć i~po~kolei przekazuje je
  obiektowi \texttt{Parser}.

Warto może w~tym miejscu powiedzieć czym w~ogóle jest parser.
W~ogólnym znaczeniu to program\sidenote{Lub jego część.} odpowiedzialny za~przetwarzanie tekstu
  \--- jego odczytywanie, wydobycie z~niego informacji i~przedstawienie ich w~języku i~formie,
  które zrozumie komputer\sidenote{%
    Takie wyjaśnienie jest, oczywiście, dalekie od~formalnej definicji parsera jako analizatora
      składniowego, ale myślę, że na~potrzeby niniejszej dysertacji jest wystarzcające.}.
W~przypadku danych zawartych w~plikach wynikowych programu Gaussian ważny jest również kontekst
  odczytywanych informacji.
Parser musi nie tylko rozpoznać, że dany ciąg znaków to liczba, ale również przypisać jej znaczenie
  \--- czy ta liczba to wartość energii czy ładunku cząsteczki; czy te koordynaty reprezentują
  geometrię zoptymalizowaną czy nie?

Aby umożliwić wydaje odczytywanie informacji z~takiej kontekstowej treści, zaimplementowałem
  obiekt \texttt{Parser} w~formie automatu skończonego.
Angielski odpowiednik tego terminu \--- \textit{finite state machine} \--- zdradza więcej
  informacji o~naturze tego modelu.
Jest to rodzaj \enquote{maszyny}, która może znajdować się w~pewnej skończonej liczbie stanów
  i~reagować na~bodźce w~sposób zależny od~tego, w~jakim stanie się akurat znajduje.
W~przypadku obiektu \texttt{Parser} bodźcami są informacje odczytywane z~przetwarzanego pliku,
  a~stany są tożsame z~kontekstem, w~jakim te informacje się znajdują.
Taki sposób przetwarzania danych pozwala nie tylko \enquote{zrozumieć} znaczenie odczytanych
  danych, ale również zminimalizować liczbę testów, które należy przeprowadzić w~danym momencie.

Przykładowo \--- jeśli \texttt{Parser} napotyka frazę \enquote{\texttt{Optimization completed}},
  to wejdzie w~stan \enquote{odczytywania zoptymalizowanej geometrii}.
Będzie wiedział, że ciągi liczb, które odczytuje, są reprezentacją atomów w~przestrzeni,
  i~że ma zapisać je w~pamięci jako geometrię optymalną właśnie.
Będzie wiedział też, że nie musi sprawdzać czy w~danym fragmencie znajduje się informacja
  o~wartości energii swobodnej Gibbsa, ani o~przejściach elektronowych w~stanie wzbudzonym.
Zacznie szukać tych innych treści dopiero gdy opuści obecny stan pod wpływem konkretnego bodźca
  \--- w~przykładowej sytuacji byłaby to pusta linia po~koordynatach ostatniego z~atomów.

\subsection{Analiza rozkładem Boltzmanna}\label{implementation:boltzmann}
Symulacji finalnego widma spektroskopowego, \tesliper{} dokonuje, uśredniając widma teoretyczne,
  obliczone dla każdego z~konformerów.
Aby jak najlepiej odwzorować dane eksperymentalne, liczy średnią ważoną tych widm, za~wagę biorąc
  powszechność występowania poszczególnych konformerów w~badanym systemie.
Liczbowo wyrazić ją można jako procentową zawartość każdego z~możliwych konformerów w~mieszaninie,
  inaczej mówiąc jako wkład w~populację.
Populacja konformerów może natomiast być oszacowana na~podstawie rozkładu Boltzmanna\sidenote{%
  Nazywanego też rozkładem Gibbsa. Boltzmann zaproponował go po~raz pierwszy podczas swoich
    studiów nad mechaniką statystyczną gazów w~równowadze cieplnej (\colorcite{boltzmann68}),
    ale w~formie używanej obecnie został wyrażony przez Gibbsa właśnie (\colorcite{gibbs02}).
}.

Jest to rozkład prawdopodobieństwa, opisujący szansę na~znalezienie układu w~określonym stanie,
  na~podstawie energii tego stanu i~temperatury układu.
Prawdopodobieństwo obsadzenia stanu rośnie w~nim wykładniczo wraz ze~zmianą tych parametrów,
  zgodnie z~poniższym równaniem.

\begin{equation}
  p_i \propto e^{-E_i/kT}
\end{equation}

$p_i$ w~tym równaniu jest wspomnianym prawdopodobieństwem występowania układu w~stanie $i$,
  $E_i$ jest wartością energii tego stanu wyrażona w~\si{\kcalpm},
  $k = \SI{0.0019872041}{\kcalpm\per\kelvin}$ to stała Boltzmanna,
  a~$T$ to temperatura układu w~Kelvinach.
Symbol $\propto$ oznacza proporcjonalność.
Stosunek prawdopodobieństw dwóch stanów nazywany jest współczynnikiem Boltzmanna tych stanów
  i~wyrażony jest poniższym równianinem, a~zatem jest zależny przede wszystkim od~różnicy
  energii między stanami.

\begin{equation}
  B^a_b = \frac{p_a}{p_b} = e^{(E_b-E_a)/kt}
\end{equation}

W~przypadku zastosowania tego rozkładu do~opisu konformerów, pod prawdopodobieństwem $p_i$
  możemy rozumieć szansę, że losowo wybrana cząsteczka badanego związku będzie miała
  konformację $i$.
A~zatem współczynnik Boltzmanna pokazuje na~ile bardziej prawdopodobne jest, że cząsteczka
  ta przybierze raczej konformację $a$ niż konformację $b$\sidenote{%
    Zakładając, że $a$ ma niższą energię niż $b$.}.
Znając wartość tego współczynnika względem konformeru o~najniższej energii dla każdego
  z~konformerów można znaleźć bezwzględną wartość prawdopodobieństwa występowania cząsteczki
  w~formie każdego konformeru.
W~poniższym równaniu $B^i_0$ oraz $B^j_0$ są właśnie współczynnikami Boltzmanna względem
  konformeru o~najniższej energii, oznaczonego umownie stanem $0$.

\begin{equation}
  p_i = \frac{B^i_0}{\sum\limits^N_{j=0}B^j_0}
\end{equation}

Jak obliczeń tych można dokonać za~pomocą kodu Pythona pokazuje \cref{lst:boltzmann}.
Po~przygotowaniu pustych list do~przechowania wyników obliczeń w~linii 4 oraz ustaleniu
  wartości stałych w~danych warunkach w~liniach 5 i~6, przedstawiam właściwy algorytm,
  wykonany w~trzech kolejnych pętlach \lstinline|for ... in ...|\ .
W~pierwszej pętli obliczam różnicę energii każdego z~danych konformerów względem konformeru
  o~najniższej energii, przechowując je w~liście \texttt{deltas}.
Na~podstawie tych wartości obliczam następnie wartości odpowiednich współczynników Boltzmanna,
  korzystając z~funkcji eksponencjonalnej \texttt{math.exp()}, dostępnej w~bibliotece
  standardowej Pythona.
Wartości współczynników sumuję w~linii 15, po~czym używam tej sumy do~obliczenia ułamka
  populacji, jaki stanowi każdy z~konformerów w~ostatnich liniach algorytmu.
Wynik przechowuję w~liście \texttt{populations}.

\begin{listing}
  \begin{lstlisting}[emph={energies, temperature, BOLTZMANN_CONSTANT}]
    # DANE: energies, temperature
    # STAŁE: BOLTZMANN_CONSTANT
    # puste listy na przechowanie wyników
    deltas, factors, populations = [], [], []
    # wartości stałe w danych warunkach
    energy_zero = min(energies)
    kt = temperature * BOLTZMANN_CONSTANT
    # właściwy algorytm
    for energy in energies:
        # oblicz różnice energii
        deltas += [energy - energy_zero]
    for delta in deltas:
        # oblicz współczynniki Boltzmanna
        factors += [math.exp(-delta / kt)]
    factors_sum = sum(factors)
    for factor in factors:
        # oblicz populacje konformerów
        populations += [factor / factors_sum]
  \end{lstlisting}
  \caption{
    Algorytm obliczania populacji konformerów na~podstawie rozkładu Boltzmanna.
    Po~jego wykonaniu, zmienna \lstinline|populations| zawiera listę ułamkowych wartości populacji
      każdego z~konformerów wziętych pod uwagę, sumujących się do~jedności.
    Algorytm zakłada, że dostarczona zostanie lista wartości energii tych konformerów jako
      zmienna \lstinline|energies| oraz temperatura układu jako zmienna \lstinline|temperature|.
    Zakłada także, że znana wartość stałej Boltzmanna przechowywana jest w~zmiennej
      \lstinline|BOLTZMANN_CONSTANT|.
    Wyrażenie \lstinline|lista += [wartość]| oznacza dodanie danej wartości na~koniec listy.
  }\label{lst:boltzmann}
\end{listing}

\begin{listing}
  \begin{lstlisting}[emph={energies, temperature, BOLTZMANN_CONSTANT}]
    # DANE: energies, temperature
    # STAŁE: BOLTZMANN_CONSTANT
    import numpy as np  # przygotuj paczkę NumPy
    kt = temperature * BOLTZMANN_CONSTANT
    deltas = np.array(energies) - np.min(energies)
    factors = np.exp(-deltas / kt)
    populations = factors / factors.sum()
  \end{lstlisting}
  \caption{
    Rzeczywisty sposób obliczania populacji konformerów na~podstawie rozkładu Boltzmanna
      przy użyciu paczki NumPy do~wydajnych obliczeń numerycznych na~tablicach liczb.
    Obowiązują takie same założenia jak opisałem w~\protect\cref{lst:boltzmann}.
  }\label{lst:boltzmann-numpy}
\end{listing}

\subsection{Metoda porównania geometrii}\label{implementation:rmsd}
\subsection{Obliczanie symulowanego widma}\label{implementation:spectra}
