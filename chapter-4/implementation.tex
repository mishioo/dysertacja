\section{Detale implementacji}\label{tesliper:implementation}
\subsection{Wybór języka programowania}\label{implementation:language}
\subsection{Ogólna architektura kodu}\label{implementation:architecture}
Program \tesliper{} ma strukturę obiektową\sidenote{%
    Dla ścisłości muszę tu zaznaczyć, że \tesliper{} nie jest w~pełni utrzymany w~stylu obiektowym.
    Python nie wymusza konkretnego paradygmatu programowania,
      pozwalając mieszać ze~sobą różne wzorce.
    Te jego części, które odpowiadają za~przeliczanie konkretnych danych w~inne \---
      bez uwzględniania kontekstu wcześniejszych działań \--- są napisane w~formie niezależnych
      funkcji, czerpiąc częściowo z~paradygmatu programowania funkcjonalnego.
    Takie podejście jest typowe dla~programów napisanych w~języku Python, i~można traktować
      je jako szczegół w~organizacji kodu.
    W~związku z~tym nie opisuję tej kwestii bardziej drobiazgowo.
  }, to znaczy wykorzystuje konstrukcje zwane obiektami,
  które zawierają zarówno dane, jak i~kod odpowiedzialny za~działania.
Ten styl programowania próbuje do~pewnego stopnia naśladować rzeczywisty świat \---
  obiekty mogą \enquote{posiadać} i~\enquote{robić}; reprezentują rzeczy lub idee.
Gdy obiekty \enquote{posiadają}, posiadane przez nich dane nazywa się atrybutami,
  natomiast możliwe działania obiektów są reprezentowane przez fragmenty kodu nazywane metodami.

Obiekty mogą posiadać inne obiekty \--- taka relacja nazywana jest kompozycją.
Innym rodzajem relacji między obiektami jest dziedziczenie.
Mówi się o~nim, gdy obiekt implementuje swoją funkcjonalność na~bazie innego obiektu.
Po~obiektach, będących w~relacji dziedziczenia względem tego samego \enquote{rodzica},
  można spodziewać się, że realizują podobne zadania i~\--- zależnie od~kontekstu \--- 
  można ich używać zamiennie.

Charakterystyczną cechą obiektów jest to, że mają pewien czas istnienia w~obrębie używającego ich
  programu komputerowego \--- są tworzone w~konkretnym celu i~przechowywane na~czas jego realizacji.
Programista, pisząc kod zorientowany obiektowo, przygotowuje szablony tworzenia obiektów,
  nazywane klasami.
Konkretny obiekt stworzony przez program na~podstawie takiego szablonu to egzemplarz klasy.

\begin{figure}
  \includesvg{class-diagram}
  \caption{
    Schematyczne przedstawienie architektury programu \tesliper{} oraz przepływu danych,
      począwszy od~plików wynikowych programu Gaussian do~zapisu przetworzonych danych
      na~dysk twardy.
    Legenda po~lewej stronie przybliża znaczenie poszczególnych elementów.
  }
  \label{fig:class-diagram}
\end{figure}
\subsection{Sposób przetwarzania danych}\label{implementation:parsing}
\subsection{Analiza rozkładem Boltzmanna}\label{implementation:boltzmann}
\subsection{Metoda porównania geometrii}\label{implementation:rmsd}
\subsection{Obliczanie symulowanego widma}\label{implementation:spectra}
