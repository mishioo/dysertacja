\chapter{Program komputerowy tesliper}\label{chapter:tesliper}
Symulacja widm optycznych związków organicznych wymaga analizy dużych ilości danych.
W~przypadku prób opisanych w~poprzednim rozdziale\sidenote{W~sekcji \secref{stereo:galacto}.}
  analiza konformacyjna obydwu badanych diastereomerów\sidenote{
    I~to nie nadwyraz dokładna analiza, gdyż z~trudności wynikających z~labilności układu
      wiedziałem zwczasu \--- nie zdawałem sobie jedynie sprawy z~ich skali.
  } prowadziła do~zapisania 3316 różnych struktur, a~wygenerowane w~procesie modelowania
  dane miały sumaryczną objętość około 16 gigabajtów.
Podczas~ich przetwarzania korzystałem ze~swoich zdolności programistycznych, automatyzując
  znaczną część pracy.
Ręczna analiza takiej ilości danych nie byłaby możliwa do~przeprowadzenia w~rozsądnym czasie.

Muszę zaznaczyć, że przypadek tysięcy konformerów wykorzystanych do~modelowania jest
  ekstremalny, a~tym samym niezbyt reprezentacyjny.
Problem czasochłonności i~mozolności procesu pozostaje jednak w~mocy,
  nawet przy typowej ilości materiału.
Ręczne sprawdzenie, porównanie, selekcja, ekstrakcja danych i~ich przeliczenie z~choćby
  kilkudziesięciu plików wynikowych programu do~obliczeń kwantowo-chemicznych wymaga
  niemało pracy.
Chcąc ułatwić to zadanie innym badaczom, postanowiłem uporządkować i~usystematyzować
  zestaw skryptów, których używałem podczas prac przy modelowaniu \--- wynikiem tego jest
  program komputerowy \texttt{tesliper}\sidenote{
    Nazwa programu jest skrótowcem angielskiej frazy \textit{theoretical spectroscopist's
    little helper}, czyli, w~wolnym tłumaczeniu, \enquote{mały pomocnik spektroskopisty-teoretyka}.
  }, który opisuję w~tej części niniejszej dysertacji.

% \section{Korzystanie z~programu}\label{tesliper:use}
% \subsection{Interfejs graficzny}\label{tesliper:use:GUI}
% \subsection{Interfejs linii komend}\label{tesliper:use:CLI}
% \subsection{Interfejs programistyczny}\label{tesliper:use:API}
% \section{Implementacja}\label{tesliper:implementation}
% \subsection{Wybór języka programowania}\label{implementation:language}


