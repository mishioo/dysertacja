\chapter{Program komputerowy tesliper}\label{chapter:tesliper}
Symulacja widm spektroskopowych związków organicznych wymaga analizy dużych ilości danych.
W~przypadku prób opisanych w~poprzednim rozdziale\sidenote{W~sekcji \secref{stereo:galacto}.}
  analiza konformacyjna obydwu badanych diastereoizomerów\sidenote{
    I~to nie nadwyraz dokładna analiza, gdyż z~trudności wynikających z~labilności układu
      wiedziałem zwczasu \--- nie zdawałem sobie jedynie sprawy z~ich skali.
  } prowadziła do~zapisania 3316 różnych struktur, a~wygenerowane w~procesie modelowania
  dane miały sumaryczną objętość około 16 gigabajtów.
Podczas~ich przetwarzania korzystałem ze~swoich zdolności programistycznych, automatyzując
  znaczną część pracy.
Ręczna analiza takiej ilości danych nie byłaby możliwa do~przeprowadzenia w~rozsądnym czasie.

Muszę zaznaczyć, że przypadek tysięcy konformerów wykorzystanych do~modelowania jest
  ekstremalny, a~tym samym niezbyt reprezentacyjny.
Problem czasochłonności i~mozolności procesu pozostaje jednak w~mocy,
  nawet przy typowej ilości materiału.
Ręczne sprawdzenie, porównanie, selekcja, ekstrakcja danych i~ich przeliczenie z~choćby
  kilkudziesięciu plików wynikowych programu do~obliczeń kwantowo-chemicznych wymaga
  niemało pracy.
Chcąc ułatwić to zadanie innym badaczom, postanowiłem uporządkować i~usystematyzować
  zestaw skryptów, których używałem podczas prac przy modelowaniu \--- wynikiem tego jest
  program komputerowy \texttt{tesliper}\sidenote{
    Nazwa programu jest skrótowcem angielskiej frazy \textit{theoretical spectroscopist's
    little helper}, czyli, w~wolnym tłumaczeniu, \enquote{mały pomocnik spektroskopisty-teoretyka}.
  }, który opisuję w~tej części niniejszej dysertacji\sidenote{
    Oprócz tego, przygotowałem też jego instrukcję użycia i~obszerną dokumentację,
      dostępne w~języku angielskim pod adresem internetowym
      \href{https://tesliper.readthedocs.io/}{www.tesliper.readthedocs.io}.
    Załączam je również, w~formie pliku PDF, do~materiałów dodatkowych.
  }.

\subimport{./}{essence}
\subimport{./}{implementation}
