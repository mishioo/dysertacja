\section{Istota programu}\label{tesliper:essence}
Zadaniem stworzonego przeze mnie oprogramowania jest wsadowe przetwarzanie plików wynikowych
  programu Gaussian\sidecite{gaussian09,gaussian16}, w~szczególności takich plików, które zawierają
  dane pochodzące z~optymalizacji struktury oraz obliczania jej aktywności optycznej\sidenote{
    Gaussian nie jest jedynym programem zdolnym do~przeprowadzenia takich obliczeń,
      jednak ograniczam się do~niego, gdyż jest on powszechnie stosowany.}.
Finalnym efektem działania \tesliper{}a jest symulowane widmo spektroskopowe badanego związku,
  powstałe przez uśrednienie teoretycznych widm każdego z~wybranych konformerów.
Program asystuje w~realizacji tych kroków procesu symulacji, które wymagają analizy danych
  ze~wspomnianych plików oraz w~przygotowaniu kolejnego etapu obliczeń.

\subsection{Proces symulacji widm}\label{essence:simulation}
Jak wywnioskować można z~poprzednich akapitów, symulacja widma spektroskopowego jest
  procesem kilkuetapowym.
Dokładnie opisują go \citeauthor{pescitelli16}, podając wskazówki i~sugerując dobrą
  praktykę w~tej pracy.\sidecite{pescitelli16}.
Na~potrzeby dyskusji zawartej w~niniejszej dysertacji poniższe ogólne przedstawienie
  stosowanej metodologi wydaje mi się wystarczające.

Punktem wyjścia jest proponowana konfiguracja absolutna badanego związku.
Należy poddać ją wnikliwej analizie konformacyjnej, to znaczy znaleźć możliwie dużo
  potencjalnie stabilnych konformerów związku o~zakładanej strukturze.
Dokonuje się tego zwykle za~pomocą dedykowanych programów komputerowych, operujących w~obrębie
  modelu mechaniki molekularnej.
Ich geometrie optymalizuje się korzystając z~metod \gls{dft}, a~później porównuje się ze~sobą,
  odrzucając duplikaty\sidenote{
    Zadarza się bowiem, że różne struktury uzyskane w~wyniku analizy konformacyjnej zostaną
      zoptymalizowane do~takiej samej geometrii.}.

Następnie należy przeprowadzić obliczenia częstości, aby potwierdzić, że uzyskane
  struktury są punktami stacjonarnymi, oraz obliczenia energii, niezbędnych do~przeprowadzenia
  analizy rozkładu Boltzmanna\sidenote{
    Prowadzącego do~oszacowania udziału każdego z~konforemrów w~populacji.}.
Jako, że kolejny etap obliczeń jest kosztowny\sidenote{W~sensie wymaganego czasu obliczeniowego.},
  zwykle liczbę wziętych do~niego konformerów ogranicza się, odrzucając te, które mają marginalny
  udział w~populacji.
Wtedy dopiero poddaje się pozostałe konformery obliczeniom wybranej aktywności spektroskopowej
  w~funkcji częstości albo długości falowej\sidenote{Zależnie od~typu symulowanego widma.}.
Wartości te nie są jeszcze symulowanym widmem.
Jego otrzymanie wymaga wykonania dalszej pracy \--- przeliczenia ich na~intensywność pików,
  symulacji kształtu pików, i~w~końcu uśrednienia uzyskanych w~ten sposób teoretycznych widm
  poszczególnych konformerów.

\begin{figure*}
  \includesvg{simulation-flow}
  \caption{
    Schemat przedstawiający proces symulacji widma spektroskopowego metodami numerycznymi.
    Zaznaczyłem na~nim etapy, które mogą zostać wykonane w~programie Gaussian i~autorskim
      programie \tesliper{}.
  }\label{fig:simulation-flow}
\end{figure*}

\Cref{fig:simulation-flow} przedstawia schematycznie ten proces, wyszczególniając kolejne etapy.
Jedynie niektóre z~tych etapów można wykonać za~pomocą programu do~obliczeń kwantowo-chemicznych,
  o~realizację pozostałych musi zadbać badacz.
Na~wspomnianym schemacie zaznaczam kolorowymi obszarami etapy, które mogą być zrealizowane
  w~programie Gaussian oraz te, w~których można użyć opisywanego programu \tesliper{}.

\subsection{Podobne narzędzia}\label{essence:simmilar}
\subsection{Funkcje programu}\label{essence:features}
