\section{Istota programu}\label{tesliper:essence}
Zadaniem stworzonego przeze mnie oprogramowania jest wsadowe przetwarzanie plików wynikowych
  programu Gaussian\sidecite{gaussian09,gaussian16}, w~szczególności takich plików, które zawierają
  dane pochodzące z~optymalizacji struktury oraz obliczania jej aktywności optycznej\sidenote{
    Gaussian nie jest jedynym programem zdolnym do~przeprowadzenia takich obliczeń,
      jednak ograniczam się do~niego, gdyż jest on powszechnie stosowany.}.
Finalnym efektem działania \tesliper{}a jest symulowane widmo spektroskopowe badanego związku,
  powstałe przez uśrednienie teoretycznych widm każdego z~wybranych konformerów.
Program asystuje w~realizacji tych kroków procesu symulacji, które wymagają analizy danych
  ze~wspomnianych plików oraz w~przygotowaniu kolejnego etapu obliczeń.

\subsection{Proces symulacji widm}\label{essence:simulation}
\subsection{Podobne narzędzia}\label{essence:simmilar}
\subsection{Funkcje programu}\label{essence:features}
