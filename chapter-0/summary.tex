\chapter{Streszczenie}\label{chapter:summary:pl}
\marginnote[-8em]{\textsf{%
  \nohyphenation\fontsize{15}{17}\selectfont%
  \raggedleft\makebox[\linewidth][s]{\textsc{Michał M. Więcław}}\\\vspace{0.5pc}%
  \fontsize{10}{12}\selectfont\noindent\raggedright%
  \textit{Reduktywna aktywacja amidów jako metoda otrzymywania sfunkcjonalizowanych amin:
    badania syntetyczne wspomagane metodami komputerowymi}\\\vspace{0.5pc}%
  \textsc{\fontsize{8}{10}\selectfont%
    \noindent\raggedleft\textbf{Praca doktorska}\\
    \noindent przygotowana pod kierunkiem\\
    \noindent \mbox{prof. dr. hab. Bartłomieja Furmana}\\
  }
}}
\begin{marginscheme}
  \includesvg{summary-pl}
\end{marginscheme}
\begin{marginfigure}
  \includesvg{banner}
\end{marginfigure}
\begin{marginlisting}
  \begin{lstlisting}[emph={recursive,fmt}]
    # przykład użycia programu
    from tesliper import Tesliper
    t = Tesliper("/input/path")
    t.extract(recursive=True)
    t.calculate_spectra()
    t.export_spectra(fmt="csv")
  \end{lstlisting}
\end{marginlisting}


\chapter{Summary}\label{chapter:summary:en}
\marginnote[-8em]{\textsf{%
  \nohyphenation\fontsize{15}{17}\selectfont%
  \raggedleft\makebox[\linewidth][s]{\textsc{Michał M. Więcław}}\\\vspace{0.5pc}%
  \fontsize{10}{12}\selectfont\noindent\raggedright%
  \textit{Reductive activation of amides as a method of obtaining functionalized amies:
    computer-aided synthetic investigations}\\\vspace{0.5pc}%
  \textsc{\fontsize{8}{10}\selectfont%
    \noindent\raggedleft\textbf{Doctoral thesis}\\
    \noindent under supervision of\\
    \noindent prof. dr hab. Bartłomiej Furman\\
  }
}}
\begin{marginscheme}
  \includesvg{summary-en}
\end{marginscheme}
\begin{marginfigure}
  \includesvg{banner}
\end{marginfigure}
\begin{marginlisting}
  \begin{lstlisting}[emph={recursive,fmt}]
    # program's usage example
    from tesliper import Tesliper
    t = Tesliper("/input/path")
    t.extract(recursive=True)
    t.calculate_spectra()
    t.export_spectra(fmt="csv")
  \end{lstlisting}
\end{marginlisting}

