\chapter{Streszczenie}\label{chapter:summary:pl}
\marginnote[-8em]{\textsf{%
  \nohyphenation\fontsize{15}{17}\selectfont%
  \raggedleft\makebox[\linewidth][s]{\textsc{Michał M. Więcław}}\\\vspace{0.5pc}%
  \fontsize{10}{12}\selectfont\noindent\raggedright%
  \textit{Reduktywna aktywacja amidów jako metoda otrzymywania sfunkcjonalizowanych amin:
    badania syntetyczne wspomagane metodami komputerowymi}\\\vspace{0.5pc}%
  \textsc{\fontsize{8}{10}\selectfont%
    \noindent\raggedleft\textbf{Praca doktorska}\\
    \noindent przygotowana pod kierunkiem\\
    \noindent \mbox{prof. dr. hab. Bartłomieja Furmana}\\
  }
}}
\begin{marginscheme}
  \includesvg{summary-pl}
\end{marginscheme}
\begin{marginfigure}
  \includesvg{banner}
\end{marginfigure}
\begin{marginlisting}
  \begin{lstlisting}[emph={recursive,fmt}]
    # przykład użycia programu
    from tesliper import Tesliper
    t = Tesliper("/input/path")
    t.extract(recursive=True)
    t.calculate_spectra()
    t.export_spectra(fmt="csv")
  \end{lstlisting}
\end{marginlisting}

Niniejsza dysertacja poświęcona jest w~głównej mierze eksperymentom syntetycznym z~zakresu
  chemii amidów, ale istotna jej część dotyczy badań przeprowadzonych metodami obliczeniowymi
  oraz programowania komputerowego, które wykonałem z~przyczyny tych badań.
Moim celem było sprawdzenie, jak współcześnie dostępne metody reduktywnej aktywacji
  amidów sprawdzają się w~roli narzędzi do~syntezy sfunkcjonalizowanych amin w~złożonych
  układach reakcyjnych.
Bazując na~wnikliwych studiach literatury fachowej wyłoniłem cztery takie metody \---
  wykorzystującą bezwodnik triflowy, odczynnik Schwartza, kompleks Vaski oraz kompleks van der Enta.

Badania rozpocząłem używając \textbeta{}-amidoestrów jako substratów w~prostych funkcjonalizacjach.
Przemiany te okazały się stanowić większe wyzwanie, niż oczekiwałem \--- udało mi ię ich dokonać
  jedynie przy użyciu odczynnika Schwartza i~z~przeciętnym sukcesem.
Lwią część prac poświęciłem zastosowaniu reduktywnej aktywacji laktamów wywiedzionych z~cukrów
  prostych w~wariancie wieloskładnikowej reakcji Ugiego.
Otrzymałem na~tej drodze kilka tetrazolowych pochodnych iminocukrów z~doskonałą
  diastereoselektywnością.

Podczas tych eksperymentów zaobserwowałem po~raz pierwszy, że reakcja azydo-Ugiego może
  przebiegać w~aprotycznym środowisku.
To zaskakujące odkrycie skłoniło mnie do~pochylenia się nad przebiegiem omawianej reakcji.
Zaproponowałem jej mechanizm, opierając się na~poczynionych obserwacjach oraz badaniach
  \textit{in silico}, prowadzonych metodami DFT.

Ze~wsparcia metod komputerowych korzystałem również podczas ustalania konfiguracji absolutnej
  jednego z~produktów.
Wyzwania, którym musiałem przy tym sprostać były dla mnie przyczynkiem do~napisania programu
  komputerowego \tesliper{}, ułatwiającego analizę wyników obliczeń kwantowo-chemicznych \---
  przede wszystkim związanych z~optymalizacją struktury i~symulacją widm spektroskopowych.
Stworzony przeze mnie program udostępniłem na~licencji otwartego
  oprogramowania, a~instrukcja jego użytkowania umieściłem w~Internecie.
W~niniejszej dysertacji opisuję jego strukturę i~podstawę funkcjonowania.


\chapter{Summary}\label{chapter:summary:en}
\marginnote[-8em]{\textsf{%
  \nohyphenation\fontsize{15}{17}\selectfont%
  \raggedleft\makebox[\linewidth][s]{\textsc{Michał M. Więcław}}\\\vspace{0.5pc}%
  \fontsize{10}{12}\selectfont\noindent\raggedright%
  \textit{Reductive activation of amides as a method of obtaining functionalized amies:
    computer-aided synthetic investigations}\\\vspace{0.5pc}%
  \textsc{\fontsize{8}{10}\selectfont%
    \noindent\raggedleft\textbf{Doctoral thesis}\\
    \noindent under supervision of\\
    \noindent prof. dr hab. Bartłomiej Furman\\
  }
}}
\begin{marginscheme}
  \includesvg{summary-en}
\end{marginscheme}
\begin{marginfigure}
  \includesvg{banner}
\end{marginfigure}
\begin{marginlisting}
  \begin{lstlisting}[emph={recursive,fmt}]
    # program's usage example
    from tesliper import Tesliper
    t = Tesliper("/input/path")
    t.extract(recursive=True)
    t.calculate_spectra()
    t.export_spectra(fmt="csv")
  \end{lstlisting}
\end{marginlisting}

