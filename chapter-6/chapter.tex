\chapter{Detale techniczne}\label{chapter:experimental}

\section{Ogólne procedury aktywacji amidów}\label{experimental:activation}
\subsubsection{Procedura wykorzystująca odczynnik Schwartza}\label{experimental:activation:schwartz}
Do wygrzanego w~płomieniu palnika i~wypełnionego gazem obojętnym naczynia Schlenka odważyłem
\SI{0.2}{\mmol} odczynnika Schwartza oraz \SI{0.2}{\mmol} amidu\sidenote[][-2\baselineskip]{
  W~przypadku amidu będącego cieczą lub olejem dodawałem przygotowany osobno roztwór amidu
    do~zawiesiny odczynnika Schwartza.
},
  dodałem \SI{1.0}{\mL} suchego \gls{thf}\sidenote{
    W przypadku prowadzenia reakcji w~obniżonej temperaturze dodałem rozpuszczalnik do~samego
      odczynnika Schwartza, ochłodziłem w~łaźni izopropanol\--suchy lód, i~dopiero dodałem amid.
  }.
Mieszałem do~sklarowania roztworu.
W~przepływie argonu dodałem \SI{0.4}{\mmol} wybranego kwasu oraz \SI{0.2}{\mmol} nukleofila\sidenote{
  W przypadku allilotributylocyny użyłem \SI{0.6}{\mmol}.
}.
Metodę terminacji reakcji oraz oczyszczania produktów dobierałem do~każdego przypadku osobno.

\subsubsection{Procedura wykorzystująca kompleks Vaski}\label{experimental:activation:vaska}
Do wygrzanego w~płomieniu palnika i~wypełnionego gazem obojętnym naczynia Schlenka odważyłem
  \SI{0.002}{\mmol} \ch{IrCl(CO)(PPh3)2} oraz \SI{0.2}{\mmol} amidu.
Rozpuściłem w~\SI{2.0}{\ml} suchego toluenu.
Dodałem \SI{0.4}{\mmol} \gls{tmds} i~mieszałem przez noc.
Po tym czasie dodałem \SI{0.6}{\mmol} allilotributylocyny oraz \SI{0.4}{\mmol} \ch{BF3.OEt2}.
Mieszałem jeszcze przez \SI{3}{\day}, po~czym wylałem na~\SI{4}{\ml} \ch{NaHCO3_{(aq)}},
  ekstrahowałem \SI[product-units = single]{2 x 4}{\mL} \ch{Et2O}, zebrane frakcje organiczne
  przemyłem \SI[product-units = single]{3 x 5}{\mL} \SI{10}{\percent} \ch{NH4F_{(aq)}},
  suszyłem \ch{MgSO4}, po~czym odparowałem rozpuszczalnik za~pomocą wyparki rotacyjnej.

\subsubsection{Procedura wykorzystująca kompleks van der Enta}\label{experimental:activation:van-der-ent}
Do wygrzanego w~płomieniu palnika i~wypełnionego gazem obojętnym naczynia Schlenka odważyłem
  \SI{0.002}{\mmol} \ch{[Ir(coe)2Cl]2} oraz \SI{0.2}{\mmol} amidu.
Rozpuściłem w~\SI{2.0}{\ml} \ch{CH2Cl2}.
Dodałem \SI{0.4}{\mmol} \ch{Et2SiH2} i~mieszałem przez noc.
Po tym czasie dodałem \SI{0.6}{\mmol} allilotributylocyny oraz \SI{0.4}{\mmol} \ch{BF3.OEt2}.
Mieszałem jeszcze przez \SI{3}{\day}, po~czym wylałem na~\SI{5}{\ml} \ch{NaHCO3_{(aq)}},
  ekstrahowałem \SI[product-units = single]{2 x 4}{\mL} \ch{Et2O}, zebrane frakcje organiczne
  przemyłem \SI[product-units = single]{3 x 5}{\mL} \SI{10}{\percent} \ch{NH4F_{(aq)}},
  suszyłem \ch{MgSO4}, po~czym odparowałem rozpuszczalnik za~pomocą wyparki rotacyjnej.

\subsubsection{Procedura wykorzystująca bezwodnik triflowy}\label{experimental:activation:triflic}
Do wygrzanego w~płomieniu palnika i~wypełnionego gazem obojętnym naczynia Schlenka odważyłem
  \SI{0.2}{\mmol} amidu i~rozpuściłem w~\SI{2.0}{\ml} \ch{CH2Cl2}.
Dodałem \SI{0.24}{\mmol} \iupac{2-fluoropirydyny}, po czym ochłodziłem do~\SI{0}{\degC}
  w~łaźni izopropanol-suchy lód i~dodałem \SI{0.22}{\mmol} \ch{Tf2O}.
Mieszałem przez \SI{20}{\min}, po~czym dodałem \SI{0.22}{\mmol} \ch{Et3SiH} w~\SI{0}{\degC},
  a~następnie pozwoliłem mieszaninie ogrzać się do temperatury pokojowej i~mieszałem przez noc.
Ponownie ochłodziłem mieszaninę do~\SI{0}{\degC}, dodałem \SI{0.3}{\mmol} \ch{BF3.Et2O},
  a~po~\SI{40}{\min} dodałem jeszcze \SI{0.6}{\mmol} allilotributylocyny.
Mieszałem przez noc, pozwalając by mieszanina ogrzała się do temperatury pokojowej.
Po~tym czasie wylałem na~\SI{5}{\ml} \ch{NaHCO3_{(aq)}},
  ekstrahowałem \SI[product-units = single]{3 x 4}{\mL} \ch{CH2Cl2}, zebrane frakcje organiczne
  przemyłem \SI[product-units = single]{3 x 5}{\mL} \SI{10}{\percent} \ch{NH4F_{(aq)}},
  suszyłem \ch{MgSO4}, po~czym odparowałem rozpuszczalnik za~pomocą wyparki rotacyjnej.

\subimport{./}{amidoesters}
\subimport{./}{sugars}
\subimport{./}{calculations}
