\section{Badanie mechanizmu reakcji metodami obliczeniowymi}\label{experimental:mech}

% \renewcommand{\bottomfraction}{0.5}  % to allow a bigger bottom float
\begin{table}[b!]
  \caption{
    Podane w~jednostkach Hartree wartości sum energii elektronowych i~punktu zerowego,
      tak jak je podaje program Gaussian~09,
      obliczone dla struktur zaangażowanych w~przebieg badanej reakcji przy użyciu teorii
      na~poziomie B3LYP/Def2TZVP oraz z~uwzględnieniem empirycznej poprawki dyspersyjnej GD3.
    \textsuperscript{a}~Z~zastosowaniem modelu solwatacyjnego PCM dla \gls{thf}.
  }\label{tab:energies}
  \begin{tabular}{ c l S S }
    \toprule
    \textnumero & Struktura & {W~próżni} & {W~rozpuszczalniku \textsuperscript{a}} \\
    \midrule
    \rownumber & \refcmpd{int-1-a} & -1221.100727 & -1221.110627 \\
    \rownumber & \refcmpd{ts-1-a} & -1221.056261 & -1221.074579 \\
    \rownumber & \refcmpd{int-1-b} & -1221.097576 & -1221.106030 \\
    \rownumber & \refcmpd{ts-1-b} & -1220.997720 & -1221.010227 \\
    \rownumber & \refcmpd{int-2-a} & -1221.095618 & -1221.106694 \\
    \rownumber & \refcmpd{int-2-b} & -1220.996618 & -1221.009687 \\
    \rownumber & \refcmpd{int-3} & -250.657212 & -250.661194 \\
    \rownumber & \refcmpd{ts-3} & -824.136544 & -824.174398 \\
    \rownumber & \refcmpd{int-4} & -659.723031 & -659.784677 \\
    \rownumber & \refcmpd{ts-4} & -1098.757589 & -1098.809155 \\
    \rownumber & \refcmpd{int-5} & -1098.761665 & -1098.814285 \\
    \rownumber & \refcmpd{ts-5} & -1263.191306 & -1263.213899 \\
    \rownumber & \refcmpd{int-6} & -1263.263938 & -1263.270157 \\
    \rownumber & \refcmpd{ts-6} & -1263.244227 & -1263.250449 \\
    \rownumber & \refcmpd{int-7} & -1263.282628 & -1263.293985 \\
    \rownumber & \refcmpd{int-8} & -854.597569 & -854.609153 \\
    \rownumber & \ch{PhCH2NC} & -439.032036 & -439.038580 \\
    \rownumber & \ch{TMSN3} & -573.532370 & -573.536386 \\
    \rownumber & \ch{Me3SiOH} & -485.145316 & -485.149910 \\
    \rownumber & \ch{Cp2Zr(OH)Cl} & -970.420570 & -970.431956 \\
    \rownumber & \ch{H2O} & -76.441875 & -76.448012 \\
    \rownumber & \ch{N3-} & -164.290536 & -164.379062 \\
    \bottomrule
  \end{tabular}
\end{table}

Obliczenia związane z~symulacją mechanizmu badanego wariantu reakcji azydo-Ugiego przeprowadziłem
  korzystając z~oprogramowania Gaussian~09\sidecite{gaussian09}.
Konstruując początkowe geometrie stanów stacjonarnych \refcmpd{int-1-a, int-1-b, int-2-a},
  a~także stanu przejściowego \refcmpd{ts-1-a}, wzorowałem się na~strukturach kompleksów
  cyrkonowych zaproponowanych przez Wanga i~in. w~ich badaniach \gls{dft} mechanizmu
  redukcji amidów trzeciorzędowych odczynnikiem Schwartza\sidecite{wang10}.
W~przypadku struktur \refcmpd{ts-4, int-5, ts-5, int-6, ts-6} za~punkt odniesienia posłużyły
  mi geometrie zaprezentowane przez zespół Sharplessa w~ramach ich prac nad ustaleniem mechanizmu
  formowania się tetrazoli w~wyniku reakcji azydków z~nitrylami\sidecite{sharpless02}.
  
Wszystkie struktury zoptymalizowałem używając funkcjonału B3LYP i~stosując bazę LANL2DZ
  do~opisu atomów \ch{Zr} oraz 6-31G(d,p) do~opisu pozostałych atomów,
  z~uwzględnieniem empirycznej poprawki dyspersyjnej GD3.
Potwierdziłem, że proponowane stany przejściowe rzeczywiście łączą postulowane stany stacjonarne,
  wyznaczając ścieżkę reakcji w~obydwu kierunkach\sidenote{%
    W~języku angielskim nazywa się tego typu eksperyment
      \textit{following the intristic reaction coordinate (IRC)}. 
  }.
Geometrie zoptymalizowanych struktur dostępne są w~fw~ormie elektronicznej w~dołączonych
  do~dysertacji materiałach elektronicznych oraz w~repozytorium\sidenote{\repourl}
  z~kodem źródłowym tej pracy w~katalogu \enquote{supplementary}.

Zoptymalizowane struktury poddałem powtórnej analizie, używając większej bazy Def2TZVP.
Porównałem wyniki symulacji w~próżni z~uzyskanymi przy~zastosowaniu modelu solwatacyjnego
  PCM dla \gls{thf}, tak jak został on zaimplementowany w~programie Gaussian~09.
W~pobliskiej \cref{tab:energies} prezentuję wartości energii poszczególnych
  struktur, uzyskanych w~obydwu przypadkach.

\section{Analiza widm chiralooptycznych}\label{experimental:spectra}

Widma \gls{uv} związków \refcmpd{gal-tet.cy,gal-epi-tet.cy} zostały zarejestrowane przy użyciu
  spektrofotometru Jasco~V\-/670 w~\ch{CH3CN}.
Widma te, wraz z~ich symulowanymi odpowiednikami, prezentuję na~\cref{fig:spectra-uv}.
Widma \gls{ecd} również zostały zarejestrowane w~\ch{CH3CN}, ale za~pomocą spektropolarymetru
  Jasco~J\-/815 w~zakresach \SIrange{195}{450}{\nano\meter} oraz \SIrange{225}{450}{\nano\meter}
  w~kuwetach kwarcowych o~drodze optycznej odpowiednio \SIrange{0.02}{2.0}{\centi\meter}.
Widma rejestrowane były przy stężeniu próbki \SI{0.00029}{\molar},
  z~szybkością \SI{100}{\nm\per\minute}, szerokością szczeliny \SI{1}{\nm}, liczbą akumulacji 5
  i~5~punktami na~nanometr.

\begin{figure}
  \begin{tikzpicture}
    \begin{axis}[
      ylabel={$\epsilon\ /\si{\deci\meter\cubed\per\mole\per\centi\meter}$},
      xlabel={$\lambda\ /\si{\nm}$},
      ymin=0, ymax=1.95,
      xmin=180, xmax=400,
      axis lines=left,
      axis line style={-},
      legend style={at={(0.97,0.9)}, anchor={north east}, draw=none},
      width=\textwidth,
      height=13em,
    ]
    
    \addplot[color=wongvermillion]
      table {chapter-3/stereochemistry/gal-uv-data.txt};
    \addlegendentry{\refcmpd{gal-tet.cy}}
      
    \addplot[color=wongblue]
      table {chapter-3/stereochemistry/epi-uv-data.txt};
    \addlegendentry{\refcmpd{gal-epi-tet.cy}}

    \addplot[color=wongpurple]
      table [x expr=\thisrow{x-s-uv}-15, y expr=\thisrow{y-s-uv}/10000]
      {chapter-3/stereochemistry/simulated-data.txt};
    \addlegendentry{Sym. \iupac{\cip{2S}}$\ /10^4$}
    
    \addplot[color=wonggreen]
      table [x expr=\thisrow{x-r-uv}-15, y expr=\thisrow{y-r-uv}/10000]
      {chapter-3/stereochemistry/simulated-data.txt};
    \addlegendentry{Sym. \iupac{\cip{2R}}$\ /10^4$}
      
    \end{axis}
  \end{tikzpicture}

  \caption{
    Zarejestrowane i~symulowane widma \gls{uv} badanych związków
      \refcmpd{gal-tet.cy, gal-epi-tet.cy}.
    Dla lepszej czytelności wartości intensywności poszczególnych widm zostały przeskalowane
      o~współczynnik podany w~legendzie.
  }
  \label{fig:spectra-uv}
\end{figure}

Pierwszym krokiem do~symulacji widm było przeprowadzenie analizy konformacyjnej
  proponowanych struktur związków \refcmpd{gal-tet.cy,gal-epi-tet.cy}, którą wykonałem
  za~pomocą programu CONFLEX\sidecite{conflex00}.
Otrzymane struktury konformerów zoptymalizowałem korzystając z~programu
  Gaussian~09\sidecite{gaussian09} na~poziomie teorii B3LYP/6-31G**.
Zoptymalizowane, najniżej energetyczne struktury (do~\SI{3}{\kcalpm}) wykorzystałem w~obliczeniach
  właściwości chiralooptycznych w~programie Gaussian~16\sidecite{gaussian16}, stosując teorię
  na~poziomie B3LYP/TZVP z~zastosowaniem modelu rozpuszczalnikowego PCM dla \ch{CH3CN}.
Na~podstawie tych właściwości obliczyłem widma symulowane do~porównania z~eksperymentalnymi
  przy użyciu autorskiego programu komputerowego \texttt{tesliper}\sidenote{%
    Program ten opisałem w~niniejszej dysertacji w~rozdziale \secref{chapter:tesliper}.}.
Prezentowane widma zostały uzyskane z~użyciem rozkładu Gaussa o~szerokości \SI{0.15}{\electronvolt}
  do~opisu kształtu pików i~skorygowane hipsochromowo o~\SI{15}{\nano\meter}.
