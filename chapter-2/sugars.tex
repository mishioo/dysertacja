\section{Laktamy wywiedzione z~cukrów}\label{synthesis:sugars}
Jak wspomniałem w~rozdziale \textit{\nameref{chapter:literature}}\sidenote{%
    A dokładnie w~sekcji \secref{literature:schwartz:other}.
  }, z~grupy badawczej, w~której realizowałem badania zawarte w~niniejszej dysertacji,
  pochodzą prace poświęcone reduktywnej aktywacji laktamów wywiedzionych z~cukrów prostych.
Przedstawiony w~nich proces pozwala na~syntezę \textalpha{}-funkcjonalizowanych iminocukrów
  \--- związków nietrywialnych do~otrzymania klasycznymi metodami.
Zainteresowany możliwościami oferowanymi przez tę metodę postanowiłem kontynuować badania nad nią.

Za~punkt wyjścia posłużyło mi zawierające kilka przykładów doniesienie o~połączeniu
  reduktywnej aktywacji odczynnikiem Schwartza z~wieloskładnikową reakcją Ugiego\sidecite{furman15}.
Wykorzystałem jej modyfikację \--- reakcję Jouli{\'e}-Ugiego, aby otrzymać tetrazolowe pochodne
  iminocukrów.
W~dalszej części tekstu opisuję pracę włożoną w~realizację tego przedsięwzięcia,
  wyzwania jakie przy tym napotkałem i~dokonane w~tym procesie obserwacje.
Pokazuję również dalsze przekształcenia otrzymanych związków oraz inne ścieżki,
  którymi próbowałem dotrzeć do~obranego celu.

\subsection{Bioizosteryzm}\label{synthesis:sugars:bioisosterizm}
\subsection{Metody syntezy tetrazoli}\label{synthesis:sugars:synthesis}
