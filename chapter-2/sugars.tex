\section{Laktamy wywiedzione z~cukrów}\label{synthesis:sugars}
Jak wspomniałem w~rozdziale \textit{\nameref{chapter:literature}}\sidenote{%
    A dokładnie w~sekcji \secref{literature:schwartz:our}.
  }, z~grupy badawczej, w~której realizowałem badania zawarte w~niniejszej dysertacji,
  pochodzą prace poświęcone reduktywnej aktywacji laktamów wywiedzionych z~cukrów prostych.
Przedstawiony w~nich proces pozwala na~syntezę \textalpha{}-funkcjonalizowanych iminocukrów
  \--- związków nietrywialnych do~otrzymania klasycznymi metodami.
Zainteresowany możliwościami oferowanymi przez tę metodę postanowiłem kontynuować badania nad nią.

Za~punkt wyjścia posłużyło mi zawierające kilka przykładów doniesienie o~połączeniu
  reduktywnej aktywacji odczynnikiem Schwartza z~wieloskładnikową reakcją Ugiego\sidecite{furman15}.
Wykorzystałem jej modyfikację \--- reakcję Jouli{\'e}-Ugiego, aby otrzymać tetrazolowe pochodne
  iminocukrów.
W~dalszej części tekstu opisuję pracę włożoną w~realizację tego przedsięwzięcia,
  wyzwania jakie przy tym napotkałem i~dokonane w~tym procesie obserwacje.
Pokazuję również dalsze przekształcenia otrzymanych związków oraz inne ścieżki,
  którymi próbowałem dotrzeć do~obranego celu.
Następny rozdział niniejszej dysertacji \--- \textit{\nameref{chapter:numeric}} \---
  jest kontynuacją treści zawartej w~tej sekcji i~zawiera opis prac prowadzonych
  celem lepszego poznania opisywanego tu procesu.

\subsection{Bioizosteryzm i~inne zalety tetrazoli}\label{synthesis:sugars:bioisosterizm}
\begin{marginscheme}
  \includesvg{tetrazoles}
  \caption{
    Izomeryczne struktury tetrazoli.
    Związki \refcmpd{tetrazole.1h, tetrazole.2h} są tautomerami i,~w~przeciwieństwie
      do~izomeru \refcmpd{tetrazole.5h}, są związkami aromatycznymi.
  } \label{sch:tetrazoles}
\end{marginscheme}
Zanim jednak przejdę do~sedna, myślę, że warto pochylić się nad~tetrazolami i~ich miejscem w~chemii.
Te heterocykliczne związki, o~strukturze pokazanej na~\cref{sch:tetrazoles},
  występują w~trzech izomertcznych formach: \iupac{1\H-}, \iupac{2\H-} oraz \iupac{5\H-},
  z~których ostatnia jest niearomatyczna, a~tym samym mniej trwała i~zdecydowanie mniej
  powszechna\sidenote{
    Wg bazy Reaxys około \SI{1.3}{\percent} znanych tetrazoli to \iupac{5\H-tetrazole}
      (dostęp: 13.10.2021).
  }.
Podobnie jak ich trójazotowe analogi, wykazują aktywność organokatalityczną w~różnych
  typach przekształceń.
Można znaleźć doniesienia o~ich zastosowaniu w~reakcji aldolowej\sidecite{hartikka04},
  addycji Michaela\sidecite{chen13}, reakcji Mannicha\sidecite{kumar13},
  czy reakcjach uwodornienia\sidecite{mirabal12}.
