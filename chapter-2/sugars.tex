\section{Synteza tetrazolowych pochodnych iminocukrów}\label{synthesis:sugars}
Jak wspomniałem w~rozdziale \textit{\nameref{chapter:literature}}\sidenote{%
    A dokładnie w~sekcji \secref{literature:schwartz:our}.
  }, z~grupy badawczej, w~której realizowałem studia zawarte w~niniejszej dysertacji,
  pochodzą prace poświęcone reduktywnej aktywacji laktamów wywiedzionych z~cukrów prostych.
Przedstawiony w~nich proces pozwala na~syntezę \textalpha{}-funkcjonalizowanych iminocukrów
  \--- związków nietrywialnych do~otrzymania klasycznymi metodami.
Zainteresowany możliwościami oferowanymi przez tę metodę postanowiłem kontynuować badania nad nią.

Za~punkt wyjścia posłużyło mi, zawierające kilka przykładów, doniesienie o~połączeniu
  reduktywnej aktywacji odczynnikiem Schwartza z~wieloskładnikową reakcją Ugiego\sidecite{furman15}.
Wykorzystałem modyfikację tej przemiany \--- mariaż reakcji azydo-Ugiego i~Joulli{\'e}-Ugiego \---
  aby otrzymać tetrazolowe pochodne iminocukrów.
W~dalszej części tekstu opisuję pracę włożoną w~realizację tego przedsięwzięcia,
  wyzwania jakie przy tym napotkałem i~dokonane w~tym procesie obserwacje.
Pokazuję również dalsze przekształcenia otrzymanych związków oraz inne ścieżki,
  którymi próbowałem dotrzeć do~obranego celu.
Następny rozdział niniejszej dysertacji \--- \textit{\nameref{chapter:numeric}} \---
  jest kontynuacją treści zawartej w~tej sekcji i~zawiera omówienie prac prowadzonych
  celem lepszego poznania przedstawionego tu procesu.

\subsection{Bioizosteryzm i~inne zalety tetrazoli}\label{synthesis:sugars:bioisosterizm}
\begin{marginscheme}
  \includesvg{tetrazoles}
  \caption{
    Tautomeryczne struktury tetrazolu.
    Izomery \refcmpd{tetrazole.1h, tetrazole.2h},~w~przeciwieństwie
      do~izomeru \refcmpd{tetrazole.5h}, są związkami aromatycznymi.
  } \label{sch:tetrazoles}
\end{marginscheme}
Zanim jednak przejdę do~sedna, myślę, że warto pochylić się nad~tetrazolami i~ich miejscem w~chemii.
Te heterocykliczne związki, o~strukturze pokazanej na~\cref{sch:tetrazoles},
  występują w~trzech tautomerycznych formach: \iupac{1\H-}, \iupac{2\H-} oraz \iupac{5\H-}.
Dwie pierwsze są strukturami aromatycznymi \--- para elektronowa atomu azotu jest
  zaangażowana w~delokalizację w~obrębie pierścienia, a~izomery \iupac{1\H-} i~\iupac{2\H-}
  występują w~roztworach w~równowadze\sidecite{kiselev11}.
\iupac{5\H-Tetrazol} jest natomiast niearomatyczny \--- tym samym mniej trwały,
  oraz zdecydowanie mniej powszechny\sidenote{
    Wg bazy Reaxys około \SI{1.3}{\percent} znanych tetrazoli to \iupac{5\H-tetrazole}
      (dostęp: 13.10.2021).
  }.
Podobnie jak ich trójazotowe analogi, związki tetrazolowe wykazują aktywność
  organokatalityczną w~różnych typach przekształceń.
Można znaleźć doniesienia o~ich zastosowaniu w~reakcji aldolowej\sidecite{hartikka04},
  addycji Michaela\sidecite{chen13}, reakcji Mannicha\sidecite{kumar13},
  czy reakcjach uwodornienia\sidecite{mirabal12}.

\begin{marginfigure}
  \includesvg{losartan}
  \caption{Pierwszy zarejestrowany lek z~grupy sartanów o~działaniu przeciwnadciśnieniowym.}
  \label{fig:losartan}
\end{marginfigure}
Ważniejsza wydaje mi się jednak rola tetrazoli w~kontekście aktywności biologicznej,
  jaką nadają zawierającym je cząsteczkom.
Najbardziej znaną grupą leków posiadających w~strukturze to ugrupowanie są sartany,
  czyli antagoniści receptora angiotensyny II, stosowane w~terapii nadciśnienia tętniczego.
Pierwszym zarejestrowany lekiem tego typu jest Losartan \refcmpd{losartan},
  przedstawiony na~\cref{fig:losartan}.
Pochodne tetrazoli wykazują jednak znacznie szerszą gamę aktywności \---
  znaleźć można liczne doniesienia o~ich działaniu przeciwbakteryjnym i~przeciwgrzybiczym,
  ale dostępne są też prace poświęcone przeciwnowotworowym, przeciwzapalnym, przeciwwirusowym
  i~wielu innym właściwościom tych związków\sidecite{wei15}.

Uważam, że najciekawszą cechą pierścienia tetrazolowego jest jego bioizosteryzm.
Termin ten używany jest przede wszystkim w~chemii medycznej, a~oznacza podobieństwo atomów
  lub grup funkcyjnych w~kontekście ich aktywności biochemicznej\sidecite{meanwell11}.
Zaproponował go \citeauthor{friedman50} w~\citeyear{friedman50}\sidecite{friedman50}
  biorąc za~podstawę wcześniejsze prace definiujące pojęcie izosteryczności \---
  czyli podobieństwa elektronowego i~fizykochemicznego, zwłaszcza małych cząsteczek\sidenote{%
    Na tej podstawie \citeauthor{langmuir19} przewidział właściwości fizyczne ketenu
      na~18 lat przed jego scharakteryzowaniem: \colorcite{langmuir19}.
  }.
Starszy z~konceptów zdaje się być nieco zapomniany, a~obydwa pojęcia bywają \---
  czasem niepoprawnie \--- używane wymiennie.

\begin{marginfigure}
  \includesvg{ezetimibe}
  \caption{
    Ezetymib \--- wykorzystywany w~formie leku laktam, w~przypadku którego obecność atomów fluoru
      w~strukturze jest kluczowa dla jego aktywności biologicznej.
  }
  \label{fig:ezetimibe}
\end{marginfigure}
Zamiana fragmentu cząsteczki na~jego bioizoster pozwala zmodyfikować jej niektóre właściwości,
  przy zachowaniu czynności farmakologicznej.
Dzięki rozważnemu wyborowi takiej grupy na~etapie projektowania cząsteczki możliwe jest na~przykład
  zmniejszenie toksyczności, dostosowanie lipofilowości czy wydłużenie czasu metabolizmu
  nowego kandydata na~lek.
Chyba najbardziej znanym i~jednym z~najprostszych przykładów jest atom fluoru jako
  bioizoster atomu wodoru.
Zamiana \ch{H} na~\ch{F} w~strukturze powoduje zazwyczaj wzrost lipofilowości
  związku\sidecite{meanwell11}, a~w~przypadku Ezetymibu \refcmpd{ezetimibe},
  pokazanego na~\cref{fig:ezetimibe}, wprowadzenie dwóch atomów fluoru
  okazało się kluczowe do~uzyskania odpowiedniej stabilności metabolicznej\sidecite{clader04}.

\Cref{fig:bioisosteres} przedstawia bioizosteryczne relacje tetrazolu z~innymi grupami funkcyjnymi:
  pierścień \iupac{\N\H-tetrazolowy} jest bioizosterem kwasu karboksylowego, lub \---
  w~formie zdeprotonowanej \--- odpowiedniego anionu, natomiast \iupac{1-podstawiony} tetrazol
  uważa się za~bioizoster drugorzędowego amidu.
Najczęściej oczekiwaną korzyścią z~wprowadzenia tej grupy do~biologicznie aktywnej cząsteczki
  jest zwiększenie jej biodostępności, czy łatwości przyswajania substancji przez organizm.
Wspomniany wcześniej Losartan (\cref{fig:losartan}) jest tego przykładem \---
  obecność tetrazolu w~jego strukturze zwiększa 10\=/krotnie jego aktywność w~stosunku
  do~pochodnej posiadającej w~tej pozycji podstawnik \ch{-COOH}\sidecite{carini98}.
\begin{figure}
  \includesvg{bioisosteres}
  \caption{
    Bioizosteryczne pary tetrazoli: kwas karboksylowy, odpowiedni anion, oraz drugorzędowy amid.
  }
  \label{fig:bioisosteres}
\end{figure}

Ze względu na~rosnące zainteresowanie pochodnymi tetrazolu, dostępnych jest coraz więcej metod
  syntezy tych związków.
Większość z~nich to wariacje najpopularniejszego podejścia, czyli reakcji cykloaddycji
  azydku do~nitryli.
Obszerny opis tych, jak i~alternatywnych, sposobów otrzymywania tetrazoli można znaleźć
  w,~wydanym podczas przygotowywania niniejszej dysertacji, przeglądowym artykule\sidecite{leyva21}.

\subsection{Wieloskładnikowe reakcje typu reakcji Ugiego}
Wśród różnych metod syntezy tetrazoli wyróżniają się reakcje wieloskładnikowe \---
  cechuje je wysoka ekonomia atomowa\sidenote{%
    Patrz: przypis \ref{note:atom-economy}, str.~\pageref{note:atom-economy}.
  }, pozwalają stosować strategię syntezy zbieżnej, a~jako przebiegające w~jednym naczyniu
  reakcyjnym są wygodne dla eksperymentatora\sidecite{neochoritis19}.
Najbardziej znaną i~najpowszechniej stosowaną reakcją tego typu jest bez wątpienia
  reakcja Ugiego\sidecite{sunderhaus09}.
Przemiana ta została zaproponowana przez Ivara Ugiego w~\citeyear{ugi59} roku\sidecite{ugi59}.
Biorą w~niej udział cztery komponenty:
  kwas karboksylowy~\refcmpd{carboxylic-acid}, pierwszorzędowa amina~\refcmpd{amine},
  keton~\refcmpd{ketone} lub aldehyd oraz izocyjanek~\refcmpd{isocyanide}.
Jak pokazuje \cref{sch:ugi}, formalnie jest reakcją kondensacji,
  a~w~jej wyniku powstaje diamid~\refcmpd{ugi-product}.
\begin{scheme}
  \includesvg{ugi}
  \caption{Uproszczony schemat czteroskładnikowej reakcji Ugiego.}
  \label{sch:ugi}
  \setfloatalignment{b}
\end{scheme}

Jak zazwyczaj w~przypadku reakcji wieloskładnikowych, przemiana ta prawdopodobnie zachodzi
  według różnych mechanizmów jednocześnie.
Badania prowadzone nad tym zagadnieniem są zgodne co do~pierwszego i~ostatniego etapu reakcji,
  czyli rozpoczęcia przemiany powstawaniem iminy~\refcmpd{imine} oraz jej zakończenia
  przegrupowaniem Mumma imidatu~\refcmpd{ugi-imidate}.
Ostatnie doniesienia w~tej materii sugerują, że przeważająca jest ścieżka przedstawiona
  na~\cref{sch:ugi-mech}, przebiegająca przez kation iminiowy~\refcmpd{iminium} oraz
  nitryliowy~\refcmpd{nitrilium}\sidecite{rocha20}.
Zazwyczaj reakcja ta przebiega najwydajniej w~polarnym protonowym rozpuszczalniku,
  takim jak metanol \--- możliwe, że odpowiada za~to stabilizacja polarnych stanów przejściowych.
W~niektórych przypadkach dobre rezultaty może również przynieść użycie polarnego
  aprotycznego rozpuszczalnika, najczęściej \ch{CH2Cl2}\sidecite{rocha20}.
\begin{scheme*}
  \includesvg{ugi-mech}
  \caption{Jeden z~możliwych mechanizmów tworzenia produktu w~reakcji Ugiego.}
  \label{sch:ugi-mech}
\end{scheme*}

Produkt reakcji Ugiego może być postrzegany jako oligopeptyd, z~czego wynikają długotrwałe
  wysiłki badaczy w~przeprowadzeniu tej reakcji w~sposób enancjoselektywny.
Pierwsza zakończona powodzeniem próba miała miejsce dopiero w~\citeyear{zhang18} \---
  po~niemal 60 latach od~jej odkrycia.
Udało się tego dokonać, prowadząc reakcję w~obecności chiralnej pochodnej kwasu fosforowego
  jako katalizatora\sidecite[\baselineskip]{zhang18}.
Używając jednego z~dwóch zaproponowanych katalizatorów, autorzy pracy byli w~stanie sterować
  stereochemią przemiany, otrzymując \iupac{\R-} lub \iupac{\S-}produkty z~doskonałą wydajnością
  i~wysokim nadmiarem enancjomerycznym.

Duże zainteresowanie reakcją Ugiego zaowocowało jej rozmaitymi modyfikacjami.
Wiele z~nich proponuje wykorzystanie substratów o~strukturze pozwalającej na~przeprowadzenie
  wewnątrzcząsteczkowej cyklizacji po,~przebiegającym standardowo, etapie kondensacji.
Spośród licznych przykładów można wymienić reakcję Ugiego-Dielsa-Aldera, kończącą się
  spontaniczną reakcją Dielsa-Aldera, albo reakcję Ugiego-Hecka, dodającą etap cyklizacji
  produktu z~wykorzystaniem wewnątrzcząsteczkowej reakcji Hecka\sidecite{sunderhaus09}.
Inne natomiast ingerują w~przebieg samego procesu kondensacji przez zmianę grup funkcyjnych
  biorących udział w~przemianie \--- na~przykład w~reakcji Ugiego-Smilesa finalne przegrupowanie
  Mumma jest zastąpione przegrupowaniem Smilesa, ze~względu na~użycie fenolu zamiast
  kwasu karboksylowego\sidecite{kaim06}.

Jednym z~wariantów tego drugiego typu jest reakcja azydo-Ugiego\sidenote{%
    Ang. \textit{Ugi-Azide Reaction}, czasem określana też akronimem \gls{azido-ugi}.
  }, w~której kwas karboksylowy zastąpiony jest azydkiem.
Została ona zaproponowana przez Ugiego i~Steinbr{\"u}cknera dwa lata po~ukazaniu się
  doniesienia o~oryginalnej wersji czteroskładnikowej reakcji Ugiego\sidecite{ugi61}.
\Cref{sch:ugi-azide} pokazuje przebieg tej reakcji \--- azydek wchodzi w~reakcję
  z~kationem nitryliowym~\cmpd{nitrilium}, prowadząc do~powstania \iupac{2-podstawionej}
  pochodnej tetrazolu~\refcmpd{aminotetrazole}.
Jest to zdecydowanie najczęściej wykorzystywana wieloskładnikowa metoda syntezy
  tetrazoli\sidecite{neochoritis19}.
\begin{scheme}
  \includesvg{ugi-azide}
  \caption{Przebieg reakcji azydo-Ugiego.}
  \label{sch:ugi-azide}
\end{scheme}

\subsection{Model i~optymalizacja}
W~\citeyear{nenajdenko13} \citeauthor{nenajdenko13} pokazali, że możliwe jest przeprowadzenie
  reakcji azydo-Ugiego w~wariancie Joulli{\'e}-Ugiego\sidenote{%
    Czyli wariancie, w~którym aminę i~keton/aldehyd zastępuje się przygotowaną wcześniej iminą.
  }\textsuperscript{,\thinspace}\sidecite{nenajdenko13}.
W~zespole badawczym, w~którym realizowana była niniejsza dysertacja, w~podobnym czasie
  z~sukcesem zastosowano tę wersję reakcji Ugiego do~funkcjonalizacji imin otrzymanych
  z~laktamów na~drodze reduktywnej aktywacji odczynnikiem Schwartza\sidenote{%
    \See{sch:our-joullie-ugi} w~sekcji \nameref{literature:schwartz:our}.}.
Doniesienia te dają dobrą podstawę, by spodziewać się, że reakcja azydo-Ugiego poprzedzona
  reduktywną aktywacją laktamu również będzie owocna.
Trafność tej hipotezy udowodniłem, przeprowadzając przemianę zobrazowaną na~\cref{sch:preliminary}.
Wychodząc z~wywiedzionego~z glukozy laktamu~\refcmpd{glu-lactam} i~stosując ustalone wcześniej,
  optymalne warunki prowadzenia reduktywnej aktywacji odczynnikiem Schwartza\sidecite{furman14}
  oraz typowe warunki prowadzenia reakcji azydo-Ugiego\sidecite{nenajdenko13},
  otrzymałem z~zadowalającą, jak na~pierwszą próbę, wydajnością diastereomeryczne tetrazolowe
  pochodne: \refcmpd{glu-tet.cy, glu-epi-tet.cy}.
\begin{scheme*}
  \includesvg{preliminary}
  \caption{
    Pierwszy eksperyment sprawdzający możliwość syntezy tetrazolowych pochodnych iminocukrów
      w~sekwencji aktywacja amidu\--reakcja azydo-Ugiego.
    Użyłem \SI{1.6}{\equiv} \schwartz{}, \SI{1.1}{\equiv} \ch{CyNC} oraz
      \SI{1.1}{\equiv} \ch{TMSN3}.
    \acrshort{cy} \--- \acrlong{cy}; \acrshort{tms} \--- \acrlong{tms}.
  }
  \label{sch:preliminary}
\end{scheme*}

Podjąłem próbę optymalizacji procesu, szukając najlepszego donora protonu do~aktywacji \ch{TMSN3},
  będącego źródłem anionów azydkowych, odpowiedzialnych za~powstawanie pierścienia tetrazolowego.
\Cref{tab:sugars-opt} zawiera zbiór prób przeprowadzonych w~tym zakresie (pozycje 1. do~7.).
Zaskakująco, oczekiwany produkt powstaje, nawet jeśli żaden donor protonów nie zostanie użyty.
Co więcej, takie warunki okazały się najlepsze dla przebiegu reakcji \--- zapewniają najwyższą
  wydajność oraz diastereoselektywność procesu.
Przeprowadziłem również próbę wydzielenia powstającej jako związek pośredni iminy i~realizacji
  reakcji azydo-Joulli{\'e}-Ugiego w~osobnym etapie (pozycje 8. oraz 9.).
Prowadziło to jednak do~znacznego spadku wydajności, prawdopodobnie ze~względu na~niską trwałość
  imin o~strukturze takiej jak \refcmpd{glu-imine}.

\begin{table}
  {\includesvg{sugars-opt}}
  \centering
  \begin{tabular}{ccccc}
    \toprule
    \textnumero & Dodatek & Rozpuszczalnik & Wydajność /\si{\percent} & \textit{d.r.} \textsuperscript{a} \\ \midrule
    \rownumber & \ch{MeOH} \textsuperscript{b} & \ch{THF} & \num{65} & $43:57$ \\
    \rownumber & \ch{CF3CO2H} & \ch{THF}  & \num{24} & $43:57$ \\
    \rownumber & \ch{AcOH} & \ch{THF} & \num{47} & $80:20$ \\
    \rownumber & \ch{Et3N*HCl} & \ch{THF} & \num{45} & $74:26$ \\
    \rownumber & \ch{H2O} & \ch{THF} & \num{34} & $>95:5$ \\
    \rownumber & \ch{(CF3)2CHOH} & \ch{THF} & \num{35} & $>95:5$ \\
    \rowcolor{\tablemarkecolor}
    \rownumber & brak & \ch{THF} & \num{73} & $>95:5$ \\
    \rownumber & brak & \ch{MeOH} & \num{19} \textsuperscript{c} & $>95:5$ \\
    \rownumber & brak & \ch{DCM} & \num{36} \textsuperscript{c} & $>95:5$ \\
    \bottomrule
  \end{tabular}
  \caption{%
    Optymalizacja syntezy \iupac{2-(1\hydrogen-tetrazol-5-ylo)-iminocurów} poprzez
      redukcję laktamu odczynnikiem Schwartza i~reakcję azydo-Ugiego.
    A: \SI{1.6}{\equiv} \ch{Cp2Zr(H)Cl} w~\ch{THF} w~atmosferze argonu;
    B: \SI{1.6}{\equiv} dodatku (jeśli występuje), \SI{1.1}{\equiv} \ch{CyNC}
      oraz \SI{1.1}{\equiv} \ch{TMSN3}.
    \textsuperscript{a}\iupac{2-\R} do \iupac{2-\S}, wg wydajności wydzielonych produktów.
    \textsuperscript{b}Dodatek użyty w~nadmiarze.
    \textsuperscript{c}Imina wydzielona po etapie redukcji.
  }\label{tab:sugars-opt}
\end{table}

\begin{marginfigure}
  \begin{tikzpicture}
    \begin{axis}[
      ylabel={Wydajność /\si{\percent}},
      xlabel={Ilość \ch{TMSN3} i \ch{CyNC} /\si{\equiv}},
      ymin=0, ymax=100,
      xmin=0.8, xmax=2.4,
      ytick={0,50,100},
      extra y ticks={86},
      extra y tick labels={\textcolor{wongvermillion}{$86$}},
      axis lines=left,
      axis line style={-},
      x label style={at={(0.5,0.05)}},
      y label style={at={(0.1,0.5)}},
      legend style={at={(0.05,0.4)}, anchor={north west}, draw=none},
      mark options={scale=0.5},
      width=1.1*\textwidth,
    ]
      \addplot[color=wongblue,mark=*,only marks,error bars/y dir=both,error bars/y explicit]
        coordinates {
          (1.0,86) +- (0,5)
          (1.3,84) +- (0,1)
          (1.6,84) +- (0,7)
          (1.9,91) +- (0,3)
          (2.2,86) +- (0,3)
        };
      \addplot[dashed,color=wongvermillion,no markers]
        coordinates {(0.8,86)(2.4,86)};
      \legend{,Średnia wydajność}
    \end{axis}
  \end{tikzpicture}
  \caption{
    Wykres przedstawiający wydajność tworzenia związku \refcmpd{glu-tet.cy}, w~zależności
      od~użytej ilości \ch{TMSN3} i~\ch{CyNC}, wg analizy \NMR*.
  }
  \label{fig:sugars-opt-plot}
\end{marginfigure}
Prace źródłowe dotyczące reakcji azydo-Ugiego proponują użycie różnych proporcji reagentów \---
  zwykle od~1 do~3 ekwiwalentów \ch{TMSN3} i~\ch{CyNC} względem iminy.
Postanowiłem sprawdzić, czy parametr ten ma wpływ na~wydajność w~przypadku badanej przez mnie
  wersji przemiany.
Przeprowadziłem szereg eksperymentów w~warunkach uznanych za~optymalne, ale zmieniając ilość
  użytych w~ostatnim etapie odczynników.
Aby zminimalizować wpływ innych czynników na~przebieg tego eksperymentu, wydajność reakcji
  określałem na~podstawie analizy \NMR*{} surowej mieszaniny \---
  w~tym celu dodałem do~reakcji \SI{1.0}{\equiv} trifenylometanu\sidenote[][1\baselineskip]{%
    Sprawdziwszy uprzednio, że nie ulega on żadnym przemianom w~warunkach reakcji,
      ani nie wpływa na jej przebieg.
  } jako wzorca wewnętrznego.
Wyniki tych testów, razem z~odchyleniem standardowym z~trzech prób w~postaci słupków błędu,
  przedstawiłem na~\cref{fig:sugars-opt-plot}.
Sugerują one, że w~badanym zakresie, czyli powyżej \SI{1}{\equiv}, ilość użytych \ch{TMSN3}
  i~\ch{CyNC} nie ma wpływu na~wydajność powstawania produktu\sidenote{%
    Choć przedstawiony powyżej wykres zdaje się nie pozostawiać wątpliwości co do~trafności
      takiej interpretacji, to stawianie hipotez na~temat danych bez ich analizy raczej 
      nie jest dobrą praktyką.
    Z~tego powodu zamieszczam adekwatne detale techniczne, a~także szczegóły przyjętej metodologii,
      w~dalszej części tej pracy, w~sekcji \secref{experimental:sugars-opt}. 
  }.
Średnia wydajność, również naniesiona na~\cref{fig:sugars-opt-plot}, wynosi \SI{86 \pm 5}{\percent}.

Bardzo niedawno ukazała się praca łącząca reakcję azydo-Ugiego z~protokołem aktywacji amidów
  opartym o~kompleks Vaski\sidecite{dixon18}.
Poza doskonałymi wynikami prób prowadzonych z~użyciem amidów liniowych,
  \citeauthor{dixon18} pokazali w~niej tylko jeden przykład funkcjonalizacji laktamu,
  a~efekt tej próby był raczej przeciętny \--- otrzymali produkt z~wydajnością \SI{41}{\percent},
  wychodząc z~\iupac{\N-tertbutylokaprolaktamu}.
Zaznaczają też, że metoda jest ograniczona do~laktamów trzeciorzędowych.
Podczas prac optymalizacyjnych \--- jeszcze przed ukazaniem się wspomnianego doniesienia \---
  próbowałem zastosować zarówno tę, jak i~wykorzystującą kompleks van der Enta, metodę aktywacji
  amidów do~funkcjonalizacji laktamu~\refcmpd{glu-lactam}, jednak bez powodzenia.
Z~wcześniejszych prac\sidecite{furman14} wiadomo natomiast, że procedura wykorzystująca
  bezwodnik triflowy również nie znajduje zastosowania w~przypadku takich substratów.

\subsection{Wyniki i~zakres stosowalności}
Opracowane przeze mnie optymalne warunki prowadzenia reakcji zastosowałem do~otrzymania kilku
  tetrazolowych pochodnych iminocukrów \--- ich spis, wraz z~wydajnościami wydzielania,
  znajduje się w~\cref{tab:sugars-scope}.
Zmieniając izocyjanek użyty do~tej przemiany byłem w~stanie otrzymać ugrupowanie pierścienia
  tetrazolowego z~różnymi podstawnikami na~atomie azotu (pozycje 1. do~7.).
Użycie niektórych z~nich wymagało wydłużenia czasu prowadzenia addycji \--- reakcja z~izocyjankiem
  tertbutylu (pozycja 6.) zakończyła się dopiero po~12 dniach.
Wydajności otrzymanych produktów nie są tak wysokie jak w~przypadku modelowej reakcji \---
  utrzymują się przeważnie na~poziomie około \SI{40}{\percent}.
Zwraca uwagę przypadek izocyjanku \iupac{\para-metoksyfenylu}, który jako jedyny spowodował spadek
  diastereoselektywności procesu.

Jako substratów do~tych prac użyłem laktamów \cmpdref{glu-lacta, gal-lactam}, wywiedzionych
  odpowiednio z~glukozy (pozycje 1. do~7.) oraz galaktozy (pozycje 8. i~9.).
Przeprowadziłem również liczne próby otrzymania iminocukrów o~pięcioczłonowym pierścieniu \---
  przede wszystkim wywiedzionych z~rybozy oraz arabinozy.
Niestety, eksperymenty te nie były owocne \--- mimo, że obserwowałem powstawanie pożądanych
  związków za~pomocą analizy \gls{ms} surowych mieszanin, to nie udało mi się ich wydzielić.
Zastosowanie alternatywnych procedur aktywacji pięcioczłonowych laktamów również nie przyniosło
  rezultatu \--- tetrazolowe pochodne iminocukrów nie powstawały w~ogóle,
  tak samo jak w~przypadku reakcji modelowej.
Przypuszczam, że takie rezultaty wynikają z~małej trwałości imin wywiedzionych z~pięcioczłonowych
  laktamów, jednak nie prowadziłem dalszych badań mających na~celu wyjaśnienie tego fenomenu.

\begin{table}
  {\includesvg{sugars-scope}}

  \vspace{.2\baselineskip}  % apparently needs to be in its own paragraph to work

  \begin{tabular}{rccccc}
    \toprule
    \textnumero  & Substrat                 & \ch{-R}    & \makecell{Produkt\\główny} & Wydajność /\si{\percent}
      & \textit{d.r.} \textsuperscript{a} \\ \midrule
    \rownumber  & \refcmpd{glu-lactam} & \ch{-Cy}       & \cmpd{glu-tet.cy}   & \num{73} & $>95:5$ \\
    \rownumber  & \refcmpd{glu-lactam} & \ch{-CH2CO2Et} & \cmpd{glu-tet.est}  & \num{49} & $>95:5$ \\
    \rownumber  & \refcmpd{glu-lactam} & \ch{-Bn}       & \cmpd{glu-tet.benz} & \num{18} & $>95:5$ \\
    \rownumber  & \refcmpd{glu-lactam} & \ch{-PMP}      & \cmpd{glu-tet.pmp}  & \num{29} & $79:21$ \\
    \rownumber  & \refcmpd{glu-lactam} & \ch{-PMB}      & \cmpd{glu-tet.pmb}  & \num{42} & $>95:5$ \\
    \rownumber  & \refcmpd{glu-lactam} & \ch{-\textit{^t}Bu} & \cmpd{glu-tet.tbu} & \num{40} & $>95:5$ \\
    \rownumber  & \refcmpd{glu-lactam} & \ch{-\textit{^t}Oct} & \cmpd{glu-tet.oct} & \num{48} & $>95:5$ \\
    \rownumber  & \refcmpd{gal-lactam} & \ch{-Cy}       & \cmpd{gal-tet.cy}   & \num{33} & $>95:5$ \\
    \rownumber  & \refcmpd{gal-lactam} & \ch{-CH2CO2Et} & \cmpd{gal-tet.est}  & \num{16} & $>95:5$ \\
    \bottomrule
  \end{tabular}
  \caption{%
    Synteza różnych \iupac{2-(1\H-tetrazol-5-ylo)-iminocukrów} przy użyciu optymalnych warunków
      prowadzenia reakcji, opracowanych przeze mnie w~toku opisywanych tu badań.
    \textsuperscript{a}\iupac{2-\R} do~\iupac{2-\S}, według wydajności wydzielonych produktów.
  }
  \label{tab:sugars-scope}
\end{table}

\subsection{Dalsze przekształcenia produktów}
Struktura niektórych związków otrzymanych w~toku tych badań umożliwia ich dalsze,
  interesujące przemiany.
Związek~\refcmpd{glu-tet.est} posiada grupę estrową, potencjalnie zdolną do~wewnątrzcząsteczkowego
  przekształcenia w~amid z~iminocukrowym atomem azotu.
Jak przedstawia \cref{sch:ugi-tricyclo}, transformację tę udało mi się przeprowadzić niemal
  ilościowo, działając na~substrat kwasem benzoesowym w~podwyższonej temperaturze.
Aby przeprowadzić deoksygenację otrzymanego związku musiałem
  uciec się do~jego aktywacji odczynnikiem Schwartza i~redukcji z~użyciem \ch{NaBH4} \---
  standardowa metoda wykorzystująca \ch{LiAlH4} okazała się nieskuteczna.
Uzyskany na~drodze tych przemian związek~\refcmpd{glu-tet-tricyclo} posiada trzy pierścienie
  skondensowane i~może być postrzegany jako nieznana dotąd klasa nienaturalnych, chiralnych
  alkaloidów, być może aktywnych biologicznie\sidecite{finsinger15}.

\begin{scheme*}
  \includesvg{ugi-tricyclo}
  \caption{%
    Synteza nowej klasy alkaloidu \refcmpd{glu-tet-tricyclo} z~iminocukru \refcmpd{glu-tet.est},
      otrzymanego przy użyciu opracowanej przeze mnie metodologii.
  }
  \label{sch:ugi-tricyclo}
\end{scheme*}

\begin{marginfigure}[3\baselineskip]
  \includesvg{pmb-rearr}
  \caption{%
    Produkt przegrupowania związku \refcmpd{glu-tet.pmb} następującego pod wpływem \gls{tfa}.
  }
  \label{fig:pmb-rearr}
\end{marginfigure}

Drugim celem syntetycznym, który chciałem osiągnąć, było otrzymanie \iupac{\N\H-tetrazolowej}
  pochodnej iminocukru, naśladującej układ chiralnej \iupac{1,2-diaminy}.
Liczne próby uzyskania takiego wyniku przez usunięcie grupy \iupac{\para-metoksybenzylowej}
  ze~związku~\refcmpd{glu-tet.pmb} nie powiodły się.
Nieoczekiwanie, jeden z~tych eksperymentów doprowadził do~przegrupowania w~obrębie pierścienia
  tetrazolowego i~powstania związku \refcmpd{glu-tet-pmb-rearr},
  przedstawionego na~\cref{fig:pmb-rearr}.
Pożądany związek~\refcmpd{glu-tet-free} udało mi się uzyskać, poddając pochodną
  \iupac{\tert-oktylow}ą \refcmpd{glu-tet.oct} działaniu suchego \ch{HCl} w~podwyższonej
  temperaturze.
Przemiana ta, przedstawiona na~\cref{sch:ugi-diamine}, biegnie z~dobrą wydajnością.
Jej produkt wart jest uwagi ze względu na~podobieństwo do~układów stosowanych jako
  organokatalizatory\sidenote{Wspomniałem o~nich w~sekcji \secref{synthesis:sugars:bioisosterizm}.},
  może być również postrzegany jako tetrazolowa pochodna \iupac{1-deoksynojirimycyny}.

\begin{scheme}
  \includesvg{ugi-diamine}
  \caption{%
    Otrzymywanie \iupac{\N-niezabezpiecznoej} pochodnej
      \iupac{2-(1\hydrogen-tetrazol-5-ylo)-iminocukru}.
    }
    \label{sch:ugi-diamine}
\end{scheme}

\subsection{Nietrafione alternatywy}
Do otrzymania tetrazolowych pochodnych iminocukrów próbowałem podejść również inaczej
  niż na~drodze wariantu reakcji Ugiego.
Pierwsze próby zakładały przeprowadzenie syntezy w~dwóch etapach \--- reduktywnej addycji
  anionu cyjankowego do~aktywowanego laktamu, a~następnie cykloaddycji azydku do~grupy nitrylowej.
Funkcjonalizacja anionem cyjankowym przebiega dość wydajnie, dając tylko jeden diastereomer
  \refcmpd{glu-cn}.
Niestety, jak widać na~\cref{sch:gluco-nitrile}, związek \refcmpd{glu-cn} nie ulega reakcji
  cykloaddycji azydku.

\begin{scheme*}
  \includesvg{gluco-nitrile}
  \caption{
    Planowana metoda dwuetapowej syntezy niepodstawionej tetrazolowej pochodnej iminocukru
      okazałą się nieskuteczna ze~względu na~niepowodzenie w~cykloaddycji azydku.
  }
  \label{sch:gluco-nitrile}
\end{scheme*}

Niepowodzenie w~tej materii było niejakim zaskoczeniem, ponieważ przemiana tego typu jest
  znana i~wielokrotnie opisana w~literaturze.
Wypróbowałem różne spośród dostępnych metod prowadzenia reakcji cykloaddycji azydku do~grupy
  nitrylowej, wykorzystujące jako aktywator \ch{Bu2SnO}\sidecite{quinodoz16},
  \ch{Bu4NF}\sidecite{liu16}, \ch{Zn} w~butanolu\sidecite{vorona14}, czy jod\sidecite{das10}.
Eksperymentowałem również z~próbami wzbudzenia reakcji podwyższoną temperaturą
  oraz promieniowaniem mikrofalowym.
W~obliczu nieskuteczności każdego z~przetestowanych podejść, zrezygnowałem z~dalszych prac
  w~tym kierunku.

Choć było dla mnie jasnym, że szanse powodzenia takiego przedsięwzięcia są nikłe \---
  i~to mówiąc oględnie \--- pozwoliłem sobie rozważyć odwrócenie kolejności etapów tej
  nieudanej syntezy.
Anion azydkowy nie wykazuje znacznych właściwości nukleofilowych, ale jednak może
  ulegać zarówno klasycznej, jak i~aromatycznej nukleofilowej substytucji\sidecite{brase05}.
W~literaturze odnalazłem jednak tylko dwa przykłady addycji nukleofilowej \ch{N3-} do~iminy,
  obydwa przebiegające z~niezwykle aktywnym, fluorowanym partnerem\sidecite{burger87,douglas66}.
Dla spokoju sumienia przeprowadziłem próbę reduktywnej funkcjonalizacji laktamu aktywowanego
  odczynnikiem Schwartza za~pomocą \ch{TMSN3}.
Jej wynik, zgodnie z~oczekiwaniami, był negatywny.

\begin{scheme*}[t]
  \includesvg{azide-nucleophile}
  \caption[][-6\baselineskip]{
    Dwa przykłady nukleofilowej addycji anionu azydkowego do~iminy, które udało mi się znaleźć.
    Na~górze addycja z~przegrupowaniem do~oksadiazolu, na~dole addycja katalizowana chinoliną.
  }
  \label{sch:azide-nucleophile}
\end{scheme*}

\begin{scheme*}[b]
  \includesvg{cp-addition}
  \caption[][-7em]{
    Formalna cykloaddycja cyklopentadienu do~iminy \refcmpd{glu-imine}, następująca zamiast
      funkcjonalizacji indolem w~badanych warunkach.
  }
  \label{sch:cp-addition}
  \setfloatalignment{t}
\end{scheme*}

Szukając bardziej odległych koncepcyjnie alternatyw dla \iupac{2-tetrazolowych} pochodnych
  iminocukrów próbowałem przeprowadzić reduktywną funkcjonalizację laktamu \refcmpd{glu-lactam},
  używając indolu jako nukleofila.
Przemiana tego typu została opisana w~jednej z~prac pochodzących z~zespołu badawczego,
  w~którym realizowałem badania\sidecite{czerwinski19}.
Próba ta nie przyniosła jednak pożądanego rezultatu.
Zamiast tego, w~mieszaninie poreakcyjnej znalazłem związek, który formalnie jest produktem
  sprzęgania z~cyklopentadienem.
Dwa diastereomery tego związku, oznaczone wspólnym numerem \refcmpd{glu-cp-di}
  na~\cref{sch:cp-addition}, powstają w~proporcji 1:1 z~sumaryczną wydajnością \SI{23}{\percent}.
Izomerów tych nie udało mi się ich rozdzielić \--- ani za pomocą chromatografii klasycznej,
  ani wysokociśnieniowej.

\citeauthor{ulikowski16} zaobserwowali, jako reakcję uboczną w~reduktywnej
  funkcjonalizacji \iupac{3-hydroksy-2-oksoindolu} z~użyciem \schwartz{} i~\ch{TMSCN},
  uwalnianie się cyklopentadienu z~kompleksu cyrkonu\sidenote{
    A~dokładnie z~\ch{Cp2ZrO}, powstającego po~redukcji amidu odczynnikiem Schwartza.
  }.
Autorzy postulowali, że następuje to pod wpływem anionu cyjankowego, który zastępuje
  ligand cyklopentadienylowy w~otoczeniu atomu cyrkonu.
Uwolniony cyklopentadien wstępuje następnie w~reakcję z~aktywnym kationem
  iminiowym, w~tym wypadku jako dienofil\sidecite{ulikowski16}.

W osobnym eksperymencie przeprowadziłem próbę funkcjonalizacji laktamu~\refcmpd{glu-lactam},
  używając świeżo przedestylowanego cyklopentadienu zamiast indolu.
Związek~\refcmpd{glu-cp-di} powstał w~tej reakcji w~niewielkiej ilości, otrzymanym
  w~przewadze był produkt reduktywnej nukleofilowej addycji cyklopentadienu do~iminy,
  oznaczony na~\cref{sch:cp-nucleophile} numerem~\refcmpd{glu-cp}.
Istotną obserwacją jest, że w~tym wypadku powstaje tylko jeden izomer.
Konfigurację absolutną centrum stereogenicznego w~pozycji 2. ustaliłem za~pomocą
  eksperymentu \gls{noesy}.
Nieznana pozostaje konfiguracja centrum znajdującego się w~obrębie pierścienia cyklopentenu,
  oznaczonego gwiazdką na~wspomnianym schemacie.
Analiza \NMR*{} świadczy, że powstaje tylko tylko jeden diastereomer\sidenote{%
    Przynajmniej w~granicach dokładności metody pomairu.} ale konfiguracji absolutnej
  takiego układu wirującego nie można ustalić na~podstawie efektów \gls{noe}.

\begin{scheme}
  \includesvg{cp-nucleophile}
  \caption{
    Przebieg funkcjonalizacji z~dodatkiem cyklopentadienu.
    Przemiana do~zwiąku \refcmpd{glu-cp-di} następuje pod nieobecność innych czynników
      nukleofilowych, ale głównym produktem reakcji jest, nie obserwowany wcześniej,
      związek~\refcmpd{glu-cp}.
  }
  \label{sch:cp-nucleophile}
\end{scheme}

O~addycji prowadzącej do~struktury takiego typu jak \refcmpd{glu-cp} donosi chyba jedynie
  praca Coffineta i~in. z~\citeyear{coffinet16} roku.
Pokazuje ona, że taka przemiana może zachodzić pod wpływem katalitycznej ilości
  \ch{Cp2TiCl2} w~obecności \gls{dibal}\sidecite{coffinet16}.
W~zastosowanym przeze mnie układzie reakcyjnym można dostrzec znaczne podobieństwo,
  szczególnie, jeśli wziąć pod uwagę, że odczynnik Schwartza można generować \insitu{} 
  z~\ch{Cp2ZrCl2} w~połączeniu z~\gls{dibal}\sidenote{
    Wspominałem o~tym wcześniej, we~wstępie do~sekcji \secref{literature:schwartz}.
  }.
Mechanizm tej reakcji mógłby więc być analogiczny do~zaproponowanego we~wspomnianej pracy,
  z~tą różnicą, że \schwartz{} musiałby odpowiadać zarówno za~redukcję laktamu do~iminy,
  jak i~aktywację cyklopentadienu poprzez hydrocyrkonowanie jednego z~wiązań podwójnych.
\Cref{sch:cp-mech} przedstawia graficznie tę propozycję, razem z~porównaniem do~przebiegu
  oryginalnej przemiany.

\begin{scheme}
  \includesvg{cp-mech}
  \caption{
    Propozycja mechanizmu powstawania związku \refcmpd{glu-cp} (na~górze) w~porównaniu
      z~przebiegiem metaloallilowania imin według Coffineta (na~dole).
  }
  \label{sch:cp-mech}
\end{scheme}

Bardziej zagadkowe wydaje się przekształcenie pokazane na~\cref{sch:cp-addition}.
Obecność produktu \refcmpd{glu-cp-di} w~reakcji bez żadnego nukleofila, mogącego zastąpić
  ligand cyklopentadienylowy w~otoczeniu cyrkonu, sugeruje, że mechanizm tej reakcji jest inny
  niż w~przypadku reakcji ubocznej opisanej przez Ulikowskiego i~Furmana\sidecite{ulikowski16}.
Można by przypuszczać, że kompleks cyrkonu po redukcji rozpada się samoczynnie, uwalniając
  cyklopentadien, lub, że proces ten następuje pod wpływem \ch{Yb(OTf)3}.
Takie założenie nasuwa jednak pytanie \--- dlaczego związek~\refcmpd{glu-cp} obserwuję
  tylko w~reakcji z~dodatkiem cyklopentadienu?
Inną możliwością jest powstawanie produktu~\refcmpd{glu-cp-di} bezpośrednio w~wyniku
  degradacji kompleksu \refcmpd{glu-zr}.
\citeauthor{spletstoser07} pokazali, że kompleksy tego typu są nietrwałe i~ulegają rozpadowi
  z~utworzeniem dimerycznego kompleksu \ch{ClCp2Zr-O-ZrCp2Cl}\sidecite{spletstoser07}.
Być może obecność kwasu Lewisa ma wpływ na~ten proces i~promuje rozpad tego kompleksu
  z~przeniesieniem grupy cykolpentadienylowej?

Choć takie spekulacyjne rozważania są na~pewno interesujące, zaczynają wykraczać poza
  tematykę niniejszej dysertacji.
Nie przeprowadziłem dodatkowych eksperymentów w~tym kierunku, a~literatura nie dostarcza
  więcej wskazówek w~tej materii.
Postawione powyżej pytania pozostaną więc, niestety, bez odpowiedzi.
Dalsza część mojej pracy skupia się na~zgłębianiu innych zagadnień, związanych z~przebiegiem
  opisanego wcześniej wariantu reakcji azydo-Ugiego.
