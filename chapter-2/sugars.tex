\section{Synteza tetrazolowych pochodnych iminocukrów}\label{synthesis:sugars}
Jak wspomniałem w~rozdziale \textit{\nameref{chapter:literature}}\sidenote{%
    A dokładnie w~sekcji \secref{literature:schwartz:our}.
  }, z~grupy badawczej, w~której realizowałem studia zawarte w~niniejszej dysertacji,
  pochodzą prace poświęcone reduktywnej aktywacji laktamów wywiedzionych z~cukrów prostych.
Przedstawiony w~nich proces pozwala na~syntezę \textalpha{}-funkcjonalizowanych iminocukrów
  \--- związków nietrywialnych do~otrzymania klasycznymi metodami.
Zainteresowany możliwościami oferowanymi przez tę metodę postanowiłem kontynuować badania nad nią.

Za~punkt wyjścia posłużyło mi, zawierające kilka przykładów, doniesienie o~połączeniu
  reduktywnej aktywacji odczynnikiem Schwartza z~wieloskładnikową reakcją Ugiego\sidecite{furman15}.
Wykorzystałem modyfikację tej przemiany \--- mariaż reakcji azydo-Ugiego i~Joulli{\'e}-Ugiego \---
  aby otrzymać tetrazolowe pochodne iminocukrów.
W~dalszej części tekstu opisuję pracę włożoną w~realizację tego przedsięwzięcia,
  wyzwania jakie przy tym napotkałem i~dokonane w~tym procesie obserwacje.
Pokazuję również dalsze przekształcenia otrzymanych związków oraz inne ścieżki,
  którymi próbowałem dotrzeć do~obranego celu.
Następny rozdział niniejszej dysertacji \--- \textit{\nameref{chapter:numeric}} \---
  jest kontynuacją treści zawartej w~tej sekcji i~zawiera omówienie prac prowadzonych
  celem lepszego poznania przedstawionego tu procesu.

\subsection{Bioizosteryzm i~inne zalety tetrazoli}\label{synthesis:sugars:bioisosterizm}
\begin{marginscheme}
  \includesvg{tetrazoles}
  \caption{
    Tautomeryczne struktury tetrazolu.
    Izomery \refcmpd{tetrazole.1h, tetrazole.2h},~w~przeciwieństwie
      do~izomeru \refcmpd{tetrazole.5h}, są związkami aromatycznymi.
  } \label{sch:tetrazoles}
\end{marginscheme}
Zanim jednak przejdę do~sedna, myślę, że warto pochylić się nad~tetrazolami i~ich miejscem w~chemii.
Te heterocykliczne związki, o~strukturze pokazanej na~\cref{sch:tetrazoles},
  występują w~trzech tautomerycznych formach: \iupac{1\H-}, \iupac{2\H-} oraz \iupac{5\H-}.
Dwie pierwsze są strukturami aromatycznymi \--- para elektronowa atomu azotu jest
  zaangażowana w~delokalizację w~obrębie pierścienia, a~izomery \iupac{1\H-} i~\iupac{2\H-}
  występują w~roztworach w~równowadze\sidecite{kiselev11}.
\iupac{5\H-Tetrazol} jest natomiast niearomatyczny \--- tym samym mniej trwały,
  oraz zdecydowanie mniej powszechny\sidenote{
    Wg bazy Reaxys około \SI{1.3}{\percent} znanych tetrazoli to \iupac{5\H-tetrazole}
      (dostęp: 13.10.2021).
  }.
Podobnie jak ich trójazotowe analogi, związki tetrazolowe wykazują aktywność
  organokatalityczną w~różnych typach przekształceń.
Można znaleźć doniesienia o~ich zastosowaniu w~reakcji aldolowej\sidecite{hartikka04},
  addycji Michaela\sidecite{chen13}, reakcji Mannicha\sidecite{kumar13},
  czy reakcjach uwodornienia\sidecite{mirabal12}.

\begin{marginfigure}
  \includesvg{losartan}
  \caption{Pierwszy zarejestrowany lek z~grupy sartanów o~działaniu przeciwnadciśnieniowym.}
  \label{fig:losartan}
\end{marginfigure}
Ważniejsza wydaje mi się jednak rola tetrazoli w~kontekście aktywności biologicznej,
  jaką nadają zawierającym je cząsteczkom.
Najbardziej znaną grupą leków posiadających w~strukturze to ugrupowanie są sartany,
  czyli antagoniści receptora angiotensyny II, stosowane w~terapii nadciśnienia tętniczego.
Pierwszym zarejestrowany lekiem tego typu jest Losartan \refcmpd{losartan},
  przedstawiony na~\cref{fig:losartan}.
Pochodne tetrazoli wykazują jednak znacznie szerszą gamę aktywności \---
  znaleźć można liczne doniesienia o~ich działaniu przeciwbakteryjnym i~przeciwgrzybiczym,
  ale dostępne są też prace poświęcone przeciwnowotworowym, przeciwzapalnym, przeciwwirusowym
  i~wielu innym właściwościom tych związków\sidecite{wei15}.

Uważam, że najciekawszą cechą pierścienia tetrazolowego jest jego bioizosteryzm.
Termin ten używany jest przede wszystkim w~chemii medycznej, a~oznacza podobieństwo atomów
  lub grup funkcyjnych w~kontekście ich aktywności biochemicznej\sidecite{meanwell11}.
Zaproponował go \citeauthor{friedman50} w~\citeyear{friedman50}\sidecite{friedman50}
  biorąc za~podstawę wcześniejsze prace definiujące pojęcie izosteryczności \---
  czyli podobieństwa elektronowego i~fizykochemicznego, zwłaszcza małych cząsteczek\sidenote{%
    Na tej podstawie \citeauthor{langmuir19} przewidział właściwości fizyczne ketenu
      na~18 lat przed jego scharakteryzowaniem: \colorcite{langmuir19}.
  }.
Starszy z~konceptów zdaje się być nieco zapomniany, a~obydwa pojęcia bywają \---
  czasem niepoprawnie \--- używane wymiennie.

Zamiana fragmentu cząsteczki na~jego bioizoster pozwala zmodyfikować jej niektóre właściwości,
  przy zachowaniu czynności farmakologicznej.
Dzięki rozważnemu wyborowi takiej grupy na~etapie projektowania cząsteczki możliwe jest na~przykład
  zmniejszenie toksyczności, dostosowanie lipofilowości czy wydłużenie czasu metabolizmu
  nowego kandydata na~lek.
Chyba najbardziej znanym i~jednym z~najprostszych przykładów jest atom fluoru jako
  bioizoster atomu wodoru.
Zamiana \ch{H} na~\ch{F} w~strukturze powoduje zazwyczaj wzrost lipofilowości
  związku\sidecite{meanwell11}, a~w~przypadku Ezetimibu wprowadzenie dwóch atomów fluoru
  okazało się kluczowe do~uzyskania odpowiedniej stabilności metabolicznej\sidecite{clader04}.

\begin{figure}[b]
  \includesvg{bioisosteres}
  \caption{
    Bioizosteryczne pary tetrazoli: kwas karboksylowy, odpowiedni anion, oraz drugorzędowy amid.
  }
  \label{fig:bioisosteres}
\end{figure}
\Cref{fig:bioisosteres} przedstawia bioizosteryczne relacje tetrazolu z~innymi grupami funkcyjnymi:
  pierścień \iupac{\N\H-tetrazolowy} jest bioizosterem kwasu karboksylowego, lub \---
  w~formie zdeprotonowanej \--- odpowiedniego anionu, natomiast \iupac{1-podstawiony} tetrazol
  uważa się za~bioizoster drugorzędowego amidu.
Najczęściej oczekiwaną korzyścią z~wprowadzenia tej grupy do~biologicznie aktywnej cząsteczki
  jest zwiększenie jej biodostępności, czy łatwości przyswajania substancji przez organizm.
Wspomniany wcześniej Losartan (\cref{fig:losartan}) jest tego przykładem \---
  obecność tetrazolu w~jego strukturze zwiększa 10\=/krotnie jego aktywność w~stosunku
  do~pochodnej posiadającej w~tej pozycji podstawnik \ch{-COOH}\sidecite{carini98}.

Ze względu na~rosnące zainteresowanie pochodnymi tetrazolu, dostępnych jest coraz więcej metod
  syntezy tych związków.
Większość z~nich to wariacje najpopularniejszego podejścia, czyli reakcji cykloaddycji
  azydku do~nitryli.
Obszerny opis tych, jak i~alternatywnych, sposobów otrzymywania tetrazoli można znaleźć
  w,~wydanym podczas przygotowywania niniejszej dysertacji, przeglądowym artykule\sidecite{leyva21}.

\subsection{Wieloskładnikowe reakcje typu reakcji Ugiego}
Wśród różnych metod syntezy tetrazoli wyróżniają się reakcje wieloskładnikowe \---
  cechuje je wysoka ekonomia atomowa\sidenote{%
    Patrz: przypis \ref{note:atom-economy}, str.~\pageref{note:atom-economy}.
  }, pozwalają stosować strategię syntezy zbieżnej, a~jako przebiegające w~jednym naczyniu
  reakcyjnym są wygodne dla eksperymentatora\sidecite{neochoritis19}.
Najbardziej znaną i~najpowszechniej stosowaną reakcją tego typu jest bez wątpienia
  reakcja Ugiego\sidecite{sunderhaus09}.
Przemiana ta została zaproponowana przez Ugiego i~in. w~\citeyear{ugi59} roku\sidecite{ugi59}.
Biorą w~niej udział cztery komponenty:
  kwas karboksylowy~\refcmpd{carboxylic-acid}, pierwszorzędowa amina~\refcmpd{amine},
  keton~\refcmpd{ketone} lub aldehyd oraz izocyjanek~\refcmpd{isocyanide}.
Jak pokazuje \cref{sch:ugi}, formalnie jest reakcją kondensacji,
  a~w~jej wyniku powstaje diamid~\refcmpd{ugi-product}.
\begin{scheme}
  \includesvg{ugi}
  \caption{Uproszczony schemat czteroskładnikowej reakcji Ugiego.}
  \label{sch:ugi}
  \setfloatalignment{b}
\end{scheme}

Jak zazwyczaj w~przypadku reakcji wieloskładnikowych, przemiana ta prawdopodobnie zachodzi
  według różnych mechanizmów jednocześnie.
Badania prowadzone nad tym zagadnieniem są zgodne co do~pierwszego i~ostatniego etapu reakcji,
  czyli rozpoczęcia przemiany powstawaniem iminy~\refcmpd{imine} oraz jej zakończenia
  przegrupowaniem Mumma imidatu~\refcmpd{ugi-imidate}.
Ostatnie doniesienia w~tej materii sugerują, że przeważająca jest ścieżka przedstawiona
  na~\cref{sch:ugi-mech}, przebiegająca przez kation iminiowy~\refcmpd{iminium} oraz
  nitryliowy~\refcmpd{nitrilium}\sidecite{rocha20}.
Zazwyczaj reakcja ta przebiega najwydajniej w~polarnym protonowym rozpuszczalniku,
  takim jak metanol \--- możliwe, że odpowiada za~to stabilizacja polarnych stanów przejściowych.
W~niektórych przypadkach dobre rezultaty może również przynieść użycie polarnego
  aprotycznego rozpuszczalnika, najczęściej \ch{CH2Cl2}\sidecite{rocha20}.
\begin{scheme*}
  \includesvg{ugi-mech}
  \caption{Jeden z~możliwych mechanizmów tworzenia produktu w~reakcji Ugiego.}
  \label{sch:ugi-mech}
\end{scheme*}

Produkt reakcji Ugiego może być postrzegany jako oligopeptyd, z~czego wynikają długotrwałe
  wysiłki badaczy w~przeprowadzeniu tej reakcji w~sposób enancjoselektywny.
Pierwsza zakończona powodzeniem próba miała miejsce dopiero w~\citeyear{zhang18} \---
  po~niemal 60 latach od~jej odkrycia.
Udało się tego dokonać, prowadząc reakcję w~obecności chiralnej pochodnej kwasu fosforowego
  jako katalizatora\sidecite[\baselineskip]{zhang18}.
Używając jednego z~dwóch zaproponowanych katalizatorów, autorzy pracy byli w~stanie sterować
  stereochemią przemiany, otrzymując \iupac{\R-} lub \iupac{\S-}produkty z~doskonałą wydajnością
  i~wysokim nadmiarem enancjomerycznym.

Duże zainteresowanie reakcją Ugiego zaowocowało jej rozmaitymi modyfikacjami.
Wiele z~nich proponuje wykorzystanie substratów o~strukturze pozwalającej na~przeprowadzenie
  wewnątrzcząsteczkowej cyklizacji po,~przebiegającym standardowo, etapie kondensacji.
Spośród licznych przykładów można wymienić reakcję Ugiego-Dielsa-Aldera, kończącą się
  spontaniczną reakcją Dielsa-Aldera, albo reakcję Ugiego-Hecka, dodającą etap cyklizacji
  produktu z~wykorzystaniem wewnątrzcząsteczkowej reakcji Hecka\sidecite{sunderhaus09}.
Inne natomiast ingerują w~przebieg samego procesu kondensacji przez zmianę grup funkcyjnych
  biorących udział w~przemianie \--- na~przykład w~reakcji Ugiego-Smilesa finalne przegrupowanie
  Mumma jest zastąpione przegrupowaniem Smilesa, ze~względu na~użycie fenolu zamiast
  kwasu karboksylowego\sidecite{kaim06}.

Jednym z~wariantów tego drugiego typu jest reakcja azydo-Ugiego\sidenote{%
    Ang. \textit{Ugi-Azide Reaction}, czasem określana też akronimem \gls{azido-ugi}.
  }, w~której kwas karboksylowy zastąpiony jest azydkiem.
Została ona zaproponowana przez Ugiego i~Steinbr{\"u}cknera dwa lata po~ukazaniu się
  doniesienia o~oryginalnej wersji czteroskładnikowej reakcji Ugiego\sidecite{ugi61}.
\Cref{sch:ugi-azide} pokazuje przebieg tej reakcji \--- azydek wchodzi w~reakcję
  z~kationem nitryliowym~\cmpd{nitrilium}, prowadząc do~powstania \iupac{2-podstawionej}
  pochodnej tetrazolu~\refcmpd{aminotetrazole}.
Jest to zdecydowanie najczęściej wykorzystywana wieloskładnikowa metoda syntezy
  tetrazoli\sidecite{neochoritis19}.
\begin{scheme}
  \includesvg{ugi-azide}
  \caption{Przebieg reakcji azydo-Ugiego.}
  \label{sch:ugi-azide}
\end{scheme}

\subsection{Model i~optymalizacja}
W~\citeyear{nenajdenko13} \citeauthor{nenajdenko13} pokazali, że możliwe jest przeprowadzenie
  reakcji azydo-Ugiego w~wariancie Joulli{\'e}-Ugiego\sidenote{%
    Czyli wariancie, w~którym aminę i~keton/aldehyd zastępuje się przygotowaną wcześniej iminą.
  }\textsuperscript{,\thinspace}\sidecite{nenajdenko13}.
W~zespole badawczym, w~którym realizowana była niniejsza dysertacja, w~podobnym czasie
  z~sukcesem zastosowano tę wersję reakcji Ugiego do~funkcjonalizacji imin otrzymanych
  z~laktamów na~drodze reduktywnej aktywacji odczynnikiem Schwartza\sidenote{%
    \See{sch:our-joullie-ugi} w~sekcji \nameref{literature:schwartz:our}.}
Doniesienia te dają dobrą podstawę, by spodziewać się, że reakcja azydo-Ugiego poprzedzona
  reduktywną aktywacją laktamu również będzie owocna.
Trafność tej hipotezy udowodniłem, przeprowadzając przemianę zobrazowaną na~\cref{sch:preliminary}.
Wychodząc z~wywiedzionego~z glukozy laktamu~\refcmpd{glu-lactam} i~stosując ustalone wcześniej,
  optymalne warunki prowadzenia reduktywnej aktywacji odczynnikiem Schwartza\sidecite{furman14}
  oraz typowe warunki prowadzenia reakcji azydo-Ugiego\sidecite{nenajdenko13},
  otrzymałem z~zadowalającą, jak na~pierwszą próbę, wydajnością diastereomeryczne tetrazolowe
  pochodne: \refcmpd{glu-tet.cy, glu-epi-tet.cy}.
\begin{scheme*}
  \includesvg{preliminary}
  \caption{
    Pierwszy eksperyment sprawdzający możliwość syntezy tetrazolowych pochodnych iminocukrów
      w~sekwencji aktywacja amidu\--reakcja azydo-Ugiego.
    Użyłem \SI{1.6}{\equiv} \schwartz{}, \SI{1.1}{\equiv} \ch{CyNC} oraz
      \SI{1.1}{\equiv} \ch{TMSN3}.
    \acrshort{cy} \--- \acrlong{cy}; \acrshort{tms} \--- \acrlong{tms}.
  }
  \label{sch:preliminary}
\end{scheme*}

Podjąłem próbę optymalizacji procesu, szukając najlepszego donora protonu do~aktywacji \ch{TMSN3},
  będącego źródłem anionów azydkowych, odpowiedzialnych za~powstawanie pierścienia tetrazolowego.
\Cref{tab:sugars-opt} zawiera zbiór prób przeprowadzonych w~tym zakresie (pozycje 1. do~7.).
Zaskakująco, oczekiwany produkt powstaje, nawet jeśli żaden donor protonów nie zostanie użyty.
Co więcej, takie warunki okazały się najlepsze dla przebiegu reakcji \--- zapewniają najwyższą
  wydajność oraz diastereoselektywność procesu.
Przeprowadziłem również próbę wydzielenia powstającej jako związek pośredni iminy i~realizacji
  reakcji azydo-Joulli{\'e}-Ugiego w~osobnym etapie (pozycje 8. oraz 9.).
Prowadziło to jednak do~znacznego spadku wydajności, prawdopodobnie ze~względu na~niską trwałość
  imin o~strukturze takiej jak \refcmpd{glu-imine}.

\begin{table}
  {\includesvg{sugars-opt}}
  \centering
  \begin{tabular}{ccccc}
    \toprule
    \textnumero & Dodatek & Rozpuszczalnik & Wydajność /\% & \textit{d.r.} \textsuperscript{a} \\ \midrule
    \rownumber & \ch{MeOH} \textsuperscript{b} & \ch{THF} & \num{65} & $43:57$ \\
    \rownumber & \ch{CF3CO2H} & \ch{THF}  & \num{24} & $43:57$ \\
    \rownumber & \ch{AcOH} & \ch{THF} & \num{47} & $80:20$ \\
    \rownumber & \ch{Et3N*HCl} & \ch{THF} & \num{45} & $74:26$ \\
    \rownumber & \ch{H2O} & \ch{THF} & \num{34} & $>95:5$ \\
    \rownumber & \ch{(CF3)2CHOH} & \ch{THF} & \num{35} & $>95:5$ \\
    \rowcolor{\tablemarkecolor}
    \rownumber & brak & \ch{THF} & \num{73} & $>95:5$ \\
    \rownumber & brak & \ch{MeOH} & \num{19} \textsuperscript{c} & $>95:5$ \\
    \rownumber & brak & \ch{DCM} & \num{36} \textsuperscript{c} & $>95:5$ \\
    \bottomrule
  \end{tabular}
  \caption{%
    Optymalizacja syntezy \iupac{2-(1\hydrogen-tetrazol-5-ylo)-iminocurów} poprzez
      redukcję laktamu odczynnikiem Schwartza i~reakcję azydo-Ugiego.
    A: \SI{1.6}{\equiv} \ch{Cp2Zr(H)Cl} w~\ch{THF} w~atmosferze argonu;
    B: \SI{1.6}{\equiv} dodatku (jeśli występuje), \SI{1.1}{\equiv} \ch{CyNC}
      oraz \SI{1.1}{\equiv} \ch{TMSN3}.
    \textsuperscript{a}\iupac{2-\R} do \iupac{2-\S}, wg wydajności wydzielonych produktów.
    \textsuperscript{b}Dodatek użyty w~nadmiarze.
    \textsuperscript{c}Imina wydzielona po etapie redukcji.
  }\label{tab:sugars-opt}
\end{table}



\subsection{Wyniki i~zakres stosowalności}
\subsection{Dalsze przekształcenia produktów}
\subsection{Nietrafne alternatywne metody}
