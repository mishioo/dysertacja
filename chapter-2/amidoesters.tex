\section{Funkcjonaliacja amidoestrów}\label{synthesis:amidoesters}

Niezwykle istotną cechą metod aktywacji amidów jest ich selektywność
  względem ugrupowania amidowego, wspomniana wielokrotnie we~wstępie.
Wysoka selektywność danej metody funkcjonalizacji dodaje jej atrakcyjności w~oczach
  chemika-syntetyka, ponieważ pozwala na~przeprowadzenie pożądanej transformacji
  w~mniejszej liczbie kroków syntetycznych, bez potrzeby zabezpieczania wrażliwych grup,
  często z~lepszą wydajnością.
Co~więcej, obecność innych grup funkcyjnych w~cząsteczce umożliwia jej dalsze modyfikacje.
Interesującym typem substratu z~tego punktu widzenia mogą być
  \iupac{\b-karboksyamidy}~\refcmpd{amidoester}.
Ich struktura pozwala na~dwutorową modyfikację \---
  układ \iupac{1,3-dikarbonylowy} może być funkcjonalizowany elektrofilem dzięki
  zwiększonej kwasowości atomu wodoru w~pozycji 2, natomiast wiązanie amidowe może być
  reduktywnie aktywowane i~funkcjonalizowane nukleofilem.
Jak widać na~\cref{sch:divergent}, prowadziłoby to do~otrzymania pochodnych
  \iupac{1,2-di|funkc|jo|na|li|zo|wa|nych-\b-amino|kwasów}~\refcmpd{difunc-aminoester}.
\begin{scheme}
  \includesvg{divergent}
  \caption{Schematyczne przedstawienie dwutorowej funkcjonalizacji amidoestrów.}
  \label{sch:divergent}
\end{scheme}

Większość opisanych metod aktywacji amidów toleruje obecność grupy estrowej w~cząsteczce,
  na~przykład w~postaci grupy \ch{N-\gls{Boc}}\sidecite{nakajima14}
  lub jako odległy od~amidu podstawnik\sidecite{spletstoser07,huang15joc}.
Nie ma jednak dotąd doniesień o~próbie przeprowadzenia za~ich pomocą funkcjonalizacji amidoestrów
  o~strukturze malonianu, takich jak \refcmpd{amidoester}.
Postanowiłem wypełnić tę lukę, zaczynając od~prób prowadzonych na~prostej pochodnej kwasu
  malonowego, która nie posiada atomów wodoru w~pozycji 2.
W~serii prostych przekształceń, przedstawionych na~\cref{sch:amidoester-cycloprop-synthesis},
  przygotowałem amidoester \refcmpd{amidoester-cycloprop} z~grupą cyklopropylową w~pozycji 2.
\begin{scheme}
  \includesvg{amidoester-cycloprop-synthesis}
  \caption{
    Synteza związku modelowego~\refcmpd{amidoester-cycloprop} do~prób aktywacji i~reduktywnej
      funkcjonalizacji amidosetrów o~strukturze kwasu malonowego.
    Detale prowadzenia reakcji znajdują się w~sekcji \textit{\nameref{experimental:procedures}}.
    % either of below doesn't work because new-chapter-page has no number displayed
    % TODO: see if page number can be added
    % na~stronach~\cpagerefrange{syn:dimethyl-malonate}{syn:amidoester-cycloprop}.
    % \cpageref{
    %   syn:dimethyl-malonate, syn:cyclopropyl-dimethyl-malonate,
    %   syn:cyclopropyl-monomethyl-malonate, syn:cyclopropyl-methyl-malonate-chloride,
    %   syn:amidoester-cycloprop
    % }.
  }
  \label{sch:amidoester-cycloprop-synthesis}
\end{scheme}


\subsection{Optymalizacja}

\begin{table}
  {\includesvg{amidoester-opt}}  % wrap in brackets to prevent font settings leaking

  \vspace{.2\baselineskip}  % apparently needs to be in its own paragraph to work
  
  \begin{tabular}{ r c c c c c }
    \toprule
    \textnumero & \makecell{\ch{Cp2ZrHCl}\\/\si{\equiv}} & Kwas Lewisa
      & \makecell{Konwersja\\/\si{\percent}} & \makecell{Wydajność\\/\si{\percent}} & Temp. \\
    \midrule
    \rownumber & \num{0.9} & \ch{Yb(OTf)3} & \num{55.5} & \num{20.7} & \SI{-10}{\degreeCelsius} \\
    \rownumber & \num{0.9} & \ch{Yb(OTf)3} & \num{53.1} & \num{28.4} & RT \\
    \rownumber & \num{1.0} & \ch{Yb(OTf)3} & \num{58.1} & \num{27.5} & \SI{-10}{\degreeCelsius} \\
    \rowcolor{\tablemarkecolor}
    \rownumber & \num{1.0} & \ch{Yb(OTf)3} & \num{84.4} & \num{30.7} & RT \\
    \rownumber & \num{1.3} & \ch{Yb(OTf)3} & \num{66.9} & \num{21.2} & \SI{-10}{\degreeCelsius} \\
    \rownumber & \num{1.8} & \ch{Yb(OTf)3} & \num{75.2} & \num{16.9} & \SI{-10}{\degreeCelsius} \\
    \rownumber & \num{1.0} & \ch{Sc(OTf)3} & \num{55.9} & \num{22.0} & \SI{-10}{\degreeCelsius} \\
    \rownumber & \num{1.0} & \ch{Sn(OTf)2} & \multicolumn{2}{c}{\---} & \SI{-10}{\degreeCelsius} \\
    \rownumber & \num{1.0} & \ch{TMSOTf} & \multicolumn{2}{c}{\---} & \SI{-10}{\degreeCelsius} \\
    \rownumber & \num{1.0} & \ch{TFA} & \num{49.1} & \num{21.6} & \SI{-10}{\degreeCelsius} \\
    \rownumber & \num{1.0} & \ch{TFA} & \num{55.5} & \num{26.2} & RT \\
    \rownumber & \num{1.0} & \ch{BF3.OEt2} & \multicolumn{2}{c}{ślady} & \SI{-10}{\degreeCelsius} \\
    \rownumber & \num{1.0} & \ch{TiCl4} & \multicolumn{2}{c}{\---} & \SI{-10}{\degreeCelsius} \\
    \rownumber & \num{1.0} & \ch{(PhO)2PO2H} & \multicolumn{2}{c}{\---} & RT \\
    \rownumber & \num{1.0} & \ch{PhCO2H} & \num{49.7} & \num{19.8} & RT \\
    \bottomrule
  \end{tabular}
  \caption{Optymalizacja reduktywnej funkcjonalizacji amidoestru odczynnikiem Schwartza.}
  \label{tab:amidoester-opt}
\end{table}

\subsection{Zakres stosowalności}
\begin{margintable}
  \begin{tabular}{ r c c }
    \toprule
    \textnumero & Nukleofil & Wydajność /\si{\percent} \\
    \midrule
    \rownumber & {\includesvg{allylbutylstanne}} & \num{30.7} \\
    \rownumber & \ch{\acrshort{tms}CN} & \num{39.2} \\
    \rownumber & \ch{PhNH2} & \---\textsuperscript{a} \\
    \rownumber & \ch{PhMgBr} & \---\textsuperscript{b} \\
    \rownumber & indol & \---\textsuperscript{b} \\
    \rownumber & skatol\textsuperscript{c} & \---\textsuperscript{b} \\
    \bottomrule
  \end{tabular}
  \caption{
    Różne substraty.
    \textsuperscript{a}Brak reakcji.
    \textsuperscript{b}Złożona mieszanina.
    \textsuperscript{c}Czyli \iupac{3-metyloindol}.
    }
  \label{tab:amidoester-scope}
\end{margintable}