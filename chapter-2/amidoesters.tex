\section{Funkcjonaliacja amidoestrów}\label{synthesis:amidoesters}

Niezwykle istotną cechą metod aktywacji amidów jest ich selektywność
  względem ugrupowania amidowego, wspomniana wielokrotnie we~wstępie.
Wysoka selektywność danej metody funkcjonalizacji dodaje jej atrakcyjności w~oczach
  chemika-syntetyka, ponieważ pozwala na~przeprowadzenie pożądanej transformacji
  w~mniejszej liczbie kroków syntetycznych, bez potrzeby zabezpieczania wrażliwych grup,
  często z~lepszą wydajnością.
Co~więcej, obecność innych grup funkcyjnych w~cząsteczce umożliwia jej dalsze modyfikacje.
Interesującym typem substratu z~tego punktu widzenia mogą być
  \iupac{\b-karboksyamidy}~\refcmpd{amidoester}.
Ich struktura pozwala na~dwutorową modyfikację \---
  układ \iupac{1,3-dikarbonylowy} może być funkcjonalizowany elektrofilem dzięki
  zwiększonej kwasowości atomu wodoru w~pozycji 2, natomiast wiązanie amidowe może być
  reduktywnie aktywowane i~funkcjonalizowane nukleofilem.
Jak widać na~\cref{sch:divergent}, prowadziłoby to do~otrzymania pochodnych
  \iupac{1,2-di|funkc|jo|na|li|zo|wa|nych-\b-amino|kwasów}~\refcmpd{difunc-aminoester}.
\begin{scheme}
  \includesvg{divergent}
  \caption{Schematyczne przedstawienie dwutorowej funkcjonalizacji amidoestrów.}
  \label{sch:divergent}
\end{scheme}

Większość opisanych metod aktywacji amidów toleruje obecność grupy estrowej w~cząsteczce,
  na~przykład w~postaci grupy \ch{N-\gls{Boc}}\sidecite{nakajima14}
  lub jako odległy od~amidu podstawnik\sidecite{spletstoser07,huang15joc}.
Nie ma jednak dotąd doniesień o~próbach tak prowadzonej funkcjonalizacji amidoestrów
  o~strukturze malonianu, takich jak \refcmpd{amidoester}.

\subsection{Optymalizacja}

\begin{table}
  \begin{tabular}{ c c c c c }
    \ch{Cp2ZrHCl}/ekwiw.   & Kwas Lewisa     & Konwersja                 & Wydajność           & Temperatura \\
    \midrule
    \num{0.9}              & \ch{Yb(OTf)3}   & \SI{55.5}{\percent}       & \SI{20.7}{\percent} & \SI{-10}{\degreeCelsius} \\
    \num{0.9}              & \ch{Yb(OTf)3}   & \SI{53.1}{\percent}       & \SI{28.4}{\percent} & RT \\
    \num{1.0}              & \ch{Yb(OTf)3}   & \SI{58.1}{\percent}       & \SI{27.5}{\percent} & \SI{-10}{\degreeCelsius} \\
    \num{1.3}              & \ch{Yb(OTf)3}   & \SI{66.9}{\percent}       & \SI{21.2}{\percent} & \SI{-10}{\degreeCelsius} \\
    \num{1.8}              & \ch{Yb(OTf)3}   & \SI{75.2}{\percent}       & \SI{16.9}{\percent} & \SI{-10}{\degreeCelsius} \\
    \num{1.0}              & \ch{Yb(OTf)3}   & \SI{84.4}{\percent}       & \SI{30.7}{\percent} & RT \\
    \num{1.0}              & \ch{Sc(OTf)3}   & \SI{55.9}{\percent}       & \SI{22.0}{\percent} & \SI{-10}{\degreeCelsius} \\
    \num{1.0}              & \ch{Sn(OTf)2}   & \multicolumn{2}{c}{\---}                        & \SI{-10}{\degreeCelsius} \\
    \num{1.0}              & \ch{TMSOTf}     & \multicolumn{2}{c}{\---}                        & \SI{-10}{\degreeCelsius} \\
    \num{1.0}              & \ch{TFA}        & \SI{49.1}{\percent}       & \SI{21.6}{\percent} & \SI{-10}{\degreeCelsius} \\
    \num{1.0}              & \ch{TFA}        & \SI{55.5}{\percent}       & \SI{26.2}{\percent} & RT \\
    \num{1.0}              & \ch{BF3.OEt2}   & \multicolumn{2}{c}{ślady}                       & \SI{-10}{\degreeCelsius} \\
    \num{1.0}              & \ch{TiCl4}      & \multicolumn{2}{c}{\---}                        & \SI{-10}{\degreeCelsius} \\
    \num{1.0}              & \ch{(PhO)2PO2H} & \multicolumn{2}{c}{\---}                        & RT \\
    \num{1.0}              & \ch{PhCO2H}     & \SI{49.7}{\percent}       & \SI{19.8}{\percent} & RT \\
  \end{tabular}
  \caption{Optymalizacja reduktywnej funkcjonalizacji amidoestru odczynnikiem Schwartza.}
  \label{tab:amidoester-opt}
\end{table}
