\section{Funkcjonaliacja amidoestrów}\label{synthesis:amidoesters}

% reaktywność 1,3-dikarbonylowych oparta głównie o łatwość enolizacji
% reaktywność wobec elektrofila, zwłaszcza w przypadku dwuanionów
% próba opracowania nowej reaktywności 1,3-karb: "takie jakby przebiegunowanie"
% sprzedać jako częściowy sukces
% addycja do dimetylo nieudana bo imina z układem neopentylowym jest nieaktywna na nukleofile

Niezwykle istotną cechą metod aktywacji amidów jest ich selektywność
  względem ugrupowania amidowego, wspomniana wielokrotnie we~wstępie.
Wysoka selektywność danej metody funkcjonalizacji dodaje jej atrakcyjności w~oczach
  chemika-syntetyka, ponieważ pozwala na~przeprowadzenie pożądanej transformacji
  w~mniejszej liczbie kroków syntetycznych, bez potrzeby zabezpieczania wrażliwych grup,
  często z~lepszą wydajnością.
Co~więcej, obecność innych grup funkcyjnych w~cząsteczce umożliwia jej dalsze modyfikacje.
Interesującym typem substratu z~tego punktu widzenia mogą być
  \iupac{\b-karboksyamidy}~\refcmpd{amidoester}.

\begin{marginscheme}
  \includesvg{dimedone}
  \caption{Forma ketonowa i~enolowa dimedonu w~roztworze \ch{CDCl3}.}
  \label{sch:dimedone}
\end{marginscheme}
Reaktywność związków \iupac{1,3-dikarbonylowych} opiera się przede wszystkim na~łatwości
  ich enolizacji.
W~przypadku wielu tego typu związków forma enolowa jest stabilna \---
  \cref{sch:dimedone} obrazuje przykład dimedonu\sidenote{
    Nazwa systematyzczna to~\iupac{5,5-dimetylocykloheksan-1,3-dion}.
  }, który w~roztworze \ch{CDCl3} przyjmuje postać ketonową~\refcmpd{dimedone}
  i~enolową~\refcmpd{dimedone-enol} w~proporcji 2~do~1\sidecite{natsuko65}.
Struktura takich związków ułatwia też deprotonowanie, ponieważ powstający anion jest
  dobrze stabilizowany przez obydwie grupy karbonylowe.
Zarówno taki anion, jak i~sam enol chętnie wstępują w~reakcje z~elektrofilami,
  pozwalając na~łatwą funkcjonalizację w~pozycji 2.

Struktura \iupac{\b-karboksyamidu} daje możliwość dwutorowej modyfikacji \---
  oprócz tradycyjnej funkcjonalizacji układu \iupac{1,3-dikarbonylowego} elektrofilem w~pozycji 2,
  obecne w~cząsteczce wiązanie amidowe może być poddane reduktywnej funkcjonalizacji nukleofilem.
Jak widać na~\cref{sch:divergent}, prowadziłoby to do~otrzymania pochodnych
  \iupac{1,2-di|funkc|jo|na|li|zo|wa|nych-\b-amino|kwasów}~\refcmpd{difunc-aminoester}.
Taka reaktywność, choć niewątpliwie atrakcyjna, nie została dotąd w~ogóle opisana w~literaturze,
  jej osiągnięcie może się więc okazać niejakim wyzwaniem.
Podstawę do~optymizmu daje natomiast fakt, że większość metod aktywacji amidów toleruje obecność
  grupy estrowej w~cząsteczce, na~przykład w~postaci grupy \ch{N-\gls{Boc}}\sidecite{nakajima14}
  lub jako odległy od~amidu podstawnik\sidecite{spletstoser07,huang15joc}.
\begin{scheme}[t]
  \includesvg{divergent}
  \caption{Schematyczne przedstawienie dwutorowej funkcjonalizacji amidoestrów.}
  \label{sch:divergent}
\end{scheme}

Postanowiłem sprawdzić, czy dwutorowa funkcjonalizacja amidoestrów~\refcmpd{amidoester} jest możliwa.
Zacząłem od~prób aktywacji amidu prowadzonych na~prostej pochodnej kwasu
  malonowego, która nie posiada atomów wodoru w~pozycji 2.
W~serii prostych przekształceń przygotowałem amidoester \refcmpd{amidoester-cycloprop} z~grupą
  cyklopropylową w~pozycji 2.
Schemat tej syntezy, detale prowadzenia reakcji oraz opis analiz tego i~innych substratów
  znajduje się w~sekcji \textit{\nameref{experimental:amidoester-substrates}}.

\subsection{Optymalizacja}
\begin{table}[b!]
  {\includesvg{amidoester-opt}}  % wrap in brackets to prevent font settings leaking

  \vspace{.2\baselineskip}  % apparently needs to be in its own paragraph to work
  
  \begin{tabular}{ r c c c c c }
    \toprule
    \textnumero & \makecell{\ch{Cp2ZrHCl}\\/\si{\equiv}} & Aktywator
      & \makecell{Konwersja\\/\si{\percent}} & \makecell{Wydajność\\/\si{\percent}} & Temp. \\
    \midrule
    \rownumber & \num{0.9} & \ch{Yb(OTf)3} & \num{56} & \num{21} & \SI{-10}{\degreeCelsius} \\
    \rownumber & \num{1.0} & \ch{Yb(OTf)3} & \num{58} & \num{28} & \SI{-10}{\degreeCelsius} \\
    \rownumber & \num{1.3} & \ch{Yb(OTf)3} & \num{67} & \num{21} & \SI{-10}{\degreeCelsius} \\
    \rownumber & \num{1.8} & \ch{Yb(OTf)3} & \num{75} & \num{17} & \SI{-10}{\degreeCelsius} \\
    \rownumber & \num{0.9} & \ch{Yb(OTf)3} & \num{53} & \num{28} & RT \\
    \rowcolor{\tablemarkecolor}
    \rownumber & \num{1.0} & \ch{Yb(OTf)3} & \num{84} & \num{31} & RT \\
    \rownumber & \num{1.0} & \ch{Sc(OTf)3} & \num{56} & \num{22} & \SI{-10}{\degreeCelsius} \\
    \rownumber & \num{1.0} & \ch{Sn(OTf)2} & \multicolumn{2}{c}{\---} & \SI{-10}{\degreeCelsius} \\
    \rownumber & \num{1.0} & \ch{TMSOTf} & \multicolumn{2}{c}{\---} & \SI{-10}{\degreeCelsius} \\
    \rownumber & \num{1.0} & \ch{TFA} & \num{49} & \num{22} & \SI{-10}{\degreeCelsius} \\
    \rownumber & \num{1.0} & \ch{TFA} & \num{56} & \num{26} & RT \\
    \rownumber & \num{1.0} & \ch{BF3.OEt2} & \multicolumn{2}{c}{ślady} & \SI{-10}{\degreeCelsius} \\
    \rownumber & \num{1.0} & \ch{TiCl4} & \multicolumn{2}{c}{\---} & \SI{-10}{\degreeCelsius} \\
    \rownumber & \num{1.0} & \ch{(PhO)2PO2H} & \multicolumn{2}{c}{\---} & RT \\
    \rownumber & \num{1.0} & \ch{PhCO2H} & \num{49} & \num{20} & RT \\
    \bottomrule
  \end{tabular}
  \caption{
    Optymalizacja reduktywnej funkcjonalizacji amidoestru odczynnikiem Schwartza.
    Błękitnym kolorem zaznaczyłem wiersz zawierający optymalne warunki.
    Eksperymenty prowadziłem w~atmosferze gazu obojętnego (argonu), na~skalę \SI{2}{\milli\mol}
      substratu \refcmpd{amidoester-cycloprop}.
    W~etapie funkcjonalizacji używałem \SI{2}{\equiv} aktywatora oraz \SI{3}{\equiv}
      allilotributylocyny.
    Stosowałem \gls{thf} jako rozpuszczalnik.
    Pełen opis procedury znajduje się w~sekcji \textit{\nameref{experimental:amidoester-products}}.
  }
  \label{tab:amidoester-opt}
\end{table}

Badania zacząłem od~próby optymalizacji metody wykorzystującej odczynnik Schwartza.
Modelowy związek \refcmpd{amidoester-cycloprop} poddałem funkcjonalizacji za~pomocą
  allilotributylocyny, nukleofila często wykorzystywanego w~próbnych reakcjach tego typu.
Sprawdziłem wpływ ilości \schwartz{}, rodzaju użytego aktywatora (kwasu Lewisa lub protonowego) oraz,
  w~ograniczonym stopniu, temperatury na~przebieg reakcji.
Wyniki tych studiów zebrałem w~\cref{tab:amidoester-opt}.

Spośród przetestowanych aktywatorów jedynie \ch{Yb(OTf)3}, \ch{Sc(OTf)3}, \gls{tfa}
  oraz kwas benzoesowy pozwoliły otrzymać oczekiwany produkt funkcjonalizacji.
Pozostałe, których użycie w~tego typu reakcjach znalazłem w~literaturze,
  prowadziły do~rozkładu substratu lub \--- w~przypadku \ch{BF3.OEt2} \--- powstania jedynie
  śladowych ilości związku \refcmpd{b-aminoester-cycloprop.allyl}.
Wbrew doświadczeniu uzyskanemu w~toku badań prowadzonych wcześniej w~grupie Furmana,
  zwiększenie ilości użytego do~redukcji odczynnika Schwartza wpłynęło negatywnie
  na~wydajność procesu.
Podobnie, niesprzyjające było obniżenie temperatury prowadzenia reakcji.

\begin{marginfigure}
  \begin{tikzpicture}
    \begin{axis}[
      ylabel={Wydajność/Konwersja /\si{\percent}},
      xlabel={Ilość \schwartz{} /\si{\equiv}},
      ymin=0, ymax=100,
      xmin=0.8, xmax=1.9,
      ytick={0,50,100},
      axis lines=left,
      axis line style={-},
      x label style={at={(0.5,0.05)}},
      y label style={at={(0.1,0.5)}},
      legend style={at={(0.05,1.1)}, anchor={north west}, draw=none},
      mark options={scale=0.7},
      width=1.1*\textwidth,
    ]
    
    \addplot[color=wongvermillion,mark=*]
      coordinates {(0.9,56)(1.0,58)(1.3,67)(1.8,75)};
    \addlegendentry{Konwersja}
      
    \addplot[color=wonggreen,mark=square*]
      coordinates {(0.9,21)(1.0,28)(1.3,21)(1.8,17)};
    \addlegendentry{Wydajność}
            
    \end{axis}
  \end{tikzpicture}
  \caption{
    Konwersja i~wydajność reduktywnej funkcjonalizacji amidoestru~\refcmpd{amidoester-cycloprop}
      w~zależności od~ilości użytego odczynnika Schwartza.
    Naniesione na~wykres dane pochodzą z~serii eksperymentów prowadzonych w~obniżonej temperaturze
      (wiersze \numlist{1; 3; 5; 6} w~\cref{tab:amidoester-opt}).
  }
  \label{fig:amidoester-opt-plot}
\end{marginfigure}
Najlepszy rezultat uzyskałem stosując \SI{1.0}{\equiv} \schwartz{} w~połączeniu
  z~\ch{Yb(OTf)3} w~temperaturze pokojowej.
Niestety, nawet w~tym wypadku wydajność pożądanego produktu jest dość niska,
  pomimo wysokiej konwersji substratu.
Takie obserwacje sugerują, że selektywność odczynnika Schwartza w~tym układzie
  jest gorsza, niż można by spodziewać się na~podstawie opisanych w~literaturze przykładów.
Świadczy o~tym spadek wydajności produktu połączony ze~wzrostem konwersji substratu,
  następujące wraz~ze zwiększaniem powyżej \SI{1}{\equiv} ilości użytego reduktora.
Problem ten zobrazowany jest w~postaci wykresu na~\cref{fig:amidoester-opt-plot}.

Trudno z~przekonaniem powiedzieć jakim reakcjom ubocznym ulega substrat w~badanych warunkach.
Testy wykonane techniką \gls{tlc} sugerowały, że w~mieszaninie poreakcyjnej znajdują się
  związki o~znacznie większej polarności niż substrat czy spodziewany produkt,
  ale nie udało mi się ich wydzielić typowymi metodami chromatograficznymi\sidenote[][-1\baselineskip]{%
    Mam tu na myśi chromatografię kolumnową na~żelu krzemionkowym.
    Nie próbowałem stosować chromatografii na~fazach odwróconych ani innych specjalistycznych metod.
  }.
Przypuszczalnie były to związki zawierające grupy \ch{-OH} oraz \ch{-NH2},
  powstałe w~skutek nadmiarowej redukcji lub hydrolizy produktów pośrednich.

\subsection{Zakres stosowalności}
Mimo niezadowalających wyników optymalizacji postanowiłem poddać metodę dalszym próbom.
Przetestowałem jej kompatybilność z~innymi nukleofilami, a~także wpływ struktury substratu
  na~przebieg reakcji.
Niektóre wyniki tych badań, zebranych poniżej, były dość zaskakujące, zwłaszcza w~kontekście
  serii eksperymentów poświęconych innym substratom.

\begin{margintable}
  {\includesvg{b-aminoester-cycloprop}}  % wrap in brackets to prevent font settings leaking
  \begin{tabular}{ r c c }
    \toprule
    \textnumero & Nukleofil & Wydajność /\si{\percent} \\
    \midrule
    \rownumber & {\includesvg{allylbutylstanne-small}} & \num{31} (\refcmpd{b-aminoester-cycloprop.allyl}) \\
    \rownumber & \ch{\acrshort{tms}CN} & \num{39} (\cmpd{b-aminoester-cycloprop.cn}) \\
    \rownumber & \ch{PhNH2} & \---\textsuperscript{a} \\
    \rownumber & \ch{PhMgBr} & \---\textsuperscript{b} \\
    \rownumber & indol & \---\textsuperscript{a} \\
    \rownumber & skatol & \---\textsuperscript{a} \\
    \bottomrule
  \end{tabular}
  \caption{
    Wyniki studiów nad kompatybilnością różnych typów nukleofili z~badaną metodą.
    W~każdym przypadku zastosowałem warunki reakcji ustalone podczas opisanej wcześniej optymalizacji.
    Na~górze tabeli znajduje się ogólna struktura oczekiwanego produktu.
    \textsuperscript{a}Brak reakcji.
    \textsuperscript{b}Rozkład substratu.
    }
  \label{tab:amidoester-scope}
\end{margintable}
Spis użytych do~testów nukleofili, wraz z~rezultatami, znajduje się w~\cref{tab:amidoester-scope}.
Oprócz allilotributylocyny z~modelowego przykładu,
  sprawdziłem też anion cyjankowy (\ch{\acrshort{tms}CN}), anilinę,
  odczynnik Grignarda (\ch{PhMgBr}) oraz \textpi{}-nukleofile o~różnym charakterze
  (indol i~skatol\sidenote{Czyli \iupac{3-metyloindol}.}).
Spośród nich tylko użycie cyjanku trimetylosililu zaowocowało powstaniem oczekiwanego produktu.
Pozostałe prowadziły do~rozkładu substratu (\ch{PhMgBr}) lub w~ogóle nie wchodziły
  w~reakcję z~aktywowanym amidem~\refcmpd{amidoester-cycloprop} (\ch{PhNH2}, indol i~skatol).

Podstawowe testy poświęcone wpływowi struktury substratu na~przebieg reakcji,
  zebrane na~\cref{sch:amidoester-other}, również zaowocowały umiarkowanym sukcesem.
Prosty, wywiedziony z~kwasu malonowego amid~\refcmpd{amidoester-plain} przekształciłem
  w~odopwiednią alliloaminę~\refcmpd{aminoester-plain-allyl} z~wydajnością nieco lepszą
  niż w~przypadku modelowego substratu.
Trochę gorszy wynik uzyskałem w~reakcji \iupac{2-benzylowej} pochodnej~\refcmpd{amidoester-bn},
  ale oczekiwany związek~\refcmpd{aminoester-bn-allyl} udało mi się wydzielić i~scharakteryzować.
% TODO: comment on only one diastereomer

\begin{scheme}
  \includesvg{amidoester-other}
  \caption{
    Zastosowanie metodologii reduktywnej aktywacji amidów odczynnikiem Schwartza
      do~funkcjonalizacji innych \textbeta{}-amidoestrów.
    A~symbolizuje optymalne warunki prowadzenia reakcji według wykonanej wcześniej
      optymalizacji, z~użyciem allilotributylocyny jako nukleofila.
    Szczegółowy opis procedury znajduje się w~\protect\secref{experimental:amidoester-products}.
  }
  \label{sch:amidoester-other}
\end{scheme}

Zaskoczeniem był dla mnie wynik próby przekształcenia dimetylowej
  pochodnej~\refcmpd{amidoester-dime}.
Choć związek ten ma strukturę bardzo podobną do~modelowego substratu~\refcmpd{amidoester-cycloprop},
  jego funkcjonalizacja w~opracowanych warunkach nie powiodła się, a~oczekiwany
  produkt~\refcmpd{aminoester-dime-allyl} w~ogóle nie powstał.
Jak pokazali pionierzy metody\sidecite{schedler93} oraz jej wnikliwi badacze\sidecite{nakajima14},
  rozpad kompleksu powstałego po~addycji odczynnika Schwartza prowadzi
  do~iminy\sidenote{Albo kationu iminiowego, jeśli redukcji poddawany jest amid trzeciorzędowy.}.
Dopiero ten produkt pośredni ulega addycji nukleofila.
Fenomen ten można więc próbować wyjaśnić przez porównanie do~iminy o~układzie neopentylowym,
  która, ze względów sterycznych, jest nieaktywna wobec czynników nukleofilowych\todo{Citation needed}.
% TODO: Rephrase neopentyl part

Odnotowania wymaga też niespodziewany przebieg redukcji laktonu~\refcmpd{amidoester-cyclo}
  z~funkcją amidową w~łańcuch bocznym.
Analiza mieszaniny reakcyjnej pzy użyciu technik \gls{ms} pozwala sądzić,
  że w~tym wypadku redukcji pod wpływem \schwartz{} uległ fragment laktonowy zamiast amidu,
  prowadząc do~powstania hemiacetalu~\refcmpd{amido-cyclo-hemiacetal}.
Pierwsze próby użycia odczynnika Schwartza w~syntezie, wykonane przez zespół Ganema,
  zarejestrowały pewną jego aktywność wobec związków dikarbonylowych, ale przeprowadzenie
  redukcji wymagało wstępnej aktywacji\sidecite{godfrey92}.
Zbliżoną reaktywność \--- a~dokładnie powstawanie hemiaminalu \--- zaobserwowano również
  w~przypadku niektórych pochodnych hydantoiny oraz izooksazolidynonu\sidenote{
    Patrz: \cref{sch:nboc-hydantoin-reduction-zr} oraz \ref{sch:isoxazolidinone-reduction-zr}
    na~str.~\pageref{sch:nboc-hydantoin-reduction-zr}
  }.
Preferencja redukcji laktamu nad grupą amidową byłaby bezprecedensowa,
  ale pozostaje jedynie hipotezą, ponieważ zaobserwowanego w~eksperymencie \gls{ms}
  związku~\refcmpd{amido-cyclo-hemiacetal} nie udało mi się wydzielić.

\subsection{Inne metody aktywacji}
Wobec nie w~pełni zadowalających wyników osiągniętych przy użyciu metody opartej o~\schwartz{}
  sięgnąłem po inne dostępne procedury \--- wykorzystujące katalityczne ilości kompleksów irydu.
W~eksperymentach tych wykorzystałem dwa spośród używanych dotąd substratów.
W~\cref{tab:amidoester-methods} zebrałem wyniki prób przeprowadzonych w~tym zakresie,
  wraz z~wynikami uzyskanymi poprzednio, dla porównania.
Prowadziły one do~rozkładu substratu, z~wyjątkiem zastosowania procedury wykorzystującej
  kompleks Vaski do~redukcji związku~\refcmpd{amidoester-cyclo} \---
  w~tym przypadku zaobserwowałem otwarcie laktonu.

\begin{table*}
  \begin{tabular}{ 
    C{0.14\textwidth}
    C{0.18\textwidth}
    C{0.18\textwidth}
    C{0.18\textwidth}
    C{0.18\textwidth}
  }
    \toprule
    \multirow{2}{*}{Substrat} & \multicolumn{4}{c}{Zastosowana procedura aktywacji} \\\cmidrule{2-5}
                              & Schwartz & Vaska & van der Ent & $2$-\ch{F}-\ch{Py.Tf2O} \\
    \midrule\addlinespace
    \refcmpd{amidoester-bn} & \refcmpd{aminoester-bn-allyl}, \SI{22}{\percent} & rozkład & rozkład & \refcmpd{aminoester-bn-allyl}, ślady \\\addlinespace
    \refcmpd{amidoester-cyclo} & \refcmpd{amido-cyclo-hemiacetal}, \SI{15}{\percent} & {\includesvg{tab-methods-opened}} & rozkład & {\includesvg{tab-methods-cyclo-allyl}} \\
    \bottomrule
  \end{tabular}
  \caption[][\baselineskip]{
    Próby zastosowania do~redukcji amidoestrów procedur wykorzystujących katalityczne
      ilości kompleksów irydu.
    \enquote{Schwartz}, \enquote{Vaska}, \enquote{van der Ent} oraz $2$-\ch{F}-\ch{Py.Tf2O}
      w~nagłówku reprezentują metody wykorzystujące odpowiadające kompleksy.
    Szczegółowe procedury znajdują się w~\protect\secref{experimental:amidoester-products}.
  }\label{tab:amidoester-methods}
\end{table*}

Nadzieję na~lepszy wynik dawała początkowo metoda oparta o~bezwodnik triflowy.
Według analizy techniką \gls{ms} w~obydwu przypadkach w~mieszanina reakcyjna zawierała
  pożądany związek funkcjonalizowany grupą allilową,
  jednak w~zarejestrowanych widmach \NMR*{} tych mieszanin nieobecne były charakterystyczne
  sygnały pochodzące od~protonów położonych przy wiązaniu \ch{C=C}.
Na~tej podstawie wnioskuję, że w~reakcji powstały jedynie śladowe ilości związków 
  \refcmpd{aminoester-bn-allyl, aminoester-cyclo-allyl}.

Takim samym próbom poddałem prostą pochodną hydantoiny \--- heterocyklicznego związku
  \iupac{1,3-dikarbonylowego} o~zupełnie innej budowie elektronowej niż badane dotąd amidoestry.
Przywołany niedawno fakt powstawania hemiaminalu pod wpływem odczynnika Schwartza pozwala sądzić,
  że w~sprzyjających warunkach związki te mogłyby być poddane reduktywnej funkcjonalizacji.
Mimo poczynionych w~tym kierunku starań, nie udało mi się jednak tego osiągnąć.
Zarówno użycie \schwartz{}, metod wykorzystujących kompleksy irydu, jak i~procedury z~bezwodnikiem
  triflowym nie zaowocowało powstaniem pożądanego produktu~\refcmpd{hydantoin-bn-allyl},
  widocznego na~\cref{sch:hydantoin-attempt}.

\begin{marginscheme}
  \includesvg{hydantoin-attempt}
  \caption{
    Schematyczne przedstawienie prób funkcjonalizacji hydantoin, zakończonych niepowodzeniem.
    Jako metodę aktywacji przetestowałem procedury wykorzystujące odczynnik Schwartza,
      kompleks Vaski, kompleks van der Enta oraz bezwodnik triflowy.
    Przetestowałem dwa substraty o~strukturze~\refcmpd{hydantoin-bn}:
      niepodstawiony oraz metylowany (odpowiednio \ch{R} = \ch{H} oraz \ch{R} = \ch{CH3}).
  }\label{sch:hydantoin-attempt}
\end{marginscheme}

Funkcjonalizacja \iupac{\b-karbonylowych} amidów okazała się zadaniem trudniejszym
  niż wskazywałyby na~to przesłanki z~literatury.
Znane są przykłady selektywnej aktywacji amidów posiadających izolowaną grupę karbonylową
  lub estrową w~strukturze, jednak zastosowane w~nich procedury nie pozwoliły na~osiągnięcie
  podobnych wyników w~przypadku badanych związków.
Jedynie użycie odczynnika Schwartza zaowocowało powstaniem oczekiwanych produktów
  funkcjonalizacji, choć z~niewysoką wydajnością i~w~ograniczonym zakresie substratowym.
W~obliczu zupełnej nieskuteczności innych metod aktywacji poczytuję to za~mały sukces,
  choć niedostateczny, by kontynuować badania na~tym polu.
