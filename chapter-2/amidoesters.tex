\section{Funkcjonaliacja amidoestrów}\label{synthesis:amidoesters}

Niezwykle istotną cechą metod aktywacji amidów jest ich selektywność
  względem ugrupowania amidowego, wspomniana wielokrotnie we~wstępie.
Wysoka selektywność danej metody funkcjonalizacji dodaje jej atrakcyjności w~oczach
  chemika-syntetyka, ponieważ pozwala na~przeprowadzenie pożądanej transformacji
  w~mniejszej liczbie kroków syntetycznych, bez potrzeby zabezpieczania wrażliwych grup,
  często z~lepszą wydajnością.
Co~więcej, obecność innych grup funkcyjnych w~cząsteczce umożliwia jej dalsze modyfikacje.
Interesującym typem substratu z~tego punktu widzenia mogą być
  \iupac{\b-karboksyamidy}~\refcmpd{amidoester}.
Taka struktura pozwalałaby na~dwutorową modyfikację \---
  układ \iupac{1,3-dikarbonylowy} może być funkcjonalizowany elektrofilem dzięki
  zwiększonej kwasowości atomu wodoru w~pozycji 2, natomiast wiązanie amidowe może być
  reduktywnie aktywowane i~funkcjonalizowane nukleofilem.
Jak widać na~\cref{sch:divergent}, prowadziłoby to do~otrzymania pochodnych
  \iupac{1,2-di|funkc|jo|na|li|zo|wa|nych-\b-amino|kwasów}~\refcmpd{difunc-aminoester}.
\begin{scheme}
  \includesvg{divergent}
  \caption{Schematyczne przedstawienie dwutorowej funkcjonalizacji amidoestrów.}
  \label{sch:divergent}
\end{scheme}

Większość opisanych metod aktywacji amidów toleruje obecność grupy estrowej w~cząsteczce,
  na~przykład w~postaci grupy \ch{N-\gls{Boc}}\sidecite{nakajima14}
  lub jako odległy od~amidu podstawnik\sidecite{spletstoser07,huang15joc}.
Nie ma jednak dotąd doniesień o~próbach tak prowadzonej funkcjonalizacji amidoestrów
  o~strukturze malonianu.
