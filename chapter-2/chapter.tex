\chapter{Badania syntetyczne}\label{chapter:synthesis}

Podczas prac nad syntezą sfunkcjonalizowanych amin poprzez aktywację amidów
  stosowałem najczęściej metodologię opartą o~użycie odczynnika Schwartza.
Wybór ten nie był arbitralny \--- w~przypadku każdego typu badanych przeze mnie
  amidów opisuję też próby wykorzystujące inne procedury,
  jednak zwykle próby te nie były owocne.
Studia opisane w~niniejszej dysertacji obejmują dwa typy amidów:
  związki, w~których wiązanie amidowe jest częścią układu \iupac{1,3-dikarbonylowego}
  oraz, w~znacznie większym stopniu, laktamy wywiedzione z~cukrów prostych.
Obydwu klasom związków poświęciłem oddzielną sekcję w~tym rozdziale.

\subimport{./}{amidoesters}
\subimport{./}{sugars}
