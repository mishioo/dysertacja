\chapter{Badania syntetyczne}\label{chapter:synthesis}

Podczas prac nad syntezą sfunkcjonalizowanych amin poprzez aktywację amidów
  stosowałem najczęściej metodologię opartą o~użycie odczynnika Schwartza.
Wybór ten nie był arbitralny \--- w~przypadku każdego typu badanych przeze mnie
  amidów opisuję też próby wykorzystujące inne procedury,
  jednak zwykle próby te nie były owocne.
Studia opisane w~niniejszej dysertacji obejmują trzy typy amidów:
  amidoestry będące pochodnymi kwasu malonowego,
  \iupac{imidazolino-2,4-diony} nazywane zwyczajowo hydantoinami,
  oraz laktamy wywiedzione z~cukrów prostych.
Każdej z~wymienionych klas związków poświęciłem oddzielną sekcję w~tym rozdziale.

\subimport{./}{amidoesters}
\subimport{./}{hydantoins}
\subimport{./}{sugars}
