\documentclass[a4paper, 12pt]{book}
\usepackage[total={6in, 9in}, left=1in, right=1in]{geometry}

%% font encoding
\usepackage[utf8]{inputenc}
\usepackage[T1]{fontenc}

%% normal font
\usepackage{lmodern}

%% language dependent
\usepackage[polish]{babel}
\usepackage{polski}
\usepackage[autostyle]{csquotes}
\usepackage{indentfirst}


% pretty tabs
% do I need it?
% \usepackage{booktabs}


%% Bibliography and citations setup
% based on https://tex.stackexchange.com/q/139158
\usepackage[
    citestyle=numeric,
    % bibstyle=authoryear,  % needed only for traditional bibliography
    sorting=none,
    backref=true,
    citetracker,
    backend=biber,
    % autocite=footnote,  % not wanted in draft
    labeldateparts,  % make year of publication available
    giveninits=true,  % names as initials
    url=false,  % suppress url everywhere but in `online'
    maxbibnames=3,  % if more than 3 autors, produce et al.
]{biblatex}
% suppress `In:' in article citations
% from https://tex.stackexchange.com/a/10686/223674
\renewbibmacro{in:}{%
  \ifentrytype{article}{}{\printtext{\bibstring{in}\intitlepunct}}}
% suppress title of articles in citations
\DeclareFieldFormat[article]{title}{}
% suppress ISSN
\AtEveryCitekey{\clearfield{issn}}
% punctuation in autocite
\DeclareAutoCiteCommand{footnote}[r]{\footcite}{\footcites}
% references location
\addbibresource{refs.bib}

%% define shortcuts for colors from Wong's palette
\usepackage{xcolor}
\definecolor{wongblack}{HTML}{000000}
\definecolor{wongorange}{HTML}{E69F00}
\definecolor{wongsky}{HTML}{56B4E9}
\definecolor{wonggreen}{HTML}{009E73}
\definecolor{wongyellow}{HTML}{F0E442}
\definecolor{wongblue}{HTML}{0072B2}
\definecolor{wongvermillion}{HTML}{D55E00}
\definecolor{wongpurple}{HTML}{CC79A7}


%% TODOs in form of inline notes
\newcommand{\todo}[2][]{{ \color{wongpurple}[TODO: #2]}}


%% setup code snippets
\usepackage{inconsolata}  % monospaced font
\usepackage{listings}  % code listings
\lstset{
    breaklines=true,
    % numbers=left,
    numberstyle=\tiny,
    basicstyle=\ttfamily,
    keywordstyle=\color{wongvermillion},
    commentstyle=\color{wongsky},
    stringstyle=\color{wonggreen},
    extendedchars=\true,
    inputencoding=utf8,
    literate={ą}{{\k{a}}}1
             {Ą}{{\k{A}}}1
             {ę}{{\k{e}}}1
             {Ę}{{\k{E}}}1
             {ó}{{\'o}}1
             {Ó}{{\'O}}1
             {ś}{{\'s}}1
             {Ś}{{\'S}}1
             {ł}{{\l{}}}1
             {Ł}{{\L{}}}1
             {ż}{{\.z}}1
             {Ż}{{\.Z}}1
             {ź}{{\'z}}1
             {Ź}{{\'Z}}1
             {ć}{{\'c}}1
             {Ć}{{\'C}}1
             {ń}{{\'n}}1
             {Ń}{{\'N}}1
            %  {\-}{}{0\discretionary{-}{}{}
}  % last one to enable word breaks


%% Create scheme floating environment
\usepackage{newfloat}
% must be done before loading chemmacros to prevent name conflict
\DeclareFloatingEnvironment{scheme}

% continuos numbering of floats
\usepackage{chngcntr}
\counterwithout{scheme}{chapter}
\counterwithout{figure}{chapter}
\counterwithout{table}{chapter}

%% all tufte floats behave as standard
\newenvironment{marginscheme}[1][]{\begin{scheme}[h!]\centering}{\end{scheme}}
\newenvironment{marginfigure}[1][]{\begin{figure}[h!]\centering}{\end{figure}}
\newenvironment{margintable}[1][]{\begin{table}[h!]\centering}{\end{table}}
\renewenvironment{scheme*}[1][]{\begin{scheme}[h!]}{\end{scheme}}
\renewenvironment{figure*}[1][]{\begin{figure}[h!]}{\end{figure}}
\renewenvironment{table*}[1][]{\begin{table}[h!]}{\end{table}}
\newenvironment{fullwidth}[1][]{}{}
\newenvironment{doublespace}[1][]{}{}
% setfloatalignment do nothing
\newcommand{\setfloatalignment}[1]{\ignorespaces}
% set includegraphics to do nothing
% \renewcommand{\includegraphics}[1]{\centering [RYSUNEK]}
\renewcommand{\footnote}[2][]{ {\color{wongblue}[PRZYPIS: #2]}}

%% enable access to nested subdirs
\usepackage{import}
% signal a relative path - produces path of current import dir
\makeatletter\def\relpath{\import@path}\makeatother

%% provide images replacement

% patch for importing svg file functionality
% extracts cmpd declarations from pdf_tex, converts to labelcmpd
% requires --shell-escape
% #1 - path to file, optional
% #2 - name of the file to be imported
\usepackage[unique=true,dir=bashout]{bashful}
\makeatletter
\newcommand{\includesvg}[2][]{%
  \begingroup%
  \color{wongvermillion}
  \catcode`\_=12\relax%
  \splice{%
    if grep -qP '\@backslashchar\@backslashchar cmpd{[a-zA-Z0-9:.\@backslashchar-]*}' ../\relpath#1#2.pdf_tex; then
      echo '[NA RYSUNKU ZWI\@backslashchar k{A}ZKI';
      grep -oP '\@backslashchar\@backslashchar cmpd{[a-zA-Z0-9:.\@backslashchar-]*}' ../\relpath#1#2.pdf_tex;
      echo ']';
    else
      echo "[RYSUNEK BEZ NUMEROWANYCH ZWI\@backslashchar k{A}ZK\@backslashchar 'OW]";
    fi
    % only print \cmpd{...}
    % grep -oP '\@backslashchar\@backslashchar cmpd{[a-zA-Z0-9:.\@backslashchar-]*}' ../\relpath#1#2.pdf_tex%
    % convert to \labelcmpd{...}
    % sed -n 's/.*\@backslashchar\@backslashchar\@backslashchar(cmpd{[a-zA-Z0-9:.\@backslashchar-]*}\@backslashchar).*/%
    % \@backslashchar\@backslashchar label\@backslashchar1/p' ../\relpath#1#2.pdf_tex%
  }%
  \endgroup%
}
\makeatother
% from http://www.tex.ac.uk/tex-archive/info/svg-inkscape/InkscapePDFLaTeX.pdf
% and https://tex.stackexchange.com/a/17491/223674

\usepackage{nopageno}  % no page numbers


%% chemistry packages
\usepackage{chemmacros}  % various chemistry typesetting macros (e.g. \iupac)
\usechemmodule{spectroscopy}  % adds /NMR command for typesetting of experimental data
\chemsetup{greek = upgreek}  % for upright Greek letters in chemical formulas
\usechemmodule{units}  % siunitx for typesetting units
\DeclareSIUnit{\volume}{vol.}
\DeclareSIUnit{\equiv}{equiv.}

\usepackage{chemformula}  % for typesetting chemical equations
\setchemformula{circled=all}

\usepackage{chemnum}  % for automated structures numeration


%% list of acronyms
\usepackage[acronym, nonumberlist, style=list, nogroupskip]{glossaries}
% make list of abbrev a section rather than a chapter
\makeatletter
\renewcommand*{\@@glossarysec}{section}%
\makeatother
% acronyms definitions in separate file
\makeglossaries
\setacronymstyle{short-long}

% TODO: show formula if defined
% TODO: on first use explain in sidenote
\glsaddkey*{formula}% key
  {}% default value
  {\glsentryformula}% command analogous to \glsentrytext
  {\Glsentryformula}% command analogous to \Glsentrytext
  {\glsformula}% command analogous to \glstext
  {\Glsformula}% command analogous to \Glstext
  {\GLSformula}% command analogous to \GLStext

\glsnoexpandfields  % compatibility with \iupac
% \newacronym{<label>}{<acronym>}{<long name>}
\newacronym{dcm}{DCM}{dichlorometan}
\newacronym{thf}{THF}{tetrahydrofuran}
\newacronym[formula={\ch{CF3SO3\bond{single}}}]
  {TfO}{TfO}{grupa triflowa}
\newacronym{dbu}{DBU}{\iupac{1,8-diazabicyklo[5.4.0]undek-7-en}}
\newacronym{nbs}{NBS}{\iupac{\N-bromosukcynoimid}}
\newacronym{mcpba}{\iupac{\meta{}CPBA}}{kwas \iupac{\meta-chloroperoksybenzoesowy}}
\newacronym[formula={\ch{\textit{^t}BuPh2Si\bond{single}}}]
  {tbdps}{TBDPS}{grupa \iupac{\tert-butylo-di-fenylosililowa}}
\newacronym{dtbmp}{DTBMP}{\iupac{2,6-di-\tert-butylo-4-metylopirydyna}}
\newacronym{dtbp}{DTBP}{\iupac{2,6-di-\tert-butylopirydyna}}
\newacronym{sphos}{SPhos}{\iupac{2-dicycloheksylo-fosfino-2\chemprime,6\chemprime-dimetoksy-1,1\chemprime-bifenyl}}
\newacronym[formula={\ch{\textit{^t}BuMe2Si\bond{single}}}]
  {tbs}{TBS}{grupa \iupac{\tert-butylo(dimetylo)sililowa}}
\newacronym{tmds}{TMDS}{\iupac{1,1,3,3-tetrametylodisiloksan}}
\newacronym{heh}{HEH}{ester Hantzscha (\iupac{ester dietylowy 3,5-dikarboksylanu 1,4-dihydro-2,6-dimetylopirydyny})}
\newacronym[formula={\ch{\textit{^i}Bu2AlH}}]
  {dibal}{DIBAL\=/H}{wodorek diizobutyloglinu}
\newacronym{Boc}{Boc}{grupa \iupac{\tert-butoxykarbonylowa}}
\newacronym[formula={\ch{-SiMe3}}]
  {tms}{TMS}{grupa trimetylosililowa}
\newacronym[formula={\ch{LiN(SiMe3)2}}]
  {lihmds}{LiHMDS}{heksametylodisilazan litu}
\newacronym{dft}{DFT}{teoria funkcjonału gęstości, ang. \textit{density functional theory}}
\newacronym[formula={\ch{CF3CO2H}}]{tfa}{TFA}{kwas trifluorooctowy}
\newacronym{thp}{THP}{grupa \iupac{2-tetrahydropiranowa}}
\newacronym[formula={\ch{-CH2OCH3}}]{mom}{MOM}{grupa metoksymetylowa}
\newacronym{csd}{CSD}{\textit{Cambridge Structural Database}}
\newacronym{ts}{Ts}{grupa tosylowa}
\newacronym{sm}{SM}{sita molekularne}
\newacronym{dmf}{DMF}{dimetyloformamid}
\newacronym{tlc}{TLC}{chromatografia cienkowarstwowa, ang. \textit{thin-layer chromatography}}
\newacronym{iupac}{IUPAC}{
  Międzynarodowa Unia Chemii Czystej i~Stosowanej,
  ang. \textit{International Union of Pure and Applied Chemistry}
}
\newacronym{ms}{MS}{spektrometria masowa, ang. \textit{mass spectrometry}}
\newacronym{dmso}{DMSO}{dimetylosulfotlenek}


%% LOCALIZATION
%% Adjust babel translations of floats names
\addto\captionspolish{\renewcommand{\schemename}{Schemat}}  % for Scheme -> Schemat; babel doesn't change it
\addto\captionspolish{\renewcommand{\tablename}{Tabela}}  % change babel's Tablica -> Tabela

%% Adjust floats names in references
\usepackage[nameinlink]{cleveref}
\crefname{figure}{rys.}{rys.}
\Crefname{figure}{Rys.}{Rys.}
\crefname{scheme}{schem.}{schem.}
\Crefname{scheme}{Schem.}{Schem.}
% and translate ceveref's conjunctions
\newcommand{\crefrangeconjunction}{ do~}
\newcommand{\crefpairconjunction}{ i~}
\newcommand{\creflastconjunction}{, oraz }

% chemnum package related
\DeclareTranslation{polish}{chemnum-sep-two}{~oraz~}
\DeclareTranslation{polish}{chemnum-sep-last-two}{,~oraz~}

% biblatex related
\DefineBibliographyStrings{polish}{
  urlseen = {dostęp}
}

%% typesetting dashes
\usepackage[shortcuts,shortemdash]{extdash}
\sisetup{range-phrase = {\--}, range-units=single}


%% vertical space between marginals
% \setlength\marginparpush{12pt}
% not needed in draft


%% slightly more loose typesetting of whitespaces
\tolerance 1414
\hbadness 1414
\emergencystretch 1.5em
\hfuzz 0.3pt
\widowpenalty=10000
\vfuzz \hfuzz
% glues text up rather than distributing on partially empty page
\raggedbottom


%% Title page layout setup
% define subtitle
\makeatletter
\newcommand{\plainsubtitle}{}%     plain-text-only subtitle
\newcommand{\subtitle}[1]{%
    \gdef\@subtitle{#1}%
    \renewcommand{\plainsubtitle}{#1}% use provided plain-text title
}
\newcommand{\plainpublisher}{}%     plain-text-only subtitle
\newcommand{\publisher}[1]{%
    \gdef\@publisher{#1}%
    \renewcommand{\plainpublisher}{#1}% use provided plain-text title
}

% full title page
\renewcommand{\maketitle}[0]{%
    {%
    \sffamily%
    \fontsize{16}{18}\selectfont\noindent\@author\par%
    \vspace{11.5pc}%
    \fontsize{28}{34}\selectfont\noindent\nohyphenation\textit{\@title}\par%
    \vspace{5pc}%
    \fontsize{18}{20}\selectfont\noindent{\plainsubtitle}\par%
    \vspace{2pc}%
    \fontsize{14}{16}\selectfont\noindent\plainpublisher\par%
    }
}
\makeatother

\usepackage{graphicx}
% \newsavebox{\titleimage}
% \savebox{\titleimage}{\includegraphics[height=7\baselineskip]{example-image}}

\title{Badania nad chemoselektywnymi metodami aktywacji amidowych grup karbonylowych na~czynniki nukleofilowe}
\subtitle{\textbf{Praca doktorska} \\ przygotowana pod kierunkiem \\ prof. Bartłomieja Furmana}
\author{Michał M. Więcław}
\publisher{Instytut Chemii Organicznej \\ Polskiej Akademii Nauk}

%% TABLE OF CONTENTS SETUP
% show chapters and sections in table of contents
\setcounter{tocdepth}{1}

\begin{document}

\frontmatter
\maketitle

% acknowledgements
% \cleardoublepage
% \thispagestyle{empty}
% ~\vfill
% \vfill
% \begin{fullwidth}
% \begin{doublespace}
% \raggedleft\noindent\fontsize{16}{20}\selectfont\itshape
% \nohyphenation
% Gorąco dziękuję Magdalenie,\\
% bez wsparcia której ta praca nie zostałaby ukończona.
% \end{doublespace}
% \end{fullwidth}
% \vfill

% \tableofcontents

\chapter{Przedsłowie}\label{chapter:intro}

\section{Konwencje przyjęte w~niniejszej dysertacji}\label{intro:conventions}

Każdy chemik zauważy, że forma graficzna niniejszej rozprawy doktorskiej różni się
  od~zwykle spotykanych w~naukach chemicznych.
Zapewne najbardziej zwraca uwagę format tekstu \--- 
  jest on~inspirowany pracami Edwarda R.~Tuftego\sidecite{Tufte2001,Tufte1990,Tufte1997,Tufte2006},
  uznanego za~eksperta w~dziedzinie prezentowania informacji i~pioniera wizualizacji
  danych\sidecite{Yaffa2011}.
Tufte zauważa, że odnośniki do~spodu strony, a~tym bardziej końca tekstu,
  utrudniają czytanie, rozpraszając uwagę czytelnika.
Zamiast tego umieszcza przypisy i~komentarze na~szerokim marginesie,
  komentując żartobliwie, że \enquote{to miejsce zaplanował dla nich Bóg}.
Na~takiej szacie graficznej korzysta również główny tekst \---
  zawarty w~kolumnie węższej niż cała strona, ma szerokość uznawaną za~optymalną
  do czytania\sidecite{nanavati05}.

Esencją podejścia Tuftego do prezentowania danych jest położenie nacisku
  na~informatywność, a~nie estetykę przekazu.
Wizualizacja powinna ułatwić jak najlepsze zrozumienie danych w~jak najkrótszym czasie,
  w~jak najmniejszej przestrzeni i~przy użyciu jak najprostszej formy.
Sposób przedstawienia danych musi być jednoznaczny i~nie może ich zniekształcać
  ani wymuszać ich interpretacji.
Dane nie powinny być ozdabiane, a~wszystkie zbędne elementy, takie jak ramki czy tło,
  nie powinny znajdować się na~rysunkach, ponieważ odwracają uwagę od~treści.
Projektując graficzną stronę tego dokumentu starałem się stosować do~tych zasad.

\begin{marginfigure}
  \includesvg{palettes}
  \caption{
    Wykorzystane w~niniejszej dysertacji palety kolorów,
    będące przyjazne osobom z~zaburzeniem rozpoznawania barw.
  }
  \label{fig:palettes}
\end{marginfigure}
Jeden z paradygmatów kształcenia mówi, że nauka, będąc narzędziem poznania,
  powinna być przystępna.
Jako że kolor może nieść istotną część informacji podczas prezentacji danych,
  przy tworzeniu grafik zawartych w~tej pracy użyłem palety przyjaznej osobom
  z~zaburzeniem rozpoznawania barw.
Jako paletę jakościową użyłem schematu kolorów zaproponowanego przez Wonga\sidecite{wong11},
  natomiast jako paletę ilościową wykorzystałem viridis\sidecite{Smith2015} lub BrBG,
  zależnie od~kontekstu.
Wszystkie je prezentuję na~\cref{fig:palettes}.

Na~przystępność tekstu w~ogromnym stopniu wpływa również wybór kroju pisma.
Uważny czytelnik może zwrócić uwagę, że nie jest on jednakowy w~całej objętości
  niniejszej dysertacji.
Wzory chemiczne zapisuję czcionką zalecaną przez organizację \gls{iupac}
  zamiast dominującym, klasycznym krojem szeryfowym.
W~tym miejscu dodam, że doprecyzowując podstawniki ogólne\sidenote{%
    Czyli grupy i~atomy oznaczone jako \ch{X}, \ch{R} lub \ch{R^n}.},
  używam znaku równości (\enquote{$=$}) pokazując konkretne grupy lub atomy,
  a~znaku tożsamości (\enquote{$\equiv$}) prezentując koncepcje\sidenote{%
  Na przykład \enquote{\ch{R}~$\equiv$~alkil} albo \enquote{\ch{R^1}~$\equiv$~\ch{R^2}}.}.

Cyfry występujące w~tekście również nie zawsze wyglądają tak samo.
W~większości są to tak zwane cyfry nautyczne, zaprojektowane tak, aby wizualnie współgrały
  z~minuskułami\sidenote{Czyli małymi litrami.},
  ale do~przedstawienia wielkości fizycznych i~matematycznych użyłem cyfr zwykłych,
  aby zwiększyć ich czytelność.
Podobny zabieg zastosowałem w~przypadku numerów związków występujących w~tekście,
  które są dodatkowo wyróżnione za~pomocą pogrubienia.
Kolejnym odstępstwem jest użycie jaśniejszego koloru do~zapisu cytowań,
  co~pozwala skupić uwagę na~treści, a~nie detalach technicznych.

Pewnie jak większość ludzi parających się naukami ścisłymi ulegam pokusie używania skrótów.
Większość z~nich to~skrótowce standardowo używane przez chemików,
  jednak, dla jasności, wszystkie rozwijam przy ich pierwszym wystąpieniu.
Zgodnie z~konwencją zamieszczam również wykaz tych akronimów,
  wraz z~ich znaczeniem, na~następnych stronach.

Choć główna część tej dysertacji poświęcona jest badaniom z~dziedziny syntezy organicznej,
  podczas pracy nad nią moje zainteresowania poszerzyły się.
Stąd też czytelnik natrafi na~fragmenty dotyczące obliczeń kwantowo\-/chemicznych,
  a~nawet programowania komputerowego.
Zwłaszcza ten ostatni temat wymaga dodatkowego komentarza jako najbardziej oddalony
  od~podstawowej dyscypliny.

Gdy w~tekście pojawia się odniesienie do~nazw elementów opisywanego kodu źródłowego,
  sygnalizuję to~używając kroju czcionki \lstinline!o stałej szerokości!.
Większe bloki kodu są wydzielone z~tekstu, jak ten poniżej.
Dla poprawienia czytelności 
  \lstinline[basicstyle=\ttfamily\color{wongvermillion},columns=fixed]!słowa kluczowe!%
  \footnote{
    Słowa kluczowe to ciągi znaków zarezerwowane w~danym języku programowania,
      stanowiące część jego składni.
    Mają one z~góry określone znaczenie, definiowane przez ten język.
  },
  \lstinline[basicstyle=\ttfamily\color{wongsky},columns=fixed]!komentarze!,
  \lstinline[basicstyle=\ttfamily\color{wonggreen},columns=fixed]!dane tekstowe!, a~także
  \lstinline[basicstyle=\ttfamily\color{wongpurple},columns=fixed]!niektóre zmienne!
  są wyróżnione przy użyciu koloru.
Linie tych bloków są ponumerowane na~lewym marginesie.
Niestety, nie doczekały się one polskiego terminu i~nazywane są \--- 
  z~języka angielskiego \--- listingami.
\Cref{lst:example} jest przykładem takiego bloku kodu.

\begin{listing}
  \begin{lstlisting}
    if is_first_program():
        print('Hello world!')
    else:
        pass  # nic nie rób
  \end{lstlisting}
\caption{Przykład formatowania bloku zawierającego kod źródłowy.}
\label{lst:example}
\end{listing}

Tekst niniejszej rozprawy doktorskiej został przygotowany przy użyciu oprogramowania \LaTeX,
  a~jej kod źródłowy dostępny jest na~dołączonej płycie CD oraz w~Internecie pod adresem \repourl{}.

\section{Cel pracy}\label{intro:goal}
Ramowym celem niniejszej pracy było sprawdzenie, jak współcześnie dostępne metody
  reduktywnej aktywacji amidów sprawdzają się w~roli narzędzi do~syntezy
  sfunkcjonalizowanych amin w~złożonych układach reakcyjnych.
Takie ujęcie zagadnienia, choć obejmuje meritum, zdecydowanie wymaga doprecyzowania.
Wśród badanych przeze mnie \enquote{złożonych układów reakcyjnych} znajdują się dwie,
  wybrane arbitralnie, kategorie.
Pierwszą są reakcje z~amidami posiadającymi grupy estrowe, które w~reakcji reduktywnej aktywacji
  mogą być potencjalnymi konkurentami wobec amidu.
Drugą są reakcje wieloskładnikowe laktamów na~przykładzie pochodnych reakcji Ugiego.
Natomiast do~współczesnych metod reduktywnej aktywacji amidów zaliczam metodę opartą
  o~wykorzystanie bezwodnika triflowego, reakcję z~odczynnikiem Schwartza oraz katalityczne
  redukcje wobec kompleksów irydu \--- kompleksu Vaski i~kompleksu van der Enta.

Za~główny cel postawiłem sobie sprawdzenie, które z~tych metod aktywacji mogą być zastosowane
  do~przekształcenia amidu w~funkcjonalizowaną aminę w~każdym ze~wspomnianych układów
  reakcyjnych oraz jaki jest zakres stosowalności skutecznych procedur.
Podczas jego realizacji natrafiłem na~trudności i~wyzwania, którym starałem się zaradzić,
  wykorzystując metody wykraczające poza standardowo wykorzystywane w~syntezie organicznej.
Dzięki moim zainteresowaniom obejmującym dziedzinę nauk komputerowych,
  mogłem zastosować obliczenia metodami numerycznymi oraz programowanie komputerowe
  do~uzyskania odpowiedzi na~niektóre z~powstałych wątpliwości i~pokonania pewnych przeszkód.
Dociekania te były dla mnie przyczynkiem do~postawienia sobie celów dodatkowych \---
  ustalenia przebiegu wariantu reakcji Ugiego, który badałem oraz stworzenie oprogramowania
  komputerowego, ułatwiającego analizę wyników obliczeń kwantowo-chemicznych.

\section{Publikacje}\label{intro:publications}
\begin{itemize}
  \item \cite{wieclaw21}
  \item \cite{stecko18}
  \item \cite{wieclaw22}
\end{itemize}

\section{Konferencje}\label{intro:conferences}
\begin{fullwidth}
\begin{itemize}
  \item 19\textsuperscript{th}~International Symposium \enquote{Advances in~the Chemistry of~Heteroorganic Compounds}, poster: \enquote{Captodative functionalization of~amidoesters}, Poland, Łódź, 19.10.2016~r.
  \item Ogólnopolskie Studenckie Mikrosympozjum Chemików, wystąpienie ustne: \enquote{Aktywacja amidów na~atak nukleofila jako metoda selektywnej funkcjonalizacji}, Białystok, 30.03.\-–1.04.2017~r.
  \item V~Łódzkie Sympozjum Doktorantów Chemii, poster: \enquote{Aktywacja amidów na~atak nukleofila jako metoda selektywnej funkcjonalizacji}, Polska, Łódź, 11.05.\-–12.05.2017~r.
  \item XIV Warszawskie Seminarium Doktorantów Chemików - ChemSession’17, poster: \enquote{Captodative functionalization of~amidoesters}, Polska , Warszawa, 9.06.2017~r.
  \item 26\textsuperscript{th}~ISHC Congress, poster, Regensburg, poster: \enquote{Chemoselective activation of~amide carbonyls towards nucleophilic reagents}, Niemcy, Ratyzbona, 3.\-–8.09.2017~r.
  \item XX~International Symposium \enquote{Advances in~the Chemistry of~Heteroorganic Compounds} and XVII International Symposiumon on Selected Problems of Chemistry of Acyclic and Cyclic Heteroorganic Compounds, poster: \enquote{Schwartz’s reagent mediated nojirimycin derivatives synthesis}, Polska, Łódź, 23.\-–24.11.2017~r.
  \item XI~Ogólnopolskie Sympozjum Chemii Organiczne, poster: \enquote{Synteza cukrowych pochodnych tetrazoli z~użyciem odczynnika Schwartza}, Polska, Warszawa, 8.\-–11.04.2018~r.
  \item International Congress of~Young Chemists YoungChem~2018, wystąpienie ustne: \enquote{Short and sweet: An~approach to direct synthesis of~iminosugar-derived tetrazoles}, Polska, Bydgoszcz, 10.\-–14.10.2018~r.
  \item XXI~International Symposium \enquote{Advances in~the Chemistry of~Heteroorganic Compounds}, poster: \enquote{An~approach to~direct synthesis of~iminosugar derived tetrazoles}, Polska, Łódź, 23.11.2018~r.
  \item International Symposium on Synthesis and Catalysis~2019, wystąpienie ustne: \enquote{An~approach to~direct synthesis of~iminosugar derived tetrazoles}, Portugalia, Evora, 3.\-–6.09.2019~r.
  \item XXII~International Symposium \enquote{Advances in~the Chemistry of~Heteroorganic Compounds}, poster: \enquote{Iminosugar derived tetrazoles: direct synthesis and mechanistic insights}, Polska, Łódź, 22.11.2019~r.
  \item Virtual Winter Workshop \enquote{Multiscale modeling in materials science, chemistry, and biology: How to meet, greet, and beat scale-bridging challenges}, poster: \enquote{Tesliper: Spectral Simulations Simplified}, Niemcy, Karlsruhe, 22.\--23.11.2021~r.
  
\end{itemize}
\end{fullwidth}

\section{Finansowanie}\label{intro:founding}
\begin{fullwidth}
\begin{itemize}
  \item Grant Preludium \textnumero~2017/25/N/ST5/00079 Narodowego Centrum Nauki
  \item Grant obliczeniowy PL\=/Grid
\end{itemize}
\end{fullwidth}

\begin{fullwidth}
  \printglossary[title=Wykaz skrótów, type=\acronymtype]
\end{fullwidth}
  

\mainmatter
% widths of text elements
% zwykły tekst \printinunitsof{cm}\prntlen{\textwidth}\\  % 10.69847 cm
% margines \printinunitsof{cm}\prntlen{\marginparwidth}\\ %  4.93929 cm
% przerwy \printinunitsof{cm}\prntlen{\marginparsep}      %  0.81987 cm
%                                                 SUMA:     16.45763 cm

\import{chapter-1}{chapter}

% \chapter{In vitro}

% \chapter{In silico}
% \section{Obliczenia mechanizmu}
% \section{Symulacja widm}

% \chapter{In machina}
% \section{Istota programu}
% \section{Wybór języka programowania}

% \chapter{In detail}
% \section{Procedury}
% \section{Analizy}

\backmatter

\printbibliography

\end{document}
