\documentclass{book}
\usepackage[a4paper, total={6in, 8in}]{geometry}

%% font encoding
\usepackage[utf8]{inputenc}
\usepackage[T1]{fontenc}


%% language dependent
\usepackage[polish]{babel}
\usepackage{polski}
\usepackage[autostyle]{csquotes}
\usepackage{indentfirst}


% preety tabs
% do I need it?
% \usepackage{booktabs}


%% Helper for TODOs
\usepackage[
    linecolor=red,
    backgroundcolor=white,
    bordercolor=red,
    textcolor=red,
    textsize=footnotesize,
]{todonotes}


%% Bibliography and citations setup
% based on https://tex.stackexchange.com/q/139158
\usepackage[
    citestyle=numeric,
    % bibstyle=authoryear,  % needed only for traditional bibliography
    backref=true,
    citetracker,
    backend=biber,
    % autocite=footnote,  % cooperate with tufte-latex
    labeldateparts,  % make year of publication available
    giveninits=true,  % names as initials
    url=false,  % suppress url everywhere but in `online'
    maxbibnames=3,  % if more than 3 autors, produce et al.
]{biblatex}
% suppress `In:' in article citations
% from https://tex.stackexchange.com/a/10686/223674
\renewbibmacro{in:}{%
  \ifentrytype{article}{}{\printtext{\bibstring{in}\intitlepunct}}}
% suppress title of articles in citations
\DeclareFieldFormat[article]{title}{}
% suppress ISSN
\AtEveryCitekey{\clearfield{issn}}
% redefine cite:short to give author-year
\renewbibmacro*{cite:short}{%
  \printnames{labelname}%
  \setunit*{\printdelim{nameyeardelim}}%
  \iffieldundef{labelyear}
    {}
    {\printtext[parens]{\printlabeldateextra}},  % comma is printed
  \printtext[bibhyperlink]{%  make it a link to citation
    \bibstring{backrefpage}\ppspace%  "cyt. na s. "
    \printlist[][-1]{pageref}}%  page number of first occurrence
}
% punctuation in autocite
\DeclareAutoCiteCommand{footnote}[r]{\footcite}{\footcites}
% references location
\addbibresource{refs.bib}

%% define shortcuts for colors from Wong's palette
\usepackage{xcolor}
\definecolor{wongblack}{HTML}{000000}
\definecolor{wongorange}{HTML}{E69F00}
\definecolor{wongsky}{HTML}{56B4E9}
\definecolor{wonggreen}{HTML}{009E73}
\definecolor{wongyellow}{HTML}{F0E442}
\definecolor{wongblue}{HTML}{0072B2}
\definecolor{wongvermillion}{HTML}{D55E00}
\definecolor{wongpurple}{HTML}{CC79A7}


%% setup code snippets
\usepackage{inconsolata}  % monospaced font
\usepackage{listings}  % code listings
\lstset{
    breaklines=true,
    % numbers=left,
    numberstyle=\tiny,
    basicstyle=\ttfamily,
    keywordstyle=\color{wongvermillion},
    commentstyle=\color{wongsky},
    stringstyle=\color{wonggreen},
    extendedchars=\true,
    inputencoding=utf8,
    literate={ą}{{\k{a}}}1
             {Ą}{{\k{A}}}1
             {ę}{{\k{e}}}1
             {Ę}{{\k{E}}}1
             {ó}{{\'o}}1
             {Ó}{{\'O}}1
             {ś}{{\'s}}1
             {Ś}{{\'S}}1
             {ł}{{\l{}}}1
             {Ł}{{\L{}}}1
             {ż}{{\.z}}1
             {Ż}{{\.Z}}1
             {ź}{{\'z}}1
             {Ź}{{\'Z}}1
             {ć}{{\'c}}1
             {Ć}{{\'C}}1
             {ń}{{\'n}}1
             {Ń}{{\'N}}1
            %  {\-}{}{0\discretionary{-}{}{}
}  % last one to enable word breaks


%% Create scheme floating environment
\usepackage{newfloat}
% must be done before loading chemmacros to prevent name conflict
\DeclareFloatingEnvironment{scheme}

%% all tufte floats behave as standard
\newenvironment{marginscheme}[1][]{\begin{scheme}[h]}{\end{scheme}}
\newenvironment{marginfigure}[1][]{\begin{figure}[h]}{\end{figure}}
\newenvironment{margintable}[1][]{\begin{table}[h]}{\end{table}}
\newenvironment{fullwidth}[1][]{}{}
\newenvironment{doublespace}[1][]{}{}
% setfloatalignment do nothing
\newcommand{\setfloatalignment}[1]{\ignorespaces}
% set includegraphics to do nothing
\renewcommand{\includegraphics}[1]{\centering [TUTAJ RYSUNEK]}
\renewcommand{\footnote}[2][]{ [PRZYPIS: #2]}

\usepackage{nopageno}  % no page numbers


%% chemistry packages
\usepackage{chemmacros}  % various chemistry typesetting macros (e.g. \iupac)
\usechemmodule{spectroscopy}  % adds /NMR command for typesetting of experimental data
\chemsetup{greek = upgreek}  % for upright Greek letters in chemical formulas
\usechemmodule{units}  % siunitx for typesetting units
\DeclareSIUnit{\volume}{vol.}
\DeclareSIUnit{\equiv}{equiv.}

\usepackage{chemformula}  % for typesetting chemical equations
\setchemformula{circled=all}

\usepackage{chemnum}  % for automated structures numeration
\setchemnum{replace-style = \fontsize{10}{12}}  % change size of tags in schemes


%% list of acronyms
\usepackage[acronym, nonumberlist, style=list, nogroupskip]{glossaries}
% make list of abbrev a section rather than a chapter
\makeatletter
\renewcommand*{\@@glossarysec}{section}%
\makeatother
% acronyms definitions in separate file
\makeglossaries
\setacronymstyle{long-short}

\glsnoexpandfields  % compatibility with \iupac
% \newacronym{<label>}{<acronym>}{<long name>}
\newacronym{dcm}{DCM}{dichlorometan}
\newacronym{thf}{THF}{tetrahydrofuran}
\newacronym{TfO}{TfO}{grupa triflowa (\ch{CF3SO3\bond{single}})}
\newacronym{dbu}{DBU}{\iupac{1,8-diazabicyklo[5.4.0]undek-7-en}}
\newacronym{nbs}{NBS}{\iupac{\N-bromosukcynoimid}}
\newacronym{mcpba}{\ch{\meta CPBA}}{kwas \iupac{\meta-chloroperoksybenzoesowy}}
\newacronym{tbdps}{TBDPS}{grupa \iupac{\tert-butylo-di-fenylosililowa} (\ch{\textit{^t}BuPh2Si\bond{single}})}
\newacronym{dtbmp}{DTBMP}{\iupac{2,6-di-\tert-butylo-4-metylopirydyna}}
\newacronym{dtbp}{DTBP}{\iupac{2,6-di-\tert-butylopirydyna}}
\newacronym{sphos}{SPhos}{\iupac{2-dicycloheksylo-fosfino-2\chemprime,6\chemprime-dimetoksy-1,1\chemprime-bifenyl}}
\newacronym{tbs}{TBS}{grupa \iupac{\tert-butylo-dimetylosililowa} (\ch{\textit{^t}BuMe2Si\bond{single}})}
\newacronym{tmds}{TMDS}{\iupac{1,1,3,3-tetrametylodisiloksan}}
\newacronym{heh}{HEH}{estr Hantzscha (\iupac{ester dietylowy 3,5-dikarboksylanu 1,4-dihydro-2,6-dimetylopirydyny})}
\newacronym[]{dibal}{DIBAL\=/H}{\ch{ "\textit{i-}" Bu2AlH}}


%% LOCALIZATION
%% Adjust babel translations of floats names
\addto\captionspolish{\renewcommand{\schemename}{Schemat}}  % for Scheme -> Schemat; babel doesn't change it
\addto\captionspolish{\renewcommand{\tablename}{Tabela}}  % change babel's Tablica -> Tabela

%% Adjust floats names in references
\usepackage[nameinlink]{cleveref}
\crefname{figure}{rys.}{rys.}
\Crefname{figure}{Rys.}{Rys.}
\crefname{scheme}{schem.}{schem.}
\Crefname{scheme}{Schem.}{Schem.}

% chemnum package related
\DeclareTranslation{polish}{chemnum-sep-two}{~oraz~}
\DeclareTranslation{polish}{chemnum-sep-last-two}{,~oraz~}

% biblatex related
\DefineBibliographyStrings{polish}{
  urlseen = {dostęp}
}

%% typesetting dashes
\usepackage[shortcuts,shortemdash]{extdash}
\sisetup{range-phrase = {\--}, range-units=single}


%% vertical space between marginals
% \setlength\marginparpush{12pt}
% not needed in draft


%% slightly more loose typesetting of whitespaces
\tolerance 1414
\hbadness 1414
\emergencystretch 1.5em
\hfuzz 0.3pt
\widowpenalty=10000
\vfuzz \hfuzz
% glues text up rather than distributing on partially empty page
\raggedbottom


%% Title page layout setup
% define subtitle
\makeatletter
\newcommand{\plainsubtitle}{}%     plain-text-only subtitle
\newcommand{\subtitle}[1]{%
    \gdef\@subtitle{#1}%
    \renewcommand{\plainsubtitle}{#1}% use provided plain-text title
}
\newcommand{\plainpublisher}{}%     plain-text-only subtitle
\newcommand{\publisher}[1]{%
    \gdef\@publisher{#1}%
    \renewcommand{\plainpublisher}{#1}% use provided plain-text title
}

% full title page
\renewcommand{\maketitle}[0]{%
    {%
    \sffamily%
    \fontsize{16}{18}\selectfont\noindent\@author\par%
    \vspace{11.5pc}%
    \fontsize{28}{34}\selectfont\noindent\nohyphenation\textit{\@title}\par%
    \vspace{5pc}%
    \fontsize{18}{20}\selectfont\noindent\textsc{\plainsubtitle}\par%
    \vspace{2pc}%
    \fontsize{14}{16}\selectfont\noindent\plainpublisher\par%
    }
}
\makeatother

% \usepackage{graphicx}
% \newsavebox{\titleimage}
% \savebox{\titleimage}{\includegraphics[height=7\baselineskip]{example-image}}

\title{Badania nad chemoselektywnymi metodami aktywacji amidowych grup karbonylowych na~czynniki nukleofilowe}
\subtitle{\textbf{Praca doktorska} \\ przygotowana pod kierunkiem \\ prof. Bartłomieja Furmana}
\author{Michał M. Więcław}
\publisher{Instytut Chemii Organicznej \\ Polskiej Akademii Nauk}

\begin{document}

\frontmatter
\maketitle

% acknowledgements
% \cleardoublepage
% \thispagestyle{empty}
% ~\vfill
% \vfill
% \begin{fullwidth}
% \begin{doublespace}
% \raggedleft\noindent\fontsize{16}{20}\selectfont\itshape
% \nohyphenation
% Gorąco dziękuję Magdalenie,\\
% bez wsparcia której ta praca nie zostałaby ukończona.
% \end{doublespace}
% \end{fullwidth}
% \vfill

% \tableofcontents

\chapter{Przedsłowie}\label{chapter:intro}

\section{Konwencje przyjęte w~niniejszej dysertacji}\label{intro:conventions}

Każdy chemik zauważy, że forma graficzna niniejszej rozprawy doktorskiej różni się
  od~zwykle spotykanych w~naukach chemicznych.
Zapewne najbardziej zwraca uwagę format tekstu \--- zawarty w~kolumnie węższej niż
  szerokość strony, z~przypisami i~komentarzami na~szerokim marginesie.
Jest on~inspirowany pracami Edwarda~R. Tuftego\sidecite{Tufte2001,Tufte1990,Tufte1997,Tufte2006},
  uznanego za~eksperta w~dziedzinie prezentowania informacji i~pioniera wizualizacji
  danych\sidecite{Yaffa2011}.
Projektując graficzną stronę tego dokumentu starałem się stosować do~zasad
  proponowanych w~jego pracach.
% TODO: describe Tufte rules of good layout

Jeden z paradygmatów kształcenia mówi, że nauka, będąc narzędziem poznania,
  powinna być przystępna.
Jako że kolor może nieść istotną część informacji podczas prezentacji danych,
  przy tworzeniu grafik zawartych w~tej pracy użyłem palety przyjaznej osobom
  z~zaburzeniem rozpoznawania barw.
Jako paletę jakościową użyłem schematu kolorów zaproponowanego przez Wonga\sidecite{wong11},
  natomiast jako paletę ilościową wykorzystałem viridis\sidecite{Smith2015} lub BrBG,
  zależnie od~kontekstu.
% TODO: add marginfigure with palettes used

Pewnie jak większość ludzi parających się naukami ścisłymi ulegam pokusie używania skrótów.
Większość z~nich to~skrótowce standardowo używane przez chemików,
  jednak, dla jasności, wszystkie rozwijam przy ich pierwszym wystąpieniu.
Zgodnie z~konwencją zamieszczam również wykaz tych akronimów,
  wraz z~ich znaczeniem, na~następnych stronach.

Choć główna część tej dysertacji poświęcona jest badaniom z~dziedziny syntezy organicznej,
  podczas pracy nad nią moje zainteresowania poszerzyły się.
Stąd też czytelnik natrafi na~fragmenty dotyczące obliczeń kwantowo\-/chemicznych,
  a~nawet programowania komputerowego.
Zwłaszcza ten ostatni temat wymaga dodatkowego komentarza jako najbardziej oddalony
  od~podstawowej dyscypliny.

Gdy w~tekście pojawia się odniesienie do~nazw elementów opisywanego kodu źródłowego,
  sygnalizuję to~używając kroju czcionki \lstinline!o stałej szerokości!.
Większe bloki kodu są wydzielone z~tekstu, jak ten poniżej.
Dla poprawienia czytelności 
  \lstinline[basicstyle=\ttfamily\color{wongvermillion},columns=fixed]!słowa kluczowe!%
  \footnote{
    Słowa kluczowe to ciągi znaków zarezerwowane w~danym języku programowania,
      stanowiące część jego składni.
    Mają one z~góry określone znaczenie, definiowane przez ten język.
  },
  \lstinline[basicstyle=\ttfamily\color{wongsky},columns=fixed]!komentarze!, oraz 
  \lstinline[basicstyle=\ttfamily\color{wonggreen},columns=fixed]!dane tekstowe!
  są wyróżnione przy użyciu koloru.
Linie tych bloków są ponumerowane na~lewym marginesie.
Niestety, nie doczekały się one polskiego terminu i~nazywane są \--- 
  z~języka angielskiego \--- listingami.
\Cref{lst:example} jest przykładem takiego bloku kodu.

\begin{listing}
\begin{lstlisting}[language=Python]
if is_first_program():
    print('Hello world!')
else:
    pass  # nic nie rób
\end{lstlisting}
\caption{Przykład formatowania bloku zawierającego kod źródłowy.}
\label{lst:example}
\end{listing}

Tekst niniejszej rozprawy doktorskiej został przygotowany przy użyciu oprogramowania \LaTeX,
  a~jej kod źródłowy dostępny jest w~Internecie pod adresem
  \url{https://github.com/Mishioo/dysertacja}.

\section{Cel pracy}\label{intro:goal}
  \todo{Uzupełnić cel pracy.}

\section{Publikacje}\label{intro:publications}
% TODO: force black color of citations here
\begin{itemize}
  \item \cite{wieclaw21}
  \item \cite{stecko18}
\end{itemize}

\section{Konferencje}\label{intro:conferences}
\begin{fullwidth}
\begin{itemize}
  \item 19\textsuperscript{th}~International Symposium \enquote{Advances in~the Chemistry of~Heteroorganic Compounds}, poster: \enquote{Captodative functionalization of~amidoesters}, Poland, Łódź, 19.10.2016~r.
  \item Ogólnopolskie Studenckie Mikrosympozjum Chemików, wystąpienie ustne: \enquote{Aktywacja amidów na~atak nukleofila jako metoda selektywnej funkcjonalizacji}, Białystok, 30.03.\-–1.04.2017~r.
  \item V~Łódzkie Sympozjum Doktorantów Chemii, poster: \enquote{Aktywacja amidów na~atak nukleofila jako metoda selektywnej funkcjonalizacji}, Polska, Łódź, 11.05.\-–12.05.2017~r.
  \item XIV Warszawskie Seminarium Doktorantów Chemików - ChemSession’17, poster: \enquote{Captodative functionalization of~amidoesters}, Polska , Warszawa, 9.06.2017~r.
  \item 26\textsuperscript{th}~ISHC Congress, poster, Regensburg, poster: \enquote{Chemoselective activation of~amide carbonyls towards nucleophilic reagents}, Niemcy, Ratyzbona, 3.\-–8.09.2017~r.
  \item XX~International Symposium \enquote{Advances in~the Chemistry of~Heteroorganic Compounds} and XVII International Symposiumon on Selected Problems of Chemistry of Acyclic and Cyclic Heteroorganic Compounds, poster: \enquote{Schwartz’s reagent mediated nojirimycin derivatives synthesis}, Polska, Łódź, 23.\-–24.11.2017~r.
  \item XI~Ogólnopolskie Sympozjum Chemii Organiczne, poster: \enquote{Synteza cukrowych pochodnych tetrazoli z~użyciem odczynnika Schwartza}, Polska, Warszawa, 8.\-–11.04.2018~r.
  \item International Congress of~Young Chemists YoungChem~2018, wystąpienie ustne: \enquote{Short and sweet: An~approach to direct synthesis of~iminosugar-derived tetrazoles}, Polska, Bydgoszcz, 10.\-–14.10.2018~r.
  \item XXI~International Symposium \enquote{Advances in~the Chemistry of~Heteroorganic Compounds}, poster: \enquote{An~approach to~direct synthesis of~iminosugar derived tetrazoles}, Polska, Łódź, 23.11.2018~r.
  \item International Symposium on Synthesis and Catalysis~2019, wystąpienie ustne: \enquote{An~approach to~direct synthesis of~iminosugar derived tetrazoles}, Portugalia, Evora, 3.\-–6.09.2019~r.
  \item XXII~International Symposium \enquote{Advances in~the Chemistry of~Heteroorganic Compounds}, poster: \enquote{Iminosugar derived tetrazoles: direct synthesis and mechanistic insights}, Polska, Łódź, 22.11.2019~r.
\end{itemize}
\end{fullwidth}

\section{Finansowanie}\label{intro:founding}
\begin{fullwidth}
\begin{itemize}
  \item Grant Preludium \textnumero~2017/25/N/ST5/00079 Narodowego Centrum Nauki
  \item Grant obliczeniowy PL\=/Grid
\end{itemize}
\end{fullwidth}

\begin{fullwidth}
  \printglossary[title=Wykaz skrótów, type=\acronymtype]
\end{fullwidth}
  

\mainmatter
% widths of text elements
% zwykły tekst \printinunitsof{cm}\prntlen{\textwidth}\\  % 10.69847 cm
% margines \printinunitsof{cm}\prntlen{\marginparwidth}\\ %  4.93929 cm
% przerwy \printinunitsof{cm}\prntlen{\marginparsep}      %  0.81987 cm
%                                                 SUMA:     16.45763 cm

\chapter{In litterae}

\section{Trwałość amidów}
Wiązanie amidowe występuje w naturze niezwykle powszechnie.
Można nawet pokusić się o stwierdzenie, że jest ono jednym z budulców życia ---
w końcu peptydy, podstawowa struktura biocheniczna złożonych organizmów,
to łańcuchy aminowkasów, połączonych wiązaniami amidowymi.

{\color{wongpurple} [rysunek: peptyd]}  % TODO

Ugrupowanie to można też znaleźć w wielu związkach biologicznie czynnych.
Za prosty przykład niech posłuży lidokaina, powszechnie stosowana jako środek miejscowo znieczulający.

{\color{wongpurple} [rysunek: lidokaina]}  % TODO

Przykładów takich możnaby przytoczyć wiele, bo jak pokazuje analiza produkcji farmaceutyków,
\SI{66}{\percent} leków syntezuje się tworząc wiązanie amidowe\autocite{carey06}.

Tę powszechność amidy zawdzięczają między innymi wyjątkowo niskiej reaktywności.
Związki te ulegają niewielu przemianom chemicznym ze względu  % TODO

Związki te zawdzięczają tę niezwykłą trwałosć bardzo efentywnemu nakładaniu się orbitali molekularnych atomu azotu oraz $\pi$ wiązania podwójnego \ch{C=O}.
Pozwala to na wydajną delokalizację elektronów w obrębie wiązania i znaczny udział dwóch możliwych struktur zwiterionowych, jak widać na \autoref{sch:resonance}.

Ze względu na swoje właściwości amidy znalazły zastosowanie także w przemyśle.
{\color{wongpurple} [nylon, uretany]}  % TODO

\section{Prezkształcenia amidów}
Przez długi czas chemia amidów była raczej uboga ---
ograniczała się przede wszystkim do prostych reakcji, dziś uznanych za podręcznikowe.
W pierwszej kolejności można wymienić ich redukcję do amin oraz hydrolizę.

\section{Odczynnik Schwartza}


% \chapter{In vitro}

% \chapter{In silico}
% \section{Obliczenia mechanizmu}
% \section{Symulacja widm}

% \chapter{In machina}
% \section{Istota programu}
% \section{Wybór języka programowania}

% \chapter{In detail}
% \section{Procedury}
% \section{Analizy}

\backmatter

% \printbibliography

\end{document}
