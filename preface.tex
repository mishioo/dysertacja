\chapter{Praefatio}

\section{Publikacje}

\section{Konferencje}

\section{Finansowanie}

\begin{itemize}
  \item Grant Preludium \textnumero~2017/25/N/ST5/00079 Narodowego Centrum Nauki
  \item Grant obliczeniowy PL\=Grid
\end{itemize}


\section{Konwencje przyjęte w~niniejszej dysertacji}

Prawdopodobnie każdy obyty chemik szybko zauważy, że forma niniejszej rozprawy doktorskiej różni się od~zwykle spotykanej w~naukach chemicznych.
Zapewne najbardziej zwraca uwagę format tekstu \--- zawarty w~kolumnie węższej niż szerokość strony, z~przypisami i~komentarzami na~szerokim marginesie.
Jest on~inspirowany pracami\autocite{Tufte2001,Tufte1990,Tufte1997,Tufte2006} Edwarda R. Tuftego,
uznanego za~eksperta w~dziedzinie prezentowania informacji i~pioniera wizualizacji danych\autocite{Yaffa2011}.
Projektując graficzną stronę tego dokumentu starałem się stosować do~proponowanych w~jego pracach zasad.

Uważam ponadto, że nauka powinna być jak najbardziej przystępna jak największej liczbie osób.
Jako, że kolor może nieść istotną część informacji podczas prezentacji danych,
przy tworzeniu grafik zawartych w~pracy tej użyłem palety przyjaznej osobom z~zaburzeniem rozpoznawania barw.
Jako palety jakościowej użyłem zaproponowanej przez Wonga\autocite{wong11},
natomiast jako paletę ilościową wykorzystałem viridis\autocite{Smith2015} lub BrBG, zależnie od~kontekstu.

Również z~powodu czytelności, staram się nie nadużywać żargonu podczas opisywania eksperymentów.
Mimo wszystko \--- pewnie jak większość umysłów ścisłych \--- ulegam pokusie używania skrótów.
Większość z~nich to~skrótowce standardowo używane w~chemicznym światku,
jednak, dla jasności, wszystkie wyjaśniam przy ich pierwszym wystąpieniu.
Zgodnie z~konwencją zamieszczam również wykaz tych akronimów, wraz z~wyjaśnieniem, na~następnych stronach.

Choć główna część tej dysertacji poświęcona jest badaniom z~dziedziny syntezy organicznej,
podczas pracy nad nią moje zainteresowania poszerzyły się.
Stąd też, czytelnik natrafi również na~fragmenty dotyczące obliczeń kwantowo\-/chemicznych, a~nawet programowania komputerowego.
Zwłaszcza ten ostatni temat wymaga dodatkowego komentarza, jako najbardziej oddalony od~podstawowej dyscypliny.

Gdy w~tekście pojawia się odniesienie do~nazw elementów opisywanego kodu,
sygnalizuję to~używając kroju czcionki \lstinline!o stałej szerokości!.
Większe bloki kodu są wydzielone z~tekstu, jak ten poniżej.
Dla poprawienia czytelności 
\lstinline[basicstyle=\ttfamily\color{wongvermillion},columns=fixed]!słowa kluczowe!, 
\lstinline[basicstyle=\ttfamily\color{wongsky},columns=fixed]!komentarze!, oraz 
\lstinline[basicstyle=\ttfamily\color{wonggreen},columns=fixed]!dane tekstowe!
są wyróżnione przy użyciu koloru.
Linie tych bloków są ponumerowane na~lewym marginesie.

\begin{lstlisting}[language=Python]
if is_first_program():
    print('Hello world!')
else:
    pass  # nic nie rób
\end{lstlisting}

Tekst niniejszej rozprawy doktorskiej został przygotowany przy użyciu oprogramowania \LaTeX,
a~jej kod źródłowy dostępny jest w~Internecie pod adresem \url{https://github.com/Mishioo/dysertacja}.

\begin{fullwidth}
\printglossary[title=Wykaz skrótów, type=\acronymtype]
\end{fullwidth}

\section{Cel pracy}
