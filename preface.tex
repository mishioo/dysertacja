\chapter{Przedsłowie}\label{chapter:intro}

\section{Konwencje przyjęte w~niniejszej dysertacji}\label{intro:conventions}

Każdy chemik zauważy, że forma graficzna niniejszej rozprawy doktorskiej różni się
  od~zwykle spotykanych w~naukach chemicznych.
Zapewne najbardziej zwraca uwagę format tekstu \--- 
  jest on~inspirowany pracami Edwarda R.~Tuftego\sidecite{Tufte2001,Tufte1990,Tufte1997,Tufte2006},
  uznanego za~eksperta w~dziedzinie prezentowania informacji i~pioniera wizualizacji
  danych\sidecite{Yaffa2011}.
Tufte zauważa, że odnośniki do~spodu strony, a~tym bardziej końca tekstu,
  utrudniają czytanie, rozpraszając uwagę czytelnika.
Zamiast tego umieszcza przypisy i~komentarze na~szerokim marginesie,
  komentując żartobliwie, że \enquote{to miejsce zaplanował dla nich Bóg}.
Na~takiej szacie graficznej korzysta również główny tekst \---
  zawarty w~kolumnie węższej niż cała strona, ma szerokość uznawaną za~optymalną
  do czytania\sidecite{nanavati05}.

Esencją podejścia Tuftego do prezentowania danych jest położenie nacisku
  na~informatywność, a~nie estetykę przekazu.
Wizualizacja powinna ułatwić jak najlepsze zrozumienie danych w~jak najkrótszym czasie,
  w~jak najmniejszej przestrzeni i~przy użyciu jak najprostszej formy.
Sposób przedstawienia danych musi być jednoznaczny i~nie może ich zniekształcać
  ani wymuszać ich interpretacji.
Dane nie powinny być ozdabiane, a~wszystkie zbędne elementy, takie jak ramki czy tło,
  nie powinny znajdować się na~rysunkach, ponieważ odwracają uwagę od~treści.
Projektując graficzną stronę tego dokumentu starałem się stosować do~tych zasad.

\begin{marginfigure}
  \includesvg{palettes}
  \caption{
    Wykorzystane w~niniejszej dysertacji palety kolorów,
    będące przyjazne osobom z~zaburzeniem rozpoznawania barw.
  }
  \label{fig:palettes}
\end{marginfigure}
Jeden z paradygmatów kształcenia mówi, że nauka, będąc narzędziem poznania,
  powinna być przystępna.
Jako że kolor może nieść istotną część informacji podczas prezentacji danych,
  przy tworzeniu grafik zawartych w~tej pracy użyłem palety przyjaznej osobom
  z~zaburzeniem rozpoznawania barw.
Jako paletę jakościową użyłem schematu kolorów zaproponowanego przez Wonga\sidecite{wong11},
  natomiast jako paletę ilościową wykorzystałem viridis\sidecite{Smith2015} lub BrBG,
  zależnie od~kontekstu.
Wszystkie je prezentuję na~\cref{fig:palettes}.

Na~przystępność tekstu w~ogromnym stopniu wpływa również wybór kroju pisma.
Uważny czytelnik może zwrócić uwagę, że nie jest on jednakowy w~całej objętości
  niniejszej dysertacji.
Wzory chemiczne zapisuję czcionką zalecaną przez organizację \gls{iupac}
  zamiast dominującym, klasycznym krojem szeryfowym.
W~tym miejscu dodam, że doprecyzowując podstawniki ogólne\sidenote{%
    Czyli grupy i~atomy oznaczone jako \ch{X}, \ch{R} lub \ch{R^n}.},
  używam znaku równości (\enquote{$=$}) pokazując konkretne grupy lub atomy,
  a~znaku tożsamości (\enquote{$\equiv$}) prezentując koncepcje\sidenote{%
  Na przykład \enquote{\ch{R}~$\equiv$~alkil} albo \enquote{\ch{R^1}~$\equiv$~\ch{R^2}}.}.

Cyfry występujące w~tekście również nie zawsze wyglądają tak samo.
W~większości są to tak zwane cyfry nautyczne, zaprojektowane tak, aby wizualnie współgrały
  z~minuskułami\sidenote{Czyli małymi litrami.},
  ale do~przedstawienia wielkości fizycznych i~matematycznych użyłem cyfr zwykłych,
  aby zwiększyć ich czytelność.
Podobny zabieg zastosowałem w~przypadku numerów związków występujących w~tekście,
  które są dodatkowo wyróżnione za~pomocą pogrubienia.
Kolejnym odstępstwem jest użycie jaśniejszego koloru do~zapisu cytowań,
  co~pozwala skupić uwagę na~treści, a~nie detalach technicznych.

Pewnie jak większość ludzi parających się naukami ścisłymi ulegam pokusie używania skrótów.
Większość z~nich to~skrótowce standardowo używane przez chemików,
  jednak, dla jasności, wszystkie rozwijam przy ich pierwszym wystąpieniu.
Zgodnie z~konwencją zamieszczam również wykaz tych akronimów,
  wraz z~ich znaczeniem, na~następnych stronach.

Choć główna część tej dysertacji poświęcona jest badaniom z~dziedziny syntezy organicznej,
  podczas pracy nad nią moje zainteresowania poszerzyły się.
Stąd też czytelnik natrafi na~fragmenty dotyczące obliczeń kwantowo\-/chemicznych,
  a~nawet programowania komputerowego.
Zwłaszcza ten ostatni temat wymaga dodatkowego komentarza jako najbardziej oddalony
  od~podstawowej dyscypliny.

Gdy w~tekście pojawia się odniesienie do~nazw elementów opisywanego kodu źródłowego,
  sygnalizuję to~używając kroju czcionki \lstinline!o stałej szerokości!.
Większe bloki kodu są wydzielone z~tekstu, jak ten poniżej.
Dla poprawienia czytelności 
  \lstinline[basicstyle=\ttfamily\color{wongvermillion},columns=fixed]!słowa kluczowe!%
  \footnote{
    Słowa kluczowe to ciągi znaków zarezerwowane w~danym języku programowania,
      stanowiące część jego składni.
    Mają one z~góry określone znaczenie, definiowane przez ten język.
  },
  \lstinline[basicstyle=\ttfamily\color{wongsky},columns=fixed]!komentarze!, oraz 
  \lstinline[basicstyle=\ttfamily\color{wonggreen},columns=fixed]!dane tekstowe!
  są wyróżnione przy użyciu koloru.
Linie tych bloków są ponumerowane na~lewym marginesie.
Niestety, nie doczekały się one polskiego terminu i~nazywane są \--- 
  z~języka angielskiego \--- listingami.
\Cref{lst:example} jest przykładem takiego bloku kodu.

\begin{listing}
\begin{lstlisting}[language=Python]
if is_first_program():
    print('Hello world!')
else:
    pass  # nic nie rób
\end{lstlisting}
\caption{Przykład formatowania bloku zawierającego kod źródłowy.}
\label{lst:example}
\end{listing}

Tekst niniejszej rozprawy doktorskiej został przygotowany przy użyciu oprogramowania \LaTeX,
  a~jej kod źródłowy dostępny jest na~dołączonej płycie CD oraz w~Internecie pod adresem \repourl{}.

\section{Cel pracy}\label{intro:goal}
  \todo{Uzupełnić cel pracy.}

\section{Publikacje}\label{intro:publications}
\begin{itemize}
  \item \cite{wieclaw21}
  \item \cite{stecko18}
\end{itemize}

\section{Konferencje}\label{intro:conferences}
\begin{fullwidth}
\begin{itemize}
  \item 19\textsuperscript{th}~International Symposium \enquote{Advances in~the Chemistry of~Heteroorganic Compounds}, poster: \enquote{Captodative functionalization of~amidoesters}, Poland, Łódź, 19.10.2016~r.
  \item Ogólnopolskie Studenckie Mikrosympozjum Chemików, wystąpienie ustne: \enquote{Aktywacja amidów na~atak nukleofila jako metoda selektywnej funkcjonalizacji}, Białystok, 30.03.\-–1.04.2017~r.
  \item V~Łódzkie Sympozjum Doktorantów Chemii, poster: \enquote{Aktywacja amidów na~atak nukleofila jako metoda selektywnej funkcjonalizacji}, Polska, Łódź, 11.05.\-–12.05.2017~r.
  \item XIV Warszawskie Seminarium Doktorantów Chemików - ChemSession’17, poster: \enquote{Captodative functionalization of~amidoesters}, Polska , Warszawa, 9.06.2017~r.
  \item 26\textsuperscript{th}~ISHC Congress, poster, Regensburg, poster: \enquote{Chemoselective activation of~amide carbonyls towards nucleophilic reagents}, Niemcy, Ratyzbona, 3.\-–8.09.2017~r.
  \item XX~International Symposium \enquote{Advances in~the Chemistry of~Heteroorganic Compounds} and XVII International Symposiumon on Selected Problems of Chemistry of Acyclic and Cyclic Heteroorganic Compounds, poster: \enquote{Schwartz’s reagent mediated nojirimycin derivatives synthesis}, Polska, Łódź, 23.\-–24.11.2017~r.
  \item XI~Ogólnopolskie Sympozjum Chemii Organiczne, poster: \enquote{Synteza cukrowych pochodnych tetrazoli z~użyciem odczynnika Schwartza}, Polska, Warszawa, 8.\-–11.04.2018~r.
  \item International Congress of~Young Chemists YoungChem~2018, wystąpienie ustne: \enquote{Short and sweet: An~approach to direct synthesis of~iminosugar-derived tetrazoles}, Polska, Bydgoszcz, 10.\-–14.10.2018~r.
  \item XXI~International Symposium \enquote{Advances in~the Chemistry of~Heteroorganic Compounds}, poster: \enquote{An~approach to~direct synthesis of~iminosugar derived tetrazoles}, Polska, Łódź, 23.11.2018~r.
  \item International Symposium on Synthesis and Catalysis~2019, wystąpienie ustne: \enquote{An~approach to~direct synthesis of~iminosugar derived tetrazoles}, Portugalia, Evora, 3.\-–6.09.2019~r.
  \item XXII~International Symposium \enquote{Advances in~the Chemistry of~Heteroorganic Compounds}, poster: \enquote{Iminosugar derived tetrazoles: direct synthesis and mechanistic insights}, Polska, Łódź, 22.11.2019~r.
  \item Virtual Winter Workshop \enquote{Multiscale modeling in materials science, chemistry, and biology: How to meet, greet, and beat scale-bridging challenges}, poster: \enquote{Tesliper: Spectral Simulations Simplified}, Niemcy, Karlsruhe, 22.\--23.11.2021~r.
  
\end{itemize}
\end{fullwidth}

\section{Finansowanie}\label{intro:founding}
\begin{fullwidth}
\begin{itemize}
  \item Grant Preludium \textnumero~2017/25/N/ST5/00079 Narodowego Centrum Nauki
  \item Grant obliczeniowy PL\=/Grid
\end{itemize}
\end{fullwidth}

\begin{fullwidth}
  \printglossary[title=Wykaz skrótów, type=\acronymtype]
\end{fullwidth}
  